%%%%%%%%%%%%%%%%%%%%%%%%%%%%%%%%%%%%%%%%%
% Stylish Article
% LaTeX Template
% Version 2.2 (2020-10-22)
%
% This template has been downloaded from:
% http://www.LaTeXTemplates.com
%
% Original author:
% Mathias Legrand (legrand.mathias@gmail.com) 
% With extensive modifications by:
% Vel (vel@latextemplates.com)
%
% License:
% CC BY-NC-SA 3.0 (http://creativecommons.org/licenses/by-nc-sa/3.0/)
%
%%%%%%%%%%%%%%%%%%%%%%%%%%%%%%%%%%%%%%%%%

%----------------------------------------------------------------------------------------
%	PACKAGES AND OTHER DOCUMENT CONFIGURATIONS
%----------------------------------------------------------------------------------------

\documentclass[fleqn,10pt]{SelfArx} % Document font size and equations flushed left

\usepackage[spanish]{babel} % Specify a different language here - english by default

\usepackage{lipsum} % Required to insert dummy text. To be removed otherwise
\usepackage{tikz}
%\usepackage{newtxmath} % Cambia la fuente (pero mola)

\newcommand{\parentesis}[1]{\left( #1  \right)} 
\newcommand{\parciales}[2]{\frac{\partial #1}{\partial #2}}
\newcommand{\pparciales}[2]{\parentesis{\parciales{#1}{#2}}}
\newcommand{\ccorchetes}[1]{\left[ #1  \right]}
\newcommand{\D}{\mathrm{d}}
\newcommand{\derivadas}[2]{\frac{\D #1}{\D #2}}

\newcommand{\In}{\mathbf{I}}
\newcommand{\Jn}{\mathbf{J}}

\newcommand{\Rn}{\mathbf{R}}
\newcommand{\Ln}{\mathbf{L}}
\newcommand{\Un}{\mathbf{U}}
\newcommand{\Bn}{\mathbf{B}}
\newcommand{\Omegan}{\boldsymbol{\Omega}}


\newcommand{\rn}{\mathbf{r}}

\newcommand{\Ical}{\mathcal{I}}
\newcommand{\Ocal}{\mathcal{O}}
%----------------------------------------------------------------------------------------
%	COLUMNS
%----------------------------------------------------------------------------------------

\setlength{\columnsep}{0.55cm} % Distance between the two columns of text
\setlength{\fboxrule}{0.75pt} % Width of the border around the abstract

%----------------------------------------------------------------------------------------
%	COLORS
%----------------------------------------------------------------------------------------

\definecolor{color1}{RGB}{0,0,90} % Color of the article title and sections
\definecolor{color2}{RGB}{0,20,20} % Color of the boxes behind the abstract and headings

\definecolor{darkgreen}{rgb}{0.15, 0.4, 0.0} % Valores de 0 a 1

\definecolor{naranja}{RGB}{255, 171, 0} % Valores de 0 a 1
\definecolor{morado}{RGB}{151, 0, 151} % Valores de 0 a 1

%----------------------------------------------------------------------------------------
%	HYPERLINKS
%----------------------------------------------------------------------------------------

\usepackage{hyperref} % Required for hyperlinks

\hypersetup{
	hidelinks,
	colorlinks,
	breaklinks=true,
	urlcolor=color2,
	citecolor=color1,
	linkcolor=color1,
	bookmarksopen=false,
	pdftitle={Title},
	pdfauthor={Author},
}

\usepackage[font=small, justification=centering]{caption}  % Configura las captions

%----------------------------------------------------------------------------------------
%	ARTICLE INFORMATION
%----------------------------------------------------------------------------------------

\JournalInfo{Trabajo voluntario} % Journal information
\Archive{Fisica nuclear y partículas} % Additional notes (e.g. copyright, DOI, review/research article)

\PaperTitle{Difusión elástica resonante} % Article title

\Authors{Daniel Vázquez Lago\textsuperscript{1}*} % Authors
\affiliation{\textsuperscript{1}\textit{Facultad de Física, Universidad Santiago de Compostela, Galicia, España}} % Author affiliation 
\affiliation{*\textbf{Correo del autor}: danielvazquezlago@gmail.com, daniel.vazquez.lago@rai.usc.es} % Corresponding author

\Keywords{Difusión elástica resonante, Formalismo R-Martricial, Autoestados, Breit-Wigner} % Keywords - if you don't want any simply remove all the text between the curly brackets
\newcommand{\keywordname}{Palabras clave} % Defines the keywords heading name

%----------------------------------------------------------------------------------------
%	ABSTRACT
%----------------------------------------------------------------------------------------

\Abstract{En este trabajo estudiaremos las difusiones elásticas resonantes y sus aplicaciones en el ámbito particular de la física nuclear, tratando de responder a las siguientes preguntas: ¿Qué es una difusión elástica resonante?¿Cómo se comporta la sección eficaz en una difusión elástica resonante?¿Cual es el interés de las dispersiones elásticas resonantes?¿Qué las diferencia de una difusión inelástica? Tratando de responder a estas preguntas estudiaremos entonces diferentes modelos matemáticos que nos permiten introducir las secciones eficaces y diferentes aplicaciones para ver cual es la potencia e interés de las difusiones elásticas resonantes.}

%----------------------------------------------------------------------------------------

\begin{document}

\maketitle % Output the title and abstract box

\tableofcontents % Output the contents section

\thispagestyle{empty} % Removes page numbering from the first page


\setlength{\parskip}{1.5mm} % Cambia el espacio entre párrafos


%----------------------------------------------------------------------------------------
%	ARTICLE CONTENTS
%----------------------------------------------------------------------------------------

\section{Introducción}

Para tratar de entender un poco el fenómeno de la difusión elástica resonante primero tenemos que tratar un poco la historia de las difusiones. Las difusiones o colisiones entre partículas se empezaron a estudiar a principios del siglo XX por Marsden, Geiger, Rutherford, estudiando como se comportaba un haz partículas $\alpha$ al colisionar con diferentes planchas de metal, observando que parte de las partículas interaccionaban fuertemente con el metal cuando incidía con ángulos más allá de los 90$^\circ$. De este resultado Rutherford dedujo que los núcleos estaban formados por un núcleo cargado positivamente y un halo de partículas poco pesadas cargadas negativamente orbitando a su alrededor, que denominó electrones. Mas tarde Darwin derivó una fórmula (usando la mecánica clásica) que describía la sección eficaz de la difusión elástica de Rutherford, que es el nombre que se le da a las difusiones elásticas entre partículas cargadas. Esta viene dada por:

\begin{equation}
	\parentesis{\derivadas{\sigma(\theta)}{\Omega}}_{\text{Rutherford}} = \parentesis{\frac{1}{ 4 \pi \epsilon_0}}^2 \frac{Z_1Z_2 e^2}{(4E)^2\sin^4 (\theta/2)}
\end{equation}
donde $Z_2$ es la carga el haz incidente y $E$ su energía. Aunque esta sección eficaz se dedujo usando la mecánica clásica, también es posible deducirla a través de la mecánica cuántica (y se obtiene la misma fórmula, siempre que no se tenga en cuenta el espín y correcciones relativistas). Sin embargo pocos años después en el estudio de otras difusiones se obtuvieron secciones eficaces con anomalías. Estas anomalías se manifestaron generalmente en forma de picos para energías muy particulares.

El descubrimiento del neutrón (1932) cambio por completo el paradigma de la física nuclear, ya que estas partículas con carga neutra no podían seguir la dispersión de Rutheford. El comportamiento de las reacciones en las que se usan haces de neutrones deberían venir dadas por el desarrollo de ondas parciales que se puede hacer a través de la ecuación de Schrödinger (que no habría sido publicada hasta 1926) por lo que dichas reacciones ahora son un fenómeno probabilístico, de tal modo que en los valores medios obtenidos dependen de las diferentes posibles reacciones (de \textit{todas} las posibles reacciones y de las \textit{interacciones estadísticas} entre estas. De esta forma el estudio de las difusiones pasa de ser un problema sencillo en la mecánica clásica a un problema colosal en el que intervienen varios procesos a la vez que además interfieren entre sí.

Por ejemplo, cuando un neutrón $n$ incide en un núcleo atómico $X$ ¿Que podría pasar? Una posibilidad sería la típica difusión elástica $n+X\rightarrow  n+X$ en el que los momentos de cada partícula cambian pero no el momento global. Otra posibilidad es que el neutrón incidente entre en el núcleo formando un isótopo de $X$ tal que $n+X\rightarrow Y$. A esto lo llamamos la difusión inelástica. En el caso de los nuetrones incidiendo en una partícula solo hay una posible sección eficaz. Las secciones eficaces de ambos procesos son diferentes, y la sección eficaz global es la suma de ambas:
\begin{equation}
	\sigma_{tot} = \sigma_{el} + \sigma_{inel}
\end{equation}
Ahora surge la pregunta. ¿Solo existe una forma de que ocurra la reacción $n+X\rightarrow n +X$?¿O por el contrario hay alguna otra forma de que ocurra? En realidad uno podría pensar 



\section{Comportamiento de la difusión elástica resonante} 


%----------------------------------------------------------------------------------------
%	REFERENCE LIST
%----------------------------------------------------------------------------------------

\newpage
\phantomsection
\bibliographystyle{unsrt}
\bibliography{sample.bib}

%----------------------------------------------------------------------------------------

\end{document}