%%%%%%%%%%%%%%%%%%%%%%%%%%%%%%%%%%%%%%%%%
% Stylish Article
% LaTeX Template
% Version 2.2 (2020-10-22)
%
% This template has been downloaded from:
% http://www.LaTeXTemplates.com
%
% Original author:
% Mathias Legrand (legrand.mathias@gmail.com) 
% With extensive modifications by:
% Vel (vel@latextemplates.com)
%
% License:
% CC BY-NC-SA 3.0 (http://creativecommons.org/licenses/by-nc-sa/3.0/)
%
%%%%%%%%%%%%%%%%%%%%%%%%%%%%%%%%%%%%%%%%%

%----------------------------------------------------------------------------------------
%	PACKAGES AND OTHER DOCUMENT CONFIGURATIONS
%----------------------------------------------------------------------------------------

\documentclass[fleqn,10pt]{SelfArx} % Document font size and equations flushed left

\usepackage[spanish]{babel} % Specify a different language here - english by default

\usepackage{lipsum} % Required to insert dummy text. To be removed otherwise

%----------------------------------------------------------------------------------------
%	COLUMNS
%----------------------------------------------------------------------------------------

\setlength{\columnsep}{0.55cm} % Distance between the two columns of text
\setlength{\fboxrule}{0.75pt} % Width of the border around the abstract

%----------------------------------------------------------------------------------------
%	COLORS
%----------------------------------------------------------------------------------------

\definecolor{color1}{RGB}{0,0,90} % Color of the article title and sections
\definecolor{color2}{RGB}{0,20,20} % Color of the boxes behind the abstract and headings

%----------------------------------------------------------------------------------------
%	HYPERLINKS
%----------------------------------------------------------------------------------------

\usepackage{hyperref} % Required for hyperlinks

\hypersetup{
	hidelinks,
	colorlinks,
	breaklinks=true,
	urlcolor=color2,
	citecolor=color1,
	linkcolor=color1,
	bookmarksopen=false,
	pdftitle={Title},
	pdfauthor={Author},
}

%----------------------------------------------------------------------------------------
%	ARTICLE INFORMATION
%----------------------------------------------------------------------------------------

\JournalInfo{Trabajo voluntario} % Journal information
\Archive{Fisica nuclear y partículas} % Additional notes (e.g. copyright, DOI, review/research article)

\PaperTitle{Difusión elástica resonante} % Article title

\Authors{Daniel Vázquez Lago\textsuperscript{1}*} % Authors
\affiliation{\textsuperscript{1}\textit{Facultad de Física, Universidad Santiago de Compostela, Galicia, España}} % Author affiliation 
\affiliation{*\textbf{Correo del autor}: danielvazquezlago@gmail.com, daniel.vazquez.lago@rai.usc.es} % Corresponding author

\Keywords{Difusión elástica resonante } % Keywords - if you don't want any simply remove all the text between the curly brackets
\newcommand{\keywordname}{Keywords} % Defines the keywords heading name

%----------------------------------------------------------------------------------------
%	ABSTRACT
%----------------------------------------------------------------------------------------

\Abstract{En este trabajo vamos a hablar de las difusiones elásticas resonantes, aplicaciones y ejemplos.}

%----------------------------------------------------------------------------------------

\begin{document}

\maketitle % Output the title and abstract box

\tableofcontents % Output the contents section

\thispagestyle{empty} % Removes page numbering from the first page

%----------------------------------------------------------------------------------------
%	ARTICLE CONTENTS
%----------------------------------------------------------------------------------------


\section{Introducción}

Pequeña introducción sobre por qué las difusiones (scattering) son importantes, y por que hay necesidad de estudiarlas (poner ejemplos). 


\section{Repaso histórico}

Hay que comentar como se descubrió la dipsersión elástica resonante, entender cual fue la anomalía vista y como se trato de solucionar.

\subsection{Dispersión de Rutherford}

\subsection{Anomalía en la sección eficaz}

\section{Difusión elástica resonante}

%Aquí tenemos que desarrollar la matemática necesaria para solucionar el problema: como debe ser, por que así, ver como es la forma teórica y la experimental (comparación). 

La difusión elástica resonante es un tipo de scattering en el que una partícula (átomo, fotón, electrón...) colisiona con otra generando un estado resonante, es decir, un estado con un tiempo de vida medio corto, que se desintegra emitiendo las mismas partículas incidentes pero con diferente momento y ángulo, es decir, actúa como una difusión elástica mediada por una partícula resonante, de ahí el nombre del fenómeno. 

\section{Aplicaciones}

Molaría ver aplicaciones industriales reales, cuanto dinero se mueve con esto, futuro de la industria...

\subsection{Aplicaciones: física nuclear}

Cuales son las aplicaciones en la física nuclear (estados nucleares excitados, valores de espines nucleares, ver si se han verificado (o contradicho) resultados experimentales y resultados teóricos. 


\section{Conclusiones}


%------------------------------------------------







%----------------------------------------------------------------------------------------
%	REFERENCE LIST
%----------------------------------------------------------------------------------------

\newpage
\phantomsection
\bibliographystyle{unsrt}
\bibliography{sample.bib}

%----------------------------------------------------------------------------------------

\end{document}