\chapter{Introducción y detección de partículas}


\section{Introducción a la física de partículas}


La materia está formada por los \textbf{fermiones} que son las partículas elementales de espín 1/2. Los \textbf{bosones vectoriales} con espín 1 son mediadores de las fuerzas \textbf{fuerte, débil y electromagnética}. El \textbf{bosón escalar de Higgs}, que tiene espín 0, dota de masa a los fermiones y a los bosones vectoriales de la fuerza débil.

Los fermiones a su vez se dividen en \textbf{quarks} si  interaccionan fuertemente y \textbf{leptones} si no interaccionan fuertemente. Los fermiones se rigen por la ecuación de Dirac, en última instancia están evaluados por espinores, y por tanto tienen asociado una \textbf{antipartícula}. Su $f$ es un fermión, denotamos por $\bar{f}$ al antifermión, mientras que los leptones se indican con $-$ ($e^-,\mu^-,\tau^-$) y los antileptones con $+$ ($e^+,\mu^+,\tau^+$). 

Los fermiones (tanto leptones como quarks) aparecen en \textbf{tres generaciones} o familias:

\begin{equation}
	\binom{\nu_e}{e} , 
	\binom{\nu_\mu}{\mu} , 
	\binom{\nu_\tau}{\tau} 
\end{equation}
\begin{equation}
	\binom{u}{d} , 
	\binom{c}{s} , 
	\binom{t}{b}
\end{equation}
A veces decimos que las partículas se presentan con \textbf{sabores}. Por ejemplo el neutrino $\nu$ se presentaría en sabor electrónico, muónico y tauónico. Además, en cada generación, tanto leptones como quarks, se agrupan en dupletes con posiciones arriba y abajo. Hasta la fecha no hay ninguna motivación de por qué hay tres generaciones, es un hecho experimental. Las tres generaciones de leptones se comportan con \textbf{universalidad} frente a interacciones si se tiene en cuenta que sus masas son diferentes.

\subsection{Leptones}

Como hemos dicho, tenemos 3 familias de leptones, cuya única diferencia son las masas entre ellos:

\begin{equation}
	\binom{\nu_e}{e} , 
	\binom{\nu_\mu}{\mu} , 
	\binom{\nu_\tau}{\tau} 
\end{equation}
Las interacciones conservan siempre el \textbf{número leptónico} $L$, que es el número de leptones menos el de anti-leptones. Por ejemplo, en la desintegración $\beta$:

\begin{equation*}
	\ce{n -> p + e^- + \bar{\nu}_e}
\end{equation*}
conserva el número leptónico, ya que el número leptónico del electrón $e$ es 1 y el del anti-neutrino electrónico es -1, por lo que el número leptónico inicial es cero y el final también. 

El electrón es el leptón cargado más ligero, y es estable. El muón $\mu$ tiene un tiempo de vida medio de $2.6 \times 10^{-6}$ s, mientras que el taón $\tau$ tiene un tiempo de vida medio de $2.9 \times 10^{-13}$ s. Los electrones interaccionan con los átomos de la materia, ionizando y radiando fotones por \textit{Bremsstrauhlung}. Los muones también radian fotones, pero debido a que esta radiación es inversamente proporcional a la masa no interacciona mucho, pudiendo llegar a recorrer metros (partículas penetrantes). Los tauones, debido a su baja vida media, solo pueden recorrer centímetros.

Los neutrinos por otro lado son partículas raras, ya que el modelo estándar (SM) predice que no tienen masa, aunque experimentalmente si la deben tener ya que si no no podrían observarse las oscilaciones de neutrinos. Su masa no está definida, pero si acotada entorno a 1eV (el electrón pesa 511 keV). Además los neutrinos son estables y solo interactuan débilmente, y es la única prueba de que el SM es incorrecto.

\subsection{Quarks}

Las tres generaciones de quarks son

\begin{equation}
	\binom{u}{d} , 
	\binom{c}{s} , 
	\binom{t}{b}
\end{equation}
con os siguientes nombres: $c$ encanto (charm), $s$ extrañeca (strange); $u$ arriba (up) y $d$ abajo (down); y las copias de los anteriores con más masa $t$ arriba (top), $b$ belleza (beauty). Este último no forma hadrones (protones, neutrones...).

Una de las clasificaciones mas comunes de los quarks viene dada por su masa: $u,s,d$ son los ligeros; mientras que $c,b$ son pesados. El top $t$ es incluso más pesado. Otro carácter que los diferencia de las partículas comunes es que tienen carga fraccionaria $+2/3$ los de arriba ($u,c,t$) y $-1/3$ los de abajo ($d,s,b$) en unidades de $e$.

Los quarks no son libres, sólo aparecen en partículas compuestas, los \textbf{hadrones}, en un fenómeno denominado confinamiento. Además aparecen con tres cargas fuertes o de color: rojo, verde y azul (r,g,b). n mismo quark existe en tres colores, por lo que hay 18 quarks diferentes. Los únicos hadrones que existen son neutros al color, digamos que <<blancos>>, al igual que los átomos son eléctricamente neutros.

Los \textbf{hadrones} por otro lado se clasifican en:

\begin{itemize}
	\item \textbf{Bariones}: formados por 3 quarks.
	\item \textbf{Mesones}: formados por un quark y un antiquark.
\end{itemize}
Cosecuentemente la carga de los bariones es entera. Los protones $uud$ y neutrones $udd$ son bariones. Al igual que con los leptones, las interacciones conservan el \textbf{número bariónico} $B$. El protón, que es el barión más ligero, es estable, a diferencia del resto de hadrones. La vida media depende de dos factores: del tipo de desintegración permitida (fuerte, electromagnética, débil) del espacio fásico permitido.

\subsection{Mediadores de fuerza}

La interacción se describe en la física de partículas como el intercambio de un medidor o portador de la fuerza entre corrientes de fermiones. Cada portador interviene según la carga (como la carga eléctrica o la carga de color) de los fermiones que se acopla. Los portadores son bosones vectoriales y obedecen la ecuación de Maxwell. 

La masa del bosón determina el rango (en distancia) de la interacción. El rango de las interacciones electromagnéticas es infinita porque la masa de los fotones es cero. El rango de las interacciones débiles es de $<10^{-18}$m. A esta distancia (rango de los 100 GeV del bosón Z) se conoce como \textit{escala electrodébil}. Los bosones $W^{\pm}$ son cargados, y sus interacciones se conocen como \textbf{corrientes cargadas} (CC) y las interacciones del bosón $Z^0$ se denominan \textbf{corrientes neutras} (NC). Por otro lado existen 8 tipos de gluones que cambian el color de los quarks. Estos son eléctricamente neutros y de masa nula. Como los gluones tienen carga de color e interaccionan entre sí, reducen la interacción fuerte al rango de fm.

Las interacciones electromagnéticas y fuertes conservan el sabor (es decir el número de partículas menos el de antipartículas de un sabor debe ser conservado). La interacción débil viola la paridad (las cargadas de forma máxima)´, y la carga y paridad (CP) mínimamente (solo en corrientes cargadas en hadrones).


\section{Detección de partículas}

La historia de la física de partículas está ligada a la de los aceleradores. Para producir una partícula necesitamos una energía igual a su masa $E=mc^2$, y para explorarla una longitud menos que la de su tamaño por relación $p=\hbar/\lambda$. Se pueden acelerar las partículas estables cargadas: $e^-,e^+,p,\bar{p}$. También podemos crear haces secundarios al golpear un haz de protones contra un blanco (mesones $\pi$ y $K$) que se pueden seleccionar en un rango de energía y momento. Si además se dejan desintegrase y se filtran se obtiene un haz de neutrinos.

\subsection{Tipos de experimentos}

Existen dos tipos de experimentos (principalmente):

\begin{itemize}
	\item De \textbf{blanco fijo} donde un haz se hace incidir sobre un blanco.
	\item De \textbf{colisión} donde dos haces se cruzan en determinados puntos. 
\end{itemize}
La energía disponible en el centro de masas se denota por $\sqrt{s}$ en ambos experimentos. Para un experimento de blanco fijo $s=(E,\pn)+(m_a,0)$ donde ($E,\pn$) es el cuadrimomento de la partícula incidente con masa $m_b$ y $(m_a,0)$ es la masa de la partícula del blanco (usualmente nucleones). En este caso la energía en el cendro de masas 
\begin{equation}
	\sqrt{s} = \sqrt{(E+m_b)^2 - \pn^2 } = \sqrt{m_b^2 + 2 m_a m_bE + m_a^2}
\end{equation}
que depende de $\sqrt{E}$ si las masas son despreciables. 

Para un experimento de colisión $s=(E_a,\pn_a)+(E_b,\pn_b)$. En el caso de partículas relativistas de igual masa que colisionan de frente la energía en el centro de masas $\sqrt{s}= 2 E$.


\subsection{Introducción a aceleradores}

Los haces están compuestos de \textbf{paquetes} con una densidad elevada de partículas tiene del orden de $10^{11}$ protones. Existen dos tipos de aceleradores principales:

\begin{itemize}
	\item Lineales: donde el acelerador es recto, un tubo de haz en vacío, que las paquetes recorren una vez.
	\item Circulares: los paquetes giran en el tubo de vacío del haz numerosas vueltas.
\end{itemize}

\subsection{Parámetros de un colisionador}

Existen dos parámetros fundamentales en un colisionador:

\begin{itemize}
	\item \textbf{Energía} en el centro de masas $\sqrt{s}$ que determina la masa de las partículas que podemos crear.
	\item \textbf{Luminosidad} o el número de partículas que se cruzan entre ellas por unidad de área y tiempo. Determina el número de interacciones en cada cruce-.
\end{itemize}

Para dos vagones con $n_1$ y $n_2$ partículas que se cruzan entre sí en la dirección $z$ con una frecuencia $\nu$ y que tienen una sección trasversal $\sigma_T$:

\begin{equation}
	\Lcal = \nu \frac{n_1n_2}{\sigma_T}
\end{equation}
Si un tipo de interacción tiene sección eficaz $\sigma$ esperamos una frecuencia de interacciones $n=\sigma \Lcal$. En general $\Lcal(t)$ depende del tiempo. A $\Lcal(t)$ le llamamos \textbf{luminosidad instantánea} y a $\int_t \Lcal (t)\D t$ le llamamos \textbf{luminosidad integrada}. A las interacciones que se producen en el cruce de los vagones se las conoce como evento. El número de interacciones por evento es una variable aleatoria con distribución de Poisson de media $\sigma \Lcal$ (y por tanto la incertidumbre es la propia media).

\subsection{Interacción de las partículas con la materia}

Las interacciones de las partículas pueden dividirse en:

\begin{itemize}
	\item Interacciones de las partículas cargadas.
	\item Interacciones electromangéticas de electrones y fotones.
	\item Interacciones fuertes de hadrones.
\end{itemize}

\subsubsection{Interacciones de las partículas cargadas}

Las partículas cargadas relativistas interaccionan electromagnéticamente con los electrones de los átomos de la materia y pierden energía por ionización. Esta pérdida por distancia recorrida viene dada por las fórmulas de Bethe-Block (1930s):

\begin{equation}
	\derivadas{E}{x} \simeq - \frac{4\pi \hbar^2 \alpha^2}{m_e} \frac{n_e}{\beta^2} \ccorchetes{\ln \parentesis{\frac{2m_e c^2 \beta^2 \gamma^2}{I_e}} - \beta^2}
\end{equation}
donde $v=\beta c$ es la velocidad, $\gamma$ el factor el Lorentz, $n_e$ la de densidad de electrones, $I_e$ el potencial de ionización. Se trata de un valor promedio. La distribucón $\D E / \D x$ está relacionada con la fluctuación del número de colisiones de la partículas con los electores de los átomos.

La densidad de electrones $n_e=\rho N_A Z/A$ donde $\rho$ es la densidad, $A$ la masa atómica (g/mol), $Z$ el número atómico y $N_A$ el número de Avogadro. Si reescribimos:

\begin{equation}
	\frac{1}{\rho} \derivadas{E}{x} \simeq - K \frac{Z}{A\beta^2} \ccorchetes{\ln \parentesis{\frac{2m_ec^2\beta^2 \gamma^2}{I_e}} - \beta^2}
\end{equation}
donde $K$ es el siguiente factor:

\begin{equation}
	K = \frac{4\pi \hbar^2 \alpha^2 N_A}{m_e} = 0.307 \ \unit{MeV cm^2 / mol}
\end{equation}
donde $Z/A$ es prácticamente constante en la mayoría de los átomos. La dependencia con el material es principalmente proporcional a su densidad. 

El recorrido medio de una partícula con $\beta$ hasta que se detiene en un medio se denomina \textbf{rango} de penetración. La pérdida de la energía de la velocidad de la partícula $\beta$. Distinguimos tres regiones:

\begin{itemize}
	\item La pérdida es más intensa para baja $\beta$ (de la dependencia $1/\beta^2$). La inionización es mayor al final de la trayectoria, cuando las velocidades es muy pequeña. Esta región se conoce como \textbf{pico de Bragg}.
	\item En el rango $\beta \gamma$ de 1-10, la energía perdida es mínimo, esa región se denomina \textbf{mip} (\textit{minimun ionizing particle}). Un muón de 10 GeV en hierro pierde en promedio $13$ MeV/cm y su rango es de varios metros. 
	\item En rango $\beta \gamma>100$, la perdida de aumenta de forma logaritmico. A partir aquí los efectos de radiación son relevantes.
\end{itemize}
La fórmula calcula el valor promedio. La distribución $\D E / \D x$ está relacionada con la fluctuación del número de colisiones de la partícula con los electrones de los átomos. Pérdida de energías en la TPC (\textit{Time Projection Chamber}) de ALICE. Los electrones pierden también energía por radiión \textbf{bremsstrahlung}.

\subsubsection{Interacciones de las electromagnéticas de fotones y electrones}

Las partículas cargadas pueden radiar fotones por la interacción electromagnética con los protones de los núcleos. Esta radiación se llama \textbf{bremsstrahlung}

\begin{equation}
	e^- + (A,Z) \rightarrow e^- + \gamma + (A,Z)
\end{equation}
Esta radición empieza a ser dominante a partir de una \textbf{energía crítica} $E_c \sim 800 /Z$ MeV, antes domina la ionización. Este proceso se puede calcular en QED y su sección eficaz:

\begin{equation}
	\sigma_b  \propto E/m^2 
\end{equation}
afecta más a los electrones que a los muones por un factor $(m_e/m_\mu)^2$. Los muones por debajo de 100 GeV pierden energía principalmente por ionización. La pérdida de energía por bremsstralung por encima de $E_c$ puede expresarse:

\begin{equation}
	\derivadas{E}{x} = - \frac{E}{X_0} \quad E(x)=E_0 e^{-x/X_0}
\end{equation}
donde $X_0$ se denomina \textbf{longitud de radiación} y $E_0$ es la energía inicial del electrón. $X_0$ depende del material, notar que $X_0=n\sigma_b$ donde $n$ es la densidad de núcleos.

Las interacciones de lo fotones con la materia dependen de su rango energía, por debajo MeV domino el efecto foto-eléctrico, en el rango MeVs, la dispersión Compton, y por encima 10 MeV, la producción de pares. La sección eficaz de producción de pares crece rápidamente desde el umbral de producción y puede aproximarse:

\begin{equation}
	\sigma_\gamma \simeq \frac{7}{9} \frac{1}{nX_0}
\end{equation}
donde $n$ es la densidad de núcleos. La cantidad $\lambda=1/(n\sigma_\gamma)$ es el \textbf{camino libre medio}, que vale $\lambda\simeq7/9 X_0$ y que nos indica la cantidad de fotones que se pierden en un haz monoenergético de intensidad $I$:

\begin{equation}
	\derivadas{I}{x} = - \frac{I}{\lambda} \tquad I(x) = I_0 e^{-x/\lambda}
\end{equation}
Por lo tanto la longitud de radiación caracteriza la pérdida de energía de electrones y conversión de fotones en pares para partículas por encima de $\sim$ 10 MeV. Los electrones o fotones de alta energía al atravesar un medio de $X_0$ producen una cascada electromagnética. 

\subsubsection{Interacciones fuertes de los hadrones}

Los hadrones cargadas (protones, piones, kaones) pierden energía por ionización. También por interacciones fuertes con los núcleos de la materia. Las interacciones se caracterizan con la longitud de interacción, $\lambda_I$, que es la distancia media entre interacciones fuertes $\lambda > X_0$. Las intearcciones de hadrones producen una cascada hadrónica. 

Las cascadas hadrónicas son más variables que electrmagnéticas, dado que en ellas se pueden producir más tipos de partículas, y también $\pi^0$ qeu se desintegran electromagnéticamente $\pi^0 \rightarrow \gamma \gamma$. Lo que da lugar a su vez a una cascada electromagnética dentro de la hadrónica. Lo que da lugar a una dispersión en la energía repartida entre los dos tipos de cascadas. Una parte de la energía también se pierde en forma de excitaciones y roturas nucleares.

\subsection{Detectores las partículas}

Los detectores de partículas usan como base la interacción de las partículas con la materia, principalmente ionización y radiación. Los detectores se dividen en:

\begin{itemize}
	\item \textbf{Detectores de trazas:} determinan las trayectorias de las partículas cargadas. Sirven para medir generalmente el momento y determinar los vértices de desintegración. Los detectores de trazas están habitualmente inmersos em campos magnéticos con lo que el momento se determina a partir de su curvatura.
	\item \textbf{Calorímetros:} sirven para la energía de las partículas, principalmente electrones/fotones y hadrones.
\end{itemize}


\section{Sistema de disparo, procesado y análisis de datos}

Los detectores 



\section{Fundamentos}


\section{Ecuación de Dirac}