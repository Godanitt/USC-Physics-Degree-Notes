\chapter{Introducción y detección de partículas}


\section{Introducción a la física de partículas}


La materia está formada por los \textbf{fermiones} que son las partículas elementales de espín 1/2. Los \textbf{bosones vectoriales} con espín 1 son mediadores de las fuerzas \textbf{fuerte, débil y electromagnética}. El \textbf{bosón escalar de Higgs}, que tiene espín 0, dota de masa a los fermiones y a los bosones vectoriales de la fuerza débil.

Los fermiones a su vez se dividen en \textbf{quarks} si  interaccionan fuertemente y \textbf{leptones} si no interaccionan fuertemente. Los fermiones se rigen por la ecuación de Dirac, en última instancia están evaluados por espinores, y por tanto tienen asociado una \textbf{antipartícula}. Su $f$ es un fermión, denotamos por $\bar{f}$ al antifermión, mientras que los leptones se indican con $-$ ($e^-,\mu^-,\tau^-$) y los antileptones con $+$ ($e^+,\mu^+,\tau^+$). 

Los fermiones (tanto leptones como quarks) aparecen en \textbf{tres generaciones} o familias:

\begin{equation}
	\binom{\nu_e}{e} , 
	\binom{\nu_\mu}{\mu} , 
	\binom{\nu_\tau}{\tau} 
\end{equation}
\begin{equation}
	\binom{u}{d} , 
	\binom{c}{s} , 
	\binom{t}{b}
\end{equation}
A veces decimos que las partículas se presentan con \textbf{sabores}. Por ejemplo el neutrino $\nu$ se presentaría en sabor electrónico, muónico y tauónico. Además, en cada generación, tanto leptones como quarks, se agrupan en dupletes con posiciones arriba y abajo. Hasta la fecha no hay ninguna motivación de por qué hay tres generaciones, es un hecho experimental. Las tres generaciones de leptones se comportan con \textbf{universalidad} frente a interacciones si se tiene en cuenta que sus masas son diferentes.

\subsection{Leptones}

Como hemos dicho, tenemos 3 familias de leptones, cuya única diferencia son las masas entre ellos:

\begin{equation}
	\binom{\nu_e}{e} , 
	\binom{\nu_\mu}{\mu} , 
	\binom{\nu_\tau}{\tau} 
\end{equation}
Las interacciones conservan siempre el \textbf{número leptónico} $L$, que es el número de leptones menos el de anti-leptones. Por ejemplo, en la desintegración $\beta$:

\begin{equation*}
	\ce{n -> p + e^- + \bar{\nu}_e}
\end{equation*}
conserva el número leptónico, ya que el número leptónico del electrón $e$ es 1 y el del anti-neutrino electrónico es -1, por lo que el número leptónico inicial es cero y el final también. 

El electrón es el leptón cargado más ligero, y es estable. El muón $\mu$ tiene un tiempo de vida medio de $2.6 \times 10^{-6}$ s, mientras que el taón $\tau$ tiene un tiempo de vida medio de $2.9 \times 10^{-13}$ s. Los electrones interaccionan con los átomos de la materia, ionizando y radiando fotones por \textit{Bremsstrauhlung}. Los muones también radian fotones, pero debido a que esta radiación es inversamente proporcional a la masa no interacciona mucho, pudiendo llegar a recorrer metros (partículas penetrantes). Los tauones, debido a su baja vida media, solo pueden recorrer centímetros.

Los neutrinos por otro lado son partículas raras, ya que el modelo estándar (SM) predice que no tienen masa, aunque experimentalmente si la deben tener ya que si no no podrían observarse las oscilaciones de neutrinos. Su masa no está definida, pero si acotada entorno a 1eV (el electrón pesa 511 keV). Además los neutrinos son estables y solo interactuan débilmente, y es la única prueba de que el SM es incorrecto.

\subsection{Quarks}

Las tres generaciones de quarks son

\begin{equation}
	\binom{u}{d} , 
	\binom{c}{s} , 
	\binom{t}{b}
\end{equation}
con os siguientes nombres: $c$ encanto (charm), $s$ extrañeca (strange); $u$ arriba (up) y $d$ abajo (down); y las copias de los anteriores con más masa $t$ arriba (top), $b$ belleza (beauty). Este último no forma hadrones (protones, neutrones...).

Una de las clasificaciones mas comunes de los quarks viene dada por su masa: $u,s,d$ son los ligeros; mientras que $c,b$ son pesados. El top $t$ es incluso más pesado. Otro carácter que los diferencia de las partículas comunes es que tienen carga fraccionaria $+2/3$ los de arriba ($u,c,t$) y $-1/3$ los de abajo ($d,s,b$) en unidades de $e$.

Los quarks no son libres, sólo aparecen en partículas compuestas, los \textbf{hadrones}, en un fenómeno denominado confinamiento. Además aparecen con tres cargas fuertes o de color: rojo, verde y azul (r,g,b). n mismo quark existe en tres colores, por lo que hay 18 quarks diferentes. Los únicos hadrones que existen son neutros al color, digamos que <<blancos>>, al igual que los átomos son eléctricamente neutros.

Los \textbf{hadrones} por otro lado se clasifican en:

\begin{itemize}
	\item \textbf{Bariones}: formados por 3 quarks.
	\item \textbf{Mesones}: formados por un quark y un antiquark.
\end{itemize}
Cosecuentemente la carga de los bariones es entera. Los protones $uud$ y neutrones $udd$ son bariones. Al igual que con los leptones, las interacciones conservan el \textbf{número bariónico} $B$. El protón, que es el barión más ligero, es estable, a diferencia del resto de hadrones. La vida media depende de dos factores: del tipo de desintegración permitida (fuerte, electromagnética, débil) del espacio fásico permitido.

\subsection{Mediadores de fuerza}

La interacción se describe en la física de partículas como el intercambio de un medidor o portador de la fuerza entre corrientes de fermiones. Cada portador interviene según la carga (como la carga eléctrica o la carga de color) de los fermiones que se acopla. Los portadores son bosones vectoriales y obedecen la ecuación de Maxwell. 

La masa del bosón determina el rango (en distancia) de la interacción. El rango de las interacciones electromagnéticas es infinita porque la masa de los fotones es cero. El rango de las interacciones débiles es de $<10^{-18}$m. A esta distancia (rango de los 100 GeV del bosón Z) se conoce como \textit{escala electrodébil}. Los bosones $W^{\pm}$ son cargados, y sus interacciones se conocen como \textbf{corrientes cargadas} (CC) y las interacciones del bosón $Z^0$ se denominan \textbf{corrientes neutras} (NC). Por otro lado existen 8 tipos de gluones que cambian el color de los quarks. Estos son eléctricamente neutros y de masa nula. Como los gluones tienen carga de color e interaccionan entre sí, reducen la interacción fuerte al rango de fm.

Las interacciones electromagnéticas y fuertes conservan el sabor (es decir el número de partículas menos el de antipartículas de un sabor debe ser conservado). La interacción débil viola la paridad (las cargadas de forma máxima)´, y la carga y paridad (CP) mínimamente (solo en corrientes cargadas en hadrones).



\section{Fundamentos}

\subsection{Unidades}

En física de partículas usamos el \textbf{sistema de unidades naturales}, donde $\hbar=c=1$. Cabe destacar que el factor de conversión $\hbar c=0.197$ GeV fm. Para convertir del sistema naturales al sistema internacional se añaden los factores $\hbar$ y $c$ correspondiente. La siguiente tabla muestra como cambian las unidades:

\begin{table}[h!] \centering
	\begin{tabular}{c|ccc} 
		Cantidad & kg, m, s & $\hbar,c$,GeV & NU \\ \hline
		Energía & kg m$^2$ s$^{-2}$ &  GeV & GeV \\
		Momento & kg m s$^{-1}$ & GeV/c & GeV \\
		Masa & kg &  GeV/c$^2$ & GeV \\
		Tiempo & s & $\hbar$/GeV & GeV$^{-1}$\\
		Distancia & m & $\hbar c$/GeV & GeV$^{-1}$ \\
	\end{tabular}
\end{table}

\subsection{Amplitudes de desintegración y secciones eficaces}

\subsubsection{Regla de oro de Fermi}

La \textbf{regla de oro de Fermi} establece la \textbf{frecuencia de transición}, $R$, entre los estados $i$-inicial y $f$-final regidos por un hamiltoniano $H_{\text{int}}$ es:

\begin{equation}
	R=(2\pi) |M_{fi}|^2 \rho (E)
\end{equation}
donde 

\begin{itemize}
	\item $M_{fi}$ es \textbf{elemento de matriz} de transición $M_{fi}=\langle f |H|i\rangle$.
	\item $\rho(E)$ es \textbf{densidad de estados disponibles} $\rho(E)$ que depende de las posibilidades es de momento que puedan tener los estados finales.
\end{itemize}
Notar que la física inherente a la interacción aparece en el elemento de matriz, y que la densidad de estados disponibles depende sólo de la cinemática del evento. en mecánica cuántica se calculan dichos factores en la aproximación no relativista, pero en física de partículas se precisa su versión relativista donde los factores $M_{fi}$ y $\rho(E)$ deben ser invariantes Lorentz.


\subsubsection{Elemento de Matriz relativista}

Como consecuencia de la normalización relativista de la función de ondas, el elemento de matriz relativista $M_{fi}=\langle \Psi_f |H_{\text{int}}| \Psi_i \rangle$ se relaciona con el no relativista $T_{fi}=\langle \psi_f |H_{\text{int}}|\psi_i\rangle$. por un factor que proviene de la normalización:

\begin{equation}
	M_{fi} = \langle \Psi_f |H_{\text{int}}| \Psi_i \rangle  = \parentesis{\prod_i \sqrt{2E_i}} T_{fi}
\end{equation}
donde $i$ corre en las $n$ partículas iniciales y finales. La expresión no relativista de la amplitud de desintegración es:

\begin{equation}
	\Gamma = (2\pi)^4 \int |T_{fi}|^2 \delta^4 \parentesis{\sum_k (p_k-p_a)}\prod_i \frac{\D^3 \pn_i}{(2\pi)^3}
\end{equation}
donde $p_a$ es el cuadrimomento de la partícula inicial y $p_i$ los de las finales $i=1,...,m$. Recordar que las deltas de Dirac imponen la conservación de energía y momento entre la partícula inicial y las finales. Y la expresión relativista:

\begin{equation}
	\Gamma = \frac{(2\pi)^4}{2E_a} \int |M_{fi}|^2 \delta^4 \parentesis{\sum_k (p_k-p_a)}\prod_i \frac{\D^3 \pn_i}{(2\pi)^3(2E_i)}
\end{equation}
Mientras que la expresión de la sección eficaz (relativista):

\begin{equation}
	\sigma = \frac{(2\pi)^4}{(2E_a)(2E_b)(\beta_a+\beta_b)} \int |M_{fi}|^2 \delta^4 \parentesis{\sum_k (p_k-p_a-p_b)}\prod_i \frac{\D^3 \pn_i}{(2\pi)^3(2E_i)}
\end{equation}
En la versión relativista, cada uno de los factores de la sección eficaz es invariante de Lorentz:

\begin{itemize}
	\item El elemento de matriz al cuadrado $|M_{fi}|^2 = |\langle \Psi_f |H_{\text{int}}|\Psi_i \rangle|^2 $.
	\item La densidad de estados $\frac{\D^3 \pn_i}{(2\pi)^3(2E_i)}$.
	\item El término asociado al flujo: $4E_aE_b(\beta_a+\beta_b)$.
\end{itemize}
Notar que a partir de la propiedad de la función delta de Dirac $\int \delta(E^2-\pn^2-m^2)\D E = 1/2E$ podemos reescribir la siguiente expresión:

\begin{equation}
	\frac{\D^3 \pn}{(2\pi)^3 (2E) } = \int \delta(p^2-m^2) \frac{\D^4 p}{(2\pi)^3}
\end{equation}

\subsubsection{Anchura de desintegración}

Un sistema de $N$ partículas en un volumen $V$ que se desintegran en el tiempo a una frecuencia constante:

\begin{equation}
	\derivadas{N}{t} = - \Gamma N
\end{equation}
cumple que 

\begin{equation}
	N(t) = N e^{-\Gamma t}
\end{equation}
llamando a $\Gamma$ \textbf{anchura de desintegración} y a su inverso $\tau$ \textbf{tiempo de vida medio}. En el caso que haya $n$ canales de desintegración la anchura total será la suma de las parciales:

\begin{equation}
	\Gamma = \sum_i \Gamma_i
\end{equation}
Llamando fracción de desintegración o \textbf{braching ratio} $\Bcal r_i$ de un canal al porcentaje de veces que una partícula se desintegra en ese  canal, que correspodne a:

\begin{equation}
	\Bcal r_i = \frac{\Gamma_i}{\Gamma}
\end{equation}
La relación entre la anchura de desintegración y la frecuencia de transición es simplemente $\Gamma=R$

\subsection{Invariantes de Mandelstam}

La cantidad $\sqrt{s}$ es la energía en e centro de masas (CM) de una aniquilación

\begin{equation}
	s = (p_a+p_b) = (E_a^*+E_b^*)-(\pn_a^*+\pn_b^*) = (E_a^*+E_b^*)^2
\end{equation}
que se deduce de que $\pn_i^*=\pn_a^*=-\pn_b^*$. En la literatura $\sqrt{s}$ para indicar la \textbf{energía en el centro de masas}. Lógicamente si nos dan $s$ y las masas $m$ podemos obtener los valores de $E_a^*$ y $E_b^*$. Como se puede deducir de la ecuación anterior se tiene que verificar que:

\begin{equation}
	E_a^* = \sqrt{s}-E_b^* \tquad E_b^* = \sqrt{s}-E_a^*
\end{equation}
Además sabemos que $E_a^{*2}=m_a^2+\pn_i^{*2}$ y que $E_b^{*2}=m_b^2+\pn_i^{*2}$, por lo que podemos despejar estas y obtener que:

\begin{equation*}
	E_a^{*2} = s + E_b^{*2} - 2\sqrt{s} E_b^{*2} \Rightarrow E_b^{*2} = \frac{s+m_b^2-m_a^2}{2\sqrt{s}}
\end{equation*}
de lo que se deduce:


\begin{equation*}
	E_b^{*} = \frac{s+m_b^2-m_a^2}{2\sqrt{s}} \tquad E_a^{*} = \frac{s+m_a^2-m_b^2}{2\sqrt{s}}
\end{equation*}
En el caso de la desintegración dos cuerpos.

\section{Ecuación de Dirac}



\subsection{Ecuación de Dirac y matrices $\gamma$}

\subsubsection{Ecuación de Dirac}


Para dar una verisión relativista de la mecánica cuántica Dirac propuso una propuso lineal en la primera derivada temporal y en las derivadas espaciales a los que multiplicó por los factores $\alpha,\beta$. La \textbf{ecuación de dirac} es

\begin{equation}
	\Hcal \Psi = \parentesis{\alpha \cdot \pn + \beta m} \Psi
\end{equation}
donde $\pn=-i\nabla$ el operador momento lineal y $\Hcal = i \parciales{}{t}$ el hamiltoniano de la partícula libre. Las matrices $\alpha^i$ (para $i=x,y,z$) y beta deben verificar que cuando se eleve la ecuación al cuadrado quede la relación de Einstein $E^2+p^2 + m^2$. 

Existen varias representaciones de estas matrices, una de las más comunes es la  \textbf{representación de Pauli-Dirac}:

\begin{equation}
	\beta = \begin{pmatrix}
 	I & 0 \\ 0 & I 
	\end{pmatrix} \quad 
	\alpha_i = \begin{pmatrix}
	0 & \sigma_i \\ \sigma_i & 0 
	\end{pmatrix}
\end{equation}
siendo $I$ la matriz identidad y $\sigma_i$ las matrices de Pauli. Otra representación es la quiral. 


\begin{equation}
	I = \begin{pmatrix}
		1 & 0 \\ 0 & 1
	\end{pmatrix} \quad 
	\sigma_1 = \begin{pmatrix}
		0 & 1 \\ 1 & 0 
	\end{pmatrix}
	\alpha_2 = \begin{pmatrix}
	0 & -i \\ i & 0 
	\end{pmatrix}
	\alpha_3 = \begin{pmatrix}
	1 & 0 \\ 0 & -1
	\end{pmatrix}
\end{equation}
La función de ond, solución de la ecuación de Dirac, $\Psi$, es un \textbf{cuadri-espinoer de Dirac} 

\begin{equation}
	\Psi = \begin{pmatrix}
		\psi_1 \\ \psi_ 2 \\ \psi_3 \\ \psi_4
	\end{pmatrix}
\end{equation}
que tiene cuatro componentes.

\subsubsection{Matrices $\gamma$}

Las \textbf{matrices} $\gamma$ se representan como:

\begin{equation}
	\gamma^0 = \beta, \ \gamma^1 = \beta \alpha_1, \ \gamma^2 = \beta \alpha_2, \ \gamma^3 = \beta \gamma^3
\end{equation}
A la representación de la ecuación de Dirac usando estas matrices la llamamos \textbf{representación covariante}:

\begin{equation}
	(i\gamma^\mu \partial_\mu - m) \Psi = 0
\end{equation}
donde $\partial_\mu$ es la \textbf{derivada covariante:}

\begin{equation}
	\partial_\mu = \parentesis{\parciales{}{t},\parciales{}{x},\parciales{}{y},\parciales{}{z}}
\end{equation}
Algunas propiedades de las matrices $\gamma$ son:

\begin{equation}
	(\gamma^0)^2 = I \quad  (\gamma^k) = - I \quad \gamma^\mu \gamma^\nu = - \gamma^\nu \gamma^\mu
\end{equation}
con $k=1,2,3$ y $\mu=0,1,2,3$. De manera análoga se cumplen las siguientes relaciones de anti-conmutación:

\begin{equation}
	\left\lbrace \gamma^\mu , \gamma^\nu \right\rbrace =  \gamma^\mu \gamma^\nu +  \gamma^\nu \gamma^\mu = 2 g^{\mu \nu}
\end{equation}
siendo la matriz $\gamma^0$ hermítica ($\gamma^{0\dagger}=\gamma^0$) y la matriz $\gamma^k$ anti-hermítica ($\gamma^{k\dagger}=-\gamma^k$). La matriz $\gamma^5$ se define a partir del producto de las otras 4:

\begin{equation}
	\gamma^5 = i \gamma^0\gamma^1\gamma^2\gamma^3 
\end{equation}
y juega un papel fundamental en las interacciones débiles. Teniendo las siguientes propiedades $(\gamma^5)^2=I$, $\gamma^{5\dagger}=\gamma^5$ y $\gamma^5 \gamma^\mu = - \gamma^\mu \gamma^5$.

 

\subsubsection{El espinor de Dirac adjunto}
Es conveniente también definir el \textbf{espinor adjunto} como:

\begin{equation}
	\bar{\Psi} = \Psi^\dagger \gamma^0
\end{equation}
En la representación de Pauli-Dirac:

\begin{equation}
	\bar{\Psi}=\parentesis{\psi_1^*,\psi_2^*,-\psi_3^*,-\psi_4^*}
\end{equation}
y tiene forma de vector fila.

\subsubsection{Densidad y corriente de probabilidad}

La densidad de corriente $\rho$ y la corriente $j^k$ con $k=1,2,3$; de probabilidad del espinor de Dirac son:

\begin{equation}
	\rho = \Psi^\dagger \Psi = |\psi_1|^2 +|\psi_2|^2+|\psi_3|^2+|\psi_4|^2 \quad j^k = \Psi^\dagger \alpha_k \Psi
\end{equation}
que cumplen:

\begin{equation}
	\parciales{\rho}{t} + \nabla \cdot \jn = 0
\end{equation}
Y la conservación de la probabildiad en un punto del espacio $\partial_\mu j^\mu$. Las corrientes cambian bajo paridad. 

\subsubsection{Espín en la ecuación de Dirac}

En mecánica cuántica la evolución temporal del observable $O$ viene dada por:

\begin{equation}
	\derivadas{O}{t} = i \langle \Psi | [\Hcal,O] |\Psi \rangle
\end{equation}
en el caso del operador angular no se mantiene:

\begin{equation}
	[\Hcal,\Ln] = - i \alphan  \times \pn
\end{equation}
Y si consideramos el espín $\Sn$ como un momento angular intrínseco $\Sn = \frac{1}{2}\Sigman$ siendo $\Sigman=\sigma_1\hnx+\sigma_2\hny+\sigma_3 \hnz$, tenemos que

\begin{equation}
	[\Hcal,\Sn] =  i \alphan \times \pn
\end{equation}
y tiene las mismas relaciones de conmutación que $\Ln$. 

\subsection{Soluciones de la partícula libre-espinores}

Vamos a obtener la soluciones de la ecuación de Dirac a partir de las funciones de la onda plana de una partícula libre:

\begin{equation}
	\Psi = u(E,\pn) e^{i(\pn \cdot \xn - Et)}
\end{equation}
y la ecuación para el espinor $u(E,0)$ debe verifcar

\begin{equation}
	(E\gamma^0-m)u=0
\end{equation}
En la representación de Pauli-Dirac hay 4 soluciones:


\begin{equation}
	u_1 = N \begin{pmatrix}
		1 \\ 0 \\ 0 \\ 0
	\end{pmatrix}
	\quad
 	u_2 = N  \begin{pmatrix}
 		0 \\ 1 \\ 0 \\ 0
	\end{pmatrix}
	\quad	
	u_3 = N  \begin{pmatrix}
		0 \\ 0 \\ 1 \\ 0
	\end{pmatrix}
	\quad	
	u_4 = N  \begin{pmatrix}
		0 \\ 0 \\ 0 \\ 1
	\end{pmatrix}
\end{equation}
Donde $N$ es un factor de normalización. La ecuación de Dirac aplicada a estos espinores nos hace obtener energía positiva en el caso de $u_1,u_2$ y energía negativa en el caso de $u_3,u_4$. Respecto al espín, las soluciones $u_1$ y $u_3$ tienen espín arriba $+1/2$ y $u_2$ y $u_4$ tienen el espín abajo. 


Existen dos formas de entender esta energía negativa. Tenemos la teoría del <<mar de Dirac>> en la que el vacío tiene todos los estados negativos ocupados, pero un fotón con $E>m_ec^2$ podía hacer saltar un electrón de la zona de energía positiva produción un electrón y un hueco en el mar que se interpreta como un positrón. La teoría de Feynman y Stückelberg propuso que las soluciones de energía negativa eran en realidad partículas que se propagaban hacia atrás en el tiempo, o equivalentemente antipartículas con cargas positivas. En este caso tendríamos 
\begin{equation}
	\text{Partículas:} \ \Psi = u (E,\pn) e^{i(\pn \cdot \xn - Et)} \quad
	\text{Anti-particulas:} \ \Psi = v (E,\pn) e^{-i(\pn \cdot \xn - Et)}
\end{equation}
siendo $v$ el espinor de una antipartícula. Es importante 

\subsection{Helicidad y quiralidad}

\subsubsection{Helicidad}

Definimos helicidad, $h$, como la proyección normalizada del espín sobre el momento:

\begin{equation}
	h \equiv \frac{\Sn \cdot \pn}{p}
\end{equation}
Para un espinor de Dirac el operador helicidad $h$ en la representación de Pauli-Dirac:

\begin{equation}
	h = \frac{\Sigman \cdot \pn}{2p} = \frac{1}{2p} \begin{pmatrix}
		\sigma \cdot \pn & 0 \\
		0 & \sigma \cdot \pn
	\end{pmatrix}
\end{equation}
Conmutando con el hamiltoniano y por tanto conservándose en el tiempo. Sin embargo no es invariante Lorentz, ya que siempre podemos encontrar un sistema de referencia (con una velocidad mayor) que revierta el momento y por lo tanto la helicidad. 


\subsubsection{Quiralidad}

La quiralidad juega un papel fundamental en las interacciones débiles, que corresponde con la matriz $\gamma^5$. Los autoestados de $\gamma^5$ son autoestados de quiralidad. En esta representación los espinores $u=\binom{u_a}{u_b}$ son autoestados de quiralidad si:

\begin{equation}
	u_B = \pm u_A
\end{equation}
Los espinores 

\begin{equation}
	u_R = \binom{u_A}{u_A} \quad u_L \binom{u_A}{-u_A}
\end{equation}
cumpliendo:

\begin{equation}
	\gamma^5 u_R = + u_R \quad \gamma^5 u_L = - u_L
\end{equation}
decimos que $u_R$ tiene quiralidad a derechas y $u_L$ quiralidad a izquierdas. Para los espinores de antipartículas, la situación se invierte. Los espinores son:

\begin{equation}
	v_R = \binom{-v_B}{v_B} \quad v_L \binom{v_B}{v_B}
\end{equation}
cumpliendo que:

\begin{equation}
	\gamma^5 v_R = - v_R \quad \gamma^5 v_L = - v_L
\end{equation}
Si definimos $\gamma^{5v}=-\gamma$ para los espinores $v$ tenemos:

\begin{equation}
	\gamma^{5v} v_R = + v_R \quad \gamma^{5v} v_L = - v_L
\end{equation}
Cabe decir que los \textit{estados de helicidad de partículas ultra-relativistas} también son \textit{autestados de quiralidad}, pero solo en el caso de partículas ultra-relativistas.




\section{Detección de partículas}

La historia de la física de partículas está ligada a la de los aceleradores. Para producir una partícula necesitamos una energía igual a su masa $E=mc^2$, y para explorarla una longitud menos que la de su tamaño por relación $p=\hbar/\lambda$. Se pueden acelerar las partículas estables cargadas: $e^-,e^+,p,\bar{p}$. También podemos crear haces secundarios al golpear un haz de protones contra un blanco (mesones $\pi$ y $K$) que se pueden seleccionar en un rango de energía y momento. Si además se dejan desintegrase y se filtran se obtiene un haz de neutrinos.

\subsection{Tipos de experimentos}

Existen dos tipos de experimentos (principalmente):

\begin{itemize}
	\item De \textbf{blanco fijo} donde un haz se hace incidir sobre un blanco.
	\item De \textbf{colisión} donde dos haces se cruzan en determinados puntos. 
\end{itemize}
La energía disponible en el centro de masas se denota por $\sqrt{s}$ en ambos experimentos. Para un experimento de blanco fijo $s=(E,\pn)+(m_a,0)$ donde ($E,\pn$) es el cuadrimomento de la partícula incidente con masa $m_b$ y $(m_a,0)$ es la masa de la partícula del blanco (usualmente nucleones). En este caso la energía en el cendro de masas 
\begin{equation}
	\sqrt{s} = \sqrt{(E+m_b)^2 - \pn^2 } = \sqrt{m_b^2 + 2 m_a m_bE + m_a^2}
\end{equation}
que depende de $\sqrt{E}$ si las masas son despreciables. 

Para un experimento de colisión $s=(E_a,\pn_a)+(E_b,\pn_b)$. En el caso de partículas relativistas de igual masa que colisionan de frente la energía en el centro de masas $\sqrt{s}= 2 E$.


\subsection{Introducción a aceleradores}

Los haces están compuestos de \textbf{paquetes} con una densidad elevada de partículas tiene del orden de $10^{11}$ protones. Existen dos tipos de aceleradores principales:

\begin{itemize}
	\item Lineales: donde el acelerador es recto, un tubo de haz en vacío, que las paquetes recorren una vez.
	\item Circulares: los paquetes giran en el tubo de vacío del haz numerosas vueltas.
\end{itemize}

\subsection{Parámetros de un colisionador}

Existen dos parámetros fundamentales en un colisionador:

\begin{itemize}
	\item \textbf{Energía} en el centro de masas $\sqrt{s}$ que determina la masa de las partículas que podemos crear.
	\item \textbf{Luminosidad} o el número de partículas que se cruzan entre ellas por unidad de área y tiempo. Determina el número de interacciones en cada cruce-.
\end{itemize}

Para dos vagones con $n_1$ y $n_2$ partículas que se cruzan entre sí en la dirección $z$ con una frecuencia $\nu$ y que tienen una sección trasversal $\sigma_T$:

\begin{equation}
	\Lcal = \nu \frac{n_1n_2}{\sigma_T}
\end{equation}
Si un tipo de interacción tiene sección eficaz $\sigma$ esperamos una frecuencia de interacciones $n=\sigma \Lcal$. En general $\Lcal(t)$ depende del tiempo. A $\Lcal(t)$ le llamamos \textbf{luminosidad instantánea} y a $\int_t \Lcal (t)\D t$ le llamamos \textbf{luminosidad integrada}. A las interacciones que se producen en el cruce de los vagones se las conoce como evento. El número de interacciones por evento es una variable aleatoria con distribución de Poisson de media $\sigma \Lcal$ (y por tanto la incertidumbre es la propia media).

\subsection{Interacción de las partículas con la materia}

Las interacciones de las partículas pueden dividirse en:

\begin{itemize}
	\item Interacciones de las partículas cargadas.
	\item Interacciones electromangéticas de electrones y fotones.
	\item Interacciones fuertes de hadrones.
\end{itemize}

\subsubsection{Interacciones de las partículas cargadas}

Las partículas cargadas relativistas interaccionan electromagnéticamente con los electrones de los átomos de la materia y pierden energía por ionización. Esta pérdida por distancia recorrida viene dada por las fórmulas de Bethe-Block (1930):

\begin{equation}
	\derivadas{E}{x} \simeq - \frac{4\pi \hbar^2 \alpha^2}{m_e} \frac{n_e}{\beta^2} \ccorchetes{\ln \parentesis{\frac{2m_e c^2 \beta^2 \gamma^2}{I_e}} - \beta^2}
\end{equation}
donde $v=\beta c$ es la velocidad, $\gamma$ el factor el Lorentz, $n_e$ la de densidad de electrones, $I_e$ el potencial de ionización. Se trata de un valor promedio. La distribucón $\D E / \D x$ está relacionada con la fluctuación del número de colisiones de la partículas con los electores de los átomos.

La densidad de electrones $n_e=\rho N_A Z/A$ donde $\rho$ es la densidad, $A$ la masa atómica (g/mol), $Z$ el número atómico y $N_A$ el número de Avogadro. Si reescribimos:

\begin{equation}
	\frac{1}{\rho} \derivadas{E}{x} \simeq - K \frac{Z}{A\beta^2} \ccorchetes{\ln \parentesis{\frac{2m_ec^2\beta^2 \gamma^2}{I_e}} - \beta^2}
\end{equation}
donde $K$ es el siguiente factor:

\begin{equation}
	K = \frac{4\pi \hbar^2 \alpha^2 N_A}{m_e} = 0.307 \ \unit{MeV cm^2 / mol}
\end{equation}
Hay que tener en cuenta $Z/A$ es prácticamente constante cuando tratamos átomos estables. La dependencia de la pérdida energía con las características del material están encerradas en la densidad del mismo.

El recorrido medio de una partícula con $\beta$ hasta que se detiene en un medio se denomina \textbf{rango} de penetración. Durante el recorrido la velocidad va cayedo y por tanto también decae cuanta energía se pierde. En función del valor de $\beta$  distinguimos tres regiones:

\begin{itemize}
	\item La pérdida es más intensa para baja $\beta$ (de la dependencia $1/\beta^2$). La inionización es mayor al final de la trayectoria, cuando las velocidades es muy pequeña. Esta región se conoce como \textbf{pico de Bragg}.
	\item En el rango $\beta \gamma$ de 1-10, la energía perdida es mínima. Esta región se denomina \textbf{mip} (\textit{minimun ionizing particle}). Un muón de 10 GeV en hierro pierde en promedio $13$ MeV/cm y su rango es de varios metros. 
	\item En rango $\beta \gamma>100$, la perdida de aumenta de forma logarítmica. A partir aquí los efectos de radiación son relevantes.
\end{itemize}
La fórmula calcula el valor promedio. La distribución $\D E / \D x$ está relacionada con la fluctuación del número de colisiones de la partícula con los electrones de los átomos. Los electrones pierden también energía por radiación \textbf{bremsstrahlung}, tal y como vamos a ver en el siguiente apartado

\subsubsection{Interacciones de las electromagnéticas de electrones}

Las partículas cargadas pueden radiar fotones por la interacción electromagnética con los protones de los núcleos. Esta radiación se llama \textbf{bremsstrahlung}

\begin{equation}
	e^- + (A,Z) \rightarrow e^- + \gamma + (A,Z)
\end{equation}
Esta radición empieza a ser dominante a partir de una \textbf{energía crítica} $E_c \sim 800 /Z$ MeV, antes domina la ionización. Este proceso se puede calcular en QED y su sección eficaz:

\begin{equation}
	\sigma_b  \propto E/m^2 
\end{equation}
afecta más a los electrones que a los muones por un factor $(m_e/m_\mu)^2$. Los muones por debajo de 100 GeV pierden energía principalmente por ionización. La pérdida de energía por bremsstralung por encima de $E_c$ puede expresarse:

\begin{equation}
	\derivadas{E}{x} = - \frac{E}{X_0} \quad E(x)=E_0 e^{-x/X_0}
\end{equation}
donde $X_0$ se denomina \textbf{longitud de radiación} y $E_0$ es la energía inicial del electrón. $X_0$ depende del material, notar que $X_0=n\sigma_b$ donde $n$ es la densidad de núcleos.


\subsubsection{Interacciones de las electromagnéticas de fotones}

Las interacciones de los fotones con la materia dependen de su rango energía, por debajo MeV domino el efecto foto-eléctrico, en el rango MeVs, la dispersión Compton, y por encima 10 MeV, la producción de pares. La sección eficaz de producción de pares crece rápidamente desde el umbral de producción y puede aproximarse:

\begin{equation}
	\sigma_\gamma \simeq \frac{7}{9} \frac{1}{nX_0}
\end{equation}
donde $n$ es la densidad de núcleos. La cantidad $\lambda=1/(n\sigma_\gamma)$ es el \textbf{camino libre medio}, que vale $\lambda\simeq7/9 X_0$ y que nos indica la cantidad de fotones que se pierden en un haz monoenergético de intensidad $I$:

\begin{equation}
	\derivadas{I}{x} = - \frac{I}{\lambda} \tquad I(x) = I_0 e^{-x/\lambda}
\end{equation}
Por lo tanto la longitud de radiación caracteriza la pérdida de energía de electrones y conversión de fotones en pares para partículas por encima de $\sim$ 10 MeV. Los electrones o fotones de alta energía al atravesar un medio de $X_0$ producen una cascada electromagnética. 

\subsubsection{Interacciones fuertes de los hadrones}

Los hadrones cargadas (protones, piones, kaones) pierden energía por ionización. También por interacciones fuertes con los núcleos de la materia. Las interacciones se caracterizan con la longitud de interacción, $\lambda_I$, que es la distancia media entre interacciones fuertes $\lambda > X_0$. Las intearcciones de hadrones producen una cascada hadrónica. 

Las cascadas hadrónicas son más variables que electromagnéticas, dado que en ellas se pueden producir más tipos de partículas, y también $\pi^0$ que se desintegran electromagnéticamente $\pi^0 \rightarrow \gamma \gamma$. Lo que da lugar a su vez a una cascada electromagnética dentro de la hadrónica. Esto nos lleva a que la energía ase dispersa entre los dos tipos de cascadas. Una parte de la energía también se pierde en forma de excitaciones y roturas nucleares.

\subsection{Detectores las partículas}

Los detectores de partículas usan como base la interacción de las partículas con la materia, principalmente ionización y radiación. Los detectores se dividen en:

\begin{itemize}
	\item \textbf{Detectores de trazas:} determinan las trayectorias de las partículas cargadas. Sirven para medir generalmente el momento y determinar los vértices de desintegración. Los detectores de trazas están habitualmente inmersos em campos magnéticos con lo que el momento se determina a partir de su curvatura.
	\item \textbf{Calorímetros:} sirven para la energía de las partículas, principalmente electrones/fotones y hadrones.
\end{itemize}

\subsubsection{Detectores tipo}

Los detectores tipo tienen una estructura en capas cilíndricas. En el interior están los detectores de trazas más precisos (detectores de silicio) sumergidos en un campo magnético (selenoidal). Luego le siguen un calorímetro electromagnético y hadrónico, para finalmente tener los detectores de muones.

\subsubsection{Detectores de trazas}

En los detectores de trazas se detecta ionización (electron es liberados) del paso de la partícula cargada a través de un medio para determinar puntos de paso o hits. Los detectores están sumergidos en un campo magnético $\Bn$ (T) que produce una curvatura de la partícula proporcional a su momento $\pn$ (GeV) en la dirección perpendicular a $\Bn$. Si entre ambos hay un ángulo $\theta$:

\begin{equation}
	p \cos (\theta) = 0.3B\rho
\end{equation}
donde $\rho$ (m) es el radio de curvatura ene l plano perpendicular. Llamamos al momento en el plano perpendicular el \textbf{plano transverso} $p_T = p \cos (\theta)$. 

\subsubsection{Detectores de Silicio}

Están basados en portadores libres de carga, y están formados por una oblea de silicio de aproximadamente $300 \ \mu$m de espesor donde se han dopado tiras (\textit{strips}) de tipo p separadas por  $50 \ \mu$m para craer uniones pn. 

La ionización de paso de una partícula cargada crea pares electrón/hueco. Los electrones de ionización derivan hacia las tiras $p$, donde su carga es amplificada por la electrónica. Con estos detectores se pueden reconstruir las trayectorias con precisión $\sim \ \mu$m e identificar vértices de desintegración de partícuals que pueden recorrer $\sim $1 cm.


\subsubsection{Detectores gaseosos}

En los detectores gaseosos de utiliza la ionización ($\sim$30 eV por ionización) del paso de las partículas cargadas en gases nobles como argón o xenón. Los detectores están en un rango de voltaje proporcional (no hay efecto avalancha). Los electrones derivan bajo la presencia de un campo eléctrico $\En$ hasta el ánodo que puede estar formado por hilos.  

Una de los detectores mas usados son las TPC. Las TPC (\textit{Time Projection Champber}) o cámaras de proyección temporal obtienen las trazas a partir del rastro de ionización que deja el haz. Bajo la presencia un campo eléctrico los electrones de la ionización derivan hacia el ánodo, recogido por un detector segmentado en hilos donde se amplifica su carga.  El tiempo de llegada de los electrones $\Delta t$ sirve de estimación de la posición $z$. 

\subsubsection{Calorímetros}

Los \textbf{calorímetros electromagnéticos} están constituidos con materiales de alto $Z$ y $X_0$. Suelen tener una estructura de material pasivo como plomo donde se desarrollan las cascadas, y material activo donde se detecta la ionziación. La resolución en energía $\sigma_E$ está limitada por las fluctuaciones en la producción de las partículas (proporcional a $\sqrt{E}$) en la cascada. En general:

\begin{equation}
	\frac{\sigma_E}{E} \sim \frac{3\% - 10\%}{\sqrt{E(\text{GeV})}}
\end{equation}
Los \textbf{centelleadores} contienen moléculas centelleadoras que en vez de ionizar se excitan, y al desexcitarse emiten luz en el visible que pueden detectarse con detectores de fotones. El número de fotones es proporcional a la energía absorbida, aproximadamente 100 eV por fotón de centelleo. Los centelleadores deben tener una longitud de radiación $X_0$ alta para evitar conversión de fotones.

También existen los \textbf{calorímetros hadrónicos}, constituidos de materiales de alta densidad, y pese a todo, aun ocupar mucho volumen. Por ejemplo, un hadrón de 100 GeV genera una cascada de aproximadamente 2m de profundidad y 50 cm de anchura. Están construidos habitualmente en capas de material pasivo y activo. Por ejemplo ATLAS está formado por placas de plástico y centelleador. La resolución de energía está limitada por la producción de las partículas $\sqrt{E}$, por la fracción entre la cascada electromagnética y hadrónica, y la energía perdida:

\begin{equation}
	\frac{\sigma_E}{E} \sim \frac{50\%}{\sqrt{E(\text{GeV})}}
\end{equation}

\subsection{Detectores de luz de Cherenkov}

El paso de las partículas cargadas que se mueven a una velocidad $v$ mayor que la luz en el medio ($v>c/n$) genera un frente de onda coherente con un determinado ángulo $\theta$ respecto a la dirección $\vn$. Esta onda, que se encuentra en el visible, se llama luz Cherenkov. El ángulo depende de la velocidad y el índice de refracción:

\begin{equation}
	\cos (\theta) = \frac{1}{\beta n}
\end{equation}
Si detectamos conos de luz podemos determinar la dirección $\hnv$ y $\beta$. 


\subsection{Sistema de disparo, procesado y análisis de datos}

Los detectores de partículas tienen un \textbf{sistema de disparo} o \textit{trigger} para seleccionar al momento los datos relevantes. El trigger es un sistema de diversos niveles donde se buscan las características principales de sucesos de interes (por ejemplo en el caso del LHC buscamos alto momento transverso y partículas desplazadas del vértice de interacción).

Los datos que superan el trigger se almacenan y procesan posteriormente en centros de computación interconectados. A partir de las señales de cada sensor se reconstruyen trazas ya cascadas, y de ahí partículas con su cuadrimomento. El producto final son ficheros de datos procesados que se analizan con técnicas estadísticas de análisis, muchas veces multivariables. 

El proceso de se completa con la producción de eventos simulados, donde se simula con gran precisión el detector y la física. La producción de datos simulados se realiza también en centros de computación, sirviendo de referencia y permiten estimar y prever las capacidades de un detector. Las mediciones de los experimentos se presentan como:

\begin{itemize}
	\item La \textbf{estimación de un observable}, por ejemplo, una sección eficazm una masa, la vida media o una fracción de desintegración.
	\item Un \textbf{límite en el observable}, como es la masa en los neutrinos.
\end{itemize}
Los primeros están sujetos a estimaciones de errores, y estos son de dos tipos:

\begin{itemize}
	\item Los \textbf{estadísticos}: dependen de la cantidad de sucesos relevantes disponibles.
	\item Los \textbf{sistemáticos}: que reflejan nuestras incertidumbres en parámetros que afectan al observable, que puede ser de diversas índoles: calibración, estimación de eficiencias de selección, teóricos...
\end{itemize}
La determinación de un límite se presenta siempre con un nivel de confianza, y los errores ya están incorporados en la definición del mismo límite. La interpretación de los errores y el nivel de confianza depende de si el análisis es frecuentista o bayesiano. Un frecuentista diría <<Los datos casan con la existencia del Higgs>> mientras que un bayesiano diría <<Hemos encontrado el Higgs>>.










