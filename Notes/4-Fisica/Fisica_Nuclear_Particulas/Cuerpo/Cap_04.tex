\chapter{Estructura nuclear} \label{Ch:03}

\section{Introducción a la estructura nuclear}

Un núcleo atómico es un objeto constituido por muchos cuerpos (el Uranio tiene hasta 238 nucleones). Describir las propiedaes nculeares a partir de la interacción básica entre esos cuerpos es una tarea formidable y fuera de nuestro alcanceen la actualidad por varias razones. En primer lugar la interacción entre dos nucleones no está del entendida desde el punto de vista teórico-conceptual\footnote{La Cromodinámica Cuántica (QCD) aunque ha superdo pruebas experimentales, no permite hacer cálculos preditivos, y mucho menos deducir a partir de ella las características de interacción fuerte entre hadrones.}, y en segundo lugar, aunque dispusiéramos de una teoría completa que describiese la interacción en todos sus detalles no podríamos manejar operativamente el difícil problema de la interacción entre muchos cuerpos. Esto nos lleva de modo natural a intentar a describir la estructura nuclear y el espectro de excitaciones a partir de de modelos muy simplificados, esto es, modelos construidos a partir de algunas de las propiedades de los núcleos en lugar de las funciones de onda detalladas de cada nucleón. Se utiliza un planteamiento similar en la termodinámica, dodne variables colectivas como la presión y la temperatura de un gas sustituyen a las variables cinemáticas de cada uno de sus átomos. A grandes rasgos, podemos clasificar los modelos de estructuraa nuclear en dos categorías:

\begin{itemize}
    \item \textbf{Modelos de fuerte correlación:} las propiedades nucleares sobre las que se construyen este tipo de modelos se explican como originadas a partir del comportamiento colectivo de los nucleones. Un ejemplo podrían ser los movimientos rotacionales y vibratorios nucleares, que dan lugar a espectros característicos de energía cuantizada. Es obvio que un movimiento rotatorio es un fenómeno creado por un comportamiento colectivo o correlacionado de los nucleones.
    \item \textbf{Modelo de casi nula correción:} las características nucleares propias de estos modelos se suponen originadas por el comportamiento individual de cada nucleón.
\end{itemize}
Ninguno de los dos tipos de modelo en su versión extrema pueden explicar todas las propiedades nucleares. Los modelos más realistas tratan con una mezcla de propiedes colectivas e individuales de los nucleones. Un modelo basado exclusivamente en propiedades colectivas será insuficiente para explicar parte de las propiedades nucleares, y es necesario acudir a una cierta mezcla de propiedades colectivas e individuales para conserguir modelos nucleares con un rango de aplicación más interesante.

\section{El modelo de la gota líquida}

El modelo de la gota líquida lo hemos aplicado ya en un capítulo anterior para obtener la fórmula semiempírica de masas o fórmula de Weizsäcker, qeu nos permite calcular las masas y las energías de ligaduras de los núcleos. Una gota de líquido en ausencia de campos gravitatorios (o cualquier otro tipo de campo) adquiere una forma que minimiza la energía positiva de tensión superficial. Esta forma es la forma de la esfera. Una gota de líquido es esencialmente incompresible (su densidad es constante), y por lo tanto su radio será $R\sim n^{1/3}$, donde $n$ es el número de moléculas de la gota. Consideremos que la gota tiene una energía de ligadura\footnote{Las moléculas de la superifice estarán menos ligados que las interiores, pero este efecto se introdcue como una corrección en términos de la tensión superficial.}  que podemos denotar por $a$ La energái de ligadura se debe a la interacción de la molécula con sus moléculas vecinas. Estas fuerzas de interacción se anulan a distancias grandes y se hacen repulsivas a distancias cortas comparadas con la distancia intermolecular típica\footnote{La interacción fuerte entre nucleones tiene características similares en cierto sentido, como veremos más adelante.}. Si tomamos como cero la energía la situación en la cual todas las moléculas de la gota están infinitamente separadas, podemos expresar la energía de ligadura de la gota (tomada positiva) de la siguiente manera:

\begin{equation}
    B=an - 4 \pi R^2 T = an-\beta n^{2/3} \quad (R^2 \sim n^{2/3})
\end{equation}
donde $T$ es la energía de tensión superficial del líquido. Si la gota tuviese una carga eléctrica $Q$ uniformemente distribuida en todo su volumen debemos añadir un término correspondiente a la energía potencial coulombiana:

\begin{equation}
    B = an - \beta n^{2/3} - \frac{\gamma Q^2}{n^{1/3}}
\end{equation}
donde $\gamma$ contiene todas las contantes de la energía coulombiana excepto la dependencia con $Q$ y $n$. 

Para obtener la fórmula de Weizsäcker, o fórmula semiempírica de masas, aplicamos estas ideas y algunas otras hipótesis de trabajo que recapitulamos aquí:

\begin{itemize}
    \item Suponemos un núcleo esférico.
    \item Los nucleones dentro del núcleo se comportan de modo análogo a moléculas en una gota líquida, es decir, fuerzas atractivas de corto alcance los mantienen unidos y fuerzas repulsivas, de todavía más corto alcance, los mantienen alejados unos de otros.
    \item La densidad nuclear es constante. 
    \item Existe una tendencia a mantener un número de protones muy parecidos al número de neutrons en un núcleo.
    \item Existe una \textit{fuerza de apareamiento} que favorece la existencia de núcleos con $Z$ par y $N$ par.
\end{itemize}

\section{El modelo de Gas de Fermi}

Alrededor de 1948, se comenzó a considera en serio la evidencia acumulada osbre la existencia de ciertos \textbf{números mágicos} para los valoers de $Z$ Y $N$. La energía de separación de dos protones para secuencias de isótonos ($N$ constante) graficada como desvaiciones de la predicción de la fórmula semiempírica de masas, mostraba unos picos acusados par $Z=8,20,28,50,83$; mientras que la energía de separación de dos neutrones para secuencias de isótopos, también graficada como desviaciones de la predicción de la fórmula semiempírica de masas muestra picos para $N=8,20,28,50,82,126$. La similitud de estas gráficas con la de algunas propiedades atómicas, como la energía de ionización es sorprendente.

\section{El modelo de capas}

A semejanza del modelo atómico de capas que tanto éxito a tenido en la física atómica, resulta tentador preguntarse si un modelo similar no tendría éxito también en la física nuclear. Es necesario, sin embargo, tener en cuenta las profundas diferencias entre la física atómica y la nuclear. 

Los electrones se mueven en un potencial externo aproximadamente central: el potencial que crea el núcleo junto con las correcciones oportunas debidas al resto de electrones. De este modo manera surgen natural las capas electrónicas, que se van llenando en orden de energía creciente cumpliendo el principio de exclusión de Pauli. Las capas electrónicas llenas forman una especie de zona interior neutra y los electrones de la capa semillena constituyen los electrones de valencia, que determinan la mayoría de las propiedades químicas del átomo correspondiente. Cuando vamos llenando una capa electrónica las propiedades atómicas como la energía de ionización varían suavemente, pero sufren una súbita discontinuidad cuando la capa queda llena y hemos de pasar a la siguiente. 

En un núcleo no tenemos un agente externo que cree el potencial en el que se mueven los nucleones, son ellos mismos los que configuran el potencial efectivo nuclear. Otra compilación o diferencia adicional es que, en principio, parecería que los nucleones debieran tener una probabilidad no despreciable de colisionar los unos con los otros, mientras que eso no sucede con los electrones atómicos. No resulta evidnete que podamos considerar a cada nucleón moviéndose independientemente de los demás en un potencial nuclear efectivo, pero ya hemos visto que el principio de exclusión de Pauli garantiza que, esencialmente, los nucleones se mueven libremente dentro del núcleo. 

La hipótesis principal del modelo de capas es suponer que los nucleones se mueven en el núcleo casi independientemente los unos de los otros a pesar de la interacción fuerte. Este movimiento libre significa, en última instancia, que el recorrido libre medio de un nucleón en materia nuclear es grande comparado con las dimensiones del núcleo. En el modelo de capas la interacción nucleón con sus compañeros se reduce a la interacción con un \textbf{campo autoconsciente} (\textit{self-consistent field}) creado por ellos. Generalmente se supone que este campo autoconsciente es estático y esféricamente simétrico.

Debido al corto alcance de las fuerzas nucleares, el potencial del campo autoconsciente tiene dependencia radial muy similiar a la densidad nuclear, es decir, es casi constante dentro del núcleo y se anula fuera. Por lo tanto, en primera aproximación podríamos considerar que el potencial nuclear constante en el interior del núcleo, tal como se hace en el modelo de gas de Fermi ideal\footnote{El modelo de capas incorpora la hipótesis del modelo de gas de Fermi.}, con lo cual las funciones de onda de los nucleones serían ondas planas. No obstante, la introducción den el modelo de capas de un campo autoconsistente que depende de la distancia al centro del núcleo es una mejora sustancial respecto del modelo de gas de Fermi, y modifica las funciones de onda de los nucleones, dejando de ser ondas planas. 

Existen varais versiones del modelo de capas. La más simple, conocida como \textbf{versión extrema del modelo de capas} (\textit{one-particle shell mode}), se usa para explicar las propiedades de los núcleos con $A$ impar. En esta versión se supone que todos los nucleones están apareados (incluyendo los de una hipotética capa semillena) formando una coraza interde de espín cero, y que las propiedades del núcleo se deben únicamente al estado del nucleón desapareado. Una mejora de este modelo conssite en considerar todos los nucleones de la última capa semillena, no sólo el desapareado. El siguiente paso en la elaboración de un modelo más detallado sería, obviamente, considerar todos los nucleones, tanto en las capas llenas como las semillenas.

\subsection{Evidencia experimental de capas en los núcleos}

\subsection{Modelo de capas con potnecial armónico}

\subsection{Interacción espín-órbita}

En los años 40 del siglo XX se realizaron muchos esfuerzos para constuir un modelo de estructura nuclear que explicase de los números mágicos. En 1949 Mayer, Haxel, Suess y Jensen probaron que la inclusión de un potencial de interacción espín-órbita producía el desdoblamiento correcto de los subiveles en las capas. En física atómica la intearcción espín órbita tiene su origen en la interacción del momento dipolar mangético creado por el espín del electrón con el campo mangnético que el electrón ve en su propio sistema de referencia\footnote{En el sistema de referencia en que el electrón está en reposo y el núcleo atómico está en movimiento y genera un cmapo magnético que interacciona con el momento dipolar magnético del electrón.}; los efectos aquí son típicamente del orden de una parte entre $10^5$, demasiado pequeños como para generar un desdoblamiento energético que modifique los \textit{números mágicos atómicos}. En el caso del nucleo veremos que la interacción espín-órbita si introduce cambios significativos en lo que conciernte a la secuencia de números mágicos. 

Introducimos entonces una interacción espín-órbita en el potencial nuclear de la siguiente manera:

\begin{eqnarray}
    V(r) \longrightarrow V(r) + V_{so} (r) \parentesis{\lnn \cdot \sn}
\end{eqnarray}
El factor $V_{so} (r)$ no es el más importante aquí, el que causa el desdoblamiento es el término $\ln\cdot\sn$. Los estados de cada partícula se tienen que etiquetar ahora como el número cuántico correspondiente al momento angular total $\jn =\lnn+\sn$. Como los nucleones tienen $s=1/2$, los posibles valores de $j$ para un nucleón son $j=\ell\pm 1/2$, excepto para $\ell=0$ que solo es posible $j=1/2$. Podemos calcular el valor esperado de la expresión $\lnnn \cdot \sn$ como:

\begin{eqnarray}
    \jn^2 = \lnn^2 + 2 \lnn\cdot\sn + \sn^2 \Rightarrow \lnnn \cdot \sn = \frac{1}{2} \parentesis{\jn^2 -\ln^2 -\sn^2}     
\end{eqnarray}
\begin{eqnarray}
    \langle \lnnn \cdot \sn \rangle = \frac{\hbar^2}{2} \ccorchetes{j(j+1)-\ell(\ell+1) - s(s-1)}
\end{eqnarray}
