\chapter{Propiedades de los núcleos}


En un sismtea cuántico complejo con varios componentes. Es la parte más pesada de un átomo aunque sólo ocupa una pequeña fracción de su volumen. Esta formado por A nucleones (Z protones y N neutrones).

\begin{equation}
    (A,Z) \equiv ^A X \equiv ^A_Z X \equiv^A _Z X_N
\end{equation}
Con 

\begin{itemize}
\item \textbf{Isótopos:}
\item \textbf{Isóstanos:}
\item \textbf{Isocoros:}
\end{itemize}

La masa de un sistema compuesto es la suma de las masas de los componentes menos la energía que los mantiene unidos, la llamada {\bf energía de ligadura}. En general, si el núcleo es estable, no existe ninguna combinación de sus componentes por separado que tenga una masa menor (si ese fuese el caso, el núcleo se desintegraría en esa combinación).

\begin{equation}
M(^A X) = Z \cdot m_p + (A-Z) \cdot m_n - B_N
\end{equation}

En la mayor parte de los casos, y de ls tablas, la masa medida corresponde al a masa atómica:

\begin{equation}
M(^A X) = Z \cdot m_p + (A-Z) \cdot m_n - B_N + Z m_e - \sum_i^Z B_i
\end{equation}
donde:

\begin{itemize}
    \item $B_i$ energía de ligadura del electrón $i$.
    \item $B_N$ energía de ligadura.
\end{itemize}
A menudo, podremos aproximar, teniendo en cuenta ñps valores típicos de cada componente. 

Una forma habitual de expresar la masa atómica es el llamado \textbf{defecto de masa} $\Delta$:

\begin{equation}
    \Delta  = (M(Z,A)-A\cdot u)
\end{equation}
donde $u=931.49$ MeV/c$^2$ es la \textbf{unidad de masa atómica}. Se define como la doceava parte de la masa de un átomo de carbono 12. 

La energía de separación es la cantidad de energía que se necesita (o se obtiene) para separar un neutrón o protón de un núcleo.

