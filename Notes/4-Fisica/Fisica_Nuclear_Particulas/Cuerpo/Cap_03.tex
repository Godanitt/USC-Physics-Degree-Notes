\chapter{Reacciones Nucleares}

Las reacciones nuclares más comunes tienen lugar cuando una partícula enegética incide sobre un núcleo y éste se transforma: exictándose, rompiéndose o simplemente absorbiendo la partícula incidente. Estas partículas incidentes son generalmente neutrones, protones, partículas $\alpha$ o fotones $\gamma$. Para que penetren dentro de del núcleo sobre el que inciden es necesario que lleven cierta energía. En la Tierra esa energía se puede conseguir con aceleradores o reactores nucleares, y también a partri de fuentes naturales radiactivas. Las reacciones nucleares permiten el estudio de las interacciones que gobiernan el mundo subnuclear y, por otro lado, proporcionan la mayor parte de los datos tabulados sobre las propiedades nucleares. Estas dos cosas, están obviamente relacionadas, porque sólo es posible entender las propiedades de los núcleos, si se posee al mismo tiempo una buena comprensión de las interacciones nucleares.

% Faltan cosas

\section{Tipos de reacciones}

Representamos una reacción nuclear típica de las siguientes dos maneras equivalentes:

\begin{equation}
    a + A  \longrightarrow B + b \tquad A(a,b)B \label{Ec:03-01-01}
\end{equation}
donde $a$ es el proyectil o partícula acelerada que se hace incidir sobre el núcleo blanco, $A$, en reposo en el sistema laboratorio. De las partículas en el segundo miembro de la reacción, $B$ suele ser un núcleo pesado que no abandona el material del blanco, y $b$ una partícula para referirse a un conjunto de reacciones del mismo tipo. Así diríamos reacciones ($\alpha,n$) o ($n,\gamma$), por ejemplo. Hay muchas maneras de clasificar las reacciones nucleares. He aquí unas cuantas:

\begin{itemize}
    \item Se suele hablar de una \textbf{reacción de dispersión} (\textit{scattering process}) cuando las partículas iniciales y finales son las mismas\footnote{En física de partículas de altas energías se usa el término \textit{scattering} de un modo más general, no sólo para referirse a reaccioens en las que las partículas iniciales coinciden con las finales. En el \textit{deep inelastic scattering} por ejemplo, la energía del proyectil es tan alta que se producen muchas partículas en el estado final}. La dispersión puede ser \textbf{elástica} si las patículas o núcleos $B$ y $b$ se encuentran en su estado fundamental, o \textbf{inelástica} si alguna de las dos queda en un estado excitado, que posteriormente se suele desexcitar por emisión gamma. En una dispersión elástica la energía cinética se conserva ($Q=0$) y simplemente se redistribuye entre las partículas interaccionantes.
    \item Si las partículas $a$ y $b$ son la misma, y además tenemos otro nucleón en el estado final (3 partículas como productos) se suele denominar una \textbf{reacción knockout}.
    \item Tenemos una \textbf{reacciones de transferencia} (\textit{transfer reaction}) cuando se transfieren uno o varios nucleones entre el proyectil y el blanco.
\end{itemize}

% Faltan cosas


\section{Leyes de conservación}

\subsection{Conservación de la carga eléctrica y el número bariónico}

Aunque veremos en un capítulo posterior estas leyes de conservación con algo más de detalle, conviene mencionarlas ya aquí. La carga eléctrica total de las partículas iniciales de la reacción es siempre igual a la de las partículas finales. La \textbf{conservación de la carga eléctrica} es una ley \textit{muy fundamental} en nuestro entendimiento actual de la física, al mismo nivel que la ley de conservación de la energía.

% Insertar foto tikz

Si asginamos a cada a cada barión\footnote{Un barión es un hadrón (partícula sensible a la interacción fuerte) con espín semientero. Los bariones más ligeros son los familiares protón y neutrón.} una unidad positiva de \textit{número bariónico} y a cada antibarión una unidad negativa, podemos formular una ley de \textbf{conservación del número de bariónico} en cualqueir reacción diciendo que la suma de números bariónicos para las partículas iniciales debe coincidir con la misma suma para las partículas finales.  % Falta texto

\subsection{Conservación de la energía y del momento lineal}

Las interacciones nucleares tienen lugar a distancia mucho más pequeñas que la separación típica entre los núcleos de un material ordinario, por eso se puede considerar a las partículas interaccionantes en una reacción nuclear como un sistema aislado y aplicar la ley de conservación de la energía total y del momento lineal total. De acuerdo con la notación expresada en (\ref{Ec:03-01-01}) escribimos la \textbf{conservación de la energía}

\begin{equation}
    T_a + m_a c^2 +T_A+m_A c^2 = T_b +m_bc^2 + T_b + m_Bc^2
\end{equation}
tal que $E_A = T_A + m_Ac^2$... El valor $Q$ del proceso o de la reacción se define como la difernecia entre la energía cinética inicial y final 

\begin{equation}
    Q \equiv T_B + Tb - T_A - T_a = (m_A + m_a - m_B - m_b) c^2
\end{equation}

Si, como es habitual, estamos analizando un experimento en el que el núcleo blanco se encuentra en reposo en el sistema laboratorio ($T_A=0$), entonces tenemos que $Q = T_B + T_b - T_a$. La energía cinética del núcleo $T_B$ es dfifícil de medir, y son las energías del proyectil y la de la partícula emergente ($T_a$ y $T_b$) las que suelen medirse. La De todos modos veremos que podemos encontrar una expresión para $Q$ (ecuación ())  que no incluye la energía cinética.


% Falta texto

\subsection{Energía umbral de reacción}

\subsection{Conservación del moemnto angular y de la paridad}

\subsection{Isospín}

\section{Dispersión y secciones eficaces}

Lo que usualmente se mide en las reacciones nucleares es el momento de las partículas ligeras emitidas (y por lo tanto su energía cinética, supuesto que se conozca la identidad de la partícula) y su distribución angular. % Falta texto

\subsection{Atenuación de un haz al atravesar un blanco}


\subsection{Dispersión de Coulomb}

\subsection{Dispersión nuclear}


\section{Mecanismos de reacciones}

\section{Fisión}

\section{Fusión}

La mayor dificultad para lograr la fusión nuclear a gran escala consiste en mantener el material fusible confinado a altas temperaturas durante el tiempo suficiente. Hasta el momento se está investigando en dos métodos: el confinamiento magnético y el confinamiento inercial. En el primero se hace circular plasma caliente de núcleos $^2$H y $^3$H en una región confinada por campos electromagnéticos. En el segundo se inyecta luz láser en una pequeña región que contiene el material fusible. En cualquier caso, el aprovechamiento comercial de la energía de fusión parece todavía una posibilidad lejana.

\section{Apéndices}