\chapter{Reacciones Nucleares}

Las reacciones nuclares más comunes tienen lugar cuando una partícula enegética incide sobre un núcleo y éste se transforma: exictándose, rompiéndose o simplemente absorbiendo la partícula incidente. Estas partículas incidentes son generalmente neutrones, protones, partículas $\alpha$ o fotones $\gamma$. Para que penetren dentro de del núcleo sobre el que inciden es necesario que lleven cierta energía. En la Tierra esa energía se puede conseguir con aceleradores o reactores nucleares, y también a partri de fuentes naturales radiactivas. Las reacciones nucleares permiten el estudio de las interacciones que gobiernan el mundo subnuclear y, por otro lado, proporcionan la mayor parte de los datos tabulados sobre las propiedades nucleares. Estas dos cosas, están obviamente relacionadas, porque sólo es posible entender las propiedades de los núcleos, si se posee al mismo tiempo una buena comprensión de las interacciones nucleares.

% Faltan cosas

\section{Tipos de reacciones}

Representamos una reacción nuclear típica de las siguientes dos maneras equivalentes:

\begin{equation}
    a + A  \longrightarrow B + b \tquad A(a,b)B \label{Ec:03-01-01}
\end{equation}
donde $a$ es el proyectil o partícula acelerada que se hace incidir sobre el núcleo blanco, $A$, en reposo en el sistema laboratorio. De las partículas en el segundo miembro de la reacción, $B$ suele ser un núcleo pesado que no abandona el material del blanco, y $b$ una partícula para referirse a un conjunto de reacciones del mismo tipo. Así diríamos reacciones ($\alpha,n$) o ($n,\gamma$), por ejemplo. Hay muchas maneras de clasificar las reacciones nucleares. He aquí unas cuantas:

\begin{itemize}
    \item Se suele hablar de una \textbf{reacción de dispersión} (\textit{scattering process}) cuando las partículas iniciales y finales son las mismas\footnote{En física de partículas de altas energías se usa el término \textit{scattering} de un modo más general, no sólo para referirse a reaccioens en las que las partículas iniciales coinciden con las finales. En el \textit{deep inelastic scattering} por ejemplo, la energía del proyectil es tan alta que se producen muchas partículas en el estado final}. La dispersión puede ser \textbf{elástica} si las patículas o núcleos $B$ y $b$ se encuentran en su estado fundamental, o \textbf{inelástica} si alguna de las dos queda en un estado excitado, que posteriormente se suele desexcitar por emisión gamma. En una dispersión elástica la energía cinética se conserva ($Q=0$) y simplemente se redistribuye entre las partículas interaccionantes.
    \item Si las partículas $a$ y $b$ son la misma, y además tenemos otro nucleón en el estado final (3 partículas como productos) se suele denominar una \textbf{reacción knockout}.
    \item Tenemos una \textbf{reacciones de transferencia} (\textit{transfer reaction}) cuando se transfieren uno o varios nucleones entre el proyectil y el blanco.
\end{itemize}


Se puede hacer una clasificación de las reacciones atendiendo al mecanismo por el que proceden, aunque es necesario advertir que la división entre ellas no es estricta, y que una misma reacción puede proceder de varias anteriores:

\begin{itemize}
    \item En las \textbf{reacciones directas} sólo unos cuantos nucleones intervienen en la reacción. Las reacciones de trasferencia antes citadas pertenecerían a este grupo.
    \item En las \textbf{reacciones de núcleo compuesto} el proyectil y el núcleo blanco llegan a compartir brevemente toda su energía en un único modelo compuesto antes de emitir, por ejemplo, el nucleón expulsado. Una lejana analogía de la emisión nuclear en un núcleo compuesto podría ser la evaporación de una molécula de la superficie de un líquido caliente.
    \item Las reacciones que proceden a través de \textbf{resonancias} estarían en un punto intermedio a los dos extremos anteriores. Se llega a formar un estado \textit{cuasi-ligado} antes de que se emita la partícula expulsada.
\end{itemize}
Atendiendo al tipo de partícula que induce la reacción podríamos clasificarlas como:

\begin{itemize}
	\item Reacciones inducidas por \textbf{neutrones}.
	\item Reacciones inducidas por \textbf{partículas cargadas}.
	\item Reacciones inducidas por \textbf{fotones} $\gamma$.	
\end{itemize}
Este tercer tipo de reacciones tiene ver más con la interacción electromagnética que con la nuclear, pero se suelen incluir dentro de las reacciones nucleare porque tienen lugar en las cercanías nucleares y conducen a su transformación. Otro tipo de reacciones serían:

\begin{itemize}
	\item Reacciones de fisión de núcleos pesados.
	\item Reacciones de fusión.
	\item Reacciones de creación de elementos transuránicos.
\end{itemize}

\begin{figure}[h!] \centering
\begin{pspicture}(-4,-2)(3,2.5)		
	\psline[linewidth=1.2pt,arrowscale=2]{->}(-4,0)(0,0)
	\psline[linewidth=1.2pt,arrowscale=2]{->}(0,0)(2.5,2.2)
	\psline[linewidth=1.2pt,arrowscale=2]{->}(0,0)(3,-1.5)
	\psline[linestyle=dashed](0,0)(3,0)
	
	\rput(-4,0.2){a}
	\rput(-4,-0.2){$\pn_a$}
	
	\rput[R](2.5,1.9){b}
	\rput(2.5,2.4){$\pn_b$}
	
	\rput[R](3,-1.2){B}
	\rput[R](3,-1.8){$\pn_B$}	
	
	\rput(0,-0.3){A}
	
	\rput(1,0.3){$\theta$}
	\rput(1.1,-0.2){$\phi$}
	
	\pswedge(0,0){0.7}{0}{40}
	\pswedge(0,0){0.8}{-26}{0}
	
\end{pspicture}
\caption{esquema de la reacción $a+A\rightarrow B+b$.}
\label{Fig:03-01}
\end{figure}

\section{Leyes de conservación}

\subsection{Conservación de la carga eléctrica y el número bariónico}

Aunque veremos en un capítulo posterior estas leyes de conservación con algo más de detalle, conviene mencionarlas ya aquí. La carga eléctrica total de las partículas iniciales de la reacción es siempre igual a la de las partículas finales. La \textbf{conservación de la carga eléctrica} es una ley \textit{muy fundamental} en nuestro entendimiento actual de la física, al mismo nivel que la ley de conservación de la energía.


Si asignamos a cada a cada barión\footnote{Un barión es un hadrón (partícula sensible a la interacción fuerte) con espín semientero. Los bariones más ligeros son los familiares protón y neutrón.} una unidad positiva de \textit{número bariónico} y a cada antibarión una unidad negativa, podemos formular una ley de \textbf{conservación del número de bariónico} en cualquier reacción diciendo que la suma de números bariónicos para las partículas iniciales debe coincidir con la misma suma para las partículas finales. Obsérvese que el proceso 
 
\begin{equation}
    \p + \e^- \longrightarrow  2 \gamma
\end{equation}
viola la ley de conservación del número bariónico, y desde luego no se observa. En otras palabras: el átomo de hidrógeno es estable. En reacciones ordinarias en las que la energía no es suficiente para la creación de antipartículas, esta ley se reduce a la \textbf{conservación del número total de nucleones}. En reacciones simples esto quiere decir que el número másico $A$ se conserva.

\subsection{Conservación de la energía y del momento lineal}

Las interacciones nucleares tienen lugar a distancia mucho más pequeñas que la separación típica entre los núcleos de un material ordinario, por eso se puede considerar a las partículas interaccionantes en una reacción nuclear como un sistema aislado y aplicar la ley de conservación de la energía total y del momento lineal total. De acuerdo con la notación expresada en (\ref{Ec:03-01-01}) escribimos la \textbf{conservación de la energía}

\begin{equation}
    T_a + m_a c^2 +T_A+m_A c^2 = T_b +m_bc^2 + T_b + m_Bc^2
\end{equation}
tal que $E_A = T_A + m_Ac^2\ldots$. El valor $Q$ del proceso o de la reacción se define como la difernecia entre la energía cinética inicial y final 

\begin{equation}
    Q \equiv T_B + Tb - T_A - T_a = (m_A + m_a - m_B - m_b) c^2
\end{equation}
Si, como es habitual, estamos analizando un experimento en el que el núcleo blanco se encuentra en reposo en el sistema laboratorio ($T_A=0$), entonces tenemos que $Q = T_B + T_b - T_a$. La energía cinética del núcleo $T_B$ es dfifícil de medir, y son las energías del proyectil y la de la partícula emergente ($T_a$ y $T_b$) las que suelen medirse. De todos modos veremos que podemos encontrar una expresión para $Q$ que no incluye la energía cinética $T_B$. 

La deducción de la expresión para la energía cinética de la partícula o núcleo ligero saliente, $T_b$, en función de las demás variables en el sistema de referencia del laboratorio. Usaremos mecánica no relativista, porque es suficientemente aproximada para las reacciones nucleares de baja energía. Por conservación del momento lineal deducimos que las trayectorias de la partícula proyectil, $a$, y las de las partículas salientes, $B$ y $b$, están contenidas en un mismo plano, de lo contrario no se conservaría la componente del momento perpendicular a dicho plano. De acuerdo con la figura \ref{Fig:03-01} podemos plantear las 3 ecuaciones siguientes:

\begin{equation}
	\begin{split}
		Q \ = \ & \ T_B + T_b - T_A - T_A \\
		p_a \  = \ & \ p_b \cos (\theta) + p_B \cos (\phi) \\
		0 \ = \ & \ p_b \sin (\theta) - p_B \sin (\phi)
	\end{split}
\end{equation}
De aquí, sumando el cuadrado de las dos últimas y despejando, y aplicando que $T_B=p^2_B/2m_B$ en la primera ecuación, podemos despejar $p_B^2$:

\begin{equation}
	p_B^2 = (p_a - p_b \cos (\theta))^2 + p_b^2 \sin^2 \theta \quad p_B^2 = 2m_B \parentesis{Q+T_A+T_a-T_B}
\end{equation} 
Igualando ambas podemos obtener una ecuación de segundo grado de $p_b^2$ de la cual obtenemos que está multivaluada:

\begin{equation}
	p_b = \frac{m_b}{(m_B + m_b)} \ccorchetes{2p_a \cos (\theta) \pm \sqrt{4p_a^2 \cos^2 (\theta)- 4\frac{m_B+m_b}{m_b} \ccorchetes{p_a^2  -2m_B(Q+T_a)} }}
\end{equation}
de lo que se deduce la expresión de la energía cinética:
\begin{equation}
	\sqrt{T_b} = \frac{\sqrt{m_a m_b T_a}  \cos (\theta)}{m_B+m_b} \pm \frac{\sqrt{m_am_b T_a \cos^2 (\theta) + (m_B+m_b) \ccorchetes{m_BQ+(m_B-m_a)T_a}}}{m_B+m_b}   \label{Ec:03-02-07}       
\end{equation}
Esta última expresión nos permite evaluar $T_b$  a partir de datos usualmente conocidos en las reacciones, como lo son las masas de las partículas o núcleos intervinientes, la energía de la cinética incidente y el ángulo de dispersión. Examinando esta fórmula podemos extraer las siguientes conclusiones

\begin{itemize}
	\item Si $Q<0$, existe un mínimo valor de $T_a$ por debajo del cual el discriminante de (\ref{Ec:03-02-07}) se hace negativo y la reacción no puede tener lugar. Esta energía cinética umbral del proyectil se obtiene igualando a cero este discriminante:
	
	\begin{equation}
		T_a = -Q\frac{m_B(m_B+m_b)}{(m_b+m_B)(m_B-m-a)+m_am_b\cos^2 (\theta)}
	\end{equation}
	Está claro además que el mínimo valor de $T_a$ se alcanza cuando $\cos^2 (\theta)  =1$. Por lo tanto la energía cinética umbral es:
	
	\begin{mybox}
	\begin{equation}
		(T_a)_{\text{um}} = - Q \frac{m_B + m_b}{m_B + m_b - m_a}
	\end{equation}
\end{mybox}
	La condición de energía umbral se alcanza por tanto para $\theta = 0$, y por tanto $\phi=0$. Esto quiere decir que las partículas se mueven en una misma línea recta sin que se gaste energía extra en darles una componente de momento perpendicular. Si $Q>0$ no existe energía umbral (lógicamente), es decir, la reacción podría producirse en principio para cualquier valor de $T_a$. De todos modos, en el caso de partículas cargadas, como una partícula $\alpha$, es obvio que se necesita un cierto valor mínimo de $T_a$ para vencer la barrera coulombiana y acercar suficientemente los núcleos para que la reacción nuclear puede tener lugar.
	\item El doble signo de la raíz en (\ref{Ec:03-02-07}) implica que existe una región bivaluada para $T_b$, es decir, una región donde un mismo ángulo de dispersión y una misma energía cinética de la partícula, $T_a$, puede producir dos valores distintos de $T_b$. Esto significa que no podemos seleccionar de fomra unívoca $T_b$ simplemente variando el ángulo de dispersión-observación $\theta$. No obstante, esta región bivaluada es usualmente muy pequeña.
	\item Dentro de este rango bivaluado de energías existe un máximo valor que el ángulo de dispersión puede tomar, $\theta_m$. Este valor se obtiene igualando a cero el discriminante de la ecuación \ref{Ec:03-02-07}:
	\begin{equation}
		\cos^2 (\theta_m) = - \frac{(m_B+m_b)\ccorchetes{m_B Q + (m_B-m_a) T_a}}{m_a m_b T_a}
	\end{equation}
	\item Las reacciones $Q>0$ no tienen región bivaluada.
\end{itemize}
Midiendo el ángulo de dispersión $\theta$ y las energías cinéticas $T_a$ y $T_b$ podemos determinar el valor de $Q$:
\begin{mybox}
\begin{equation}
	Q = T_b \parentesis{1+\frac{m_b}{m_B}} - T_a \parentesis{1-\frac{m_a}{m_B}} - 2 \sqrt{\frac{m_a}{m_B} \frac{m_b}{m_B} T_a T_b} \cos (\theta)  \label{Ec:03-02-11}
\end{equation}
\end{mybox}
Aunque en esta ecuación aparece la masa del núcleo $B$, es una buena proximación reemplazar $m_B$ directamente por su número másico, especialmente si se realizan las mediciones para $\theta \sim 90^{\textit{o}}$. De esta manera, conociendo $m_A$, $m_a$ y $m_b$, se puede obtener el valor de $m_B$ si determinamos $Q$ experimentalmente a partir de la ecuación (\ref{Ec:03-02-11}). \\

Hemos de consderar además la posibilidad de que en la reacción se produzca un estado excitado del núcleo $B$. En este caso el valor $Q$ del proceso debe expresarse en términos de la masa del núcleo excitado, es decir:

\begin{equation}
	Q_{ex} = \parentesis{m_A+m_a-m_{B^*} -m_b}c^2 = Q_0 - E_{ex}
\end{equation}
donde $Q_0$ es el valor $Q$ que correspondería a la reacción yendo al estado fundamental de $B$, y $E_{ex}$ es la energía del estado excitado ($m_{B^*} c^2 = m_Bc^2 + E_{ex}$). En una reacción dada, el valor más alto de $T_b$ se obtendría cuando el valor $B$ se queda en el estado fundamental, de tal modo que podríamos usar la ecuación (\ref{Ec:03-02-11}).


\subsection{Energía umbral de reacción}

Si queremos conocer qué energía cinética, $T_a$, tenemos que comunicarle al proyectil, para alcanzar un nivel excitado concreto en el núcleo final $B$, resulta más sencillo realizar los cálculos en el sistema de referencia del centro de masas (CM).  

\begin{mybox}
	\begin{equation}
		T_{um}' = (T_A'+T_a')_{um} = - Q
	\end{equation}	
\end{mybox}

\subsubsection{Derivación relativista}

La definición de la energía umbral relativista:

\begin{equation}
		\ccorchetes{\parentesis{T_a}_{\text{um}}}_{\textbf{r}} = - Q \frac{m_A + m_a +m_B + m_b}{2m_A}
\end{equation}
Se puede ver fácilmente que esta fórmula es generalizable a una reacción con un número arbitrario de partículas en el estado final:

\begin{equation}
	\ccorchetes{\parentesis{T_a}_{\text{um}}}_{\textbf{r}} = - Q \frac{\sum m_i}{2m_A}
\end{equation}

De todos modos, puesto que las expresiones dadas en () son las dos igualmente simples, no obtenemos una ventaja en el cálculo al usar la no relativista frente a la relativista, que además es más fácil de recordar. Por lo tanto adoptaremos comúnmente el uso de esta última y escribiremos simplemente:

\begin{mybox}
\begin{equation}
	(T_a)_{\text{um}} = -Q\frac{\sum m_i}{2m_A}
\end{equation}
\end{mybox}




\subsection{Conservación del momento angular y de la paridad}

El momento angular total de las partículas que interaccionan en una reacción nuclear se conserva, así como su proyección sobre una dirección seleccionada (usualmente el eje $z$, que se toma como el eje de la colisión). En una reacción del tipo 

\begin{eqnarray}
    a + A \longrightarrow B + b
\end{eqnarray}
podmeos denotar por $\In_i = \In_A + \In_a$ el espín total de las partículas iniciales y $\In_f=\In_B + \In_b$ el espín total de las partículas finales, $\lnn_{Aa}$ el momento angular orbital que caracteriza el movimiento relativo de las partículas iniciales, y $\lnn_{Bb}$ el correspondiente a las partículas finales. Con esta notación, la ley de conservación del momento angular total se escribe como 

\begin{mybox}
\begin{equation}
    \In_A + \In_a + \lnn_{Aa} = \In_B + \In_b + \lnn_{Bb} 
\end{equation}
\end{mybox}
Como es sabido, los espines nucleares (o de las partículas intervinientes), $\In_i$ e $\In_f$ pueden ser enteros o semienteros, mientras que los momentos angulares orbitales $\lnn_{Aa}$ y $\lnn_{Bb}$ deben ser enteros. \\

% Falta texto que parece relevante

La paridad de la función de ondas que describe el movimiento relativo de las partículas $A$ y $a$ viene determinada por su momento angular orbital. La paridad global del sistema inicial será el producto de las paridades intrínsecas\footnote{Más detalles sobre lo que es la paridad intrínseca puede encontrarse en la sección \ref{Sec:07-09}} de las partículas (que a su vez peuden ser subsistemas compuestos) y la paridad de su movimiento relativo:

\begin{eqnarray}
    \pi_i = \pi_A \pi_a (-1)^{l_Aa}
\end{eqnarray}
Escribiríamos entonces la ley de conservación de la paridad Como

\begin{mybox}
\begin{equation}
    \pi_A \pi_a (-1)^{l_{Aa}} = \pi_B \pi_b (-1)^{l_{Bb}}
\end{equation}
\end{mybox}

La conservación de la paridad en las reacciones nucleares no es estricta. Aunque podemos decir que la fuerza nuclear fuerte es la gran responsable de estos procesos, es necesario tener en cunta que la fuerza nuclear débil también está presente e introducirá pequeños efectos de violación de la paridad. Ya veremos la aplicación concreta de estas leyes de conservación, aunque en realidad ya las hemos estado discutiendo para los distintos tipos de desintegraciones en los núcleos.

\section{Dispersión y secciones eficaces}

Lo que usualmente se mide en las reacciones nucleares es el momento de las partículas ligeras emitidas (y por lo tanto su energía cinética, supuesto que se conozca la identidad de la partícula) y su distribución angular. Esto permite observar la partícula emergente en un cierto elemento diferencial de ángulo sólido $\D \sigma / \D \Omega$ a veces también representada por $\sigma (\theta, \phi)$. Si estamos interesandos en encontrar a la partícula emergente no sólo a un cierto ángulo, sino también con una cierta energía, hablamos de la sección eficaz diferencial doble, $\D^2 \sigma / \D \Omega \D E$, pero usualmente no se pone de modo explícita esta dependencia con la energía porque se supone implícita a menos que se diga lo contrario. Integrando la sección eficaz diferencial sobre el ángulo sólido obtendríamos la sección eficaz diferencial total a una energía dada $\D \sigma / \D E$. Por último, si también integramos esta energía, obtendríamos la sección eficaz total absoluta $\sigman_t$, que representa la probabilidad de formar al núcleo $B$ en el estado final (en reacciones del tipo $a+A\longrightarrow B + b$). 

Para ciertos estudios más detallados se suelen medir también secciones eficaces dependeindo de la orientación de los espines de las partículas emergentes. Esto, junto con la observación del espectro $\gamma$ de desexcitación de los núcleos formados, permite acumular información para asignar espines y paridades a los estados de los núcleos bajo estudio.

\subsection{Atenuación de un haz al atravesar un blanco}

Una disposición experimental frecuente en física nuclear consiste en hacer incidir un haz de partículas sobre un blanco fijo para producir cierto tipo de reacción. Dada la probabilidad de itneracción por unidad de longitud, podemos calcular en qué medida se atenuará el haz incidente al atravesar el blanco delgado \footnote{Blanco delgado queire decir que a cada núcleo del blanco está llegando aproximadamente el mismo flujo de partículas del haz. Si el blanco es demasiado grueso podría ocurrir que las partículas del haz fuesen completamente absorbidas en el interior del blanco, de modo que sobre los núcleos de la parte posterior de este último ya no incidiría ninguna partícula del haz.} Denotemos por $1/\lambda$ la probabildiad de itneracción de las partículas del haz por unidad de longitud en el material del blanco. Queremos calcular la probabilidad de que una partícula del haz atraviese un intervalo de longitud $X$ en el blanco sin sufrir ninguna interacción. Para ello dividimos el intervalo $X$ en una serie de intervalos infinitesimales cuyo tamaño podemos expresar como $X/n$ siendo $n$ un número suficientemente grandes. La probabildiad infinitesimal $\D P$ de que la partícula incidente logre atravesar el intervalo de longitud $X/n$ será $\D P = 1 - \frac{1}{\lambda} \frac{X}{n}$, haciendo tender $n$ a infinito al mismo tiempo que aplicamos $n$ veces la fórmula anterior obtenemos:  

\begin{equation}
    \D P = 1 - \frac{1}{\lambda} \frac{X}{n} \tquad P = \lim_{n\rightarrow \infty} \parentesis{1 - \frac{1}{\lambda}\frac{X}{n}}^n = e^{-X/\lambda}
\end{equation}
Por lo tanto, si $\phi_0$ es el flujo del haz incidente, representando el número de de partículas que en la unidad de tiempo atraviesan una unidad de área perpendicular a la dirección del haz, y $\phi$ es el flujo de salida (detrás) del blanco delgado de expresor $X$, deducimos que están relacionados por medio de la siguiente ecuación 

\begin{equation}
    \phi = \phi_0 e^{-X/\lambda}
\end{equation}
$\lambda$ representa el recorrido libre medio (\textit{mean free path}) de las partículas del haz en el material del blanco\footnote{Podemos comparar esto con la ley de exponencial de la desintegración radiactiva $N=N_0 e^{-t/\tau}$, donde $\tau$ es le vida media, que en el contexto actual equivaldría al recorrido libre medio, y $t$ es el tiempo trascurrido (distancia temporal), que en el contexto actual equivaldría a distancia física recorrida. El inverso de $\tau$ es la probabilidad de desintegración por unidad de tiempo, y ahora hablaríamos de la probabilidad de interacción por unidad de longitud espacial, no temporal.}. Se puede expresar en términos de la sección eficaz de esta meanera\footnote{Supongamos que una partícula atraviesa una región cúbica de lado $L$ en el blanco. Calculemos la probabilidad de que sufra alguna interacción en ese viaje. Supondremos que la trayectoria de la partícula es paralela a una de las caras en las posiciones de los núcleos, donde $\sigma$ es obviamente la sección eficaz total de interacción con un núcleo. El área total cubierta por las secciones eficaces det todos los núcleos del cubo es $A_\sigma = \Ncal L^3 \sigma$, mientras que el área total que el cubo presenta en la dirección perpendicular a la del avance de la partícula es $A_t=L^2$. el cociente entre ambas nos da la probabildiad de que la partícula interaccione al travesar el cubo $P_L  = A_\sigma / A_t = \Ncal L^3 \sigma / L^2$. La probabilidad de itneracción por unidad de longitud es entonces $P=P_L/L=\Ncal \sigma$}: \\

\begin{equation}
    \frac{1}{\lambda} = \Ncal \sigma = \frac{\rho N_A }{M_A} \sigma 
\end{equation}
donde $\Ncal$ es la densidad de átomos blanco (número de átomos por unidad de volumen), que puede expresarse como $\rho N_A /M_A$, siendo $N_A$ el número de Avogadro, $\rho$ la densidad del blanco y $M_A$ su masa atómica (en unidades compatibles con las de $\rho$, por supeusto). 

En lugar de utilizar el espesor del blanco, $X$, es habitual usar el \textit{espesor másico}, que se define como $X_m=X\rho$, es decir, se trata de la densidad por unidad de superficie. De este modo tendremos 

\begin{equation}
    \frac{X}{\lambda} = \frac{X_\rho}{\lambda_\rho} = \frac{X_m}{\lambda_\rho} = \frac{X_m}{\rho} \Ncal \sigma = \frac{X_m}{\rho} \frac{\rho N_A}{M_A} \sigma = \frac{X_m \Ncal \sigma}{M_A}
\end{equation}
También podríamos escribir $X/\lambda=X_m/\lambda_m$, siendo $\lambda_m = M_A /(N_A\sigma)=\lambda \rho$. Una de las aplicaciones de las reacciones nucleares inducidas por el hombre conssite en la creación de radioisótopos para apliaciones médicas o industriales.


% Falta texto importante

\subsection{Dispersión de Coulomb}

Cuando las partículas que intervienen en la reacción tienen carga eléctrica, se producirá siempre una dispersión del proyectil en el campo coulombiano del blanco: a esto se le denomina dispersión de Coulomb. En reacciones de partículas cargadas a baja energía el proceso dominante que tendrá lugar será la \textbf{dispersión elástica coulombiana} o dispersión de Rutherford, cuya sección eficaz hemos visto ya en otro capítulo:

\begin{equation}
    \derivadas{\sigma(\theta)}{\Omega} = \frac{4 Z^2 \alpha^2 (\hbar c)^2 E^2}{|\qn c|^4} = \parentesis{\frac{Ze^2}{4 \pi \epsilon_0}}^2 \frac{1}{(4E)^2 \sin^2(\theta/2)}
\end{equation}
Obsérvese que esta sección eficaz decrece rápidamente con la energía del proyectil ($E$) y con ángulo de dispersión ($\theta$). Cuando la energía sea suficientemente alta, podremos tener además una \textbf{dispersión inelástica coulombiana}, que también se llama excitación coulombiana. En este caso el núcleo final se encuentra en un estado que luego decae rápida,ente mediante la emisión de fotones $\gamma$. 

\subsection{Dispersión nuclear}

En el caso de que el proyectil no tenga carga eléctrica (como en dispersión de neutrones), no será posible, obviamente, que exista dispersión coulombiana. Se produce entonces la denominada dispersión nuclear, que tiene lugar en el campo nuclear no coulombiano\footnote{La fuerza nuclear fuerte es la interacción fundamental entre los quarks, y la que de alguna forma mantiene unidos a los nucleones en el núcleo y crea ese campo nuclear no coulombiano con el que interaccionan los proyectiles masivos sin carga eléctrica.} creado por el núcleo. Para partículas cargadas, aparte de la dispersión coulombiana podrá existir dispersión nuclear siempre que la energía del proyectil sea lo suficientemente alta como para vencer la barrera coulombiana del núcleo. \\

% falta texto

En una reacción de tipo $A+a\longrightarrow B+b$, la medida directa de la cantidad de partículas $b$ emergentes a través de un pequeño intervalo de ángulo sólido dado se hace, obviamente, en el sistema de referencia del laboratorio. Sin  embargo, para comparar los resultados experimentales con las predicciones teóricas es necesario que los traslados al sistema de referenia del centro de masas de las partículas \textit{entrantes} $A,a$. Resulta de especial utilidad, por ejemplo, graficar las llamdas \textbf{funciones de excitación} que muestran la dependencia de la sección eficaz $\D \sigma / \D \Omega$, respecto a la energía cinética del canal entrante en el sistema centro de masas, (que en la aproximación no relativista es $m_A T_a / (m_A + m_a)$), para la observación de una partícula concreta $b$ dispersada en cierto $\D \Omega$. 

La \textbf{distribución angular} de la emisión de la partícula seleccionada $b$ en el sistema centro de masas con respeto a la dirección de incidencia contiene información sobre el cambio de momento angular y paridad de los estados inicial y final de los núcleos, puesto que el momento angular y la paridad se conservan en reacciones nucleares gobernadas por la interacción fuerte y la electromagnética.



\section{Mecanismos de reacciones}

\section{Fisión}

La fisión inducida por neutros se descrubrió en los años 1934-1939 como un resultado de los experimentos que pretendían crear elementos transuránicos mediante el bombardeo de Uranio con neutrones. Un ejemplo concreto de fisión inducida del \ce{^235U} por la abosrción de neutrónicos podría ser:

\begin{equation}
\ce{n+^235_92 U -> ^148_57 La + ^87_35 Br + n}
\end{equation}
La enegía liberada en esta reacción es enorme comparable con la que se libera en otro tipo de procesos no nuclerares. 

\subsection{Fisión según el modelo de la gota líquida}

\subsection{Reacciones en cadena}

La fisión deja a los núcleos hijos excitados, que suelen desexcitarse mediante la emisión (\textit{evaporación}) de neutrones. Estos productos de fisión son ricos en neutrones y se desintegran luego por emisión $\beta^-$. Los neutrones emitidos en el proceso de fisión peuden a su vez generar nuevas fisiones. Como dato, la fisión \ce{^235U} genera un promedio de unos 2.5 neutrones. Resulta útil definir el siguiente parámetro para obtener una clasifiación de las reaccioens:

\begin{eqnarray}    
    k = \frac{\text{Número de neutrones producidos en la etapa (n+1) de fisión}}{\text{Número de neutrones producidos en la etapa (n) de fisión}}
\end{eqnarray}
De acuerdo con $k$ tenemos las siguientes posibilidades:

\begin{itemize}
    \item $k<1$. La reacción transcurre en modo \textit{modo subcrítico}. No puede continuar indefinidamente y en algún momento se detiene.
    \item $k=1$. La reacción transcurre en modo \textit{crítico}. El número de neutrones disponibles para inducir fisiones permanece constante, de tal modo que la reacción peude continuar indefinidamente al mismo ritmo. Es el modo adecuado de operación sostenida en un reactor nuclear productor de energía.
    \item $k>1$. La reacción transcurre en modo \textit{super crítico}. El número de neutrones disponibles crece exponencialmente con el tiempo, conduciendo a una liberación explosiva de energía.
\end{itemize}

\section{Fusión}

La mayor dificultad para lograr la fusión nuclear a gran escala consiste en mantener el material fusible confinado a altas temperaturas durante el tiempo suficiente. Hasta el momento se está investigando en dos métodos: el confinamiento magnético y el confinamiento inercial. En el primero se hace circular plasma caliente de núcleos $^2$H y $^3$H en una región confinada por campos electromagnéticos. En el segundo se inyecta luz láser en una pequeña región que contiene el material fusible. En cualquier caso, el aprovechamiento comercial de la energía de fusión parece todavía una posibilidad lejana.

\section{Apéndices}

\subsection{Energía cinética clásica en el sistema LAB y en el CM}

\subsection{Momento lineal clásico en el sistema LAB y en el CM}

\subsection{Energía total relativista en el sistema LAB y en el CM}

\subsection{Energía cinética relativista en el sistema LAB y en el CM}

\subsection{Momento lineal relativista en el sistema LAB y en el CM}


