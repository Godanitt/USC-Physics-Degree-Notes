\chapter{Reacciones Nucleares} \label{Ch:03}

Las reacciones nuclares más comunes tienen lugar cuando una partícula enegética incide sobre un núcleo y éste se transforma: exictándose, rompiéndose o simplemente absorbiendo la partícula incidente. Estas partículas incidentes son generalmente neutrones, protones, partículas $\alpha$ o fotones $\gamma$. Para que penetren dentro de del núcleo sobre el que inciden es necesario que lleven cierta energía. En la Tierra esa energía se puede conseguir con aceleradores o reactores nucleares, y también a partri de fuentes naturales radiactivas. Las reacciones nucleares permiten el estudio de las interacciones que gobiernan el mundo subnuclear y, por otro lado, proporcionan la mayor parte de los datos tabulados sobre las propiedades nucleares. Estas dos cosas, están obviamente relacionadas, porque sólo es posible entender las propiedades de los núcleos, si se posee al mismo tiempo una buena comprensión de las interacciones nucleares.

% Faltan cosas

\section{Tipos de reacciones}

Representamos una reacción nuclear típica de las siguientes dos maneras equivalentes:

\begin{equation}
    a + A  \longrightarrow B + b \tquad A(a,b)B \label{Ec:03-01-01}
\end{equation}
donde $a$ es el proyectil o partícula acelerada que se hace incidir sobre el núcleo blanco, $A$, en reposo en el sistema laboratorio. De las partículas en el segundo miembro de la reacción, $B$ suele ser un núcleo pesado que no abandona el material del blanco, y $b$ una partícula para referirse a un conjunto de reacciones del mismo tipo. Así diríamos reacciones ($\alpha,n$) o ($n,\gamma$), por ejemplo. Hay muchas maneras de clasificar las reacciones nucleares. He aquí unas cuantas:

\begin{itemize}
    \item Se suele hablar de una \textbf{reacción de dispersión} (\textit{scattering process}) cuando las partículas iniciales y finales son las mismas\footnote{En física de partículas de altas energías se usa el término \textit{scattering} de un modo más general, no sólo para referirse a reaccioens en las que las partículas iniciales coinciden con las finales. En el \textit{deep inelastic scattering} por ejemplo, la energía del proyectil es tan alta que se producen muchas partículas en el estado final}. La dispersión puede ser \textbf{elástica} si las patículas o núcleos $B$ y $b$ se encuentran en su estado fundamental, o \textbf{inelástica} si alguna de las dos queda en un estado excitado, que posteriormente se suele desexcitar por emisión gamma. En una dispersión elástica la energía cinética se conserva ($Q=0$) y simplemente se redistribuye entre las partículas interaccionantes.
    \item Si las partículas $a$ y $b$ son la misma, y además tenemos otro nucleón en el estado final (3 partículas como productos) se suele denominar una \textbf{reacción knockout}.
    \item Tenemos una \textbf{reacciones de transferencia} (\textit{transfer reaction}) cuando se transfieren uno o varios nucleones entre el proyectil y el blanco.
\end{itemize}


Se puede hacer una clasificación de las reacciones atendiendo al mecanismo por el que proceden, aunque es necesario advertir que la división entre ellas no es estricta, y que una misma reacción puede proceder de varias anteriores:

\begin{itemize}
    \item En las \textbf{reacciones directas} sólo unos cuantos nucleones intervienen en la reacción. Las reacciones de trasferencia antes citadas pertenecerían a este grupo.
    \item En las \textbf{reacciones de núcleo compuesto} el proyectil y el núcleo blanco llegan a compartir brevemente toda su energía en un único modelo compuesto antes de emitir, por ejemplo, el nucleón expulsado. Una lejana analogía de la emisión nuclear en un núcleo compuesto podría ser la evaporación de una molécula de la superficie de un líquido caliente.
    \item Las reacciones que proceden a través de \textbf{resonancias} estarían en un punto intermedio a los dos extremos anteriores. Se llega a formar un estado \textit{cuasi-ligado} antes de que se emita la partícula expulsada.
\end{itemize}
Atendiendo al tipo de partícula que induce la reacción podríamos clasificarlas como:

\begin{itemize}
	\item Reacciones inducidas por \textbf{neutrones}.
	\item Reacciones inducidas por \textbf{partículas cargadas}.
	\item Reacciones inducidas por \textbf{fotones} $\gamma$.	
\end{itemize}
Este tercer tipo de reacciones tiene ver más con la interacción electromagnética que con la nuclear, pero se suelen incluir dentro de las reacciones nucleare porque tienen lugar en las cercanías nucleares y conducen a su transformación. Otro tipo de reacciones serían:

\begin{itemize}
	\item Reacciones de fisión de núcleos pesados.
	\item Reacciones de fusión.
	\item Reacciones de creación de elementos transuránicos.
\end{itemize}

\begin{figure}[h!] \centering
\begin{pspicture}(-4,-2)(3,2.5)		
	\psline[linewidth=1.2pt,arrowscale=2]{->}(-4,0)(0,0)
	\psline[linewidth=1.2pt,arrowscale=2]{->}(0,0)(2.5,2.2)
	\psline[linewidth=1.2pt,arrowscale=2]{->}(0,0)(3,-1.5)
	\psline[linestyle=dashed](0,0)(3,0)
	
	\rput(-4,0.2){a}
	\rput(-4,-0.2){$\pn_a$}
	
	\rput[R](2.5,1.9){b}
	\rput(2.5,2.4){$\pn_b$}
	
	\rput[R](3,-1.2){B}
	\rput[R](3,-1.8){$\pn_B$}	
	
	\rput(0,-0.3){A}
	
	\rput(1,0.3){$\theta$}
	\rput(1.1,-0.2){$\phi$}
	
	\pswedge(0,0){0.7}{0}{40}
	\pswedge(0,0){0.8}{-26}{0}
	
\end{pspicture}
\caption{esquema de la reacción $a+A\rightarrow B+b$.}
\label{Fig:03-01}
\end{figure}

\section{Leyes de conservación}

\subsection{Conservación de la carga eléctrica, el número bariónico y leptónico}

Aunque veremos en un capítulo posterior estas leyes de conservación con algo más de detalle, conviene mencionarlas ya aquí. La carga eléctrica total de las partículas iniciales de la reacción es siempre igual a la de las partículas finales. La \textbf{conservación de la carga eléctrica} es una ley \textit{muy fundamental} en nuestro entendimiento actual de la física, al mismo nivel que la ley de conservación de la energía.


Si asignamos a cada a cada barión\footnote{Un barión es un hadrón (partícula sensible a la interacción fuerte) con espín semientero. Los bariones más ligeros son los familiares protón y neutrón.} una unidad positiva de \textit{número bariónico} y a cada antibarión una unidad negativa, podemos formular una ley de \textbf{conservación del número de bariónico} en cualquier reacción diciendo que la suma de números bariónicos para las partículas iniciales debe coincidir con la misma suma para las partículas finales. Obsérvese que el proceso 
 
\begin{equation}
    \p + \e^- \longrightarrow  2 \gamma
\end{equation}
viola la ley de conservación del número bariónico, y desde luego no se observa. En otras palabras: el átomo de hidrógeno es estable. En reacciones ordinarias en las que la energía no es suficiente para la creación de antipartículas, esta ley se reduce a la \textbf{conservación del número total de nucleones}. En reacciones simples esto quiere decir que el número másico $A$ se conserva.


A cada leptón se le asigna \textbf{número leptónico} (un leptón es un fermión sin ningún tipo de carga de color, como pueden ser un electrón, un neutrino, un muón... y sus correspondientes antipartículas) tal que $L=+1$ (o $L=-1$ si es la antipartícula a asociada a un leptón $L=1$). Se conserva dentro de las fmailias leptónicas en todas las interacciones fundamentales.

\subsection{Simetrías discretas}

Existen simetrías discretas que pueden o no ser invariantes dependiendo de la interacción. Por ejemplo, la paridad (P) no se conserva en las interacción nuclear débil. La conjugación de carga (C) es una operación abstracta realizable idealmente sobre un sistema de partículas consistente en cambiar cada partícula por su correspondiente antipartícula. Puesto que una partícula y su antipartícula poseen cargas eléctricas opuestas, al realizar la operación las cargas de todas las partículas no neutras se invierte, de ahí el nombre de conjugación de carga. Todas las interacciones, menos la interacción nuclear débil conservan la conjugación de carga. También es violada la simetría de inversión temporal (T).

Otra simetría que se creía invariante, y también violada por la interacción débil es la simetría CP. La simetría CP nos decía que si sometíamos un sistema a un cambio de paridad y de conjugación de carga sobre el sistema, el resultante sería indistinguible del primero, lo que resultó no ser cierto. Sin embargo, una simetría de la que todavía no se ha encontrado ninguna evidencia de rotura es la simetría CPT.

\subsection{Conservación de la energía y del momento lineal}

Las interacciones nucleares tienen lugar a distancia mucho más pequeñas que la separación típica entre los núcleos de un material ordinario, por eso se puede considerar a las partículas interaccionantes en una reacción nuclear como un sistema aislado y aplicar la ley de conservación de la energía total y del momento lineal total. De acuerdo con la notación expresada en (\ref{Ec:03-01-01}) escribimos la \textbf{conservación de la energía}

\begin{equation}
    T_a + m_a c^2 +T_A+m_A c^2 = T_b +m_bc^2 + T_b + m_Bc^2
\end{equation}
tal que $E_A = T_A + m_Ac^2\ldots$. El valor $Q$ del proceso o de la reacción se define como la difernecia entre la energía cinética inicial y final 

\begin{equation}
    Q \equiv T_B + Tb - T_A - T_a = (m_A + m_a - m_B - m_b) c^2
\end{equation}
Si, como es habitual, estamos analizando un experimento en el que el núcleo blanco se encuentra en reposo en el sistema laboratorio ($T_A=0$), entonces tenemos que $Q = T_B + T_b - T_a$. La energía cinética del núcleo $T_B$ es dfifícil de medir, y son las energías del proyectil y la de la partícula emergente ($T_a$ y $T_b$) las que suelen medirse. De todos modos veremos que podemos encontrar una expresión para $Q$ que no incluye la energía cinética $T_B$. 

La deducción de la expresión para la energía cinética de la partícula o núcleo ligero saliente, $T_b$, en función de las demás variables en el sistema de referencia del laboratorio. Usaremos mecánica no relativista, porque es suficientemente aproximada para las reacciones nucleares de baja energía. Por conservación del momento lineal deducimos que las trayectorias de la partícula proyectil, $a$, y las de las partículas salientes, $B$ y $b$, están contenidas en un mismo plano, de lo contrario no se conservaría la componente del momento perpendicular a dicho plano. De acuerdo con la figura \ref{Fig:03-01} podemos plantear las 3 ecuaciones siguientes:

\begin{equation}
	\begin{split}
		Q \ = \ & \ T_B + T_b - T_A - T_A \\
		p_a \  = \ & \ p_b \cos (\theta) + p_B \cos (\phi) \\
		0 \ = \ & \ p_b \sin (\theta) - p_B \sin (\phi)
	\end{split}
\end{equation}
De aquí, sumando el cuadrado de las dos últimas y despejando, y aplicando que $T_B=p^2_B/2m_B$ en la primera ecuación, podemos despejar $p_B^2$:

\begin{equation}
	p_B^2 = (p_a - p_b \cos (\theta))^2 + p_b^2 \sin^2 \theta \quad p_B^2 = 2m_B \parentesis{Q+T_A+T_a-T_B}
\end{equation} 
Igualando ambas podemos obtener una ecuación de segundo grado de $p_b^2$ de la cual obtenemos que está multivaluada:

\begin{equation}
	p_b = \frac{m_b}{(m_B + m_b)} \ccorchetes{2p_a \cos (\theta) \pm \sqrt{4p_a^2 \cos^2 (\theta)- 4\frac{m_B+m_b}{m_b} \ccorchetes{p_a^2  -2m_B(Q+T_a)} }}
\end{equation}
de lo que se deduce la expresión de la energía cinética:
\begin{equation}
	\sqrt{T_b} = \frac{\sqrt{m_a m_b T_a}  \cos (\theta)}{m_B+m_b} \pm \frac{\sqrt{m_am_b T_a \cos^2 (\theta) + (m_B+m_b) \ccorchetes{m_BQ+(m_B-m_a)T_a}}}{m_B+m_b}   \label{Ec:03-02-07}       
\end{equation}
Esta última expresión nos permite evaluar $T_b$  a partir de datos usualmente conocidos en las reacciones, como lo son las masas de las partículas o núcleos intervinientes, la energía de la cinética incidente y el ángulo de dispersión. Examinando esta fórmula podemos extraer las siguientes conclusiones

\begin{itemize}
	\item Si $Q<0$, existe un mínimo valor de $T_a$ por debajo del cual el discriminante de (\ref{Ec:03-02-07}) se hace negativo y la reacción no puede tener lugar. Esta energía cinética umbral del proyectil se obtiene igualando a cero este discriminante:
	
	\begin{equation}
		T_a = -Q\frac{m_B(m_B+m_b)}{(m_b+m_B)(m_B-m-a)+m_am_b\cos^2 (\theta)}
	\end{equation}
	Está claro además que el mínimo valor de $T_a$ se alcanza cuando $\cos^2 (\theta)  =1$. Por lo tanto la energía cinética umbral es:
	
	\begin{mybox}
	\begin{equation}
		(T_a)_{\text{um}} = - Q \frac{m_B + m_b}{m_B + m_b - m_a}
	\end{equation}
\end{mybox}
	La condición de energía umbral se alcanza por tanto para $\theta = 0$, y por tanto $\phi=0$. Esto quiere decir que las partículas se mueven en una misma línea recta sin que se gaste energía extra en darles una componente de momento perpendicular. Si $Q>0$ no existe energía umbral (lógicamente), es decir, la reacción podría producirse en principio para cualquier valor de $T_a$. De todos modos, en el caso de partículas cargadas, como una partícula $\alpha$, es obvio que se necesita un cierto valor mínimo de $T_a$ para vencer la barrera coulombiana y acercar suficientemente los núcleos para que la reacción nuclear puede tener lugar.
	\item El doble signo de la raíz en (\ref{Ec:03-02-07}) implica que existe una región bivaluada para $T_b$, es decir, una región donde un mismo ángulo de dispersión y una misma energía cinética de la partícula, $T_a$, puede producir dos valores distintos de $T_b$. Esto significa que no podemos seleccionar de fomra unívoca $T_b$ simplemente variando el ángulo de dispersión-observación $\theta$. No obstante, esta región bivaluada es usualmente muy pequeña.
	\item Dentro de este rango bivaluado de energías existe un máximo valor que el ángulo de dispersión puede tomar, $\theta_m$. Este valor se obtiene igualando a cero el discriminante de la ecuación \ref{Ec:03-02-07}:
	\begin{equation}
		\cos^2 (\theta_m) = - \frac{(m_B+m_b)\ccorchetes{m_B Q + (m_B-m_a) T_a}}{m_a m_b T_a}
	\end{equation}
	\item Las reacciones $Q>0$ no tienen región bivaluada.
\end{itemize}
Midiendo el ángulo de dispersión $\theta$ y las energías cinéticas $T_a$ y $T_b$ podemos determinar el valor de $Q$:
\begin{mybox}
\begin{equation}
	Q = T_b \parentesis{1+\frac{m_b}{m_B}} - T_a \parentesis{1-\frac{m_a}{m_B}} - 2 \sqrt{\frac{m_a}{m_B} \frac{m_b}{m_B} T_a T_b} \cos (\theta)  \label{Ec:03-02-11}
\end{equation}
\end{mybox}
Aunque en esta ecuación aparece la masa del núcleo $B$, es una buena proximación reemplazar $m_B$ directamente por su número másico, especialmente si se realizan las mediciones para $\theta \sim 90^{\textit{o}}$. De esta manera, conociendo $m_A$, $m_a$ y $m_b$, se puede obtener el valor de $m_B$ si determinamos $Q$ experimentalmente a partir de la ecuación (\ref{Ec:03-02-11}). \\

Hemos de consderar además la posibilidad de que en la reacción se produzca un estado excitado del núcleo $B$. En este caso el valor $Q$ del proceso debe expresarse en términos de la masa del núcleo excitado, es decir:

\begin{equation}
	Q_{ex} = \parentesis{m_A+m_a-m_{B^*} -m_b}c^2 = Q_0 - E_{ex}
\end{equation}
donde $Q_0$ es el valor $Q$ que correspondería a la reacción yendo al estado fundamental de $B$, y $E_{ex}$ es la energía del estado excitado ($m_{B^*} c^2 = m_Bc^2 + E_{ex}$). En una reacción dada, el valor más alto de $T_b$ se obtendría cuando el valor $B$ se queda en el estado fundamental, de tal modo que podríamos usar la ecuación (\ref{Ec:03-02-11}).


\subsection{Energía umbral de reacción}

Si queremos conocer qué energía cinética, $T_a$, tenemos que comunicarle al proyectil, para alcanzar un nivel excitado concreto en el núcleo final $B$, resulta más sencillo realizar los cálculos en el sistema de referencia del centro de masas (CM).  

\begin{mybox}
	\begin{equation}
		T_{um}' = (T_A'+T_a')_{um} = - Q
	\end{equation}	
\end{mybox}

\subsubsection{Derivación relativista}

La definición de la energía umbral relativista:

\begin{equation}
		\ccorchetes{\parentesis{T_a}_{\text{um}}}_{\textbf{r}} = - Q \frac{m_A + m_a +m_B + m_b}{2m_A}
\end{equation}
Se puede ver fácilmente que esta fórmula es generalizable a una reacción con un número arbitrario de partículas en el estado final:

\begin{equation}
	\ccorchetes{\parentesis{T_a}_{\text{um}}}_{\textbf{r}} = - Q \frac{\sum m_i}{2m_A}
\end{equation}

De todos modos, puesto que las expresiones dadas en () son las dos igualmente simples, no obtenemos una ventaja en el cálculo al usar la no relativista frente a la relativista, que además es más fácil de recordar. Por lo tanto adoptaremos comúnmente el uso de esta última y escribiremos simplemente:

\begin{mybox}
\begin{equation}
	(T_a)_{\text{um}} = -Q\frac{\sum m_i}{2m_A}
\end{equation}
\end{mybox}




\subsection{Conservación del momento angular y de la paridad}

El momento angular total de las partículas que interaccionan en una reacción nuclear se conserva, así como su proyección sobre una dirección seleccionada (usualmente el eje $z$, que se toma como el eje de la colisión). En una reacción del tipo 

\begin{equation}
    a + A \longrightarrow B + b
\end{equation}
podmeos denotar por $\In_i = \In_A + \In_a$ el espín total de las partículas iniciales y $\In_f=\In_B + \In_b$ el espín total de las partículas finales, $\lnn_{Aa}$ el momento angular orbital que caracteriza el movimiento relativo de las partículas iniciales, y $\lnn_{Bb}$ el correspondiente a las partículas finales. Con esta notación, la ley de conservación del momento angular total se escribe como 

\begin{mybox}
\begin{equation}
    \In_A + \In_a + \lnn_{Aa} = \In_B + \In_b + \lnn_{Bb} 
\end{equation}
\end{mybox}
Como es sabido, los espines nucleares (o de las partículas intervinientes), $\In_i$ e $\In_f$ pueden ser enteros o semienteros, mientras que los momentos angulares orbitales $\lnn_{Aa}$ y $\lnn_{Bb}$ deben ser enteros. \\

% Falta texto que parece relevante

La paridad de la función de ondas que describe el movimiento relativo de las partículas $A$ y $a$ viene determinada por su momento angular orbital. La paridad global del sistema inicial será el producto de las paridades intrínsecas\footnote{Más detalles sobre lo que es la paridad intrínseca puede encontrarse en la sección \ref{Sec:07-09}} de las partículas (que a su vez peuden ser subsistemas compuestos) y la paridad de su movimiento relativo:

\begin{equation}
    \pi_i = \pi_A \pi_a (-1)^{l_Aa}
\end{equation}
Escribiríamos entonces la ley de conservación de la paridad Como

\begin{mybox}
\begin{equation}
    \pi_A \pi_a (-1)^{l_{Aa}} = \pi_B \pi_b (-1)^{l_{Bb}}
\end{equation}
\end{mybox}

La conservación de la paridad en las reacciones nucleares no es estricta. Aunque podemos decir que la fuerza nuclear fuerte es la gran responsable de estos procesos, es necesario tener en cuenta que la fuerza nuclear débil también está presente e introducirá pequeños efectos de violación de la paridad. Ya veremos la aplicación concreta de estas leyes de conservación, aunque en realidad ya las hemos estado discutiendo para los distintos tipos de desintegraciones en los núcleos.

\subsection{Conservación de isospín}

El isospín $\Tn$ es un número cuántico ``efectivo'' que asume que el protón y el neutrón son la misma partícula con dos estados de isoespín, El isoespín cumple las mismas reglas algebraicas que un momento angular:

\begin{equation}
	T^2 |t,m_t\rangle = t(t+1) |t,m_t \rangle \tquad T_3 |t,m_t\rangle = m_t |t,m_t\rangle
\end{equation}
Asociando $m_t=+1/2$ al protón y $m_t=-1/2$ al neutrón, sus posibles valores en el estado fundamental para un núcleo con $Z$ protones y $N$ neutrones:

\begin{equation}
	T_3 = \frac{Z-N}{2} \tquad \left| \frac{Z-N}{2} \right| \leq T \leq \frac{Z+N}{2}
\end{equation}
En reacciones mediadas por la fuerza fuerte\footnote{Un ejemplo interesante podrían ser las desintegraciones $\alpha$.} (no por la débil, como las desintegraciones beta), se conserva el isospín:

\begin{equation}
	\Tn_A + \Tn_a = \Tn_B + \Tn_b
\end{equation}

\section{Dispersión y secciones eficaces}

Lo que usualmente se mide en las reacciones nucleares es el momento de las partículas ligeras emitidas (y por lo tanto su energía cinética, supuesto que se conozca la identidad de la partícula) y su distribución angular. Esto permite observar la partícula emergente en un cierto elemento diferencial de ángulo sólido $\D \sigma / \D \Omega$ a veces también representada por $\sigma (\theta, \phi)$. Si estamos interesandos en encontrar a la partícula emergente no sólo a un cierto ángulo, sino también con una cierta energía, hablamos de la sección eficaz diferencial doble, $\D^2 \sigma / \D \Omega \D E$, pero usualmente no se pone de modo explícita esta dependencia con la energía porque se supone implícita a menos que se diga lo contrario. Integrando la sección eficaz diferencial sobre el ángulo sólido obtendríamos la sección eficaz diferencial total a una energía dada $\D \sigma / \D E$. Por último, si también integramos esta energía, obtendríamos la sección eficaz total absoluta $\sigman_t$, que representa la probabilidad de formar al núcleo $B$ en el estado final (en reacciones del tipo $a+A\longrightarrow B + b$). 

Para ciertos estudios más detallados se suelen medir también secciones eficaces dependeindo de la orientación de los espines de las partículas emergentes. Esto, junto con la observación del espectro $\gamma$ de desexcitación de los núcleos formados, permite acumular información para asignar espines y paridades a los estados de los núcleos bajo estudio.

\subsection{Atenuación de un haz al atravesar un blanco}

Una disposición experimental frecuente en física nuclear consiste en hacer incidir un haz de partículas sobre un blanco fijo para producir cierto tipo de reacción. Dada la probabilidad de itneracción por unidad de longitud, podemos calcular en qué medida se atenuará el haz incidente al atravesar el blanco delgado \footnote{Blanco delgado queire decir que a cada núcleo del blanco está llegando aproximadamente el mismo flujo de partículas del haz. Si el blanco es demasiado grueso podría ocurrir que las partículas del haz fuesen completamente absorbidas en el interior del blanco, de modo que sobre los núcleos de la parte posterior de este último ya no incidiría ninguna partícula del haz.} Denotemos por $1/\lambda$ la probabildiad de itneracción de las partículas del haz por unidad de longitud en el material del blanco. Queremos calcular la probabilidad de que una partícula del haz atraviese un intervalo de longitud $X$ en el blanco sin sufrir ninguna interacción. Para ello dividimos el intervalo $X$ en una serie de intervalos infinitesimales cuyo tamaño podemos expresar como $X/n$ siendo $n$ un número suficientemente grandes. La probabildiad infinitesimal $\D P$ de que la partícula incidente logre atravesar el intervalo de longitud $X/n$ será $\D P = 1 - \frac{1}{\lambda} \frac{X}{n}$, haciendo tender $n$ a infinito al mismo tiempo que aplicamos $n$ veces la fórmula anterior obtenemos:  

\begin{equation}
    \D P = 1 - \frac{1}{\lambda} \frac{X}{n} \tquad P = \lim_{n\rightarrow \infty} \parentesis{1 - \frac{1}{\lambda}\frac{X}{n}}^n = e^{-X/\lambda}
\end{equation}
Por lo tanto, si $\phi_0$ es el flujo del haz incidente, representando el número de de partículas que en la unidad de tiempo atraviesan una unidad de área perpendicular a la dirección del haz, y $\phi$ es el flujo de salida (detrás) del blanco delgado de expresor $X$, deducimos que están relacionados por medio de la siguiente ecuación 

\begin{equation}
    \phi = \phi_0 e^{-X/\lambda}
\end{equation}
$\lambda$ representa el recorrido libre medio (\textit{mean free path}) de las partículas del haz en el material del blanco\footnote{Podemos comparar esto con la ley de exponencial de la desintegración radiactiva $N=N_0 e^{-t/\tau}$, donde $\tau$ es le vida media, que en el contexto actual equivaldría al recorrido libre medio, y $t$ es el tiempo trascurrido (distancia temporal), que en el contexto actual equivaldría a distancia física recorrida. El inverso de $\tau$ es la probabilidad de desintegración por unidad de tiempo, y ahora hablaríamos de la probabilidad de interacción por unidad de longitud espacial, no temporal.}. Se puede expresar en términos de la sección eficaz de esta manera \footnote{Supongamos que una partícula atraviesa una región cúbica de lado $L$ en el blanco. Calculemos la probabilidad de que sufra alguna interacción en ese viaje. Supondremos que la trayectoria de la partícula es paralela a una de las caras en las posiciones de los núcleos, donde $\sigma$ es obviamente la sección eficaz total de interacción con un núcleo. El área total cubierta por las secciones eficaces det todos los núcleos del cubo es $A_\sigma = \Ncal L^3 \sigma$, mientras que el área total que el cubo presenta en la dirección perpendicular a la del avance de la partícula es $A_t=L^2$. el cociente entre ambas nos da la probabilidad de que la partícula interaccione al travesar el cubo $P_L  = A_\sigma / A_t = \Ncal L^3 \sigma / L^2$. La probabilidad de interacción por unidad de longitud es entonces $P=P_L/L=\Ncal \sigma$}: \\

\begin{equation}
    \frac{1}{\lambda} = \Ncal \sigma = \frac{\rho N_A }{M_A} \sigma 
\end{equation}
donde $\Ncal$ es la densidad de átomos blanco (número de átomos por unidad de volumen), que puede expresarse como $\rho N_A /M_A$, siendo $N_A$ el número de Avogadro, $\rho$ la densidad del blanco y $M_A$ su masa atómica (en unidades compatibles con las de $\rho$, por supuesto). 

En lugar de utilizar el espesor del blanco, $X$, es habitual usar el \textit{espesor másico}, que se define como $X_m=X\rho$, es decir, se trata de la densidad por unidad de superficie. De este modo tendremos 

\begin{equation}
    \frac{X}{\lambda} = \frac{X_\rho}{\lambda_\rho} = \frac{X_m}{\lambda_\rho} = \frac{X_m}{\rho} \Ncal \sigma = \frac{X_m}{\rho} \frac{\rho N_A}{M_A} \sigma = \frac{X_m \Ncal \sigma}{M_A}
\end{equation}
También podríamos escribir $X/\lambda=X_m/\lambda_m$, siendo $\lambda_m = M_A /(N_A\sigma)=\lambda \rho$. Una de las aplicaciones de las reacciones nucleares inducidas por el hombre consiste en la creación de radioisótopos para aplicaciones médicas o industriales.


% Falta texto importante

\subsection{Dispersión de Coulomb}

Cuando las partículas que intervienen en la reacción tienen carga eléctrica, se producirá siempre una dispersión del proyectil en el campo coulombiano del blanco: a esto se le denomina dispersión de Coulomb. En reacciones de partículas cargadas a baja energía el proceso dominante que tendrá lugar será la \textbf{dispersión elástica coulombiana} o dispersión de Rutherford, cuya sección eficaz hemos visto ya en otro capítulo:

\begin{equation}
    \derivadas{\sigma(\theta)}{\Omega} = \frac{4 Z^2 \alpha^2 (\hbar c)^2 E^2}{|\qn c|^4} = \parentesis{\frac{Ze^2}{4 \pi \epsilon_0}}^2 \frac{1}{(4E)^2 \sin^2(\theta/2)}
\end{equation}
Obsérvese que esta sección eficaz decrece rápidamente con la energía del proyectil ($E$) y con ángulo de dispersión ($\theta$). Cuando la energía sea suficientemente alta, podremos tener además una \textbf{dispersión inelástica coulombiana}, que también se llama excitación coulombiana. En este caso el núcleo final se encuentra en un estado que luego decae rápida,ente mediante la emisión de fotones $\gamma$. 

\subsection{Dispersión nuclear}

En el caso de que el proyectil no tenga carga eléctrica (como en dispersión de neutrones), no será posible, obviamente, que exista dispersión coulombiana. Se produce entonces la denominada dispersión nuclear, que tiene lugar en el campo nuclear no coulombiano\footnote{La fuerza nuclear fuerte es la interacción fundamental entre los quarks, y la que de alguna forma mantiene unidos a los nucleones en el núcleo y crea ese campo nuclear no coulombiano con el que interaccionan los proyectiles masivos sin carga eléctrica.} creado por el núcleo. Para partículas cargadas, aparte de la dispersión coulombiana podrá existir dispersión nuclear siempre que la energía del proyectil sea lo suficientemente alta como para vencer la barrera coulombiana del núcleo. 

% falta texto

En una reacción de tipo $A+a\longrightarrow B+b$, la medida directa de la cantidad de partículas $b$ emergentes a través de un pequeño intervalo de ángulo sólido dado se hace, obviamente, en el sistema de referencia del laboratorio. Sin  embargo, para comparar los resultados experimentales con las predicciones teóricas es necesario que los traslados al sistema de referenia del centro de masas de las partículas \textit{entrantes} $A,a$. Resulta de especial utilidad, por ejemplo, graficar las llamdas \textbf{funciones de excitación} que muestran la dependencia de la sección eficaz $\D \sigma / \D \Omega$, respecto a la energía cinética del canal entrante en el sistema centro de masas, (que en la aproximación no relativista es $m_A T_a / (m_A + m_a)$), para la observación de una partícula concreta $b$ dispersada en cierto $\D \Omega$. 

La \textbf{distribución angular} de la emisión de la partícula seleccionada $b$ en el sistema centro de masas con respeto a la dirección de incidencia contiene información sobre el cambio de momento angular y paridad de los estados inicial y final de los núcleos, puesto que el momento angular y la paridad se conservan en reacciones nucleares gobernadas por la interacción fuerte y la electromagnética.

\subsection{Formalismo de dispersión}

Para simplificar el tratamiento vamos a considerar la dispersión de dos partículas sin espín, y despreciamos su interacción coulombiana (si son partículas cargadas), es decir, consideramos exclusivamente el potencial nuclear. Escribamos la función de odnas del sistema de dos partículas como $\Psi (\rn_1,\rn_2,t)$. Como bien sabemos, el problema lo podemos tratar en el centro de masas como, de tal modo que hacemos el cambio:

\begin{equation}
	\rn = \rn_1 - \rn_2 \quad \Rn = \frac{m_1 \rn_1 + m_2 \rn_2}{M} \tquad M = m_1 + m_2 \quad \mu = \frac{m_1 m_2}{m_1 + m_2}
\end{equation}
Entonces es encillo de ver que la función de ondas $\Psi(\rn,\Rn,t)$ viene dada por la ecuación de Schrödinger de tal modo que:
\begin{equation}
	i \hbar \parciales{\Psi (\rn,\Rn,t)}{t} = \parentesis{-\frac{\hbar2}{2M} \nabla_R^2-\frac{\hbar^2}{2\mu} \nabla_r^2 + V(r)} \Psi(\rn,\Rn,t) \tquad 
\end{equation}
Estamos suponiendo además que el potencial es estático y depende únicamente de la distancia relativa entre las dos partículas. Los operadores $\nabla_R^2$ y $\nabla_r^2$ implican la diferenciación respecto al centro de masas y las coordenadas relativas, respectivamente. Al igual que sucede en la resolución de la ecuación se Schrödinger para el átomo de hidrógeno, podemos aplicar ahora la técnica de separación de variables dos veces consecutivas: una para separa la dependencia temporal de la espacial, y otra para separar la dependencia en las coordenadas relativas de la dependencia de las coordenadas en el centro de masas. Se obtiene lo siguiente: 

\begin{equation}
	\Psi (\Rn,\rn,t) = u(\rn)v(\Rn) \exp \ccorchetes{- \frac{i(E_r+E_R)t}{\hbar}}
\end{equation}
\begin{equation}
	-\frac{\hbar ^2}{2\mu} \nabla_r^2 u(\rn) + V(r) u(\rn) = E_r u(\rn) \label{Ec:03-03-09}
\end{equation}
\begin{equation}
	-\frac{\hbar ^2}{2\mu} \nabla_r^2 v(\Rn)  = E_R v(\Rn)
\end{equation}
En estas ecuaciones $E_r$ es la energía de las dos partículas asociada a su movimiento relativo, es decir, su energía en el sistema centro de masas. $E_R$ es la energía correspondiente al movimiento del centro de masas, que está descrito por una ecuación idéntica al de una partícula libre, y por tanto no es de interés. \\

Obsérvese que la ecuación (\ref{Ec:03-03-09}), que describe el movimiento relativo de las dos partículas, es idéntica a la ecuación que describiría la dispersión de una partícula de masa $\mu$ y energía $E_r$ por un potencial de dispersión fijo $V(r)$. Es precisamente $E_r$ la que está \textit{disponible} para la formación de estados excitados o de nuevas partículas en la reacción, tal como hemos visto en las secciones anteriores.

La dispersión queda determinada por el estudio del comportamiento asintótico de $u(r,\theta,\phi)$ en la región en que $V(r)=0$, es decir, cuando las partículas se encuentran infinitamente separadas\footnote{La separación real no necesita ser macroscópicamente grande, porque la interacción nuclear es de muy corto alcance.} En esta situación la ecuación (\ref{Ec:03-03-09}) tiene el siguiente comportamiento asintótico

\begin{equation}
	\nabla_r^2 u(\rn) + k^2 u(\rn) = 0 \tquad k^2 = \frac{2\mu E_r}{\hbar^2}
\end{equation} 
que se trata de la ecuación para una partícula libre de masa $\mu$ y  energía $E_r$, que \textit{es precisamente lo que esperamos para el estado inicial}, es decir, una partícula incidiendo con cierta masa y momento sobre el centro dispersor. Obviamente, en un experimento real tendremos un haz de partículas incidente con un momento $\pn=\hbar \kn$ definido en un intervalo de cierta anchura. Supondremos que ese haz se propaga en el eje $z$, y que por tanto el estado inicial de las partículas incidentes se hace mediante una onda plana en el sentido positivo del eje $z$ con un vector de propagación de módulo $k^2 = 2 \mu E_r / \hbar ^2$. 

Después de la colisión tendremos, además de la onda plana, una onda esférica emergente del centro dispersor, de tal manera que podemos escribir la función de ondas en el estado final como una superposición de las dos:

\begin{equation} 
	u_f = A \ccorchetes{e^{ikz}+\frac{1}{r} f(\theta,\phi)e^{ikr}} \label{Ec:03-03-12}
\end{equation}
El segundo término es una onda esférica emergente del centro dispersor, y representa la partícula dispersada. Su amplitud depende de los ángulos $\theta$, $\phi$, y es inversamente proporcional al radio $r$, puesto que el flujo debe decrecer con el inverso del cuadrado de la distancia. Se puede comprobar fácilmente que la función de ondas (\ref{Ec:03-03-12}) verifica asintóticamente ($V(r)=0$) la ecuación (\ref{Ec:03-03-09}) para cualquier valor de $f(\theta,\phi)$. 

Para ver el significado físico de la función angular $f(\theta,\phi)$ y el coeficiente $A$ debemos calcular el flujo incidente y dispersado. El flujo se calcula a partir de la corriente de probabilidad dada por \footnote{Recuérdese que de la ec. de Schrödinger para una partícula libre puede derivarse la ecuación de continuidad $\parciales{\rho}{t}+\nabla \jn=0$, donde $\jn$ viene dada por la ec. (\ref{Ec:03-03-13}). Para una onda plana esta corriente sería $\jn = |A|^2 \hbar \kn /m$. En general se sobreentiende que estamos hablando de la dirección $\hnz$, de tal modo que escribimos directamente $j=|A|^2\hbar k/m=|A|^v$ donde $v=\hbar k /m$. .}

\begin{equation}
	\jn = \frac{\hbar}{2mi} \parentesis{\Psi^* \nabla \Psi - \Psi \nabla \Psi ^*} \label{Ec:03-03-13}
\end{equation}
Dado que el flujo $\phi$ a través de cierta superficie se define como 

\begin{equation}
	\phi = \int_S \jn \D \sn \tquad \D \sn = \hnn \D s
\end{equation}
donde $\hnn$ es el vector unitario normal al elemento diferencial de área $\D s$, sustituyendo cada uno de los términos de la ecuación (\ref{Ec:03-03-12}) en esta expresión obtenemos que el flujo incidente y el saliente:

\begin{equation}
	\phi_i = v|A^2| \tquad \phi_f = \frac{v|A|^2}{r^2} |f(\theta,\phi)|^2
\end{equation}
donde $v=\hbar k /\mu$ en este caso.  En estas ecuaciones la elección de $A$ no es relevante, ya que la sección eficaz se \textit{expresa en términos de cociente entre flujo de salida y el flujo de entrada}. Podemos elegirlo de tal manera que el flujo incidente sea la unidad, esto es, $|A|=1/\sqrt{v}$.
 
Para proceder ahora hacia el cálculo de la sección eficaz diferencial de dispersión hemos de considerar el flujo emergente a través del área subtendida por el diferencial de ángulo sólido $\D \Omega$ en el centro dispersor, esto es, $r^2 \D \Omega$ ($\D \Omega = \sin \theta \D \theta \D \phi$). La sección eficaz diferencial es entonces:

\begin{equation}
	\derivadas{\sigma (\theta,\phi)}{\Omega} \D \Omega = \frac{\phi_f r^2}{\phi_i} r^2 \D \Omega = |f(\theta,\phi)|^2 \D \Omega
\end{equation}
A la función $f(\theta,\phi)$ se le conoce como \textbf{amplitud de dispersión} (\textit{scattering amplitude}), porque la función de ondas dispersada viene dada por 

\begin{equation}
	 u_{sc} = u_f - u_i = \frac{1}{r} e^{ikr} f(\theta,\phi)
\end{equation}

\subsubsection{Ondas parciales}

Es imposible seleccionar un único valor del momento angular $\ell\hbar$ de la partícula incidente respecto al centro de dispersor. En la realidad tendremos por partículas incidiendo con momentos angulares distintos, y en esta situación conviene reemplazar la onda plana incidente por la superposición de una serie equivalente de ondas esféricas, cada una con momento angular $\ell \hbar$ definido. Dado que $V(r)$ es esféricamente simétrico, y que las partículas no tienen espín, los armónicos esféricos no tendrán dependencia con $\phi$. La expansión de una onda plana en función de ondas esféricas esta´relacionada con las funciones de Bessel radiales $j_{\ell}(kr)$:
\begin{equation}
	e^{ikz} = \sum_{\ell=0}^{\infty} \sqrt{4\pi (2\ell +1)} i^{\ell} j_\ell (kr) Y_{\ell}^0 (\theta)
\end{equation}
A esta ecuación se le denomina como \textbf{expansión en ondas parciales}, porque describe la función de ondas como una superposición infinita donde a cada término u onda parcial le corresponde un valor específico del momento angular $\ell \hbar$. Dado que los armónicos esféricos $Y_{\ell}^0$ vienen dados por 

\begin{equation}
	Y_l^0 = \sqrt{\frac{2\ell+1}{4\pi}} P_{\ell} (\cos\theta)
\end{equation}
donde $P_\ell$ son los polinomios de Legendre. Escribimos la onda plana como
\begin{equation}
	e^{ikz} = \sum_{\ell=0}^{\infty} i^{\ell} (2\ell +1) j_\ell (kr) P_\ell (\cos \theta)
\end{equation}
El comportamiento asintótico de la expresión tiene que ver con que las funciones de Bessel se comportan como senos cuando su argumento va a infinito. Más concretamente:

\begin{equation}
	\lim_{kr\rightarrow \infty} J_{\ell} (kr) = \frac{1}{kr} \sin \parentesis{kr - \frac{\ell \pi}{2}}
\end{equation}
Recordando que el seno podemos expresarlo como la combinación de exponenciales complejas tenemos que:

\begin{equation}
	u_i = \frac{1}{kr} \sum_{\ell=0}^{\infty} i^\ell (2\ell +1) P_\ell (\cos \theta) \frac{1}{2i} \parentesis{e^{i[kr-(l\pi/2)]}-e^{-i[kr-(l\pi/2)]}}
\end{equation}
En esta última fórmula podemos ver una serie de ondas esféricas que emerge $e^{i[kr-(l\pi/2)]}$ y otra que converge $e^{-i[kr-(l\pi/2)]}$, cada una con un momento angular definido $\ell \hbar$. La onda plana incidente se puede considerar entonces como una superposición coherente de estas dos series de ondas esféricas. 

Analicemos ahora lo que sucede después de la colisión. El principio de causalidad nos lleva a afirmar que la interacción con el centro dispersor sólo puede modificar la serie de ondas esféricas emergentes. Esta modificación con el centro dispersor elástica con cierta distribución angular, y en el segundo y tercero tenemos una dispersión inelástica. Un cambio, en la amplitud indica que tendremos un cambio en la magnitud del flujo, es decir, tendremos posiblemente menos partículas saliendo que entrando. Hay que tener en cuenta que, sin embargo, que tanto la función de ondas incidente como la dispersada representan sólo a partículas con momento $\hbar k$, de tal manera que si salen menos partículas con momento $\hbar k$ de las que entran, entonces debe suceder que aparecen partículas en el estado final con valores del módulo distintos a los de entradas. Esto es, en efecto, la dispersión inelástica, en la cual tiene lugar un cambio en el módulo del momento de la partícula saliente (que incluso puede tener naturaleza distinta a la entrante), mientras que len la dispersión elástica sólo cambia la dirección del momento\footnote{Estamos considerando aquí la colisión de una partícula muy ligera con otra muy pesada (un núcleo), de tal manera que la energía de retroceso del blanco sea despreciable. En caso contrario habrá un cambio en el módulo del momento de la partícula incidente. Por otro lado, si consideramos la dispersión de dos partículas en el sistema centro de masas, en una colisión elástica el módulo del momento relativo entre ellas ($\hbar k$) se mantiene constante, y sólo cambia si la colisión es inelástica.}.

Escribamos entonces la función de onda final estacionaria a grandes distancias distancias del centro de la siguiente manera:

\begin{equation}
	u_f = \frac{1}{kr} \sum_{\ell=0}^{\infty} i^\ell (2\ell +1 ) P_\ell (\cos \theta) \frac{1}{2i} \parentesis{a_\ell e^{i[kr-(\ell \pi /2)]} - e^{i[kr-(\ell\pi/2}]}  \label{Ec:03-03-23}
\end{equation}
donde los $a_\ell$ son constantes complejas que dan cuenta del efecto del potencial dispersor sobre la $\ell$-ésima. La \textit{parte real} de $a_\ell$ proporciona el cambio de amplitud de la onda dispersada, mientras que la parte imaginaria de $a_\ell$ proporciona el cambio de fase. Así podemos obtener que:

\begin{equation}
	u_f - u_i  = \frac{e^{ikr}}{r} \frac{1}{k} \sum_{\ell=0}^{\infty} (2\ell +1) \parentesis{\frac{a_\ell-1}{2i}} P_\ell (\cos \theta)
\end{equation}
Por lo tanto 

\begin{equation}
	f(\theta ) = \frac{1}{k} \sum_{\ell=0}^{\infty} (2\ell +1) \parentesis{\frac{a_\ell-1}{2i}} P_\ell (\cos (\theta))
\end{equation}
Como era de esperar, esto nos lleva a que la sección eficaz diferencial sea

\begin{equation}
	\derivadas{\sigma (\theta)}{\Omega} = \frac{1}{k^2}\left| \sum_{\ell=0}^{\infty} (2\ell +1) \parentesis{\frac{a_\ell-1}{2i}} P_\ell (\cos (\theta)) \right|
\end{equation}
En la función de ondas dispersada, $u_{sc}$, estamos considerando sólo las ondas parciales con un valor de $k$ igual a la de la onda incidente, por lo tanto esta sección eficaz diferencial es la que corresponde a la dispersión elástica. La integral sobre el ángulo sólido nos proporcionará la \textbf{sección eficaz elástica total}, tal que:

\begin{equation}
	\sigma_{el} = \frac{\pi}{k^2} \sum_{\ell=0}^{\infty} (2\ell+1)|1-a_\ell|^2 \label{Ec:03-03-27}
\end{equation}
La constante compleja $a_\ell$ suele expresarse en términos de una amplitud real $\eta_\ell$ denominada \textbf{\textit{parámetro de inelasticidad}} (porque su desviación respecto de la unidad nos indica la importancia relativa de la sección eficaz inelástica respecto a la elástica) y una fase $\delta_\ell$ denominada \textit{\textbf{desfaje}} (\textit{phase shift}). Es decir, 

\begin{equation*}
	a_\ell = \eta_\ell e^{2i\delta_\ell} \tquad 0 \leq \eta_\ell \leq 1
\end{equation*}
La denominación de desfaje, es, en efecto, apropiada, porque si sustituimos esta parametrización de $a_\ell$ en la expresión de $u_f$ dada por (\ref{Ec:03-03-23}) vemos que efectivamente 

\begin{equation}
	u_f = \frac{1}{kr} \sum_{\ell=0}^{\infty} i^\ell (2\ell +1 ) P_\ell (\cos \theta) \frac{1}{2i} \parentesis{\eta_\ell e^{i[kr-(\ell \pi /2)]+2\delta_\ell} - e^{i[kr-(\ell\pi/2}]} 
\end{equation}
lo cual muestra que hay un cambio de fase de $2\delta_\ell$ en la $\ell$-ésima onda parcial emergente. En el caso de dispersión elástica tenemos que $\eta_\ell = 1$ (no hay cambio en el módulo del momento) y por lo tanto:

\begin{equation}
	f(\theta) = \frac{1}{kr} \sum_{\ell=0}^{\infty} i^\ell (2\ell +1 ) P_\ell (\cos \theta) \parentesis{\frac{e^{2i\delta_\ell}-1}{2i}}
\end{equation}
Por lo que haciendo una sustitución en (\ref{Ec:03-03-27}) vemos que la \textbf{sección eficaz elástica} es:
\begin{mybox}
	\begin{equation}
		 \sigma_{\el} = \frac{4\pi}{k^2} \sum_{\ell=0}^{\infty} (2\ell +1)  \sin^2 (\delta_\ell) \label{Ec:03-03-30}
	\end{equation}
\end{mybox}

Si la dispersión es elástica esta fórmula proporciona también, claro está, la sección eficaz total. Vemos que la sección eficaz elástica asociada a un valor concreto de $\ell$ adquiere un valor máximo para $\delta_\ell = \pi/2$, se dice entonces que se tiene \textbf{resonancia}, y ese valor máximo es:

\begin{equation}
	(\sigma_{\el}^{\max})_{\ell \text{ fijo}} = \frac{4\pi}{k^2} (2\ell +1)
\end{equation}

En el caso de que exista dispersión inelástica, es decir que $\eta_{nl}<1$ (cambia el módulo del momento) entonces la ecuación (\ref{Ec:03-03-30}) ya no proporciona el valor de la sección eficaz total, que debemos calcular como la suma de la parte elástica más la inelástica. Si la amplitud de la onda parcial $\ell$-ésiima que emerge del centro dispersor se ve reducida en un factor $|a_\ell|^2$, entonces la sección eficaz inelástica será proporcional al factor ($1-|a_\ell|^2$) para esa onda $\ell$-ésima. Deducimos entonces que la \textbf{sección eficaz inelástica} es

\begin{mybox}
	\begin{equation}
		\sigma_{\inel} = \frac{2\pi}{k^2} \sum_{\ell=0}^{\infty} (2\ell +1) (1 - |a_\ell|^2)
		= \frac{2\pi}{k^2} \sum_{\ell=0}^{\infty} (2\ell +1) (1 - \eta^2_\ell)
		\label{Ec:03-03-32}
	\end{equation}
\end{mybox}

La sección eficaz inelástica máxima para una onda parcial concreta se alcanza cuando $\eta_\ell=0$. Su valor es

\begin{equation}
	(\sigma_{\inel}^{\max})_{\ell \text{ fijo}} = \frac{\pi}{k^2} (2\ell +1)
\end{equation}
La \textbf{sección eficaz total} será
\begin{mybox}
	\begin{equation}
		\sigma_{\tot} = \sigma_{\el} + \sigma_{\inel} = \frac{2\pi}{k^2} \sum_{\ell=0}^{\infty} (2\ell +1) (1 - \Re [a_\ell])\label{Ec:03-03-34}
	\end{equation}
\end{mybox}
Hemos visto entonces que si se cumple $|a_\ell |=1$, entonces la dispersión inelástica es nula. Podemos tener dispersión inelástica únicamente. Lo contrario, sin embargo, no es cierto. Es imposible qeu en una reacción se anula la dispersión elástica y tener solo dispersión inelástica. La dispersión elástica existe siempre\footnote{Esto se puede entender con la analogía de la difracción óptica: si existe dispersión inelástica de tal modo que estamos \textit{sacando} partículas del haz, entonces tenemos una especie de \textit{zona oscura} detrás del núcleo blanco, y por tanto tiene que existir algún efecto de difracción de las partículas incidentes sobre esa zona oscura u obstáculo. Dicho de otra forma: es imposible que no exista difracción cuando las partículas inciden sobre un obstáculo. Ese obstáculo es, en efecto, el centro dispersor del núcleo blanco.}, porque si tomamos un $a_\ell$ de tal modo que $\sigma_{\text{\inel}}$ no sea nula para onda esa onda parcial, entonces para $\sigma_\el$ es también no nula para esa misma onda parcial. En otras palabras, \textit{si existe abosrción de alguna odna parcial, entonces existe también una dispersión elástica asociada a la misma onda parcial}. A este fenómeno se le denomina \textbf{dispersión difractiva} (\textit{diffraction scattering}). 

La dispersión difractiva es un fenómeno bien conocido en óptica, donde se ve que la sombra que proyecta un obstáculo circular cuando es iluminado por un láser no es completamente oscura, presenta un patrón de difracción con una serie de máximos y mínimos. En el centro de la región de sombra aparece un máximo llamado punto de Poisson, justo detrás del disco, a una cierta distancia mínima del mismo.

\subsection{Teorema Óptico}
La sección eficaz total está relacionada con la parte imaginaria de la amplitud de dispersión adelante (\textit{forward scattering amplitude}), es decir, con la parte imaginaria de $f(0)$. Recordemos la expresión de la amplitud de dispersión:

\begin{equation}
	f(\theta) = \frac{1}{2ik} \sum_{\ell = 0}^{\infty} (2\ell +1 ) (a_{\ell}-1)P_{\ell} (\cos \theta)
\end{equation}
Para la dispersión hacia adelante tenemos que $\theta=0$ y $P_\ell (1)=1$, con lo que

\begin{equation}
	\Im [f(0)] = \frac{1}{2k} \sum_{\ell=0}^{\infty} (2\ell + 1) \Im \ccorchetes{i(1-a_\ell)} = \frac{1}{2k} \sum_{\ell=0}^{\infty} (2\ell+1) (1-\Re \ccorchetes{a_\ell})
\end{equation}
si comparamos esto con la expresión anterior vemos que

\begin{mybox}
	\begin{equation}
		\sigma_{\tot} = \frac{4\pi}{k} \Im [f(0)]
	\end{equation}
\end{mybox}
lo que constituye el denominado \textbf{teorema óptico}.

\subsection{Fórmula de Breit-Wigner}

Vamos a estudiar ahora el comportamiento de la sección eficaz en las proximidades de una resonancia. Hemos visto que la amplitud de dispersión viene dada por

\begin{equation}
	f(\theta) = \frac{1}{k} \sum_{\ell=0}^{\infty} (2\ell+1) \eta_\ell \parentesis{\frac{e^{2i\delta_\ell}-1}{2i}} P_\ell (\cos \theta)
\end{equation}
Resulta conveniente definir la amplitud de la onda parcial $\ell$-ésima en función de la energía total del sistema centro de masas 

\begin{equation}
	T_\ell (E) = \eta_\ell \parentesis{\frac{e^{2i\delta_\ell}-1}{2i}} \eta_\ell e^{i\delta_\ell} \sin \delta_\ell
\end{equation}
Vamos a estudiar el comportamiento de $T_\ell (E)$ frente a $E$. En el caso de que la \textbf{dispersión} sea \textbf{elástica} tendremos que $\eta_\ell$, de tal modo que 

\begin{equation}
 	T_\ell (E) = e^{i\delta_\ell}\sin(\delta_\ell) = \frac{\sin \delta_\ell}{ e^{-i\delta_\ell}}=\frac{1}{\cot \delta_\ell -i}
\end{equation}
La condición de resonancia la tendremos cuando $\delta_\ell \rightarrow \pi/2$, de tal manera que cerca de la resonancia tenemos $\cot (\delta_\ell)=0$. En consecuencia, cuando la energía total del sistema centro de asas se acerque a la energía de la resonancia, que denotaremos por $E_R$, tiene sentido hacer un desarrollo en serie de Taylor de la $\cot \delta_\ell$ en las proximidades de $E_R$, o lo que es lo mismo, de $\delta_\ell = \pi/2$:

\begin{equation}
	\begin{split}
	\cot \delta_\ell (E) \ = \ & \ \cot \delta_\ell (E_R) + (E-E_R) \parentesis{\derivadas{\cot \delta_l}{E}}_{E=E_R} + \\ & \  + \frac{1}{2} (E-E_R)^2  \parentesis{\derivadas{^2\cot \delta_l}{E^2}}_{E=E_R} +  \ldots
	\end{split} \label{Ec:03-03-41}
\end{equation}
Cuando $E=E_R$ (equivalente a $\delta_\ell = \pi/2$) la derivada de la cotangente coincide directamente con la derivada de la fase cambiada de signo:

\begin{equation}
	\parentesis{\derivadas{\cot \delta_\ell}{E}}_{E=E_R} = \parentesis{- \frac{1}{\sin^2 \delta_\ell} \derivadas{\delta_\ell}{E}} = -\parentesis{\derivadas{\delta_\ell}{E}} 
\end{equation}
Definimos ahora la \textbf{anchura} $\Gamma$ como

\begin{equation}
	\frac{1}{\Gamma} = \frac{1}{2}\parentesis{\derivadas{\delta_\ell}{E}} 
\end{equation}
de tal modo que se verifica que

\begin{equation}
	\parentesis{\derivadas{\cot \delta_l}{E}}_{E=E_R}  = - \frac{2}{\Gamma} \label{Ec:03-03-44}
\end{equation}
Dado que $\cot \delta_\ell(E_R)=0$, substituyendo (\ref{Ec:03-03-44}) en la expansión de Taylor (\ref{Ec:03-03-41}) 

\begin{equation}
	\cot \delta_\ell (E) \approx -\frac{(E-E_R)}{\Gamma/2}
\end{equation}
En consecuencia la amplitud de dispersión elástica para la onda parcial $\ell$-ésima sería:

\begin{equation}
	T_\ell (E) = \frac{\Gamma/2}{(E_R-E)-i(\Gamma/2)}
\end{equation}
Teniendo en cuenta que en las dispersiones elásticas $T_\ell(E)=e^{i\delta_\ell}\sin \delta_\ell$, se verificará que $\sin^2 \delta_\ell = |T_\ell(E)|^2$ que sustituyendo en la ecuación (\ref{Ec:03-03-30}) nos da

\begin{equation}
	\sigma_\el = \frac{4\pi}{k^2} (2\ell+1) |T_\ell (E) |^2 = \frac{4\pi}{k^2} (2\ell +1) \frac{(\Gamma/2)^2}{(E_R-E)^2 + (\Gamma/2)^2}
\end{equation}
Dado que la dispersión de partículas de espín cero el valor máximo para la dispersión elástica es $\sigma_\el^{\max} = 4 \pi (2\ell+1)/k^2$, y ocurre $E=E_R$. El cociente entre este valor máximo y su valor a cualquier otra energía será

\begin{equation}
	\frac{\sigma_\el(E)}{\sigma_\el^{\max}} = \frac{(\Gamma/2)^2}{(E_R-E)^2 + (\Gamma/2)^2}
\end{equation}
A partir de esta fórmula se ve que para las energías $E=E_R\pm (\Gamma/2)$ el cociente anterior es $1/2$, por lo tanto la definición que hemos dado de $\Gamma$ tiene perfecto sentido como la anchura a mitad de altura del máximo en la curva de resonancia. Esta curva de resonancia se conoce como \textit{curva de Breit-Wigner} puede ser el oscilador armónico forzado, circuitos de corriente alterna, absorción resonante de radiación... Hemos estudiado la dispersión de una partícula sin espín contra un blanco sin espín. Para generalizar el resultado para que las partículas interaccionantes tengan espín diferente de cero, simplemente dividimos por el número de subestados iniciales con distinta tercera componente de espín, esto es, substituir $2\ell+1$ por el factor estadístico más general:

\begin{equation}
	g(I) = \frac{2I+1}{(2s_a+1)(2s_A+1)} \tquad \text{donde} \tquad \In = \sn_a + \sn_A + \lnn
\end{equation}
de modo que
\begin{equation}
	\sigma_\el = \frac{\pi}{k^2} g(I) \frac{\Gamma_\el \Gamma_\el}{(E_R-E)^2 + (\Gamma/2)^2} \label{Ec:03-03-50}
\end{equation} 
expresión conocida como \textbf{fórmula de Breit-Wigner} que describe el comportamiento de la sección eficaz en las proximidades de una \textbf{resonancia aislada}\footnote{En el caso de que no se trate de una resonancia aislada (existen varias resonancias energéticamente próximas) será necesario tener en cuenta los correspondientes estados de interferencia. Esta fórmula también es válida en el régimen relativista.}. Usando el principio de incertidumbre $\Delta t \sim \hbar /2\Gamma$ podremos estimar la vida media de las resonancias.


A cada modo de produccion o desintegración de la resonancia se le denomina \textit{canal}. A cada modo de producción o desintegración de la resonancia se le suele llamar \textit{canal}. Al modo específico de formación que estamos consdierando se le llama \textbf{canal de entrada} (\textit{entrance channel}), y al de desintegración \textbf{canal de salida} (\textit{exit channel}). El \textbf{canal elástico} corresponde a la situación $a+A\rightarrow a+A$, mientra que el resto de los canales a través de los que se puede desintegrar la resonancia son, obviamente, los canales inelásticos, que podemos agrupar genéricamente con el término de \textbf{canal inelástico}. Es obvio que la categoría (elástica, inelástica) en la que englobamos a un canal concreto de salida depende de cuál sea el canal de entrada que estamos considerando.  Dado que normalmente tenemos varios canales de salida:

\begin{equation}
	\sigma_{lj} = \frac{\pi}{k^2} g(I) \frac{\Gamma^{in} \Gamma^{out}}{(E_R-E)^2 + (\Gamma/2)^2} \tquad \Gamma_\tot = \sum_{j} \Gamma_j^{out}
\end{equation}


\section{Mecanismos de reacciones}

\subsection{Núcleo compuesto}

En 1932, tras el descubrimiento del neutrón, se empezaron a usar en reacciones nucleares. Las observaciones experimentales muestran que los neutrones términos tienen una alta probabilidad de ser absorbidos por una gran variedad de núcleos, y además se observan resonancias relativamente estrechas (vida larga media), que luego pueden desintegrarse mediante emisión $\gamma$. Niels Bohr propuso en 1936 el mecanismo del \textbf{núcleo compuesto}, basado en el modelo de la gota líquida, y ue consideraba que la reacción tiene lugar a través de un estado nuclear intermedio, o núcleo compuesto, que es una mezcla de la partícula incidente y el núcleo blanco. De acuerdo con esto, una reacción que procediese según el mecanismo de núcleo compuesto tendría lugar en dos etapas:

\begin{equation}
	\ce{a + A -> C^* -> B + b}
\end{equation}
donde C$^*$ representa el estado intermedio de núcleo compuesto. En este escenario la partícula proyectil, a, incidiría sobre el núcleo blanco con un pequeño parámetro de impacto e iría perdiendo energía a través de sucesivas colisiones con los nucleones del blanco hasta que al final toda su energía cinética se distribuiría estadísticamente entre todos los nucleones de A. Eventualmente las fluctuaciones estadísticas harían que un nucleón reuniese suficiente energía como para escapar del núcleo, como si fuera la evaporación de un líquido. Siguiendo un poco esta analogía, si la energía comunicada al núcleo fuese lo suficientemente alta podrán evaporarse varios nucleones, lo cual efectivamente se observa. Esto implica que la reacción nuclear tiene lugar en dos etapas independientes, primero la formación del núcleo compuesto y luego la desintegración del mismo. El núcleo compuesto C$^*$ puede formarse mediante diversas reacciones y desintegrarse de diversos modos, por lo que el modelo de núcleo compuesto predice que el modo de desintegración \textit{no} depende del canal de entrada. Esto es algo que puede contrastarse experimentalmente.


Si la reacción procede a través de un mecanismo de núcleo compuesto tiene cierto sentido examinar con más detalle el espectro energético de ese núcleo intermedio. En contra de lo que uno podría estar tentado a deducir, los resultados experimentales muestran que es posible encontrar estados discretos a una energía superiro a la energía de separación. Si la energía cinética del canal de entrada entere los niveles de energía de uno de los estados resonantes del compuesto, $E_R$, y la separación entre los niveles es mucho mayor que su anchura, entonces la sección eficaz de formación de la resonancia vendrá dada por la ecuación de Breit-Wigner (\ref{Ec:03-03-50}). 

Supongamos una energía\footnote{En colisiones relativistas, esta energía es la energía relativista total del canal de entrada en el sistema del centro de masas. En el caso de reacciones nucleares de baja energía, con una partícula proyectil ligera y un núcleo blanco pesada, se suele sustituir simplemente simplemente esta energía relativista total por la energía cinética del proyectil en el sistema laboratorio.} del canal de entrada en el centro de masas, $E$, está cerca de la energía de una de esas resonancias aisladas. Entonces la sección eficaz de formación del núcleo compuesto, según el canal $\ce{a + A -> C^*}$, vendrá dada por una ecuación de Breit Wigner.

\begin{equation}
	\sigma_a = \frac{\pi}{k^2} g(I) \frac{\Gamma_a \Gamma}{(E_R-E)^2 + (\Gamma /2)^2}
\end{equation}
Que sería, en efecto, la sección eficaz de formación de una \textbf{resonancia discreta} en el núcleo compuesto. Escribimos por tanto la sección eficaz para el proceso $\ce{a+A -> b+B}$ como,

\begin{equation}
	\sigma_{ab} = \frac{\pi}{k^2} g(I) \frac{\Gamma_a  \Gamma_b}{(E_R-E)^2 + (\Gamma/2)^2}
\end{equation}
Si consideramos la producción de la resonancia mediante dos canales de entrada distintos y su desintegración en el mismo canal, tendríamos 


\begin{equation}
\begin{split}
	\text{a + A} & \rightarrow  \text{b + B} \\
	\text{a}' + \text{A}' & \rightarrow \text{b + B}
\end{split}
\end{equation}
con lo cual, 

\begin{equation}
	\sigma_{ab} = \frac{\pi}{k^2} g(I) \frac{\Gamma_a \Gamma_b}{ (E_R-E)^2 + (\Gamma /2 )^2} \tquad \sigma_{a 'b} = \frac{\pi}{k^2} g(I) \frac{ \Gamma_{a'} \Gamma_b}{(E_R-E)^2 + (\Gamma /2 )^2}
\end{equation}
y por lo tanto su cociente es una constante independiente de la energía,

\begin{equation}
	\frac{\sigma_{ab}}{\sigma_{a'b}} = \frac{g(I)}{g'(I)} \frac{\Gamma_{a'}}{\Gamma_{a'}}
\end{equation}
Vemos entonces que el modelo del núcleo compuesto predice de modo natural que la probabilidad e desintegración del núcleo formado a través de un canal específico no depende en concreto del canal de entrada.

\subsection{Combinaciones}

En las reacciones directas la partícula proyectil interacciona con uno o unos pocos nucleones cerca de la superficie del núcleo. Se tratan de procesos muy rápidos, con tiempos característicos de $\sim 10^{-22}$ segundos. También se conocen como procesos periféricos. La probabilidad de que la reacción suceda aumenta con la energía del proyectil. Un nucleón de 1 MeV tiene una longitud de onda de De Broglie del orden de 4 fermis y no interacciona con los nucleones individuales del núcleo blanco, con lo cual la interacción con unos pocos nucleones cerca de la superficie es más que probable. Es posible que tanto el mecanismo de núcleo compuesto como el de reacción directa contribuyan a una misma reacción concreta. Podemos citar dos aspectos que nos permiten distinguir las contribuciones de estos dos mecanismos y decidir cual de ellos es más importante en una reacción dada:

\begin{itemize}
	\item Los procesos directos tienen un tiempo característico de unos $\sim 10^{22}$ segundos, mientras que un proceso de núcleo transcurre durante un tiempo de unos $\sim 10^{-15}$ segundos. Existe técnicas experimentales para distinguir dos diferentes magnitudes temporales.
	\item La distribución angular de los productos en una reacción directa está normalmente picada en ciertos lugares, mientras que la correspondiente a una reacción de núcleo compuesto es generalmente isótropa.
\end{itemize}
Una dispersión inelástica puede proceder a través de una reacción directa o de núcleo compuesto, y sus probabilidades relativas dependen, entre otras cosas, de la energía del proyectil. Además, en la mayoría de los casos necesitamos superar la barrera de Coulomb para que los participantes se acerqeun a distancias típicas de la fuerza nuclear. Algunos tipos de reacciones directas son:

\begin{itemize}
	\item \textbf{Stripping:} se transfieren nucleones del proyectil al blanco. Ejemplo:
	\begin{equation}
		\ce{d + C^12  -> ^13 C + p}
	\end{equation}
	\item \textbf{Pick-up:} se trasnfieren nucleones del blanco al proyectil. Ejemplo:
	\begin{equation}
		\ce{d +^12 C  -> ^11 C + t}
	\end{equation}
	\item \textbf{Intercambio de carga:} hay trasnferencia en los dos sentidos. Ejemplo:
	\begin{equation}
		\ce{^3 He + ^12 C -> ^12 Ne + t}
	\end{equation}
	\item \textbf{Knock-out:} se arranca un nucleón sin trasferencia. Ejemplo:
	\begin{equation}
		\ce{p + ^12 C -> ^11 B + p + p}
	\end{equation}
\end{itemize}


\section{Fisión}

La fisión inducida por neutros se descubrió en los años 1934-1939 como un resultado de los experimentos que pretendían crear elementos transuránicos mediante el bombardeo de Uranio con neutrones. Un ejemplo concreto de fisión inducida del \ce{^235U} por la absorción de neutrónicos podría ser:

\begin{equation}
\ce{n+^235_92 U -> ^148_57 La + ^87_35 Br + n}
\end{equation}
La energía liberada en esta reacción es enorme comparable con la que se libera en otro tipo de procesos no nucleares. 

%\subsection{Fisión según el modelo de la gota líquida}

\subsection{Reacciones en cadena}

La fisión deja a los núcleos hijos excitados, que suelen desexcitarse mediante la emisión (\textit{evaporación}) de neutrones. Estos productos de fisión son ricos en neutrones y se desintegran luego por emisión $\beta^-$. Los neutrones emitidos en el proceso de fisión peuden a su vez generar nuevas fisiones. Como dato, la fisión \ce{^235U} genera un promedio de unos 2.5 neutrones. Resulta útil definir el siguiente parámetro para obtener una clasifiación de las reaccioens:

\begin{eqnarray}    
    k = \frac{\text{Número de neutrones producidos en la etapa (n+1) de fisión}}{\text{Número de neutrones producidos en la etapa (n) de fisión}}
\end{eqnarray}
De acuerdo con $k$ tenemos las siguientes posibilidades:

\begin{itemize}
    \item $k<1$. La reacción transcurre en modo \textit{modo subcrítico}. No puede continuar indefinidamente y en algún momento se detiene.
    \item $k=1$. La reacción transcurre en modo \textit{crítico}. El número de neutrones disponibles para inducir fisiones permanece constante, de tal modo que la reacción puede continuar indefinidamente al mismo ritmo. Es el modo adecuado de operación sostenida en un reactor nuclear productor de energía.
    \item $k>1$. La reacción transcurre en modo \textit{super crítico}. El número de neutrones disponibles crece exponencialmente con el tiempo, conduciendo a una liberación explosiva de energía.
\end{itemize}

\section{Fusión}

La mayor dificultad para lograr la fusión nuclear a gran escala consiste en mantener el material fusible confinado a altas temperaturas durante el tiempo suficiente. Hasta el momento se está investigando en dos métodos: el confinamiento magnético y el confinamiento inercial. En el primero se hace circular plasma caliente de núcleos $^2$H y $^3$H en una región confinada por campos electromagnéticos. En el segundo se inyecta luz láser en una pequeña región que contiene el material fusible. En cualquier caso, el aprovechamiento comercial de la energía de fusión parece todavía una posibilidad lejana.

\section{Apéndices}

\subsection{Energía cinética clásica en el sistema LAB y en el CM}

\subsection{Momento lineal clásico en el sistema LAB y en el CM}

\subsection{Energía total relativista en el sistema LAB y en el CM}

\subsection{Energía cinética relativista en el sistema LAB y en el CM}

\subsection{Momento lineal relativista en el sistema LAB y en el CM}


