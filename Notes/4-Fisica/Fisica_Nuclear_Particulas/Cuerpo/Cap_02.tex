\chapter{Inestabilidad nuclear}

\section{Desintegración $\alpha$}

\subsection{Energética de la desintegración $\alpha$}

\subsection{Sistemática de la desintegración $\alpha$. Regla de Geiger-Nuttal}

\subsection{Tratamiento de Gamow, Guerney y Condon}

\subsection{Momento angular y paridad en la desintegración $\alpha$}

% Falta texto

El momento angular orbital que se lleva la partícula $\alpha$ en la Desintegración debe cumplir la \textbf{regla del momento angular}

\begin{equation}
    |I_i-I_f|\leq l_\alpha \leq I_i + I_f
\end{equation}
Por otro lado, la ley de conservación de la paridad exige $\pi_i=\pi_f \times \pi_\alpha \times (_1)^{l_\alpha}$, donde $\pi_i$ y $\pi_f$ son las paridades del núcleo inicial (padre) y del núcleo final (hijo), y $\pi_\alpha$ es la paridad de la partícula $\alpha$, que resulta ser positiva\footnote{El primer estado excitado del núcleo $^4_2$He tiene una energía de 20,1 MeV y también $I^\pi = 0^+$.}, por lo tanto tenemos que $pi_i = \pi_f (-1)^{l_\alpha}$. Concluimos, finalmente, que si la paridad nuclear inicial y final es la misma, entocnes $l_\alpha$ tiene que ser par, mientras que en caso contrario tiene que ser impar:

\begin{equation}
    \pi_i = \pi_f \Longrightarrow l_\alpha \ \text{par}
\end{equation}
\begin{equation*}
    \pi_i = - \pi_f \Longrightarrow l_\alpha \ \text{par}
\end{equation*}


% Falta texto

\section{Desintegración $\beta$}

\section{Transiciones electromagnéticas}

\section{Teoría continua de la desintegración radiactiva y aplicaciones}

\section{Apéndice}