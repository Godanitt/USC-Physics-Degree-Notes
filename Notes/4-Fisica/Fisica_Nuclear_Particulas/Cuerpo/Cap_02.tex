\chapter{Interacción nuclear}

\section{Evidencias experimentales}

\subsection{Deuterio}

Las medidas del momento cuadrupolar eléctrico dan resultados distintos de cero para el deuterón (Q=2.88$\pm$0.02 mb), por po que el neutrón y el protón deben estar orbitando alrededor del centro de masas (o que el protón/neutron no son esferas uniformemente cargadas). \\

De los datos anteriores, se concluye que los espines del protón y el neutrón son paralelos. Por tanto $\Sn=1$ y estarán en un estado {\bf triplete}. Dado que tiene que mantener una paridad $+1$ y un espín global $+1$, y por tanto $l=0,2$, es imposible que los espines del sistema protón-neutron no tengan la misma orientación. \\

\subsection{Dispersión protón-neutron}

Otra fuente de inforamción nuclear nos la da la dispersión entre nucleones. La sección eficaz protón-neutrón (choque protón-neutron) es constante a bajas energías: la interacción no es sensible a la estrucutra interna y parece comportarse como una dispersión binaria. \\

¿Por qué cae la sección eficaz? En el momento que subo la energía empiezo a ser sensible a la estrucuta interna de cada partícula, abriendose otros canales (ya no solo hay dispersión elástica) de interacción, como puede ser la excitación del protón, neutrón, creación de otras partículas... A nosotros nos interesa la sección eficaz de la interacción elástica. \\

Del cálculo teórico de la sección eficaz (con el potencial anterior) se pueden deducir dos componentes, una para la configuración {\bf triplete} y otra para la {\bf singlete} (ver Krane, Saborido). De este modo tenemos que la sección eficaz es muy diferente si ambas partículas chocan con el mismo espín o con diferente espín.  \\

Se pueden deducir las {\bf longitudes de dispersión} $a$ y  los{\bf rangos efectivos}: 

\begin{itemize}
    \item \textbf{Rango efectivo:} es una media del tamaño del potencial.
    \item \textbf{Longitud de dispersión:} es una medida de la intensidad de la interacción. Se puede definir como el tamaño que tendría una sfera rígida que daría la misma sección eficaz elástica $\sigma=4 \pi a^2$. 
\end{itemize}

% Krane (referencia)