\chapter{Inestabilidad nuclear}

\section{Desintegración $\alpha$}

La desintegración $\alpha$ produce lo que originalmente se llamó rayos $\alpha$, identificados en su época como la radiación menos penetrante emitida por los elementos radioactivos naturales. \\ % Falta texto

La desintegración por emisión alfa produce un desplazamiento hacia la izquierda de dos posiciones en la tabla periódica, y reduce el número másico en 4 unidades: $\Delta Z = -2$, $\Delta A = - 4$. Esquemáticamente escribiríamos

\begin{equation}
    ^A_Z  X_N \longrightarrow ^{A-4}_{Z-2} Y_{N-2} + ^4_2\He_2
\end{equation}

% falta texto

La desintegración $\alpha$ es un fenómeno que en esencia se debe a la repulsión coulombiana de los protones en un núcleo, que crece a un ritmo más rápido (va como $Z^2$) que la energía nuclear\footnote{Recuérdese que la energía de ligadura por nucleón es aproximadamente constante $B/A \sim $cte.} (que va como $A$).  % Falta texto

\subsection{Energética de la desintegración $\alpha$}

En la desintegración $\alpha$ intervienen la interacción electromagnética y la fuerte.  Entre otras cosas que veremos más tarde, en este proceso se conserva la energía, el momento lineal, el momento angular total y la paridad. Si suponemos que el átomo está inicialmente en reposo, la energía del sistema en el estado inicial es simplemente la energía de la masa en reposo de este átomo. En el estado final tenemos una partícula $\alpha$ y un núcleo hijo con cierta energía cinética. Usando masas nucleares podríamos entonces escribir las ecuaciones de balance energético como:

\begin{equation}
    m_X c^2 = m_Y c^2 + T_Y + m_\alpha c^2 + T_\alpha \\
\end{equation}    
\begin{equation}
    m_X c^2 - m_y c^2 - m_\alpha  c^2  = T_y + T_\alpha = Q \\
\end{equation}    
\begin{equation}
    Q = (m_X - m_Y - m_\alpha)c^2
\end{equation}    
Este proceso está energéticamente permitido sólo si $Q>0$. Aunque la expresión anterior está escrito en términos de las masas nucleares, podemos calcular $Q$ usando las masas atómomicas porque la masa electronesse cancela en la operación (en las desintegraciones $\alpha$ no se eliminan ni generan electrones). Despreciando las energías de ligadura electrónicas:
\begin{eqnarray}
Q/c^2 = M(^A_Z X_N) - M(^{A-4}_{Z-2}Y_{N-2}) - M(^4_2\He)
\end{eqnarray}
Si el núcleo inicial estaba en reposo, los momentos lineales de los productos deben ser iguales en magnitud y opuestos en sentido

\begin{equation}
    p_\alpha = p_Y
\end{equation}
Las energías típicas de las partículas $\alpha$ en desintegraciones naturales suelen estar alrededor de los 5 MeV, de modo que podamos usar cinemática no relativista, y escribir $T=p^2/2$. Por lo tanto:

\begin{equation}
	Q=T_\alpha + T_Y = \frac{p_\alpha2}{2m_\alpha} + \frac{p^2_\alpha}{2m_Y} = \frac{p_\alpha2}{2} \parentesis{\frac{m_\alpha + m_Y}{m_\alpha m_Y}} = T_\alpha \parentesis{\frac{m_\alpha}{m_Y} + 1 } 
\end{equation}	
De donde se deduce la ecuación que nos relaciona energía cinética con $Q$:

\begin{equation}
	T_\alpha = \frac{Q}{1+(m_\alpha/m_Y)}
\end{equation}
La emisión $\alpha$ comienza a ser espontáneamente posible ya para $A>150$, pero la matoría de los emisores $\alpha$ tienen $A>190$. Dado que para $A$ grande:

\begin{equation}
	T_\alpha = \frac{Q}{1+(m_\alpha/m_Y)} \approx \frac{Q}{1+(4/(A-4))}= \frac{Q}{1-4/A} \approx Q \parentesis{1-\frac{4}{A}}
\end{equation}
En una desintegración típica la partícula $\alpha$ se lleva alrededor del 98\% de $Q$. El 2\% que se lleva el núcleo $Y$ no es despreciable, puede ser del orden de 100 keV y es bastante mayor que la energía de ligadura de los átomos en los sólidos, lo cual quiere decir que si el átomo está cerca de la superficie puede ser expulsado de la muestra radiactiva. En ciertas circunstancias esta dispersión del material radiactivo debe ser tenida en cuenta, pero en cualquier caso el rango de penetración en materia de núcleos tan masivos es muy pequeño. 

La energía cinética de la partícula $\alpha$ se puede determinar por medio de un espectrómetro magnético y así también el valor de $Q$ de la reacción. Esto nos da un método de determinación de masas atómicas en el caso de un núcleo $X$ longevo que se desintegra en uno $Y$ demasiado efímero para que la espectometría de masas sea aplicable, tal como hemos visto en el capítulo anterior.

\subsection{Sistemática de la desintegración $\alpha$. Regla de Geiger-Nuttal}

Geiger y Nutall observaron ya en 1911 una relación inversamente proporcional entre el logaritmo del período de semidesintegración del núcleo emisor y la raíz cuadrada de las partículas $\alpha$ emitida., esto es:

\begin{mybox}
\begin{equation}
\ln (\tau_{1/2}) = k_1 + \frac{k_2}{\sqrt{Q}}
\end{equation}
\end{mybox}

\subsection{Tratamiento de Gamow, Gurney y Condon}

En 1928 Gamow, Gurney y Condon aplicaron casi simultáneamente los principios de la nueva mecánica cuántica al estudio de la desitegración $\alpha$. La idea esencial consiste en considerar la partícula $\alpha$ \textit{preformada} en el potencial creado por el núcleo hijo, que puede aproximarse a un pozo esférico profundo\footnote{En el capítulo que trata de la estructura nuclear (capítulo \ref{Ch:04}) veremos con algo más de detalle por qué el potencial nuclear se puede aproximar a un pozo de potencial, y en el capítulo sobre la interacción nucleón-nucleón veremos cuantitativamente la profundidad estimada del pozo.} con una frontera en forma de barrera coulombiana. La partícula $\alpha$ tendrá cierta probabilidad de cruzar la barrera por efecto túnel, que puede calcularse sin muchas dificultades, al menos aproximadamente. Este modelo proporciona resultados numéricos razonables que demuestran una compresión semicuantitativa del fenómeno de la desintegración $\alpha$, pero no significa que la \textit{imagen} de una partícula $\alpha$ \textit{preformada} en el interior del núcleo sea un buen reflejo de la realidad, sino más bien que en ciertas condiciones se comporta aproximadamente como tal. La teoría de Gamow expresa la probabilidad de desintegración $\alpha$ como un producto:

\begin{equation}
	\lambda = p_\alpha f P
\end{equation}
siendo:

\begin{itemize}
	\item $p_\alpha$: la probabilidad de que el \textit{cluster} $\alpha$ preexista como tal en el interior del núcleo padre (empíricamente se estima $p_\alpha  \sim 1$).
	\item $f$: frecuencia de colisión contra la barrera coulmbiana. Puede aproximarse como la velocidad entre el diámetro.
	\item $P$: coeficiente de tramisión (probabilidad de que atraviese la barrera cada vez que la partícula alfa choca contra ella).
\end{itemize}

Vamos ahora a realizar un cálculo para estimar la probabilidad de desintegración $\alpha$, y para ello usaremos un modelo semiclásico en una dimensión. El potencial coulombiano se corta a una distancia $a$ que se puede considerar como la suma de los radios del núcleo hijo y de la partícula $\alpha$. En la región interior $(r<a)$, la partícula $\alpha$ se mueve con una energía cinética dada por $Q+|V_0|$. La región $a<r<b$ forma una barrera del potencial.

\begin{figure}[h!] \centering
	\begin{pspicture}(-1,-5)(6,3)
		\psline[arrowscale=2,linewidth=1pt]{->}(0,-4)(0,3)
		\psline[linewidth=1pt](0,-4)(2,-4)
		\psline[linewidth=1pt](2,-4)(2,0)
		\psline[arrowscale=2,linewidth=1pt]{->}(2,0)(6,0)
		
		\psline[linewidth=0.9pt,linearc=2,linecolor=red](2,1)(3.5,0.1)(5,0.1)
		\psline[linewidth=0.9pt,linearc=2,linecolor=red](2,0)(2,1)
		
		\psline[linewidth=0.8pt,linearc=2,linestyle=dashed](0,0.5)(2,0.5)
		\psline[linewidth=0.8pt,linearc=2,linestyle=dashed](2.85,0.5)(4,0.5)
		
		
		\psline[linewidth=0.5pt,linearc=2](2.85,0.0)(2.85,0.5)
		
		\psline[linewidth=0.75pt,arrowscale=1]{<->}(0.5,0)(0.5,0.5)
		\rput(0.75,0.25){{\footnotesize Q}}
		
		
		\rput(2.5,1.5){{\footnotesize \textcolor{red}{{V$_{\text{coulomb}}$=\small k/r}}}}
		\rput(1.0,-4.5){{\small Pozo de potencial}}
		
		\rput(-0.4,2.0){V(r)}
		\rput(2.86,-0.19){b}
		\rput(5.0,-0.5){r}
		
		\psline[linewidth=0.75pt,arrowscale=1]{<->}(0,-1.2)(2,-1.2)
		\rput(1.0,-1){$R_0$}
		
		\psline[linewidth=0.75pt,arrowscale=1]{<->}(2.2,-4)(2.2,0.0)
		\rput(3.1,-2){{\footnotesize  $V_0 \sim 35 \ \unit{\MeV}$}}
				
	\end{pspicture}
	\caption{Pozo de potencial donde se encuentra la partícula $\alpha$ encerrada.}
	\label{Fig:02-02}
\end{figure}


En el contexto de nuestro sencillo modelo anterior, la constante de desintegración vendrá dada por $\lambda = fP$, donde $f$ es la \textit{frecuencia con la que la partícula incide sobre la barrera} y $P$ es la \textit{probabilidad de que la cruce}. No es complicado calcular el coeficiente de trasmisión para esa simple barrera coulombiana. Recordemos que el coeficiente de trasmisión viene dado a través de una barrera arbitraria en una dimensión viene dado aproximadamente por

\begin{equation}
    T = \exp \parentesis{-\frac{2\sqrt{2\mu}}{\hbar } \int_a^b \sqrt{V(r)-Q}\D r}
\end{equation}
donde la integración se realiza entre los límites que marcan la intersección de la recta $V(r) = Q$ con el pozo de potencial nuclear por la izquierda y la cola de la barrera coulombiana por la derecha. En la imagen \ref{Fig:02-02} el valor de $a=R_0$ (que suele ser $R_0\approx r_0A$ con $r_0=1.2$ fm) Se suele denominar \textbf{factor de Gamov} a la ecuación siguiente

\begin{equation}
    G = \frac{\sqrt{2\mu}}{\hbar} \int_a^b \sqrt{V(r)-Q}
\end{equation}
de manera que el cociente de trasmisión es

\begin{equation}
    T = e^{-2G}
\end{equation}
Escribimos ahora para $V(r)$ la energía coulombiana de la partícula alfa 

\begin{equation}
    V(r) = \frac{2(Z-2)e^2}{4\pi \epsilon_0 r}
\end{equation}
y obtenemos la siguiente integral

\begin{equation}
    G = \sqrt{\frac{2\mu}{\hbar^2Q}} \frac{2(Z-2)e^2}{4 \pi \epsilon_0} \ccorchetes{\arccos \sqrt{\frac{a}{b}}- \sqrt{\frac{a}{b}\parentesis{1-\frac{a}{b}}}}
\end{equation}
donde $2e$ es la carga de la partícula $\alpha$, $(Z-2)e$ la carga del núcleo hijo y $\mu$ la masa reducida del sistema. Aunque en algunas partes de la literatura aparezca $\mu$ como $m_\alpha$, en realidad esto último no deja de ser una aproximación, ya que 

\begin{equation*}
	\mu = \frac{m_Y m_\alpha}{m_Y + m_\alpha} 
\end{equation*}
de tal modo que si $m_\alpha \ll m_Y$ se verifica, efectivamente, que $\mu=m_\alpha$. Denotando por $B$ la altura de la barrera coulombiana en el punto $r=a$, tenemos 

\begin{equation}
    B= V(r=a) = \frac{2(Z-2)e^2}{4\pi \epsilon_0 a} \tquad Q=\frac{2(Z-2)e^2}{4 \pi \epsilon_0 b}
\end{equation}
Porque $Q$ es precisamente el valor que $V(r)$ toma en el radio $R=b$. De aquí deducimos que $a/b=Q/B$ y puesto que $Q/B\ll 1$ hacemos la siguiente aproximación 

\begin{equation}
    \arccos \sqrt{\frac{a}{b}} \approx \frac{\pi}{2} - \sqrt{\frac{a}{b}}
\end{equation}
con lo cual tenemos que 

\begin{equation}
    G \approx  \sqrt{\frac{2\mu}{\hbar^2Q}} \frac{2(Z-2)e^2}{4 \pi \epsilon_0} \parentesis{\frac{\pi}{2} -2 \sqrt{\frac{a}{b}}}
\end{equation}
Ahora todo lo que tenemos que hacer es multiplicar la probabilidad de penetración ya calculada por una frecuencia $f$ que exprese, de una manera simple, el número de impactos o intentos que la partícula $\alpha$ hace contra la barrera. Esta frecuencia es del orden de $f \sim v/a$, donde $v$ es su velocidad anterior del núcleo, que podemos expresar como $v=\sqrt{2(Q+|V_0|)/\mu}$. Esto nos lleva a la siguiente expresión para la constante de desintegración $\alpha$:

\begin{eqnarray}
    \lambda & = & \frac{1}{a} \sqrt{\frac{2(Q+|V_0|)}{\mu}} \exp \ccorchetes{-2  \sqrt{\frac{2 \mu}{\hbar^2 Q}} \frac{2(Z-2)e^2}{4 \pi \epsilon_0} \parentesis{\frac{\pi}{2}-2\sqrt{\frac{a}{b}}}} \\ \vspace{5.0mm}
            & = & \frac{1}{a} \sqrt{\frac{2(Q+|V_0|)}{\mu}}  \exp \ccorchetes{4 \alpha c \sqrt{\frac{2 \mu}{Q}} (Z-2) \parentesis{\frac{\pi}{2}-2\sqrt{\frac{a}{b}}}}
\end{eqnarray}
Donde $\alpha=e^2/(4 \pi \epsilon_0 \hbar c)$ es la constante de estructura fina. El período de semidesintegración o semivida sería:
\begin{mybox}
\begin{equation}
    \tau_{1/2} = a (\ln (2)) \sqrt{\frac{2\mu}{\hbar^2Q}} \exp \ccorchetes{4 \alpha c \sqrt{\frac{2 \mu}{Q}} (Z-2) \parentesis{\frac{\pi}{2}-2\sqrt{\frac{a}{b}}}}
    \label{Ec:02-01-22}
\end{equation}
\end{mybox}
De esta ecuación se deduce\footnote{Aunque es cierto que $Q$ también aparece en la raíz que multiplica la exponencial, la dependencia más fuerte de $\tau_{1/2}$ con $Q$ proviene de la exponencial en sí misma. El factor multiplicativo, que es esencialmente la velocidad de la partícula $\alpha$, tiene una variación con $Q$ despreciable comparada con la variación generada por la exponencial.} que el logaritmo de $\tau_{1/2}$ tiene una relación lineal inversa con la raíz cuadrada de la energía cinética de la partícula $\alpha$:

\begin{equation}
    \log \tau_{1/2} = k_1 + \frac{k_2 Z_Y}{\sqrt{Q}}
\end{equation}
donde $k_1$ y $k_2$ son constantes, y $Z_Y$ el número atómico del núcleo hijo. Esta es la \textbf{ley o regla de Geiger y Nuttal} encontraron empíricamente en torno a 1911. Haciendo el logaritmo de la ecuación (\ref{Ec:02-01-22}) vemos que:

\begin{equation}
	\ln \parentesis{\tau_{1/2}} = \ln\parentesis{a \ln(2) \sqrt{\frac{2\mu}{\hbar^2Q}}}  +  \ccorchetes{4 \alpha c \sqrt{\frac{2 \mu}{Q}} (Z-2) \parentesis{\frac{\pi}{2}-2\sqrt{\frac{a}{b}}}}
	\label{Ec:02-01-24}
\end{equation}

Vemos que la fórmula deducida para $\tau_{1/2}$ mediante este sencillo modelo proporciona valores que están más o menos dentro del mismo orden de magnitud que las medidas experimentales, lo cual es un logro importante si se tiene en cuenta que éstas últimas abarcan más de 22 órdenes de magnitud y que hemos simplificado considerablemente el problema: no hemos tenido en cuenta la función de ondas inicial y final del núcleo (regla de oro de Fermi), ni tampoco el momento angular de la partícula $\alpha$. La forma y radio del núcleo tiene también una influencia importante en el valor numérico final\footnote{Por ejemplo, pasar de un radio nuclear medio de 1,25$A^{1/3}$ fm a 1,20$A^{1/3}$ fm implica introducir un cambio de hasta un factor 5 en $\tau_{1/2}$ respecto al radio nuclear se puede usar incluso como un método para estimar el radio nuclear, o más bien, el radio de un núcleo más el de la partícula alfa.}.

% falta texto

\subsection{Momento angular y paridad en la desintegración $\alpha$}

En una desintegración $\alpha$ han de conservarse el momento angular total y la paridad. En cuanto al primerm tenemos que $\In_i = \In_f + \In_\alpha + \lnn_\alpha$, donde $\lnn_\alpha$ es el momento angular orbital de la partícula $\alpha$ relativo al núcleo emisor\footnote{Consideramos que el hijo se queda aproximadamente en reposo.} y $\In_\alpha$ su espín nuclear. Ahora bien, el núcleo de Helio tiene los nucleons apareados de manera que su espín es cero, por lo tanto $\In_\alpha  =0$ y $\In_i = \In_f + \lnn_\alpha$. Con lo cual el momento angular orbital que se lleva la partícula $\alpha$ en la Desintegración debe cumplir la \textbf{regla del momento angular}

\begin{mybox}
\begin{equation}
    |I_i-I_f|\leq \lnn_\alpha \leq I_i + I_f
\end{equation}
\end{mybox}
Por otro lado, la ley de conservación de la paridad exige $\pi_i=\pi_f \times \pi_\alpha \times (-1)^{\ell_\alpha}$, donde $\pi_i$ y $\pi_f$ son las paridades del núcleo inicial (padre) y del núcleo final (hijo), y $\pi_\alpha$ es la paridad de la partícula $\alpha$, que resulta ser positiva\footnote{El primer estado excitado del núcleo $^4_2$He tiene una energía de 20,1 MeV y también $I^\pi = 0^+$.}, por lo tanto tenemos que $pi_i = \pi_f (-1)^{l_\alpha}$. Concluimos, finalmente, que si la paridad nuclear inicial y final es la misma, entonces $\ell_\alpha$ tiene que ser par, mientras que en caso contrario tiene que ser impar:

\begin{equation}
    \pi_i = \pi_f \Longrightarrow \ell_\alpha \ \text{par}
\end{equation}
\begin{equation*}
    \pi_i = - \pi_f \Longrightarrow \ell_\alpha \ \text{impar}
\end{equation*}


% Falta texto

\section{Desintegración $\beta$}

La desintegración $\beta^-$ fue descubierta ya en 1896 por Becquerel. En 1932 se descubrió el positrón en el estudio de los rayos cósmicos, y más tarde (1934) se produjeron artificialmente elementos que se desintegraban emitiendo positrones, poniendo así de manifiesto la desintegración $\beta^+$. La captura electrónica, esto es, la captura de un electrón orbital por parte del núcleo fue descubierta por Álvarez en 1938. Los rayos $\beta^-$ tiene carga negativa y un poder de penetración de alrededor de 1mm. en plomo. Durante algún tiempo hubo confusión entre estos dos electrones emitidos en la desintegración $\beta$ y los electrones emitidos en la \textbf{conversión interna} de los núcleos. En física nuclear se suelen usar los símbolos $\beta^-$ y $\beta^+$ para designar estas radiaciones. La desintegración por emisión $\beta^-$ ($\beta^+$) produce un desplazamiento hacia la derecha (izquierda) de una posición en la tabla periódica, pero no cambia esencialmente la masa: $\Delta Z= \pm 1$ , $\Delta A = 0$. Responsable de este fenómeno es la interacción débil. Las principales diferencias con la radiación $\alpha$ son:

\begin{itemize}
	\item El electrón y el neutrino se \textit{crean} en la desintegración.
	\item Debido a sus energías y masas, son relativistas.
	\item Sus distribuciones de energía son continuas.
\end{itemize}	

El \textit{reemplazo} de una neutrón por un protón, o viceversa, tiene lugar en los núcleos a través de los siguientes procesos:

\begin{equation}
	n \longrightarrow p + e^- + \bar{\nu}_e  \tquad \text{desintegración} \ \beta^-
\end{equation}
\begin{equation}
	p \longrightarrow n + e^+ + {\nu}_e \tquad \text{desintegración} \ \beta^+
\end{equation}
\begin{equation}
	p + e^- \longrightarrow n + {\nu}_e \tquad \text{captura electrónica (CE} \ \epsilon)
\end{equation}
siendo el primero de ellos energéticamente imposible para los protones libres o átomos de hidrógeno. 

\subsection{Energética de la desintegración $\beta$.}

Experimentalmente se observa que la energía cinética de las partículas $\beta$ forma una distribución continua, en marcado contraste con el carácter monoenergética de las partículas $\alpha$ emitidas rediactivamente. Recordemos que la energía cinética de la partícula $\alpha$ es igual a la diferencia de energía de ligadura del núcleo padre y del núcleo hijo (excepto por pequeñas correciones debidas al retroceso del núcleo emisor). En un principio no se conocía la existencia de neutrinos y se creía que la desintegración $\beta$ sólo intervenían los núcleos padre e hijo, y el electrón emitido. No obstante, el hecho de que el proceso fuese continuo indicaba que en el proceso deben intervenir más de dos partículas, como mínimo tres, que se repartan estadísticamente la energía cinética liberada por la desintegración. 

Esquematizamos el proceso de desintegración $\beta^-$ del siguiente modo:
\begin{equation}
	^A_Z X_N \longrightarrow _{Z_1}^A Y_{N-1} + e^- \bar{\nu}_e \quad \text{con} \quad Q_{\beta^-} = (m_X-m_Y-m_e)c^2 \ (\text{masas nucleares})
\end{equation}
Recordemos que

\begin{equation*}
	M(Z,A) = m_N + Zm_e - \frac{1}{c^2} \sum_{i=1}^Z B_i \tquad
	m(Z,A) = M_N - Zm_e + \frac{1}{c^2} \sum_{i=1}^Z B_i
\end{equation*}
donde las $B_i$ son las energías de ligadura de los electrones atómicos. Por lo tanto

\begin{equation}
	\frac{Q_{\beta^-}}{c^2} = \ccorchetes{M(^A_ZX_N) - Zm_e} - \ccorchetes{M(_{Z+1}^A Y)-(Z+1)m_e}  - m_e + \ccorchetes{\sum_{i=1}^{Z} B_i - \sum_{i=1}^{Z+1} B_i}
\end{equation}
Vemos que las masas de los electrones se cancelan en la ecuación anterior. Despreciando las pequeñas diferencias de ligadura electrónicas, expresamos la energía de la desintegración en términos de masas atómicas de la siguiente manera:

\begin{equation}
	Q_ {\beta^-} = \ccorchetes{M(X)-M(Y)} c^2 \tquad Q_{\beta^-} \approx T_e + T_\nu
\end{equation}
La aproximación de la segunda expresión consiste en despreciar la energía de retroceso del núcleo emisor. Está claro entonces que

\begin{equation}
	(T_e)_{\max} = (T_\nu)_{\max} = Q_{\beta^-}
\end{equation}
En términos de masas atómicas:
\begin{equation}
	\frac{Q_{\beta^+}}{c^2} = \ccorchetes{M(^A_ZX_N) - Zm_e} - \ccorchetes{M(_{Z-1}^A Y_ {N+1})-(Z-1)m_e}  - m_e + \ccorchetes{\sum_{i=1}^{Z} B_i - \sum_{i=1}^{Z-1} B_i}
\end{equation}
\begin{equation}
	Q_ {\beta^+} = \ccorchetes{M(X)-M(Y)-2m_e} c^2 
\end{equation}
Para la captura electrónica (denotada por C.E. o $\varepsilon$) tenemos:

\begin{equation}
	^A_ZX_N + e^- \rightarrow ^A_{Z-1} Y_{N+1} + \nu_e \tquad Q_{\textbf{CE}} = Q_\varepsilon = M(X) c^2 - \ccorchetes{M(Y)c^2+ B_n}
\end{equation}
donde $B_n$ es la energía de ligadura del electrón correspondiente a la capa $n$ (K,L,M...) que coincide con los rayos X (uno o varios) que se emiten cuando el resto de los electrones atómicos bajan en cascado a ocupar la vacante dejada por el electrón capturado\footnote{Justo a continuación de la captura, la capa electrónica del núcleo hijo queda altamente excitada porque tiene una vacante en uno de sus orbitales bajos. Al ser ocupada la vacante se emiten rayos X característicos cuya energía coincide con $B_n$, lo cual implica que la masa o contenido energética del átomo hijo justo después de la captura excede en $B_n$ a la masa atómica del átomo en su estado fundamental. Debe tenerse en cuenta que la corteza electrónica del átomo también puede desexcitarse mediante la emisión de electrones Auger. Este fenómeno compite con la emisión de los rayos X característicos, y es más probable en átomos ligeros.}. La energía de esos rayos X puede parametrizarse de acuerdo con la ley de Moseley:

\begin{eqnarray}
	\sqrt{\frac{E}{hc}} = A_n (Z-B_n)
\end{eqnarray}	
donde las $A_n$ y $B_n$ son constantes que dependen de la capa de la que haya sido capturado (o expulsado) el electrón. Esto es, tendríamos $A_K,A_L,A_M,B_K,B_L,B_M$ etc. Obsérvese que si la desintegración $\beta^+$ es energéticamente posible entonces la captura electrónica también, pero el recíproco no es cierto. Para qeu la desintegración $\beta^+$ sea posible es necesario que la diferencia entre las masas atómicas del padre y del hijo sea al menos de $2m_e c^2 \approx 1.022$ MeV. 

\begin{equation}
\begin{split}
\varepsilon \Longrightarrow \ \ & \ \ M(X)c^2 - M(Y)c^2 = Q_\varepsilon + B_n\\
\beta^+ \Longrightarrow \ \ & \ \ M(X)c^2- M(Y)c^2  = Q_{\beta^+} + 2m_ec^2
\end{split}
\end{equation}

Otra diferencia importante entre la desintegración $\beta^+$ y la captura electrónica es que los neutrinos emitidos en este último proceso son monoenergéticos, $E_\nu=Q_\varepsilon$ (retroceso del núcleo despreciable), mientras que los neutrinos emitidos en desintegración $\beta^+$ tienen un espectro continuo de energía.

Todas las ecuaciones energéticas anteriores se refieren a procesos de desintegración entre los estados fundamentales de los núcleos. Si se trata de estados excitados hemos de corregir el valor de $Q$ oportunamente, es decir, $Q_{\text{ex}} = Q_{\text{fund}}-E_{\text{ex}}$ donde $E_{\text{ex}}$ es la energía de excitación.



\subsection{Probabilidad de transición. Regla de oro de Fermi.}

Antes de profundizar en el estudio de la teoría de Fermi de la desintegración $\beta$ detengámonos un momento a examinar las transiciones entre estados energéticos de un sistema cuántico. Recordemos que en un estado estrictamente estacionario la densidad de probabilidad no cambia con el tiempo $(|\Psi|^2 = \cte)$. Además, la energía que corresponde a ese estado está perfectamente definida en el sentido de que tiene dispersión nula:

\begin{equation}
	(\Delta E)^2 = \langle E^2 \rangle - \langle E \rangle^2 = 0 \tquad \Delta E \Delta t \geq h / 2 \Rightarrow \Delta t \sim \infty
\end{equation}
Esto quiere decir que si un sistema (átomo, núcleo, etc.) se encontrase en un estado estacionario sería imposible que pudiese realizar transiciones fuera de él. % falta texto

% fatla texto

Un estado que se aparta un poco de ser estacionario ya no tiene la energía perfectamente definida, tendrá una cierta dispersión $\Delta E \neq 0$ que se llama \textit{anchura del estado} y se suele representar por $\Gamma$. Es posible relacionar, mediante el principio de incertidumbre, la vida media $\tau$ de ese estado haciéndolo corresponder con el intervalo de tiempo $\Delta t$ que uno tendría disponible para realizar la medida de la energía del estado. Esto implica que $\tau=\hbar \Gamma^{-1}$. La probabilidad de desintegración $\lambda$, es el inverso de la vida medio,

\begin{equation}
	\lambda = \frac{1}{\tau} = \frac{\Gamma}{\hbar}
\end{equation}
Es posible realizar en mecánica cuática un cálculo aproximado de la probabilidad de transición por unidad de tiempo, $\gamma$. El resultado, conocido como \textbf{regla de oro de Fermi} (que dedujo \textit{Paul Dirac}), puede escribirse como

\begin{mybox}
\begin{equation}
	\lambda = \frac{2\pi}{\hbar} |V'_{fi}|^2 \rho (E_f)	\label{Ec:02-02-14}
\end{equation}
\end{mybox}
donde el término $V_{fi}'$ es el valor esperado del débil potencial perturbador entre los estados final e inicial de transición, también conocido como elemento de matriz:

\begin{equation}
	V_{fi}'= \int \Psi_f^* V' \Psi_i \D v
\end{equation}
donde $\Psi_i$ y $\Psi_f$ son los estados nucleares cuasiestacionarios inicial y final. El término $\rho(E_f)$ es la \textbf{densidad de estados finales}, es decir, el número de estados por unidad de energía. Si el estado final es un estado aislado con energía en una estrecha distribución alrededor de $E_f$ entonces la probabilidad de transición será mucho menor que si tenemos un conjunto de muchos estados finales densamente concentrados en un pequeño intervalo alrededor de $E_f$, tal que

\begin{equation}
	\rho(E_f) = \frac{\D n}{\D E_f}
\end{equation}

\subsection{Teoría de Fermi de la desintegración $\beta$}

Existen diferencias importantes entre la desintegración $\beta$ y la desintegración $\alpha$.  En los procesos $\beta^-$ no hay barreara coulombiana que atravesar, y en los $\beta^+$ un cálculo sencillo muestra que el factor de Gamow es del orden de la unidad ($e^{-2G} \approx 1/e^2$). Otra diferencia es que en los procesos $\beta$ el electrón o positrón y el neutrino o antineutrino no existían previamente en el núcleo, son partículas creadas en el proceso de desintegración. Por último, las partículas $\beta$ tienen un espectro energético continuo, y no energías precisas. De acuerdo con la regla de oro de Fermi (\ref{Ec:02-02-14}) la probabilidad de transición por unidad de tiempo\footnote{Ya sabemos que esta probabilidad, multiplicada por el número de núcleos presente en cierto instante, da precisamente el número de desintegraciones por unidad de tiempo en ese mismo instante, la actividad.} puede calcularse como: 

\begin{equation}
	\lambda = \frac{2\pi}{\hbar} |V_{fi}'|^2 \rho (E_F)
\end{equation}
Como en el estado final del proceso tenemos el núcleo y dos leptones:

\begin{equation}
	V_{fi}'  = g \int_v \parentesis{\Psi_f^* \phi_e^* \phi_\nu^*} V' \Psi_i \D v = g M_{fi}
\end{equation}

En esta expresión hemos puesto explícita una constante de \textit{acoplamiento}, $g$, que representa la intensidad de interacción responsable de la desintegración $\beta$ (el mismo papel que la constante de estructura fina desempeña en al determinación de la intensidad de la interacción electromagnética). Suponiendo que las funciones de onda de los dos leptones finales ($\phi_e$ la del electrón, $\phi_\nu$ la del neutrino) se comportan como ondas planas :

\begin{eqnarray}
	\phi_e = \frac{1}{\sqrt{V}} \exp \parentesis{i \frac{\pn \cdot \rn}{\hbar}} = \frac{1}{\sqrt{V}} \ccorchetes{1+i\frac{\pn \cdot \rn}{\hbar}+ \ldots} \label{Ec:02-02-19}  \\
	\phi_\nu = \frac{1}{\sqrt{V}} \exp  \parentesis{i \frac{\qn \cdot \rn}{\hbar}} = \frac{1}{\sqrt{V}} \ccorchetes{1+i\frac{\qn \cdot \rn}{\hbar}+ \ldots} \label{Ec:02-02-20}
\end{eqnarray}
siendo $\pn$ y $\qn$ el momento del electrón/positrón y del neutrino/antineutrino respectivamente. El volumen $V$ es una constante de normalización, que no interferirá en el resultado final. Recordemos que el número de estados por unidad de energía confinada en un volumen $V$ viene dado por:

\begin{eqnarray}
	\D n = \frac{V}{h^3} \D^3 \pn = \frac{V}{h^3} 4 \pi p^2 \D p
\end{eqnarray}
Usaremos la segunda porque sólo estamos interesados en el módulo de $\pn$ denotado por $p$, puesto que lo que al final queremos hacer es obtener el espectro energético de los electrones, y obviamente su energía no depende de la dirección de $\pn$. 

El número de estados teniendo en cuenta el electrón/positrón y al neutrino/antineutrino simultánemente con los momentos $p$ y $q$ apropiados es:

\begin{eqnarray}
	\D^2 n = \frac{V^2}{h^6} (4\pi)^2 p^2 q^2 \D p \D q
\end{eqnarray}
En los procesos $\beta$ la energía típica del electrón es 1 MeV, lo cual implica un momento de $pc\sim1.4 MeV$, luego $p/\hbar \sim 0.007$ fm$^{-1}$ y podemos hacer la aproximación $\pn\cdot\rn\ll 1$ sobre el volumen nuclear\footnote{Al realizar la integración que aparece es el elemento de matriz, $M_{fi}$, sólo es necesario consdierar distancias del orden del tamaño nuclear, porque es obvio que sólo en esa región las funciones de onda del núcleo inicial y final toman valores significativamente distintos de cero.}. Esto nos lleva a considerar la comúnmente denominada \textbf{aproximación permitida} en la desintegración $\beta$ (\textit{allowed approximation}), que consiste en tomar únicamente el primer término en las expansiones de las ondas planas de las ecuaciones (\ref{Ec:02-02-19}) y (\ref{Ec:02-02-20}):

\begin{eqnarray}
	\exp \parentesis{i \frac{\pn \cdot \rn}{\hbar}} \approx 1 \tquad \exp \parentesis{i \frac{\qn \cdot \rn}{\hbar}} \approx 1
\end{eqnarray} 
Si tuvieramos en cuentra los otros términos no triviales tendríamos las \textit{aproximaciones prohibidas} (de segundo orden, tercer orden... en función de hasta que términos escogemos), que no es que no ocurran, solo que son menos probables. Entonces la probabilidad diferencial es\footnote{Obsérvese que el volumen $V$, que aparece en la densidad de estados finales se cancela con el que aparece en la normalización de las funciones de onda planas de los leptones $\phi_e$ y $\phi_\nu$.}:

\begin{eqnarray}
	\D \lambda = \frac{2\pi}{\hbar} g^2 |M_{fi}|^2 (4\pi)^2 \frac{p^2 q^2}{h^ 6} \frac{\D p \D q }{\D E_f} \label{Ec:02-02-24}
\end{eqnarray}
Suponemos que $M_{fi}$ es independiente de los momentos neutrino/electrón \footnote{Supongamos por ahora que $M_{fi}$ no depende de $p$, pero más tarde veremos que la inclusión de cierta dependencia con el momento nos llevará a establecer correcciones sobre la fórmula que vamos a obtener.}. Así, suponiendo que el neutrino prácticamente no tenga masa (o que es una partícula ultrarrelativista), nótese que $E_f=E_e+E_\nu=E_e+qc$ y por lo tanto qeu $\D q / \D E_f=1/c$. Agrupando todos los factores que no contienen el momento del electrón $p$, en una constante global $C$, podemos escribir:

\begin{eqnarray}
	\D\lambda = Cp^2 q^2 \D p \Longrightarrow N(p) \D p = Cp^2 q^2 \D p \label{Ec:02-02-25}
\end{eqnarray} 
donde $N(p)$ es el número de $e^{\pm}$ emitidos por unidad de momento $p$, es decir, su espectro de momentos. En esta ecuación (\ref{Ec:02-02-25}) está implícito el paso a un sistema de muchos núcleos. Esto es, si para un sólo núcleo $\D \lambda$ es la probabilidad diferencial de emisión beta con el momento del $e^\pm$ en el intervalo diferencial $\D p$, para un sistema de muchos núcleos $\D \lambda$ es precisamente el espectro diferencial de momentos.

El momento del $e^{\pm}$ y del neutrino están relacionados mediante la expresión $Q=T_e+T_\nu=T_e+qc$ (despreciamos la energía de retroceso del núcleo), por lo tanto podemos escribir el espectro únicamente en función de $p$:

\begin{eqnarray}
	q = \frac{Q-T_e}{c} = \frac{1}{c} \ccorchetes{Q- \parentesis{ \sqrt{p^2 c^2 + m_e^4 c ^4} -m_ec^2}}
\end{eqnarray}
de lo cual se deduce que:

\begin{eqnarray}
	N(p) \D p = \frac{C}{c^2} p^2 \parentesis{Q-T_e}^2 \D p = \frac{C}{c^2} p^2 \ccorchetes{Q- \parentesis{ \sqrt{p^2 c^2 + m_e^4 c ^4} -m_ec^2} } \D p
\end{eqnarray}
Obsérvese que la función se anula si $p=0$  y $T_e=Q$. Para poder obtener el espéctro de energía cinética solo tenemos que relacionar momento y energía

\begin{eqnarray}
	p=\sqrt{2m_eT_e+(T_e/c)^2}
\end{eqnarray}
de lo cual obtenemos que

\begin{eqnarray}
	N(T_e) \D T_e = \frac{C}{c^5} \sqrt{2m_e c^2 T_e + T_e^2} \parentesis{m_e c^2 + T_e} \parentesis{Q-T_e}^2 \D T_e \label{Ec:02-02-29}
\end{eqnarray}

No debería sorpreder al lector que se diga que existe una discrepancia entre los espectros de las ecuaciones deducidas con las reales, ya que no hemos tenido en cuenta, entre otros factores, la influencia del campo coulombiano del núcleo. 

El campo coulombiano repele al positrón y atrae al electrón, de manera que los respectivos espectros deben estar desplazados respecto a la predicción de la ecuación (\ref{Ec:02-02-29}). La pregunta ahora es ¿Como cuantizar este desplazamiento? La respuesta está en en la llamada \textbf{función de Fermi} $F(Z',p)$ o $F(Z',T_e)$ siendo $Z'$ el número atómico del núcleo hijo. La función de Fermi, viene dada por

\begin{eqnarray}
	F(Z',T_e) \approx \frac{2\pi \eta}{1-e^{-2\pi \eta}} \tquad \eta = \mp \frac{Z'e^2}{4\pi \varepsilon_0 \hbar v_e} = \mp \frac{Z' \alpha}{v_e/c} \quad \text{para} \ \beta^{\mp}
\end{eqnarray}

Además, si tomamos el siguiente miembro de la expansión de las ondas de los leptones, estaremos introduciendo dependencia con su momento. A las desintegraciones para las que la \textbf{aproximación permitida} ($e^{i\pn \cdot \rn}\approx 1$) es suficientemente buena las llamamos \textbf{desintegraciones permitidas}. Si es necesario añadir un término no trivial la llamamos \textbf{desintegración prohibida de primer orden}. Si hay que añadir un segundo término, diremos \textbf{desintegración de segundo orden} (y así \textit{ad infinitum}). A pesar del nombre hay que recordar que estas desintegraciones sucedes, aunque son menos probables (y menos probables en tanto en cuanto aumentamos el orden de la prohibición). Resumiendo, el \textit{espectro $\beta$ completo incluye los siguientes factores}

\begin{itemize}
	\item Un factor estadístico $p^2 (Q-T_e)^2$ que proviene del número de estados finales accesibles a las partículas finales.
	\item La función de Fermi $F(Z',p)$ que tiene en cuenta la influencia del campo coulombiano nuclear.
	\item El elemento de matriz nuclear $|M_{fi}|^2$, que tiene en cuenta los efectos de los estados nucleares inicial y final.
\end{itemize}

Al escribir las ecuaciones precedentes hemos supuesto que toda la dependencia del momento del electrón/positron está en el facto estadístico $p^2(Q-T_e)^2$. Sin embargo esto es sólo una aproximación, y en general hemos de tener en cuenta también una posible dependencia con el momento del electrón/positrón proveniente de la integral $\int_v (\Psi_f^* \phi_e^* \phi_\nu^*) V' \Psi_i \D v$en el caso de que la aproximación permitida no sea buena y tengamos que añadir más términos que el primero en la expansión de las ondas planas de los leptones. Esta dependencia del momento de los leptones se suele poner separadamente en lo que se conoce como \textit{shape factor} $S(p,q)$. Todos estos factores nos llevan a:

\begin{eqnarray}
	N(p) \D p \sim p^2 \parentesis{Q-T_e}^2 F(Z',p) |M_{fi}|^2 S(p,q) \D p
\end{eqnarray}

\subsection{Verifiación experimental de la teoría de Fermi}

\subsubsection{La forma del espectro}

En la aproximación permitida $(S(p,q)=1$) podemos escribir la expresiçon anterior como

\begin{eqnarray}
	(Q-T_e) \sim \sqrt{\frac{N(p)}{p^2 F(Z',p)}}
\end{eqnarray}
de tal manera que el gráfico de $\sqrt{N(p)/(p^2 F(Z',p))}$ frente a $T_e$ debería ser una línea recta cuya intersección con el eje de abcisas es precisamente la energía de desintegración $Q$. Esto es lo que se llama la \textbf{gráfica de Fermi-Kurie}. En el caso de desintegraciones prohibidas es necesario incluir la dependencia con el momento a través del \textit{shape factor} $S(p,q)$:
\begin{eqnarray}
	(Q-T_e) \sim \sqrt{\frac{N(p)}{p^2 F(Z',p)S(p,q)}}
\end{eqnarray}
En algunos casos simples de procesos prohibidos de primer orden el \textit{shape factor} viene dado por la sencilla expresión $S(p,q)=p^2 + q^2$.

\subsubsection{La tasa de desintegración total}

Para encontrar la tasa de desintegración total hemos de integrar la tasa de desintegración parcial o diferencial que, introduciendo la función de Fermi en la ecuación (\ref{Ec:02-02-24}), viene dada en la aproximación permitida por 

\begin{eqnarray}
	\D \lambda = \frac{4g^2 |M_{fi}|^2}{(2\pi)^3 \hbar^7 c^3} F(Z',p)p^2 (Q-T_e)^2 \D p
\end{eqnarray}
por consiguiente:

\begin{equation}
	\lambda= \frac{4g^2 |M_{fi}|^2}{2\pi^3 \hbar^7 c^3} \int_0^{p_{\max}} F(Z',p)p^2 (Q-T_e)^2 \D p
\end{equation}
donde, como antes, $Z'$ es el número atómico del núcleo hijo. Esta integral depende, en última instancia, de $Z'$ y de la energía relativista total máxima del electrón, denotada por $E_0 \approx Q_0 + m_e c^2$ siendo $cp_{\max}=\sqrt{E_0^2 - m_e^2c^4}$. Se define la \textbf{función integral de Fermi} como

\begin{equation}
	f(Z',E_0) = \frac{1}{(m_ec)^3(mec^2)^2} \int_0^{p_{\max}} F(Z',p) p^2 (E_0-E_e)^2 \D p\label{Ec:02-02-36}
\end{equation} 
que es una cantidad adimensional y se encuentra tabulada para diversos valores de $Z'$ y $E_0$. Obsérvese que en (\ref{Ec:02-02-36}) aparece la diferencia entre las energías relativistas totales, que coincide obviamente con la diferencia de energías cinéticas: $E_0 - E_e = (Q+m_ec^2)-(T_e+m_ec^2)=Q-T_e$. Con esta definición podemos escribir:

\begin{equation}
	\lambda = \frac{g^2 |M_{fi}|^2}{2\pi^3 \hbar^7 c^3} (m_ec)^3 (m_ec^2)^2 f(Z',E_0) =\frac{g m_e^5 c^4 |M_{fi}|^2}{2\pi^3 \hbar^7 c^3} f(Z',E_0) 
\end{equation}
y como $\lambda= \ln(2)/t_{1/2}$ tendremos

\begin{mybox}
\begin{equation}
	ft_{1/2} = (\ln(2)) \frac{2\pi^3 \hbar^7 c^3}{g m_e^5 c^4 |M_{fi}|^2}
\end{equation}
\end{mybox}
A estos \textbf{valores ft} se les conoce como \textbf{semividas comparativas}. Los valores $ft$ nos proporcionan una manera de comparar las probabilidades de emisión $\beta$ para diferentes núcleos. Las diferencias en los valores $ft$ de los distintos núcleos deben porvenir de diferencias en el elemento de matriz nuclear, es decir, en diferencias entre las funciones de onda nucleares.

Al igual que en el caso de la desintegración $\alpha$, tenemos un rango enorme de valores de $t_{1/2}$ (de $10^3$ a $10^7$ segundos para $ft$). Por consiguiente se suelen tabular los logaritmos decimales $\logd ft$ con el tiempo en segundos. Los procesos con las semividas comparativas más bajas tienen valores de $\logd ft \sim 3-4$, y se conocen como \textbf{desintegraciones superpermitidas} \footnote{Cuando la función de ondas nuclear inicial y final se superponen perfectamente, la probabilidad de trantiene los mismos números cuánticos que el neutrón o protón destruido, es decir, los dos estados se encuentran en el mismo multiplete de isospin. Estas desintegraciones superpermitidas, y sus valores $ft$ coinciden esencialmente con el valor $ft$ de la desintegración del neutrón libre.} (\textit{superallowed decays}). \\


Algunos de estos procesos superpermitidos son transiciones $0^+ \rightarrow 0^+$, y en estos casos la matriz nuclear es muy simple: $M_{fi}=\sqrt{2}$, lo cual quiere decir que los valores $\logd ft$ de todos ellos deberían ser iguales, lo cual es aproximadamente cierto. Sabiendo que $M_{fi}=\sqrt{2}$ para las transiciones superpermitidas, podemos usar la expresión () y los valores experimentales de $ft_{1/2}$ para deducir la magnitud de la constante de acoplamiento $g$ introducida al principio. Se obtiene que

\begin{equation}
	g\approx 0.88 \times 10^{-4} \ \unit{\MeV\cdot\fm^3}
\end{equation}
Podemos conseguir una magnitud adimensional $G$ a partir de $g$ que nos permita comparar esta con otras constantes fundamentales de la naturaleza. Esta $G$ es:

\begin{equation}
	G= g \frac{m_p ^2 c}{\hbar^3} \approx 1.0 \times 10^{-5}
\end{equation}
menor que la fuerza fuerte y la electromagnética, pero mayor que la gravitatoria. 

\subsection[Clasificación de las desintegraciones $\beta$]{Clasificación de las desintegraciones. Reglas de selección del momento angular y de la paridad.}

La clasificación de las desintegraciones $\beta$, de manera breve, se puede ver en la tabla \ref{Tab:02-02-01}, con las reglas de selección del momento angular y de la paridad.

\begin{table}[h!]\centering
	\begin{tabular}{|l||c|c|}\hline
		Tipo de desintegración & $\Delta \pi$ & $\Delta I$ \\ \hline \hline
		Permitidas & 0 & 0,1 \\ \hline 
		Prohibidas (1º orden) & $\pm$2 & 0,1,2 \\ \hline
		Prohibidas (2º orden) & 0 & 2,3 \\ 	\hline	
		Prohibidas (3º orden) & $\pm$2 & 2,3 \\ \hline	
		Prohibidas (4º orden) & 0 & 4,5 \\ 	\hline	
		Prohibidas (5º orden) & $\pm$ 2 & 5,6  \\		 \hline
	\end{tabular}
	\caption{Reglas de selección del momento angular y de la paridad de las desintegraciones $\beta$. $\Delta \pi=0$ significa que \textit{no hay cambio de paridad} y que $\Delta \pi=\pm2$ significa que \textit{hay cambio de paridad}.}
	\label{Tab:02-02-01}
\end{table}

Vamos a explicar la notación de la paridad. Sea $\Delta \pi = \pi_f - \pi_i$. Entonces tenemos las posibilidades siguientes:

\begin{itemize}
	\item No hay cambio de paridad, por lo que o bien ambas tienen paridad positiva $\pi_f=\pi_i=1$ o ambas tienen paridad negativa  $\pi_f=\pi_i=-1$, lo que lleva en ambos casos a un cambio de paridad nulo $\Delta \pi = 0$.
	\item Hay cambio de paridad, lo que arroja las siguientes posibilidades: los valores $\pi_f=1,\pi_i=-1$ lo que lleva a un cambio de paridad $\Delta \pi=2$; y los valores $\pi_f=-1,\pi_i=1$ lo que lleva a que $\Delta \pi=-2$. 
\end{itemize}


\subsubsection{Desintegraciones permitidas (\textit{allow decays})}
\addcontentsline{toc}{subsubsection}{Desintegraciones permitidas}


Recordemos que en la denominada aproximación permitida hemos sustituido las funciones de onda del electrón y del neutrino, por su valor en el origen, esto es, consideremos que fueron \textit{creados} en $r=0$. En este caso no puede llevarse ningún momento angular orbital, y el único cambio posible de espín nuclear debe provenir del espín del electrón y del neutrino, que ambos son fermiones de espín $s_e=s_v=1/2$. En la aproximación permitida ($\ell=0$), podemos tener entonces los siguientes casos

\begin{itemize}
	\item El espín del sistema de dos leptones ($e$ y $\nu$) es $S=0$ (configuración singlete). En este caso no puede haber cambio en el espín nuclear: $\Delta I=|I_i-I_f|=0$. Se les llama \textbf{Fermi decays.}
	\item El espín del sistema de dos leptones ($e$ y $\nu$) es $S=1$ (configuración triplete). En este caso $\In_i$ e $\In_f$ están acoplados a través de un vector de longitud unidad: $\In_f + \unovec = \In_i$, $I_i=|I_f-1|$,$I_f$,$|I_f+1|$, lo cual quiere $\Delta I = 0,1$. Se les llama \textbf{Gamow-Teller decays}. Obsérvese que si $\In_i=\In_f=0$ tenemos también $\Delta I=0$, pero no es posible que exista una transición Gamow-Teller en este caso, porque los leptones no pueden ser emitidas en configuración tripelete ($0\neq 0+1$).
\end{itemize}
Puesto que la paridad asociado al momento angular orbital $l$ es $(-1)^l$, se deduce que hay cambio de paridad en estas transiciones $(l=0)$. Las \textbf{reglas de selección} para las desintegraciones $\beta$ \textbf{permitidas} son entonces:

\begin{eqnarray}
	\Delta I =0,1 \tquad \Delta \pi = 0 \quad (\text{no hay cambio de paridad})
\end{eqnarray}
Ejemplos de transiciones $\beta$ permitidas:

\begin{itemize}
	\item $\ce{^14 O -> ^14 N^*} \ (0^+\rightarrow0^+)$. Es una transición Fermi (F) pura, no puede haber contribución de la transición Gamow-Teller (GT) porque $I_i=I_f=0$.
	\item $\ce{^6 He -> ^6  Li}  \ (0^+\rightarrow1^+$). Es una transición GT pura.
	\item $n\longrightarrow p \ \parentesis{\frac{1}{2}^+\rightarrow \frac{1}{2}^+}$. Tenemos $\Delta I = 0$ y se trata de una transición mezcla (F) y (GT), porque es posible que los dos leptones se encuentran tanto la configuración triplete del espín ($S=1$)como en la singlete ($S=0$)\footnote{Las dos combinaciones que satisfacen la conservación del momento angular serían $\frac{1}{2}\otimes 1=\frac{1}{2}\oplus \frac{3}{2}$, para el estado singlete: $\frac{1}{2}\otimes 0=\frac{1}{2}$}. Las proporciones exactas en que contribuyen cada una de estas transiciones al proceso global dependen de las funciones de onda nuclear inicial y final. Se suele definir el cociente entre las amplitudes de Fermi y de Gamow-Teller de la siguiente manera:
	\begin{eqnarray}
		y=\frac{g_FM_F}{g_{GT}M_{GT}}
	\end{eqnarray}
	donde $M_F$ y $M_{GT}$ son los \textit{elementos de matriz nucleares} de Fermi y de Gamow-Teller. En la constante de transición global tendríamos que sustituir $g^2 |M_{fi}|^2$ por ($g_F^2|M_F|^2+g_{GT}^2|M_{GT}|^2$). Suponemos que $g_F$ es idéntica a la $g$ deducida para las transiciones superpermitidas de Fermi ($0^+\rightarrow 0^+$). Para la desintegra del neutrón el elemento de matriz nuclear es simplemente $|M_F|=1$. Puesto que la constante de transición (decay rate) es proporcional a $g_F^2 |M_F|^2 (1+y^{-2})$, la tasa de desintegración del neutrón permite el cálculo del cociente $y$, que arroja un valor de $y=0.467\pm 0.003$ Es decir, la desintegración del neutrón libre tiene lugar en un 82\% de las ocasiones según un proceso de Gamow-Teller, y en el $18\%$ restante según un proceso de Fermi\footnote{De acuerdo con \begin{equation*} \frac{g_F^2 |M_F|^2}{g_F^2 |M_F|^2 (1-y^{-2})} = \frac{1}{1+y^{-2}} = 0.179  \end{equation*}}
	
\end{itemize}
En general el cálculo de $M_F$ y $M_{GT}$ es complicado, pero en el caso especial de núcleos espejo resulta particularmente simple porque las funciones de onda inicial y final son las mismas (excepto por pequeñas correcciones coulombianas). Un protón se convierte en neutrón más el positrón y el neutrino, y el cociente de amplitudes (F) y (GT) es similar al de la desintegración del neutrón libre antes vista. 

\subsubsection{Desintegraciones prohibidas (\textit{forbidden decays})}

Se denominan \textit{\textbf{fist forbidden decays}} o \textit{\textbf{desintegraciones prohibidas de primer orden}} a aquellas transiciones $\beta$ en la que los leptones se llevan una unidad de momento angular orbital $\ell=1$. El término \textit{desintegración prohibida} es realmente desafortunado, ya que aunque son transiciones poco probables, no son imposibles. Al igual que las desintegraciones permitidas, las dividimos en transiciones de Fermi (F) y de Gamow-Teller (GT), dependiendo de que los leptones emitidos se encuentren en la configuración singlete de espín ($S=0$) o la triplete ($S=1$), respectivamente.



\begin{itemize}
	\item Para las \textit{desintegraciones Fermi prohibidas de primer orden} hemos de acoplar $S=0$ con $\ell=1$, lo cual proporciona $I_i = I_f\oplus 1$ ($I_i=|I_f-1|,I_f,|I_f+1|$), y por lo tanto $\Delta I = 0,1$ (pero no podemos tener $0\rightarrow 0$, porque entonces los leptones no se podrían llevar una unidad de momento angular orbital y encontrarase en un estado singlete de espín).
	
	\item Para las \textit{desintegraciones de Gamow-Teller de primer orden} hemos de acoplar $S=1$ con $\ell=1$, lo cual proporciona $I_i=I_f\oplus 2$ ($I_f\in \{ |I_f-2|,|I_f-1|,I_f,|I_f+2|,|I_f+2| \}$) y por tanto $\Delta =0,1,2$. 
\end{itemize}

Por otro lado, si $\ell=1$, habrá un cambio de paridad en la transición. Tenemos, por consiguiente, las siguientes reglas de selección para as transiciones prohibidas de primer orden:

\begin{eqnarray}
	\Delta I = 0,1,2 \tquad \Delta \pi = \pm 2 \quad (\text{{\footnotesize  hay cambio de paridad}})
\end{eqnarray}
Algunos ejemplos de transiciones prohibidas de primer orden son:

\begin{itemize}
	\item $\ce{^17_7 N -> ^17_8 O} \quad \parentesis{\frac{1}{2}^- \rightarrow \frac{1}{2}^+}$
	\item $\ce{^76_35 Br -> ^76_34 Se}  \quad \parentesis{1^- \rightarrow 0^+}$
	\item $\ce{^122_51 Sb -> ^122_50 Sn^*}  \quad \parentesis{\frac{1}{2}^- \rightarrow \frac{1}{2}^+}$
\end{itemize}
Las desintegraciones $\beta$ \textbf{prohibdas de segundo orden} son aquellas en las que los leptones se lleva unidades de momento angular $\ell=2$, con lo cual no hay cambio de paridad. El proceso para obtener las reglas de selección es el mismo que el de las anteriores desintegraciones: las subdividimos en Fermi o Gamow-Teller, de tal manera que combinando tenemos que $\Delta I = 0,1,2,3$ (pero no siempre, porque las transiciones $0\rightarrow 0$ u $\frac{1}{2} \rightarrow \frac{1}{2}$ no permiten que los leptones se lleven dos unidades de momento angular). Por otro lado como los casos $\Delta = 0,1$ están comprendidos en las desintegraciones permitidas, por lo que la contribución de las prohibidas de segundo orden a la tasa de desintegración es completamente despreciable en estos casos, por lo que debemos ignorarlos. Así, las reglas de selección son:

\begin{equation}
	\Delta I = 2,3 \tquad \Delta \pi = \pm 2\quad (\text{{\footnotesize no hay cambio de paridad}})
\end{equation}
Algunos ejemplos:

\begin{itemize}
	\item $\ce{^22Na ->^22 Ne} \quad \parentesis{3^+ \rightarrow 0^+}$  
	\item $\ce{^137Cs ->^137 Ba} \quad \parentesis{\frac{7}{2}^+ \rightarrow \frac{3}{2}^+}$  
\end{itemize}

Las desintegraciones $\beta$ \textbf{prohibidas de tercer orden} son aquellas en las que los leptones se llevan un momento angular $\ell=3$. Así, tendrían las siguientes reglas de selección:

\begin{equation}
	\Delta I = 3,4 \tquad \Delta \pi = \pm 2\quad (\text{{\footnotesize  hay cambio de paridad}})
\end{equation}

\begin{itemize}
	\item $\ce{^87 Rb -> ^87 Sr} \quad (\frac{3}{2}^- \rightarrow \frac{9}{2}^+)$
	\item $\ce{^40 K -> ^40 Ca} \quad (4^- \rightarrow 0^+)$
\end{itemize}
Incluso es posible encontrar desintegraciones prohibidas de cuarto (y quinto) orden, pero cuanto más alto es el orden más improbable es que se produce. En la práctica estas transiciones son tan improbables que sólo pueden observarse cuando las otras son realmente imposibles. En cualquier caso dejamos en la tabla \ref{Tab:02-02-01} las reglas de selección de las transiciones permitidas y prohibidas hasta 5º orden.


\subsection{Desintegración doble $\beta$}

La desintegración doble $\beta$ (o $\beta \beta$) es un proceso mediante el cual dos neutrones se convierten ``simultáneamente'' en dos protones, dos electrones y dos antineutrinos. Un ejemplo es el siguiente:

\begin{eqnarray}
	\ce{^82_34 Se -> ^82_36 Kr + 2e^- + 2\bar{\nu}_e}
\end{eqnarray}
Es necesario insistir en que se trata de un proceso único y no dos desintegraciones $\beta$ sucesivas muy seguidos en el tiempo. En este caso, el proceso $
\ce{^82_34 Se -> ^82_35 Br + e^- + \bar{\nu}_e}$ está prohibido por la ley de conservación de energía ($Q<0$), por lo que la desintegración $\beta$ no puede tener lugar. Sin embargo $\beta \beta$ tiene un $Q>0$. Esta es una característica común a la mayoría de las desintegraciones doble $\beta$: el núcleo intermedio que resultaría de la desintegración $\beta$ simple es más pesado que el núcleo padre, mientras que el resultante de la doble $\beta$, no. \\

Mediciones experimentales de los períodos de semidesintegración para  algunas desintegraciones $\beta \beta$ mencionan arrojan tiempos de vida medio ridículos, 
\begin{equation}
	\ce{^82_34 Se -> ^82_36 Kr + 2e^- + 2\bar{\nu}_e} \tquad t{1/2} = (1.2\pm 0.1)\times10^{20} \ \textbf{años}
\end{equation}
\begin{equation}
\ce{^48_20 Ca -> ^48_22 Ti + 2e^- + 2\bar{\nu}_e} \tquad t{1/2} = (4.3\pm 1.4)\times10^{19} \ \textbf{años}
\end{equation}
Es decir, se trata de procesos sumamente improbables, ya que sabiendo que la probabilidad de transición de $\beta$ viene dada por

\begin{equation}
	\lambda_\beta = \frac{m_e c^2}{\hbar} \parentesis{fg^2 \frac{m_e^4 c^2 |M_{fi}|^2}{2\pi^3 \hbar^6}}
\end{equation}
podemos estimar que la tasa de desintegración $\beta \beta$, salvo por el factor inicial, el cuadrado del anterior:

\begin{equation}
	\lambda_{\beta \beta} = \frac{m_e c^2}{\hbar} \parentesis{fg^2 \frac{m_e^4 c^2 |M_{fi}|^2}{2\pi^3 \hbar^6}}^2
\end{equation}
aunque este tratamiento simplista no se debe tomar demasiado en serio a efectos de predicciones cuantitativas. La búsqueda experimental de desintegraciones doble $\beta$ sin emisión de neutrinos\footnote{Obsérvese que la desintegración doble $\beta$ sin emisión de neutrinos no está prohibida por la ley de conservación de la energía y el momento, tal y como sucede para las desintegraciones $\beta$ simple sin emisión de neutrinos.} (\textit{neutrinoless double beta decay}) es campo de investigación activo e intersante, porque su existencia sería una confirmación más de que los neutrinos no tienen masa nula\footnote{Sin entrar en detalles, se puede considerar que la desintegración doble $\beta$ sin emisión de neutrinos tiene lugar mediante dos procesos virtuales. En el primero de ellos un neutrón se desintegra y emite un neutrino que luego es absorbido por otro neutrón para iniciar una reacción beta inversa que lo convierte en protón. Para que esto pueda ocurrir es necesario que el neutrino emitido en el primer proceso virtual cambie de helicidad, lo cual implicaría que tiene masa.}.


\section{Transiciones electromagnéticas}

Los rayos $\gamma$ son capaces de penetrar varios milímetros en plomo. No son desviados por los campos electromagnéticos e interaccionan con la materia de manera similar a los rayos X.  Se trata de radiación electromagnética, e inicialmente se confundieron os rayos X emitidos por el reordenamiento de los electrones atómicos que sigue a una conversión interna. La desintegración $\gamma$ consiste en la emisión espontánea de fotones altamente energéticos cuando el núcleo pasa de un estado excitado a otro estado de menor energía o al fundamental. Es por tanto un proceso análogo al que tiene lugar cuando un átomo se desexcita emitiendo radiación, bien sea en el rango visible o en el de los rayos X. La emisión gamma suele acompañar a los otros dos tipos de radiación, porque sus productos quedan normalmente excitados. La vida media típica para una emisión $\gamma$ es de unos $10^{-9}$ segundos, pero a veces se observan vidas medias significativamente mayores. Estas transiciones se denominan \textbf{isoméricas}, y a los estados excitados de vida media larga se les llama estados \textbf{estados metaestables}, estados isoméricos, o isómeros. Se suele denotar esta característica con un superíndice en el símbolo del elemento $\ce{^110 Ag^m}$. Los rayos $\gamma$ provenientes de transiciones nucleares electromagnéticas están en un rango de energías de entre 0.1 y 10 MeV, lo cual implica una longitud de onda $\lambda$ entre $10^4$ y $10^2$ fm. 

\subsection{Energética de la radiación $\gamma$}

Si denotamos por $E_i$ y $E_f$ las energías de los estados excitados nucleares inicial y final respectivamente, la ley de conservación de energía para una transición electromagnética nos permite escribir:

\begin{equation*}
	E_i = E_f + E_\lambda + T_R, \quad \Delta E = E_i - E_f = E_\gamma + T_r = p_\gamma c+ \frac{p^2_R}{2M}  
\end{equation*}
dado que el momento se tiene que conservar:

\begin{equation*}
	\Delta E = p_\gamma c+ \frac{p^2_Rc c^2}{2Mc^2} =E_\gamma + \frac{E_\gamma^2}{2Mc^2} = E_\gamma \parentesis{1+\frac{E_\gamma}{2Mc^2}}
\end{equation*}
Es evidente que $T_R$ no puede ser otra cosa que la \textit{energía de retroceso del núcleo emisor}. Podemos ver que podemos escribir la ecuación de arriba como una ecuación de segundo grado:

\begin{eqnarray*}
	E_\gamma^2 +2 Mc^2 E_\gamma - 2Mc^2 \Delta E = 0 
\end{eqnarray*}
la cual nos  lleva a la solución (elegimos la positiva evidentemente):

\begin{equation}
	E_\gamma = Mc^2 \ccorchetes{-1+\sqrt{1+\frac{2\Delta E}{Mc^2}}}
\end{equation}
Aproximando $\sqrt{1+x} \approx 1+x/2-x^2/8...$ tenemos que:

\begin{equation}
	E_\gamma = \Delta E \parentesis{1-\frac{\Delta E}{2Mc^2}}
\end{equation}
Vemos entonces que la energía del fotón no es exactamente igual a la diferencia de energías entre los dos estados excitados, ya que para conservar el momento debemos tener en cuenta la energía de retroceso del núcleo, aunque para la mayoría de propósitos podemos despreciar esta relación y escribir $E_\gamma = \Delta  E$.

\subsection{Análisis multipolar de la radiación electromagnética.}

Las distribucioens de cargas y corrientes estacionarias crean campos eléctrico y magnéticos estacionarios que se analizan en términos de los \textbf{momentos multipolares estáticos}. Las distribuciones estáticas no crean campos de radiación, no se radia energía. Por el contrario, las distribuciones de cargas y corrientes cambiantes en el  tiempo sí crean \textbf{campos de radiación}, es decir, radian energía. Un dipolo eléctrico está formado por un par de cargas positiva y negativa, separadas una cierta distancia y crean el denominado campo dipolar eléctrico. Si las dos cargas del dipolo vibran alrededor de una posición de equilibrio con cierta frecuencia $\omega$, emitirán radiación dipolar eléctrica. Análogamente, una corriente eléctrica ciruclar uniforme crea un campo dipolar magnético y no emite radiación, pero si hacemos que la intensidad de esa corriente eléctrica oscilase sinusoidalmente (por ejemplo) en el tiempo con cierta frecuencia $\omega$\footnote{Creando un momento dipolar magnético oscilante en el tiempo del tipo $\mu(t)=iA\cos (\omega t)$, donde $i\cos (\omega t)$ es la corriente sinusoidal del anillo circular y $A$ su área.}, tendremos radiación dipolar magnética. Estos son los dos ejemplos más sencillos de distribuciones variables de cargas o corrientes, pero cualquier campo de radiación se puede analizar en términos de sus \textbf{componentes multipolares}. 

Denotaremos por EL a la \textbf{radiación multipolar eléctrica} de orden $2^L$, de tal manera que para $L$ tenemos \textit{radiación dipolar}, para $L=2$ \textit{radiación cuadrupolar}, para $L=3$ \textit{radiación octupolar}, para $L=4$ \textit{hexadecapolar}, etc. Análogamcente denotaremos por ML a la \textbf{radiación multipolar magnética} de orden $2^L$. En Mecáica Cuántica hablamos de operadores multipolares, de tal manera que la probabilidad de transición (emisión del fotón) está gobernada por el elemento de matriz del \textbf{operador multipolar} correspondiente:

\begin{equation}
	m_{fi} (\sigma L) = \int_\tau \Psi_f^* m(\sigma L) \Psi_i \D \tau \quad
\end{equation}
donde $\sigma=E,M$  y $m(\sigma L)$ es el operador multipolar. La paridad del campo de radiación eléctrica es $\pi(EL)=(-1)^L$, y la del campo de radiación magnética es $\pi(ML)=(-1)^{L+1}$.


Para calcular la probabilidad de emisión de un fotón de energía $\hbar \omega$ en la transición de dos estados es preciso utilizar el aparato matemático de la electródinamica cuántica. Sin embargo, es posible escribir la siguiente expresión aproximada reinterpretando convenientemente la fórmula clásica para la potencia radiada. Así, la \textbf{probabilidad de transición} (por unidad de tiempo):

\begin{equation}
	\lambda (\sigma L) = \frac{2(L+1)}{\epsilon_0 \hbar L [(2L+1)!!]} \parentesis{\frac{\omega}{c}}^{2L+1} \ccorchetes{m_{fi}(\sigma L)}^2
\end{equation}
donde la expresión del doble factorial significa $(2L+1)!!=(2L+1)\times(2L-1) \times \ldots \times 5 \times 3 \times 1$. Lógicamente esta $\lambda$ la podremos relacionar experimentalmente con el valor del periodo de semidesintegración ($\lambda=ln(2)/t_{1/2}$). Para obtener una expresión más concreta es necesario evaluar el elemento de matriz $m_{fi}(\sigma L)$, para lo cual necesitamos conocer las funciones de onda nucleares inicial y final. No obstante si suponemos que la emisión $\gamma$ se debe a la transición de un único protón de un estado a otro se puede hacer un cálculo simplificado cuyo resultado es:

\begin{equation}	
	\lambda(EL) \approx \frac{e^2}{4\pi\epsilon_0 \hbar c} \frac{8\pi (L+1)}{L[(2L+1)!!]^2} \parentesis{\frac{E}{\hbar c}}^{2L+1} \parentesis{\frac{3}{L+3}}^2 cR^{2L}
\end{equation}
\begin{equation}	
	\lambda(ML) \approx \frac{e^2}{4\pi\epsilon_0 \hbar c} \parentesis{\frac{\hbar}{m_pc}}^2 \frac{8\pi (L+1)}{L[(2L+1)!!]^2} \parentesis{\mu_p - \frac{1}{L+1}} \parentesis{\frac{E}{\hbar c}}^{2L+1} \parentesis{\frac{3}{L+2}}^2 cR^{2L-2}
\end{equation}
donde $E$ es la energía del fotón emitido $E_\gamma$, $R$ es el radio nuclear y $\mu_p$ es el \textit{momento magnético del protón}. Haciendo $R=R_0 A^{1/3}$ podemos hacer las siguientes estimaciones, denominadas estimaciones de Weisskopf:

\begin{equation}
	\begin{split}
	\lambda (E1) \ = \ & \unit{\num{1.0e14} \ A^{2/3}E^3} \\
	\lambda (E2) \ = \ & \unit{\num{7.3e7} \ A^{4/3}E^5} \\
	\lambda (E3) \ = \ & \unit{\num{3.4e1} \ A^{6/3}E^7} \\
	\lambda (E4) \ = \ & \unit{\num{1.1e-5} \ A^{8/3}E^9} \\
	\end{split}
\end{equation}
tal que $\lambda \ [\unit{\s^-1}]$ y $E \ [\unit{\eV}]$. Para las estimaciones de los multipolos magnéticos se suele sustiuir la expresión $[\mu_p-1/(L+1)]^2$ por 10 y obtendremos

\begin{equation}
	\begin{split}
	\lambda (M1) \ = \ & \unit{\num{5.6e13} \ A^{0/3}E^3} \\
	\lambda (M2) \ = \ & \unit{\num{3.5e7} \ A^{2/3}E^5} \\
	\lambda (M3) \ = \ & \unit{\num{1.6e1} \ A^{4/3}E^7} \\
	\lambda (M4) \ = \ & \unit{\num{4.5e-6} \ A^{6/3}E^9} \\
	\end{split}
\end{equation}
Las estimaciones de Weisskopf nos proporcionan una herramienta para comparar de modo aproximado la importancia relativa entre las distintas transiciones electromagnéticas. Esa es su utilidad, no se pretende que las comparaciones directas con datos experimentales sean de alta precisión. Para determinar la multipolaridad concreta de la radiación emitida es necesario medir su distribución angular, y para distinguir su carácter electrico o mangético es necesario medir su polarización. 


\subsection{Reglas de selección del momento angular y de la paridad}

La teoría electromagnética clásica predice la trasmisión de momento lineal y angular por parte de los campos electromagnéticas. Se han hecho experimentos precisos para medir ambos fenómenos, tanto la presión ejercida por la radiación como la trasferencia de momento angular. Si hacemos incidir fotones circularmente polarizados a derechas sobre un disco absorbente que pueda girar sobre su eje veremos que efectivamente gira para garantizar la conservación del momento angular. La tasa de \textit{radiación} del momento angular es proporcional a la tasa de radiación de energía. En la teoría cuántica esta proporcionalidad se mantiene si asignamos a cada fotón un cuanto preciso de momento angular. El operador multipolar de orden $2^L$ incluye el armónico esférico $Y_L^M (\theta,\phi)$ que está asociado con un momento angular $L$. Un multipolo de orden $2^L$ transfiere por lo tanto un momento angular $L\hbar$ un fotón\footnote{En general no es posible descomponer sin ambigüedad el momento angular total del fotón en una parte intrínseca y otra orbital, porque las dos están \textit{acopladas} y no son independientes.}.

Consideremos una transición $\gamma$ entre un estado nuclear inicial con espín $\In_i$ y un estado final con espín $\In_f$, tal que $I_i\neq I_f$. Está claro que la conservación del momento angular implica que

\begin{equation}
	\In_i = \In_f + \Ln  \rightarrow |I_i -I_f|\leq L\leq I_i + I_f
\end{equation}
Si por ejemplo $I_i=3/2$ y $I_f=5/2$, el momento angular podría valer $L=1,2,3,4$, pudiendo tener radiación dipolar, cuadrupolar, octupolar o hexadecapolar. El carácter magnético o eléctrico de la radiación emitida en cada caso dependerá de la paridad de los estados nucleares inicial y final. Recordemos que $\pi(EL)=(-1)^L$ y que $\pi(ML)=(-1)^{L+1}$, por lo que las reglas de selección para la emisión de radiación electromagnética del siguiente modo:

\begin{equation}
	\In_i = \In_f + \Ln \Longrightarrow |I_i -I_f|\leq L\leq I_i + I_f \tquad (\text{excepto} \ L=0)
\end{equation}
\begin{eqnarray*}
 	\Delta \pi = 0 \Longrightarrow EL \ \text{par} \quad \text{o} \quad M L  \ \textbf{impar} \\ 	
 	\Delta \pi = 2 \Longrightarrow EL \ \text{impar} \quad \textbf{o} \quad ML \ \text{par}
\end{eqnarray*} 
La excepción para $L=0$ se debe a que no existe radiación electromagnética multipolar de orden uno (radiación monopolar). En electromagnetismo clásico el momento monopolar eléctrico es simplemente una carga, que no varía con el tiempo y no puede dar lugar a un campo con radiación. Esto quiere decir que no es posible una transición $\gamma$ entre dos estados con espín cero\footnote{Analizando el proceso en términos de física de partículas, diríamos que una transición $\gamma$ $0^+ \rightarrow 0^+$ viola la ley de conservación del momento angular total.}. En estos casos la desexcitación nuclear tiene lugar a través de la conversión interna, un proceso que veremos más adelante en el capítulo.

\subsection{Conversión interna}

En la conversión interna el núcleo se desexcita expulsando un electrón de la corteza atómica. Es un proceso que compite con la emisión $\gamma$, y se representa por

\begin{equation}
	\ce{^A_Z X_N^*^{Ze} -> ^A_Z X_N^{(*)}^{(Z-1)e} + e^-}
\end{equation}
donde $(*)$ incida que puede estar o no excitado el núcleo resultante. Conviene en este punto recordar las difeerncias esenciales que existen entre este proceso y la desintegración $\beta$. En esta última etapa se \textit{crea} una partícula $\beta$, mientras que en la conversión interna se emite un electrón que previamente ya existía en la corteza atómica. \textbf{Es un único proceso, no tiene lugar en dos etapas}, como la emisión de un fotón por el núcleo y la absorción por efecto fotoeléctrico de este fotón por un electrón atómico. La energía nuclear se transfiere directamente al electrón atómico a través del campo coulombiano que los acopla. El balance energético vendrá dado por 

\begin{equation}
	Q = m_X + E^i + Zm_e - \sum B_i - \parentesis{m_X + E^i + (Z-1)m_e - \ccorchetes{\sum B_i - B_k}}
\end{equation}
Para que pueda ocurrir la transición debe verificarse qeu la transición de nivel tiene que ser mayor que la energía de ligadura del electrón $Q=\Delta E - B_k$. Si despreciamos la energía de retroceso del núcleo, tendremos que $Q=T_e$, y por tanto la energía cinética del electrón será la diferencia de energía entre los dos estados nucleares menos la energía de ligadura del electrón atómico, es decir:

\begin{equation}
	T_e = \Delta E - B
\end{equation}
Es evidente entonces que la energía cinética del electrón expulsado dependerá de la capa atómica concreta en que se encuentre, e incluso dependerá un poco del entorno químico en el que esté inmerso el propio átomo, porque puede alterar las energías de los orbitales atómicos. Al estar tan determinado por estas condiciones, \textit{solo se emitirán electrones muy concretos, con emisiones casi discretas (en el espectro de energías)}. Los estados nucleares excitados son habitualmente el resultado de desintegraciones $\beta$ previas, de tal manera que que el espectro energético de los electrones emitidos por fuentes $\gamma$ será la superposición de un espectro continuo y otro discreto. El primero originado por la desintegración $\beta$ y el segundo por los electrones de conversión interna.

Como ya hemos dicho, la conversión interna compite con la emisión de fotones. La intensidad relativa de las emisiones de los distintos tipos de electrones de conversión interna coincide con la probabilidad relativa de las transiciones, y estas probabilidades están íntimamente relacionadas con el orden multipolar y el carácter eléctrico o magnético de la emisión o emisiones $\gamma$ permitidas entre los dos estados nucleares. De hecho, uno de los métodos principales para determinar el carácter multipolar de la emisión $\gamma$ consiste en medir las intensidades relativas de los electrones de conversión interna. En algunos casos la conversión interna está mucho más favorecida (es mucho más probable) que la emisión $\gamma$, y en otras ocasiones al revés. En cualquier caso, para determinar la probabilidad global de transición ehmos de sumar la probabilidad de emisión $\gamma$ y de conversión interna. Tendremos por lo tanto:

\begin{eqnarray}
	\lambda_t = \lambda_\gamma + \lambda_e
\end{eqnarray}
Se define el coeficiente de conversión interna como

\begin{eqnarray}
	\alpha = \frac{\lambda_e}{\lambda_\gamma}
\end{eqnarray}
de tal manera que $\lambda_t = \lambda_\gamma (1+\alpha)$. Los coeficientes parciales de conversión interna se introducen de forma natural:

\begin{eqnarray}
	\alpha= \alpha_K + \alpha_L + \alpha_M + \ldots
\end{eqnarray}

El cálculo de los coeficientes de conversión interna es un asunto complicado. De forma aproximada, podemos decir que, puesto que ests coeficienes son cocientes de probabilidades de transición qeu están gobernadas por el mismo elemento de matriz, no dependerán de los detalles de la estrucutura nuclear sino más bien sólo del número atómico $Z$ de la energía $T_e$ y del orden multipoolar $ 2^L$. Un cálculo no relativista simplificado simplificado produce las siguientes estimaciones:

\begin{equation}
	\begin{split}
		\alpha (EL) \ \approx \ & \frac{Z^3\alpha^4}{n^3} \parentesis{\frac{L}{L+1}} \parentesis{\frac{2m_ec^2}{E}}^{L+5/2} \\
		\alpha (ML) \ \approx \ & \frac{Z^3\alpha^4}{n^3}  \parentesis{\frac{2m_ec^2}{E}}^{L+3/2} \\
	\end{split}	
\end{equation}
donde $Z$ es el número atómico del núcleo en el que tiene lugar la conversión interna\footnote{El núcleo hijo de la correspondiente desintegración $\beta$}, $\alpha$ es la constante de estructura fina, y $n$ es el número cuántico principal de la capa atómica. Estas fórmulas representan únicamente una estimación aproximada, porque el electrón emitido es relativista y el núcleo no puede ser considerado puntual\footnote{Los electrones orbitales tienen cierta probabilidad de penetrar dentro de la región nuclear.}. Existen tablas de coeficientes de conversión interna calculados de manera más rigurosa, pero en cualquier caso estas expresiones nos permiten deducir una serie de características o tendencias generales sobre el comportamiento de los coeficientes: 

\begin{itemize}
	\item Crecen con $Z^3$: la conversión interna es un proceso más importante para núcleos pesados que para núcleos ligeros.
	\item Decrecen con la energía rápidamente, al contrario de lo que sucede con la emisión $\gamma$, cuya probabilidad aumenta con ella.
	\item Crecen rápidamente con el orden multipolar. Para valores altos de $L$ la conversión interna peude ser mucho más probable que la emisión $\gamma$. 
	\item Decrecen con el número cuántico principal como $1/n^3$, de tal manera que esperamos, aproximadamente $\alpha_K/\alpha_L\approx 8$, valor que es comparable al resultado experimental en muchas ocasiones.
\end{itemize}

Resumiendo, esperamos coeficientes de conversión interna altos para electrones de la capa $K$, correspondientes a transiciones de multipolaridad alta en núcleos pesados. Estos coeficientes son muy distintos para transiciones eléctricas $EL$ y magnéticas $ML$. Por lo tanto, la medición de conversión interna es uno de los principales métodos que nos permiten determinar la paridad relativa de los estados nucleares. 

Recordamos que las transicioenes electromagnéticas con $L=0$ están prohibidas, por lo que la única forma de desexcitar el núcleo es mediante la emisión de electrones de la conversión interna. De hecho, la observación de electrones de conversión interna únicamente (sin fotones) es uno de los principales métodos experimentales usados para encontrar estados excitados $0^+$.

%\subsection{Espectroscopía $\gamma$}


\section{Teoría continua de la desintegración radiactiva.}

La \textbf{actividad} de una muestra de átomos radioactivos se define como el número de desintegraciones que tienen lugar en la unidad de tiempo. Esta definición que tiene sentido si la desintegración es un fenómeno que involucra individualmente\footnote{Algo que parece evidente hoy en día estaba lejos de serlo para los descubridores de la radiactividad. Durante varios años no se tuvo claro que los fenómenos radioactivos son procesos que afectan a cada átomo individualmente.} a cada átomo, no un fenómeno colectivo. Así definida, es obvio que la actividad es proporcional al tamaño de la muestra radioactiva. Rutherford y Soddy establecieron experimentalmente a principios del siglo XX que la actividad de una muestra disminuye de forma exponencial con el tiempo. Desde un punto de vista mecanocuántico esto tiene sencilla justificación, por que se deriva de la existencia de una probabilidad fundamental de transición por unidad de tiempo para cada núcleo, lo cual implica a su vez que la probabilidad de que un átomo se desintegra de forma independiente de los demás, el número de desintegraciones por unidad de tiempo sigue una distribución de Poisson\footnote{En realidad es una distribución binomial que en el límite de un número muy elevado de núcleos tiende a una distribución de Poisson. Supongamos que una muestra con $N$ núcleos radiactivos, cada uno con probabilidad $p$ de desintegrarse por unidad de tiempo $\Delta t$. Queremos calcular la probabilidad de que en el intervalo de tiempo $\Delta t$ se desintegren $n$ núcleos. Si las desintegraciones son realmente independientes entre sí podremos escribir:

\begin{eqnarray}
	P_n^N (\Delta T) = \frac{N!}{(N-n)!n!} p^n (1-p)^{N-n} = \binom{N}{n} p^n (1-p)^{N-n}
\end{eqnarray}

En el límite $N\rightarrow \infty$ manteniéndose $Np=m$ constante, la ecuación anterior tiende a la distribución de Poisson.}. Es decir,, la probabilidad de observar $n$ desintegraciones en n intervalo de tiempo $\Delta t$ viene dada por la siguiente expresión:

\begin{eqnarray}
	P(n,\Delta t) = \frac{m^n}{n!} \exp (-m)
\end{eqnarray}
donde $m$ es el número promedio de desintegraciones por segundo en el período $\Delta t$. La desviación de esta distribución es $\sigma = \sqrt{m}$.



\subsection{Sustancia sin ramificación} \label{Subsec:02-04-01}

De las condiciones discutidas en el párrafo anterior resulta sencillo deducir la evolución exponencial de la población de una sustancia radioactiva. Denotando por $N(t)$ el número de núcleos de la muestra en función del tiempo:

\begin{eqnarray}
	\derivadas{N}{t} = - \lambda N \Rightarrow N = N_0 e^{-\lambda t} 
\end{eqnarray}
donde $\lambda$ es la \textbf{constante de desintegración}. Definimos como \textbf{actividad} de una muestra como

\begin{eqnarray}
	A(t) = \lambda N(t)
\end{eqnarray}
Son datos de interés el valor medio del tiempo de desintegración $\tau$ llamado \textbf{vida media}, y el tiempo en que tarda una muestra en reducirse a la mitad $t_{1/2}$ llamado \textbf{período de semidesintegración}. Ambos se calculan como:

\begin{eqnarray}
	\tau = \frac{1}{\lambda} \tquad t_{1/2} = \frac{\ln 2}{\lambda}
\end{eqnarray}
En el sistema internacional la unidad de medida de la actividad es el \textbf{Becquerelio (Bq)}. El \textbf{Curio (Ci)} también se usa. Originalmente se definió como la actividad de un gramo de radio $^{226}$Ra pero hoy en día se define como $1 \unit{Ci} = 3.7 \times 10^{10} \unit{Bq}$. La actividad de una muestra radioactiva indica únicamente el número de desintegraciones que se producen por segundo. Las fuentes más radiactivas tienen una actividad de microcurios $\mu \textbf{Ci}$. Las ecuaciones para una población de núcleos radioactivos que se desintegran de un modo son muy sencillas. Suponiendo que la desintegración es $X\rightarrow Y$, tenemos que las ecuaciones diferenciales que rigen la desintegración:

\begin{eqnarray}
	\derivadas{N_X(t)}{t} = - \lambda N_x \tquad \derivadas{N_Y(t)}{t} = \lambda N_x
\end{eqnarray}
de tal modo que su solución es:

\begin{eqnarray}
	N_X (t) & = & N_X(0) e^{-\lambda t} \\
	N_Y (t) & = & N_X(0) \parentesis{1-e^{-\lambda t}} + N_Y(0)
\end{eqnarray}
donde suponemos que $Y$ es estable, esto es, no se desintegra en otra sustancia.

\subsection{Ramificación. Constantes de desintegración parciales.}

Supongamos que los núcleos de una muestra pueden desintegrarse siguiendo dos mecanismos o modos alternativos (por ejemplo la desintegración $\alpha$ y $\beta$). A cada uno de estos modos le corresponde una constante de desintegración diferente, de tal modo que la ecuación diferencial será:

\begin{eqnarray}
	\derivadas{N(t)}{t} = - \lambda_a N(t) - \lambda_b N(t) = \lambda N(t)
\end{eqnarray}
donde $\lambda = \lambda_a + \lambda_b$. 

\subsection{Varias sustancias. Cadenas radiactivas naturales.}

Es frecuente que el producto de desintegración de un núcleo radioactivo sea también radiactivo. Supongamos que en el instante inicial $t=0$ tenemos $k$ sustancias que forman una cadena radiactiva. Para calcular el número de átomos en un instante posterior hemos de resolver el siguiente sistema de ecuaciones diferenciales:

\begin{equation}
	\begin{array}{ccl}
		\dfrac{\D N_1}{\D t} & = & - \lambda_1 N_1 \vspace{1.5mm} \\ \vspace{1.5mm}
		\dfrac{\D N_2}{\D t} & = & - \lambda_2 N_2 + \lambda_1 N_1 \\
		\dfrac{\D N_3}{\D t} & = & - \lambda_3 N_3 + \lambda_2 N_2\\\vspace{1.5mm}
					\ldots & \ldots &  \ldots \\					 
		\dfrac{\D N_k}{\D t} & = & - \lambda_k N_k  + \lambda_{k-1} N_{k-1}
	\end{array}
\end{equation}
Lógicamente en función de las condiciones iniciales obtendremos una solución u otra. Una de las mas preeminentes es la de suponer que tan solo la sustancia 1 se halla presente al principio. La solución general con estas condiciones viene dada por las \textbf{ecuaciones de Bateman}. Por ejemplo para 3 sustancias, tendríamos que:

\begin{equation}
	\begin{array}{ccl} \vspace{1.5mm}
		N_1(t) & = & N_1(0) e^{-\lambda_1 t}\\ \vspace{1.5mm} 
		N_2(t) & = & N_1(0) \frac{\lambda_1}{\lambda_2 - \lambda_1} \parentesis{e^{-\lambda_1 t } - e^{\lambda_2}} \\\vspace{1.5mm}
		N_3(t) & = & N_1(0) \lambda_1 \lambda_2 \parentesis{ \frac{e^{-\lambda_1 t }}{(\lambda_2-\lambda_1)(\lambda_3-\lambda_1)}  -\frac{e^{-\lambda_2 t }}{(\lambda_1-\lambda_2)(\lambda_3-\lambda_2)} -\frac{e^{-\lambda_3 t }}{(\lambda_1-\lambda_3)(\lambda_2-\lambda_3)} }
	\end{array}
\end{equation}
Si por ejemplo el núcleo 3 fuera estable, de tal modo que la cadena fuera $1\rightarrow 2\rightarrow 3$, con suponer que $\lambda_3=0$ en la ecuación anterior sería suficiente:

\begin{equation}
	\begin{array}{ccl} \vspace{1.5mm}
		N_1(t) & = & N_1(0) e^{-\lambda_1 t}\\ \vspace{1.5mm} 
		N_2(t) & = & N_1(0) \dfrac{\lambda_1}{\lambda_2 - \lambda_1} \parentesis{e^{-\lambda_1 t } - e^{\lambda_2}} + N_2(0)e^{-\lambda_2 t} \\\vspace{3mm}
		N_3(t) & = & N_1(0)  \ccorchetes{1 + \frac{1}{\lambda_2-\lambda_1} \parentesis{\lambda_1 e^{-\lambda t} - \lambda_2 e^{-\lambda_1 t}} } + N_3 (0)
	\end{array}
\end{equation}






