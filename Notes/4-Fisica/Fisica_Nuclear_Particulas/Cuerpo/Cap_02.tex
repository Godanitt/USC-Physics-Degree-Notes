\chapter{Inestabilidad nuclear}

\section{Desintegración $\alpha$}

La desintegración $\alpha$ produce lo que originalmente se llamó rayos $\alpha$, identificados en su época como la radiación menos penetrante emitida por los elementos radioactivos naturales. \\ % Falta texto

La desintegración por emisión alfa produce un desplazamiento hacia la izquierda de dos posiciones en la tabla periódica, y reduce el número másico en 4 unidades: $\Delta Z = -2$, $\Delta A = - 4$. Esquemáticamente escribiríamos

\begin{equation}
    ^A_Z  X_N \longrightarrow ^{A-4}_{Z-2} Y_{N-2} + ^4_2\He_2
\end{equation}

% falta texto

La desintegración $\alpha$ es un fenómeno que en esencia se debe a la repulsión coulombiana de los protones en un núcleo, que crece a un ritmo más rápido (va como $Z^2$) que la energía nuclear\footnote{Recuérdese que la energía de ligadura por nucleón es aproximadamente constante $B/A \sim $cte.} (que va como $A$).  % Falta texto

\subsection{Energética de la desintegración $\alpha$}

En la desintegración $\alpha$ intervienen la interacción electromagnética y la fuerte.  Entre otras cosas que veremos más tarde, en este proceso se conserva la energía, el momento lineal, el momento angular total y la paridad. Si suponemos que el átomo está inicialmente en reposo, la energía del sistema en el estado inicial es simplemente la energía de la masa en reposo de este átomo. En el estado final tenemos una partícula $\alpha$ y un núcleo hijo con cierta energía cinética. Usando masas nucleares podríamos entonces escribir las ecuaciones de balance energético como:

\begin{equation}
    m_X c^2 = m_Y c^2 + T_Y + m_\alpha c^2 + T_\alpha \\
\end{equation}    
\begin{equation}
    m_X c^2 - m_y c^2 - m_\alpha  c^2  = T_y + T_\alpha = Q \\
\end{equation}    
\begin{equation}
    Q = (m_X - m_Y - m_\alpha)c^2
\end{equation}    
Este proceso está energéticamente permitido sólo si $Q>0$. Aunque la expresión anterior está escrito en términos de las masas nucleares, podemos calcular $Q$ usando las masas atómomicas porque la masa electronesse cancela en la operación (en las desintegraciones $\alpha$ no se eliminan ni generan electrones). Despreciando las energías de ligadura electrónicas:
\begin{eqnarray}
Q/c^2 = M(^A_Z X_N) - M(^{A-4}_{Z-2}Y_{N-2}) - M(^4_2\He)
\end{eqnarray}
Si el núcleo inicial estaba en reposo, los momentos lineales de los productos deben ser iguales en magnitud y opuestos en sentido

\begin{equation}
    p_\alpha = p_Y
\end{equation}

\Revisar

\subsection{Sistemática de la desintegración $\alpha$. Regla de Geiger-Nuttal}

Geiger y Nutall observaron ya en 1911 una relación inversamente proporcional entre el logaritmo del período de semidesintegración del núcleo emisor y la raíz cuadrada de las partículas $\alpha$ emitida., esto es:

\begin{equation}
\ln (\tau_{1/2}) = k_1 + \frac{k_2}{\sqrt{Q}}
\end{equation}

\subsection{Tratamiento de Gamow, Gurney y Condon}

En 1928 Gamow, Gurney y Condon aplicaron casi simultáneamente los principios de la nueva mecánica cuántica al estudio de la desitegración $\alpha$. La idea esencial consiste en considerar la partícula $\alpha$ \textit{preformada} en el potencial creado por el núcleo hijo, que puede aproximarse a un pozo esférico profundo\footnote{En el capítulo que trata de la estructura nuclear (capítulo \ref{Ch:04}) veremos con algo más de detalle por qué el potencial nuclear se puede aproximar a un pozo de potencial, y en el capítulo sobre la interacción nucleón-nucleón veremos cuantitativamente la profundidad estimada del pozo.} con una frontera en forma de barrera coulombiana. La partícula $\alpha$ tendrá cierta probabilidad de cruzar la barrera por efecto túnel, que puede calcularse sin muchas dificultados, al menos aproximadamente. Este modelo proporciona resultados numéricos razonables que demuestran una compresión semicuantitativa del fenómeno de la desintegración $\alpha$, pero no significa que la \textit{imagen} de una partícula $\alpha$ \textit{preformada} en el interior del núcleo sea un buen reflejo de la realidad, sino más bien que en ciertas condiciones se comporta aproximadamente como tal.

Vamos ahora a realizar un cálculo para estimar la probabilidad de desintegración $\alpha$, y para ello usaremos un modelo semiclásico en una dimensión. El potencial coulombiano se corta a una distancia $a$ que se puede considerar como la suma de los radios del núcleo hijo y de la partícula $\alpha$. En la región interior $(r<a)$, la partícula $\alpha$ se mueve con una energía cinética dada por $Q+|V_0|$. La región $a<r<b$ forma una barrera del potencial. 

En el contexto de nuestro sencillo modelo anterior, la constante de desintegración vendrá dada por $\lambda = fP$, donde $f$ es la \textit{frecuencia con la que la partícula incide sobre la barrera} y $P$ es la \textit{probabilidad de que la cruce}. No es complicado calcular el coeficiente de trasmisión para esa simple barrera coulombiana. Recordemos que el coeficiente de trasmisión viene dado a través de una barrera arbitraria en una dimensión viene dado aproximadamente por

\begin{equation}
    T = \exp \parentesis{-\frac{2\sqrt{2m}}{\hbar } \int_a^b \sqrt{V(r)-Q}\D r}
\end{equation}
donde la integración se realiza entre los límites que marcan la intersección de la recta $V(r) = Q$ con el pozo de potencial nuclear por la izquierda y la cola de la barrera coulombiana por la derecha. Se suele denominar \textbf{factor de Gamov} a 

\begin{equation}
    G = \frac{\sqrt{2m}}{\hbar} \int_a^b \sqrt{V(r)-Q}
\end{equation}
de manera que el cociente de trasmisión es

\begin{equation}
    T = e^{-2G}
\end{equation}
Escribimos ahora para $V(r)$ la energía coulombiana de la partícula alfa 

\begin{equation}
    V(r) = \frac{2(Z-2)e^2}{4\pi \epsilon_0 r}
\end{equation}
y obtenemos la siguiente integral

\begin{equation}
    G = \sqrt{\frac{2\mu}{\hbar^2Q}} \frac{2(Z-2)e^2}{4 \pi \epsilon_0} \ccorchetes{\arccos \sqrt{\frac{a}{b}}- \sqrt{\frac{a}{b}\parentesis{1-\frac{a}{b}}}}
\end{equation}
donde $2e$ es la carga de la partícula $\alpha$, $(Z-2)e$ la carga del núcleo hijo y $\mu$ la masa reducida del sistema. Denotando por $B$ la altura de la barrera coulombiana en el punto $r=a$, tenemos 

\begin{equation}
    B= V(r=a) = \frac{2(Z-2)e^2}{4\pi \epsilon_0 a} \tquad Q=\frac{2(Z-2)e^2}{4 \pi \epsilon_0 b}
\end{equation}
Porque $Q$ es precisamente el valor que $V(r)$ toma en el radio $R=b$. De aquí deducimos que $a/b=Q/B$ y puesto que $Q/B\ll 1$ hacemos la siguiente aproximación 

\begin{equation}
    \arccos \sqrt{\frac{a}{b}} \approx \frac{\pi}{2} - \sqrt{\frac{a}{b}}
\end{equation}
con lo cual tenemos que 

\begin{equation}
    G \approx  \sqrt{\frac{2\mu}{\hbar^2Q}} \frac{2(Z-2)e^2}{4 \pi \epsilon_0} \parentesis{\frac{\pi}{2} -2 \sqrt{\frac{a}{b}}}
\end{equation}
Ahora todo lo que tenemos que hacer es multiplicar la probabilidad de penetración ya calculada por una frecuencia $f$ que exprese, de una manera simple, el número de impactos o intentos que la partícula $\alpha$ hace contra la barrera. Esta frecuencia es del orden de $f \sim v/a$, donde $v$ es su velocidad anterior del núcleo, que podemos expresar como $v=\sqrt{2(Q+|V_0|)/\mu}$. Esto nos lleva a la siguiente expresión para la constante de desintegración $\alpha$:

\begin{eqnarray}
    \lambda & = & \frac{1}{a} \sqrt{\frac{2(Q+|V_0|)}{\mu}} \exp \ccorchetes{-2  \sqrt{\frac{2 \mu}{\hbar^2 Q}} \frac{2(Z-2)e^2}{4 \pi \epsilon_0} \parentesis{\frac{\pi}{2}-2\sqrt{\frac{a}{b}}}} \\ \\
            & = & \frac{1}{a} \sqrt{\frac{2(Q+|V_0|)}{\mu}}  \exp \ccorchetes{4 \alpha c \sqrt{\frac{2 \mu}{Q}} (Z-2) \parentesis{\frac{\pi}{2}-2\sqrt{\frac{a}{b}}}}
\end{eqnarray}
Donde $\alpha=e^2/(4 \pi \epsilon_0 \hbar c)$ es la constante de estructura fina. El período de semidesintegración o semivida sería:

\begin{equation}
    \tau_{1/2} = a (\ln (2)) \sqrt{\frac{2\mu}{\hbar^2Q}} \exp \ccorchetes{4 \alpha c \sqrt{\frac{2 \mu}{Q}} (Z-2) \parentesis{\frac{\pi}{2}-2\sqrt{\frac{a}{b}}}}
\end{equation}
De esta ecuación se deduce\footnote{Aunque es cierto que $Q$ también aparece en la raíz que multiplica la exponencial, la dependencia más fuerte de $\tau_{1/2}$ con $Q$ proviene de la exponencial en sí misma. El factor multiplicativo, que es esencialmente la velocidad de la partícula $\alpha$, tiene una variación con $Q$ despreciable comparada con la variación generada por la exponencial.} que el logaritmo de $\tau_{1/2}$ tiene una relación lineal inversa con la raíz cuadrada de la energía cinética de la partícula $\alpha$:

\begin{equation}
    \log \tau_{1/2} = k_1 + \frac{k_2}{\sqrt{Q}}
\end{equation}
donde $k_1$ y $k_2$ son constantes. Esta es la \textbf{ley o regla de Geiger y Nuttal} encontraron empíricamente en torno a 1911. % falta texto

Vemos que la fórmula deducida para $\tau_{1/2}$ mediante este sencillo modelo proporciona valores que están más o menos dentro del mismo orden de magnitud que las medidas experimentales, lo cual es un logro importante si se tiene en cuenta que éstas últimas abarcan más de 22 órdenes de magnitud y que hemos simplificado considerablemente el problema: no hemos tenido en cuenta la función de ondas inicial y final del núcleo (regla de oro de Fermi) ¡, ni tampoco el momento angular de la partícula $\alpha$. La forma y radio del núcleo tiene también una influencia importante en el valor numérico final\footnote{Por ejemplo, pasar de un radio nuclear medio de 1,25$A^{1/3}$ fm a 1,20$A^{1/3}$ fm implica introducir un cambio de hasta un factor 5 en $\tau_{1/2}$ respecto al radio nuclear se puede usar incluso como un método para estimar el radio nuclear, o más bien, el radio de un núcleo más el de la partícula alfa.}.

% falta texto

\subsection{Momento angular y paridad en la desintegración $\alpha$}

En una desintegración $\alpha$ han de conservarse el momento angular total y la paridad. En cuanto al primerm tenemos que $\In_i = \In_f + \In_\alpha + \lnn_\alpha$, donde $\lnn_\alpha$ es el momento angular orbital de la partícula $\alpha$ relativo al núcleo emisor\footnote{Consideramos que el hijo se queda aproximadamente en reposo.} y $\In_\alpha$ su espín nuclear. Ahora bien, el núcleo de Helio tiene los nucleons apareados de manera que su espín es cero, por lo tanto $\In_\alpha  =0$ y $\In_i = \In_f + \lnn_\alpha$. Con lo cual el momento angular orbital que se lleva la partícula $\alpha$ en la Desintegración debe cumplir la \textbf{regla del momento angular}

\begin{equation}
    |I_i-I_f|\leq \lnn_\alpha \leq I_i + I_f
\end{equation}
Por otro lado, la ley de conservación de la paridad exige $\pi_i=\pi_f \times \pi_\alpha \times (_1)^{\ell_\alpha}$, donde $\pi_i$ y $\pi_f$ son las paridades del núcleo inicial (padre) y del núcleo final (hijo), y $\pi_\alpha$ es la paridad de la partícula $\alpha$, que resulta ser positiva\footnote{El primer estado excitado del núcleo $^4_2$He tiene una energía de 20,1 MeV y también $I^\pi = 0^+$.}, por lo tanto tenemos que $pi_i = \pi_f (-1)^{l_\alpha}$. Concluimos, finalmente, que si la paridad nuclear inicial y final es la misma, entonces $\ell_\alpha$ tiene que ser par, mientras que en caso contrario tiene que ser impar:

\begin{equation}
    \pi_i = \pi_f \Longrightarrow \ell_\alpha \ \text{par}
\end{equation}
\begin{equation*}
    \pi_i = - \pi_f \Longrightarrow \ell_\alpha \ \text{impar}
\end{equation*}


% Falta texto

\section{Desintegración $\beta$}

La desintegración $\beta^-$ fue descubierta ya en 1896 por Becquerel. En 1932 se descubrió el positrón en el estudio de los rayos cósmicos, y más tarde (1934) se produjeron artificialmente elementos que se desintegraban emitiendo positrones, poniendo así de manifiesto la desintegración $\beta^+$. La captura electrónica, esto es, la captura de un electrón orbital por parte del núcleo fue descubierta por Álvarez en 1938. Los rayos $\beta^-$ tiene carga negativa y un poder de penetración de alrededor de 1mm. en plomo. Durante algún tiempo hubo confusión entre estos dos electrones emitidos en la desintegración $\beta$ y los electrones emitidos en la \textbf{conversión interna} de los núcleos. En física nuclear se suelen usar los símbolos $\beta^-$ y $\beta^+$ para designar estas radiaciones. La desintegración por emisión $\beta^-$ ($\beta^+$) produce un desplazamiento hacia la derecha (izquierda) de una posición en la tabla periódica, pero no cambia esencialmente la masa: $\Delta Z= \pm 1$ , $\Delta A = 0$. Responsable de este fenómeno es la interacción débil.

El \textit{reemplazo} de una neutrón por un protón, o viceversa, tiene lugar en los núcleos a través de los siguientes procesos:

\begin{equation}
	n \longrightarrow p + e^- + \bar{\nu}_e  \tquad \text{desintegración} \ \beta^-
\end{equation}
\begin{equation}
	p \longrightarrow n + e^+ + {\nu}_e \tquad \text{desintegración} \ \beta^+
\end{equation}
\begin{equation}
	p + e^- \longrightarrow n + {\nu}_e \tquad \text{captura electrónica (CE} \ \epsilon)
\end{equation}
siendo el primero de ellos energéticamente imposible para los protones libres o átomos de hidrógeno. 

\subsection{Energética de la desintegración $\beta$.}

Experimentalmente se observa que la energía cinética de las partículas $\beta$ forma una distribución continua, en marcado contraste con el carácter monoenergética de las partículas $\alpha$ emitidas rediactivamente. Recordemos que la energía cinética de la partícula $\alpha$ es igual a la diferencia de energía de ligadura del núcleo padre y del núcleo hijo (excepto por pequeñas correciones debidas al retroceso del núcleo emisor). En un principio no se conocía la existencia de neutrinos y se creía que la desintegración $\beta$ sólo intervenían los núcleos padre e hijo, y el electrón emitido. No obstante, el hecho de que el proceso fuese continuo indicaba que en el proceso deben intervenir más de dos partículas, como mínimo tres, que se repartan estadísticamente la energía cinética liberada por la desintegración. 

Esquematizamos el proceso de desintegración $\beta^-$ del siguiente modo:
\begin{equation}
	^A_Z X_N \longrightarrow _{Z_1}^A Y_{N-1} + e^- \bar{\nu}_e \quad \text{con} \quad Q_{\beta^-} = (m_X-m_Y-m_e)c^2 \ (\text{masas nucleares})
\end{equation}
Recordemos que

\begin{equation*}
	M(Z,A) = m_N + Zm_e - \frac{1}{c^2} \sum_{i=1}^Z B_i \tquad
	m(Z,A) = M_N - Zm_e + \frac{1}{c^2} \sum_{i=1}^Z B_i
\end{equation*}
donde las $B_i$ son las energías de ligadura de los electrones atómicos. Por lo tanto

\begin{eqnarray}
	\frac{Q_{\beta^-}}{c^2} = \ccorchetes{M(^A_ZX_N) - Zm_e} - \ccorchetes{M(_{Z+1}^A Y)-(Z+1)m_e}  - m_e + \ccorchetes{\sum_{i=1}^{Z} B_i - \sum_{i=1}^{Z+1} B_i}
\end{eqnarray}
Vemos que las masas de los electrones se cancelan en la ecuación anterior. Despreciando las pequeñas diferencias de ligadura electrónicas, expresamos la energía de la desintegración en términos de masas atómicas de la siguiente manera:

\begin{eqnarray}
	Q_ {\beta^-} = \ccorchetes{M(X)-M(Y)} c^2 \tquad Q_{\beta^-} \approx T_e + T_\nu
\end{eqnarray}
La aproximación de la segunda expresión consiste en despreciar la energía de retroceso del núcleo emisor. Está claro entonces que

\begin{eqnarray}
	(T_e)_{\max} = (T_\nu)_{\min} = Q_{\beta^-}
\end{eqnarray}
En términos de masas atómicas:
\begin{eqnarray}
	\frac{Q_{\beta^+}}{c^2} = \ccorchetes{M(^A_ZX_N) - Zm_e} - \ccorchetes{M(_{Z-1}^A Y_ {N+1})-(Z-1)m_e}  - m_e + \ccorchetes{\sum_{i=1}^{Z} B_i - \sum_{i=1}^{Z-1} B_i}
\end{eqnarray}
\begin{eqnarray}
	Q_ {\beta^-} = \ccorchetes{M(X)-M(Y)-2m_e} c^2 
\end{eqnarray}
Para la captura electrónica (denotada por C.E. o $\varepsilon$) tenemos:

\begin{equation}
	^A_ZX_N + e^- \rightarrow ^A_{Z-1} Y_{N+1} + \nu_e \tquad Q_{\textbf{CE}} = Q_\varepsilon = M(X) c^2 - \ccorchetes{M(Y)c^2+ B_n}
\end{equation}
donde $B_n$ es la energía de ligadura del electrón correspondiente a la capa $n$ (K,L,M...) que coincide con los rayos X (uno o varios) que se emiten cuando el resto de los electrones atómicos bajan en cascado a ocupar la vacante dejada por el electrón capturado\footnote{Justo a continuación de la captura, la capa electrónica del núcleo hijo queda altamente excitada porque tiene una vacante en uno de sus orbitales bajos. Al ser ocupada la vacante se emiten rayos X característicos cuya energía coincide con $B_n$, lo cual implica que la masa o contenido energética del átomo hijo justo después de la captura excede en $B_n$ a la masa atómica del átomo en su estado fundamental. Debe tenerse en cuenta que la corteza electrónica del átomo también puede desexcitarse mediante la emisión de electrones Auger. Este fenómeno compite con la emisión de los rayos X característicos, y es más probable en átomos ligeros.}. La energía de esos rayos X puede parametrizarse de acuerdo con la ley de Moseley:

\begin{eqnarray}
	\sqrt{\frac{E}{hc}} = A_n (Z-B_n)
\end{eqnarray}	
donde las $A_n$ y $B_n$ son constantes que dependen de la capa de la que haya sido capturado (o expulsado) el electrón. Esto es, tendríamos $A_K,A_L,A_M,B_K,B_L,B_M$ etc.

Obsérvese que si la desintegración $\beta^+$ es energéticamente posible entonces la captura electrónica también, pero el recíproco no es cierto. 

\subsection{Probabilidad de transición. Regla de oro de Fermi.}

Antes de profundizar en el estudio de la teoría de Fermi de la desintegración $\beta$ detengámonos un momento a examinar las transiciones entre estados energéticos de un sistema cuántico. Recordemos que en un estado estrictamente estacionario la densidad de probabilidad no cambia con el tiempo $(|\Psi|^2 = \cte)$. Además, la energía que corresponde a ese estado está perfectamente definida en el sentido de que tiene dispersión nula:

\begin{eqnarray}
	(\Delta E)^2 = \langle E^2 \rangle - \langle E \rangle^2 = 0 \tquad \Delta E \Delta t \geq \hbar / 2 \Rightarrow \Delta t \sim \infty
\end{eqnarray}
Esto quiere decir que si un sistema (átomo, núcleo, etc.) se encontrase en un estado estacionario sería imposible que pudiese realizar transiciones fuera de él. % falta texto

% fatla texto

Un estado que se aparta un poco de ser estacionario ya no tiene la energía perfectamente definida, tendrá una cierta dispersión $\Delta E \neq 0$ que se llama \textit{anchura del estado} y se suele representar por $\Gamma$. Es posible relacionar, mediante el principio de incertidumbre, la vida media $\tau$ de ese estado haciéndolo corresponder con el intervalo de tiempo $\Delta t$ que uno tendría disponible para realizar la medida de la energía del estado. Esto implica que $\tau=\hbar \Gamma$. La probabildiad de desintegración o transición $\lambda$, es el inverso de la vida medio,

\begin{equation}
	\lambda = \frac{1}{\tau}
\end{equation}
Es posible realizar en mecánica cuática un cálculo aproximado de la probabilidad de transición por unidad de tiempo, $\gamma$. El resultado, conocido como \textbf{regla de oro de Fermi}, puede escribirse como

\begin{equation}
	\lambda = \frac{2\pi}{\hbar} |V'_{fi}|^2 \rho (E_f)	
\end{equation}
donde el término $V_{fi}'$ es el valor esperado del débil potencial perturbador entre los estados final e inicial de transición, también conocido como elemento de matriz:

\begin{equation}
	V_{fi}'= \int \Psi_f^* V' \Psi_i \D v
\end{equation}
El término $\rho(E_f)$ es la densidad de estados finales, es decir, el número de estados por unidad de energía. Si el estado final es un estado aislado con energía en una estrecha distribución alrededor de $E_f$ entonces la probabilidad de transición será mucho menor que si tenemos un conjunto de muchos estados finales densamente concentrados en un pequeño intervalo alrededor de $E_f$.

\Revisar

% Falta mucho texto importante -> ¿Como se llegan a las transiciones prhibidas?

\subsection{Clasificación de las desintegraciones. Reglas de selección del momento angular y de la paridad.}

La clasificación de las desintegraciones $\beta$, de manera breve, se puede ver en la tabla \ref{Tab:02-02-01}, con las reglas de selección del momento angular y de la paridad.

\begin{table}[h!]\centering
	\begin{tabular}{|l||c|c|}\hline
		Tipo de desintegración & $\Delta \pi$ & $\Delta I$ \\ \hline \hline
		Permitidas & 0 & 0,1 \\ \hline 
		Prohibidas (1º orden) & $\pm$2 & 0,1,2 \\ \hline
		Prohibidas (2º orden) & 0 & 2,3 \\ 	\hline	
		Prohibidas (3º orden) & $\pm$2 & 2,3 \\ \hline	
		Prohibidas (4º orden) & 0 & 4,5 \\ 	\hline	
		Prohibidas (5º orden) & $\pm$ 2 & 5,6  \\		 \hline
	\end{tabular}
	\caption{Reglas de selección del momento angular y de la paridad de las desintegraciones $\beta$.}
	\label{Tab:02-02-01}
\end{table}


\subsubsection{Desintegraciones permitidas (\textit{allow deecays})}
\addcontentsline{toc}{subsubsection}{Desintegraciones permitidas}


Recordemos que en la denominada aproximación permitida hemos sustituido las funciones de onda del electrón y del neutrino, por su valor en el origen, esto es, consideremos que fueron \textit{creados} en $r=0$. En este caso no puede llevarse ningún momento angular orbital, y el único cambio posible de espín nuclear debe provenir del espín del electrón y del neutrino, que ambos son fermiones de espín $s_e=s_v=1/2$. En la aproximación permitida ($\ell=0$), podemos tener entonces los siguientes casos

\begin{itemize}
	\item El espín del sistema de dos leptones ($e$ y $\nu$) es $S=0$ (configuración singlete). En este caso no puede haber cambio en el espín nuclear: $\Delta I=|I_i-I_f|=0$. Se les llama \textbf{Fermi decays.}
	\item El espín del sistema de dos leptones ($e$ y $\nu$) es $S=1$ (configuración triplete). En este caso $\In_i$ e $\In_f$ están acoplados a través de un vector de longitud unidad: $\In_f + \unovec = \In_i$, $I_i=|I_f-1|$,$I_f$,$|I_f+1|$, lo cual quiere $\Delta I = 0,1$. Se les llama \textbf{Gamow-Teller decays}. Obsérvese que si $\In_i=\In_f=0$ tenemos también $\Delta I=0$, pero no es posible que exista una transición Gamow-Teller en este caso, porque los leptones no pueden ser emitidas en configuración tripelete ($0\neq 0+1$).
\end{itemize}
Puesto que la paridad asociado al momento angular orbital $l$ es $(-1)^l$, se deduce que hay cambio de paridad en estas transiciones $(l=0)$. Las \textbf{reglas de selección} para las desintegraciones $\beta$ \textbf{permitidas} son entonces:

\begin{eqnarray}
	\Delta I =0,1 \tquad \Delta \pi = 0 \quad (\text{no hay cambio de paridad})
\end{eqnarray}
Ejemplos de transiciones $\beta$ permitidas:

\begin{itemize}
	\item $\ce{^14 O -> ^14 N^*} \ (0^+\rightarrow0^+)$. Es una transición Fermi (F) pura, no puede haber contribución de la transición Gamow-Teller (GT) porque $I_i=I_f=0$.
	\item $\ce{^6 He -> ^6  Li}  \ (0^+\rightarrow1^+$). Es una transición GT pura.
	\item $n\longrightarrow p \ \parentesis{\frac{1}{2}^+\rightarrow \frac{1}{2}^+}$. Tenemos $\Delta I = 0$ y se trata de una transición mezcla (F) y (GT), porque es posible que los dos leptones se encuentran tanto la configuración triplete del espín ($S=1$)como en la singlete ($S=0$)\footnote{Las dos combinaciones que satisfacen la conservación del momento angular serían $\frac{1}{2}\otimes 1=\frac{1}{2}\oplus \frac{3}{2}$, para el estado singlete: $\frac{1}{2}\otimes 0=\frac{1}{2}$}. Las proporciones exactas en que contribuyen cada una de estas transiciones al proceso global dependen de las funciones de onda nuclear inicial y final. Se suele definir el cociente entre las amplitudes de Fermi y de Gamow-Teller de la siguiente manera:
	\begin{eqnarray}
		y=\frac{g_FM_F}{g_{GT}M_{GT}}
	\end{eqnarray}
	donde $M_F$ y $M_{GT}$ son los \textit{elementos de matriz nucleares} de Fermi y de Gamow-Teller. En la constante de transición global tendríamos que sustituir $g^2 |M_{fi}|^2$ por ($g_F^2|M_F|^2+g_{GT}^2|M_{GT}|^2$). Suponemos que $g_F$ es idéntica a la $g$ deducida para las transiciones superpermitidas de Fermi ($0^+\rightarrow 0^+$). Para la desintegra del neutrón el elemento de matriz nuclear es simplemente $|M_F|=1$. Puesto que la constante de transición (decay rate) es proporcional a $g_F^2 |M_F|^2 (1+y^{-2})$, la tasa de desintegración del neutrón permite el cálculo del cociente $y$, que arroja un valor de $y=0.467\pm 0.003$ Es decir, la desintegración del neutrón libre tiene lugar en un 82\% de las ocasiones según un proceso de Gamow-Teller, y en el $18\%$ restante según un proceso de Fermi\footnote{De acuerdo con \begin{equation*} \frac{g_F^2 |M_F|^2}{g_F^2 |M_F|^2 (1-y^{-2})} = \frac{1}{1+y^{-2}} = 0.179  \end{equation*}}
	
\end{itemize}
En general el cálculo de $M_F$ y $M_{GT}$ es complicado, pero en el caso especial de núcleos espejo resulta particularmente simple porque las funciones de onda inicial y final son las mismas (excepto por pequeñas correcciones coulombianas). Un protón se convierte en neutrón más el positrón y el neutrino, y el cociente de amplitudes (F) y (GT) es similar al de la desintegración del neutrón libre antes vista. 

\subsubsection{Desintegraciones prohibidas (\textit{forbidden decays})}

Se denominan \textit{\textbf{fist forbidden decays}} o \textit{\textbf{desintegraciones prohibidas de primer orden}} a aquellas transiciones $\beta$ en la que los leptones se llevan una unidad de momento angular orbital $\ell=1$. El término \textit{desintegración prohibida} es realmente desafortunado, ya que aunque son transiciones poco probables, no son imposibles. Al igual que las desintegraciones permitidas, las dividimos en transiciones de Fermi (F) y de Gamow-Teller (GT), dependiendo de que los leptones emitidos se encuentren en la configuración singlete de espín ($S=0$) o la triplete ($S=1$), respectivamente.



\begin{itemize}
	\item Para las \textit{desintegraciones Fermi prohibidas de primer orden} hemos de acoplar $S=0$ con $\ell=1$, lo cual proporciona $I_i = I_f\oplus 1$ ($I_i=|I_f-1|,I_f,|I_f+1|$), y por lo tanto $\Delta I = 0,1$ (pero no podemos tener $0\rightarrow 0$, porque entonces los leptones no se podrían llevar una unidad de momento angular orbital y encontrarase en un estado singlete de espín).
	
	\item Para las \textit{desintegraciones de Gamow-Teller de primer orden} hemos de acoplar $S=0$ con $\ell=1$, lo cual proporciona $I_i=I_f\oplus 2$ ($I_f\in \{ |I_f-2|,|I_f-1|,I_f,|I_f+2|,|I_f+2| \}$) y por tanto $\Delta =0,1,2$. 
\end{itemize}

Por otro lado, si $\ell=1$, habrá un cambio de paridad en la transición. Tenemos, por consiguiente, las siguientes reglas de selección para as transiciones prohibidas de primer orden:

\begin{eqnarray}
	\Delta I = 0,1,2 \tquad \Delta \pi = \pm 2 \quad (\text{{\footnotesize  hay cambio de paridad}})
\end{eqnarray}
Algunos ejemplos de transiciones prohibidas de primer orden son:

\begin{itemize}
	\item $\ce{^17_7 N -> ^17_8 O} \quad \parentesis{\frac{1}{2}^- \rightarrow \frac{1}{2}^+}$
	\item $\ce{^76_35 Br -> ^76_34 Se}  \quad \parentesis{1^- \rightarrow 0^+}$
	\item $\ce{^122_51 Sb -> ^122_50 Sn^*}  \quad \parentesis{\frac{1}{2}^- \rightarrow \frac{1}{2}^+}$
\end{itemize}
Las desintegraciones $\beta$ \textbf{prohibdas de segundo orden} son aquellas en las que los leptones se lleva unidades de momento angular $\ell=2$, con lo cual no hay cambio de paridad. El proceso para obtener las reglas de selección es el mismo que el de las anteriores desintegraciones: las subdividimos en Fermi o Gamow-Teller, de tal manera que combinando tenemos que $\Delta I = 0,1,2,3$ (pero no siempre, porque las transiciones $0\rightarrow 0$ u $\frac{1}{2} \rightarrow \frac{1}{2}$ no permiten que los leptones se lleven dos unidades de momento angular). Por otro lado como los casos $\Delta = 0,1$ están comprendidos en las desintegraciones permitidas, por lo que la contribución de las prohibidas de segundo orden a la tasa de desintegración es completamente despreciable en estos casos, por lo que debemos ignorarlos. Así, las reglas de selección son:

\begin{equation}
	\Delta I = 2,3 \tquad \Delta \pi = \pm 2\quad (\text{{\footnotesize no hay cambio de paridad}})
\end{equation}
Algunos ejemplos:

\begin{itemize}
	\item $\ce{^22Na ->^22 Ne} \quad \parentesis{3^+ \rightarrow 0^+}$  
	\item $\ce{^137Cs ->^137 Ba} \quad \parentesis{\frac{7}{2}^+ \rightarrow \frac{3}{2}^+}$  
\end{itemize}

Las desintegraciones $\beta$ \textbf{prohibidas de tercer orden} son aquellas en las que los leptones se llevan un momento angular $\ell=3$. Así, tendrían las siguientes reglas de selección:

\begin{equation}
	\Delta I = 3,4 \tquad \Delta \pi = \pm 2\quad (\text{{\footnotesize  hay cambio de paridad}})
\end{equation}

\begin{itemize}
	\item $\ce{^87 Rb -> ^87 Sr} \quad (\frac{3}{2}^- \rightarrow \frac{9}{2}^+)$
	\item $\ce{^40 K -> ^40 Ca} \quad (4^- \rightarrow 0^+)$
\end{itemize}
Incluso es posible encontrar desintegraciones prohibidas de cuarto (y quinto) orden, pero cuanto más alto es el orden más improbable es que se produce. En la práctica estas transiciones son tan improbables que sólo pueden observarse cuando las otras son realmente imposibles. En cualquier caso dejamos en la tabla \ref{Tab:02-02-01} las reglas de selección de las transiciones permitidas y prohibidas hasta 5º orden.

\subsection{Desintegración doble $\beta$}

La desintegración doble $\beta$ (o $\beta \beta$) es un proceso mediante el cual dos neutrones se convierten ``simultáneamente'' en dos protones, dos electrones y dos antineutrinos. Un ejemplo es el siguiente:

\begin{eqnarray}
	\ce{^82_34 Se -> ^82_36 Kr + 2e^- + 2\bar{\nu}_e}
\end{eqnarray}
Es necesario insistir en que se trata de un proceso único y no dos desintegraciones $\beta$ sucesivas muy seguidos en el tiempo. En este caso, el proceso $
\ce{^82_34 Se -> ^82_35 Br + e^- + \bar{\nu}_e}$ está prohibido por la ley de conservación de energía ($Q<0$), por lo que la desintegración $\beta$ no puede tener lugar. Sin embargo $\beta \beta$ tiene un $Q>0$. Esta es una característica común a la mayoría de las desintegraciones doble $\beta$: el núcleo intermedio que resultaría de la desintegración $\beta$ simple es más pesado que el núcleo padre, mientras que el resultante de la doble $\beta$, no. \\

Mediciones experimentales de los períodos de semidesintegración para  algunas desintegraciones $\beta \beta$ mencionan arrojan tiempos de vida medio ridículos, 
\begin{eqnarray}
	\ce{^82_34 Se -> ^82_36 Kr + 2e^- + 2\bar{\nu}_e} \tquad t{1/2} = (1.2\pm 0.1)\times10^{20} \ \textbf{años}
\end{eqnarray}
\begin{eqnarray}
\ce{^48_20 Ca -> ^48_22 Ti + 2e^- + 2\bar{\nu}_e} \tquad t{1/2} = (4.3\pm 1.4)\times10^{19} \ \textbf{años}
\end{eqnarray}
Es decir, se trata de procesos sumamente improbables, ya que sabiendo que la probabilidad de transición de $\beta$ viene dada por

\begin{eqnarray}
	\lambda_\beta = \frac{m_e c^2}{\hbar} \parentesis{fg^2 \frac{m_e^4 c^2 |M_{fi}|^2}{2\pi^3 \hbar^6}}
\end{eqnarray}
podemos estimar que la tasa de desintegración $\beta \beta$, salvo por el factor inicial, el cuadrado del anterior:

\begin{eqnarray}
	\lambda_{\beta \beta} = \frac{m_e c^2}{\hbar} \parentesis{fg^2 \frac{m_e^4 c^2 |M_{fi}|^2}{2\pi^3 \hbar^6}}^2
\end{eqnarray}
aunque este tratamiento simplista no se debe tomar demasiado en serio a efectos de predicciones cuantitativas. La búsqueda experimental de desintegraciones doble $\beta$ sin emisión de neutrinos\footnote{Obsérvese que la desintegración doble $\beta$ sin emisión de neutrinos no está prohibida por la ley de conservación de la energía y el momento, tal y como sucede para las desintegraciones $\beta$ simple sin emisión de neutrinos.} (\textit{neutrinoless double beta decay}) es campo de investigación activo e intersante, porque su existencia sería una confirmación más de que los neutrinos no tienen masa nula\footnote{Sin entrar en detalles, se puede considerar que la desintegración doble $\beta$ sin emisión de neutrinos tiene lugar mediante dos procesos virtuales. En el primero de ellos un neutrón se desintegra y emite un neutrino que luego es absorbido por otro neutrón para iniciar una reacción beta inversa que lo convierte en protón. Para que esto pueda ocurrir es necesario que el neutrino emitido en el primer proceso virtual cambie de helicidad, lo cual implicaría que tiene masa.}.


\section{Transiciones electromagnéticas}

Los rayos $\gamma$ son capaces de penetrar varios milímetros en plomo. No son desviados por los campos electromagnéticos e interaccionan con la materia de manera similar a los rayos X.  Se trata de radiación electromagnética, e inicialmente se confundieron os rayos X emitidos por el reordenamiento de los electrones atómicos que sigue a una conversión interna. La desintegración $\gamma$ consiste en la emisión espontánea de fotones altamente energéticos cuando el núcleo pasa de un estado excitado a otro estado de menor energía o al fundamental. Es por tanto un proceso análogo al que tiene lugar cuando un átomo se desexcita emitiendo radiación, bien sea en el rango visible o en el de los rayos X. La emisión gamma suele acompañar a los otros dos tipos de radiación, porque sus productos quedan normalmente excitados. La vida media típica para una emisión $\gamma$ es de unos $10^{-9}$ segundos, pero a veces se observan vidas medias significativamente mayores. Estas transiciones se denominan \textbf{isoméricas}, y a los estados excitados de vida media larga se les llama estados \textbf{estados metaestables}, estados isoméricos, o isómeros. Se suele denotar esta característica con un superíndice en el símbolo del elemento $\ce{^110 Ag^m}$. Los rayos $\gamma$ provenientes de transiciones nucleares electromagnéticas están en un rango de energías de entre 0.1 y 10 MeV, lo cual implica una longitud de onda $\lambda$ entre $10^4$ y $10^2$ fm. 

\subsection{Energétia de la radiación $\gamma$}

Si denotamos por $E_i$ y $E_f$ las energías de los estados excitados nucleares inicial y final respectivamente, la ley de conservación de energía para una transición electromagnética nos permite escribir:

% falta escribir la energética de la reacción

\Revisar

\subsection{Análisis multipolar de la radiación electromagnética.}


\subsection{Reglas de selección del momento angular y de la paridad}

\subsection{Conversión interna}


\subsection{Espectroscopía $\gamma$}


\section{Teoría continua de la desintegración radiactiva.}

La \textbf{actividad} de una muestra de átomos radioactivos se define como el número de desintegraciones que tienen lugar en la unidad de tiempo. Esta definición que tiene sentido si la desintegración es un fenómeno que involucra individualmente\footnote{Algo que parece evidente hoy en día estaba lejos de serlo para los descubridores de la radiactividad. Durante varios años no se tuvo claro que los fenómenos radioactivos son procesos que afectan a cada átomo individualmente.} a cada átomo, no un fenómeno colectivo. Así definida, es obvio que la actividad es proporcional al tamaño de la muestra radioactiva. Rutherford y Soddy establecieron experimentalmente a principios del siglo XX que la actividad de una muestra disminuye de forma exponencial con el tiempo. Desde un punto de vista mecanocuántico esto tiene sencilla justificación, por que se deriva de la existencia de una probabilidad fundamental de transición por unidad de tiempo para cada núcleo, lo cual implica a su vez que la probabilidad de que un átomo se desintegra de forma independiente de los demás, el número de desintegraciones por unidad de tiempo sigue una distribución de Poisson\footnote{En realidad es una distribución binomial que en el límite de un número muy elevado de núcleos tiende a una distribución de Poisson. Supongamos que una muestra con $N$ núcleos radiactivos, cada uno con probabilidad $p$ de desintegrarse por unidad de tiempo $\Delta t$. Queremos calcular la probabilidad de que en el intervalo de tiempo $\Delta t$ se desintegren $n$ núcleos. Si las desintegraciones son realmente independientes entre sí podremos escribir:

\begin{eqnarray}
	P_n^N (\Delta T) = \frac{N!}{(N-n)!n!} p^n (1-p)^{N-n} = \binom{N}{n} p^n (1-p)^{N-n}
\end{eqnarray}

En el límite $N\rightarrow \infty$ manteniéndose $Np=m$ constante, la ecuación anterior tiende a la distribución de Poisson.}. Es decir,, la probabilidad de observar $n$ desintegraciones en n intervalo de tiempo $\Delta t$ viene dada por la siguiente expresión:

\begin{eqnarray}
	P(n,\Delta t) = \frac{m^n}{n!} \exp (-m)
\end{eqnarray}
donde $m$ es el número promedio de desintegraciones por segundo en el período $\Delta t$. La desviación de esta distribución es $\sigma = \sqrt{m}$.


\subsection{Sustancia sin ramificación} \label{Subsec:02-04-01}

De las condiciones discutidas en el párrafo anterior resulta sencillo deducir la evolución exponencial de la población de una sustancia radioactiva. Denotando por $N(t)$ el número de núcleos de la muestra en función del tiempo:

\begin{eqnarray}
	\derivadas{N}{t} = - \lambda N \Rightarrow N = N_0 e^{-\lambda t} 
\end{eqnarray}
donde $\lambda$ es la \textbf{constante de desintegración}. Definimos como \textbf{actividad} de una muestra como

\begin{eqnarray}
	A(t) = \lambda N(t)
\end{eqnarray}
Son datos de interés el valor medio del tiempo de desintegración $\tau$ llamado \textbf{vida media}, y el tiempo en que tarda una muestra en reducirse a la mitad $t_{1/2}$ llamado \textbf{período de semidesintegración}. Ambos se calculan como:

\begin{eqnarray}
	\tau = \frac{1}{\lambda} \tquad t_{1/2} = \frac{\ln 2}{\lambda}
\end{eqnarray}
En el sistema internacional la unidad de medida de la actividad es el \textbf{Becquerelio (Bq)}. El \textbf{Curio (Ci)} también se usa. Originalmente se definió como la actividad de un gramo de radio $^{226}$Ra pero hoy en día se define como $1 \unit{Ci} = 3.7 \times 10^{10} \unit{Bq}$. La actividad de una muestra radioactiva indica únicamente el número de desintegraciones que se producen por segundo. Las fuentes más radiactivas tienen una actividad de microcurios $\mu \textbf{Ci}$. Las ecuaciones para una población de núcleos radioactivos que se desintegran de un modo son muy sencillas. Suponiendo que la desintegración es $X\rightarrow Y$, tenemos que las ecuaciones diferenciales que rigen la desintegración:

\begin{eqnarray}
	\derivadas{N_X(t)}{t} = - \lambda N_x \tquad \derivadas{N_Y(t)}{t} = \lambda N_x
\end{eqnarray}
de tal modo que su solución es:

\begin{eqnarray}
	N_X (t) & = & N_X(0) e^{-\lambda t} \\
	N_Y (t) & = & N_X(0) \parentesis{1-e^{-\lambda t}} + N_Y(0)
\end{eqnarray}
donde suponemos que $Y$ es estable, esto es, no se desintegra en otra sustancia.

\subsection{Ramificación. Constantes de desintegración parciales.}

Supongamos que los núcleos de una muestra pueden desintegrarse siguiendo dos mecanismos o modos alternativos (por ejemplo la desintegración $\alpha$ y $\beta$). A cada uno de estos modos le corresponde una constante de desintegración diferente, de tal modo que la ecuación diferencial será:

\begin{eqnarray}
	\derivadas{N(t)}{t} = - \lambda_a N(t) - \lambda_b N(t) = \lambda N(t)
\end{eqnarray}
donde $\lambda = \lambda_a + \lambda_b$. 

\subsection{Varias sustancias. Cadenas radiactivas naturales.}

Es frecuente que el producto de desintegración de un núcleo radioactivo sea también radiactivo. Supongamos que en el instante inicial $t=0$ tenemos $k$ sustancias que forman una cadena radiactiva. Para calcular el número de átomos en un instante posterior hemos de resolver el siguiente sistema de ecuaciones diferenciales:

\begin{equation}
	\begin{array}{ccl}
		\dfrac{\D N_1}{\D t} & = & - \lambda_1 N_1 \vspace{1.5mm} \\ \vspace{1.5mm}
		\dfrac{\D N_2}{\D t} & = & - \lambda_2 N_2 + \lambda_1 N_1 \\
		\dfrac{\D N_3}{\D t} & = & - \lambda_3 N_3 + \lambda_2 N_2\\\vspace{1.5mm}
					\ldots & \ldots &  \ldots \\					 
		\dfrac{\D N_k}{\D t} & = & - \lambda_k N_k  + \lambda_{k-1} N_{k-1}
	\end{array}
\end{equation}
Lógicamente en función de las condiciones iniciales obtendremos una solución u otra. Una de las mas preeminentes es la de suponer que tan solo la sustancia 1 se halla presente al principio. La solución general con estas condiciones viene dada por las \textbf{ecuaciones de Bateman}. Por ejemplo para 3 sustancias, tendríamos que:

\begin{equation}
	\begin{array}{ccl} \vspace{1.5mm}
		N_1(t) & = & N_1(0) e^{-\lambda_1 t}\\ \vspace{1.5mm} 
		N_2(t) & = & N_1(0) \frac{\lambda_1}{\lambda_2 - \lambda_1} \parentesis{e^{-\lambda_1 t } - e^{\lambda_2}} \\\vspace{1.5mm}
		N_3(t) & = & N_1(0) \lambda_1 \lambda_2 \parentesis{ \frac{e^{-\lambda_1 t }}{(\lambda_2-\lambda_1)(\lambda_3-\lambda_1)}  -\frac{e^{-\lambda_2 t }}{(\lambda_1-\lambda_2)(\lambda_3-\lambda_2)} -\frac{e^{-\lambda_3 t }}{(\lambda_1-\lambda_3)(\lambda_2-\lambda_3)} }
	\end{array}
\end{equation}
Si por ejemplo el núcleo 3 fuera estable, de tal modo que la cadena fuera $1\rightarrow 2\rightarrow 3$, con suponer que $\lambda_3=0$ en la ecuación anterior sería suficiente:

\begin{equation}
	\begin{array}{ccl} \vspace{1.5mm}
		N_1(t) & = & N_1(0) e^{-\lambda_1 t}\\ \vspace{1.5mm} 
		N_2(t) & = & N_1(0) \dfrac{\lambda_1}{\lambda_2 - \lambda_1} \parentesis{e^{-\lambda_1 t } - e^{\lambda_2}} + N_2(0)e^{-\lambda_2 t} \\\vspace{3mm}
		N_3(t) & = & N_1(0)  \ccorchetes{1 + \frac{1}{\lambda_2-\lambda_1} \parentesis{\lambda_1 e^{-\lambda t} - \lambda_2 e^{-\lambda_1 t}} } + N_3 (0)
	\end{array}
\end{equation}






