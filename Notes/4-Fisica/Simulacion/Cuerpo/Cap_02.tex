
\chapter{Introducción}

La idea general es obtener las propiedades macroscópicas de un sistema numéricamente a partir del conocimiento de su hamiltoniano. Esto implica:

\begin{itemize}
    \item Estableccer un modelo físico para el sistema real que estudiamos: escribir su hamiltoniano.
    \item Calcular la evolución temporal de todas las posiciones y velocidades (q,p) de las dinámicas constituyentes del sistema a partir de una configuración inicial. A este método lo llamamos \textbf{Dinámica Molecular}. Otra opción sería construir la colectividad representativa de mi sistema (en este caso solo q). A este método lo llamamos \textbf{Monte Carlo}.
    \item Hacer estadística de todas las (q,p) de las $N$ partículas de mi sistema para encontrar sus propiedades macroscópicas.
\end{itemize}

\section{Colectividad microcanónica}


\section{Colectividad canónica}