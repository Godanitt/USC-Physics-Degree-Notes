\chapter{Introducción a Fortran}

\section{Compilación y ejecucción}

\section{Formato código fuente}

El formato de condigo guente puede ser libre o fijo, y no deben mezclarse ambos en un fichero de código. El código fijo se considera obsoleto en Fortran95. En cualquier caso existen ciertas normas básicas y típicas de fortran, ataño obligatorias, que todavía se mantienen, por lo que es importante mencionarlas. Estas son:

\begin{itemize}
    \item Las sentencias de un programa se escribem en diferentes líneas.
    \item La posición de los caracterres dentro de las líneas es significativa.
    \item Columnas:
    \begin{itemize}
        \item 1-5. Número de etiqueta (de 1 a 5 dígitos, se usan números usualmente).
        \item 6. Carácter de continuación de línea.
        \item Resto. Sentencia.
    \end{itemize}
    \item Comentarios:
    \begin{itemize}
        \item Las líneas en blanco se ignoran. Hacen más legible el programa.
        \item Si el primer carácter de una línea es *, c o C la línea es de comentario.
        \item Si aparece el carácter ! en una línea (salvo en la columna 6) lo que sigue es un comentario.
    \end{itemize}
    \item Una línea puede contener varias sentencias separadas por punto y coma (;), el cual no puede estar en la columna 6. Sólo la primera de estas sentencias podría llevar etiqueta.
    \item Los espacios en blanco son significativos: {\tt IMPLICIT NONE, DO WHILE} (obsoleto), {\tt CASE DEFAULT}. Son opcionales en:
    \begin{itemize}
        \item Palabras clave dobles qeu comienzan por {\tt END} o {\tt ELSE}.
        \item {\tt DOUBLE PRECISION, GO TO, IN OUT, SELECT CASE.}
    \end{itemize}
    \item  El indicador de continuación de una línea es el carácter \&.
\end{itemize}

\section{Tipos de datos}

Fortran tiene los siguientes tipos de datos:

\begin{itemize}
    \item Enteros ({\tt INTEGER})
    \item Reales ({\tt REAL, DOUBLE PRECISION})
    \item Complejos ({\tt COMPLEX})
    \item Lógicos ({\tt LOGICAL})
    \item Caracteres ({\tt CHARACTER, CHARACTER(LEN=n), CHARACTER*n})
\end{itemize}

\subsection{Parámetros. Variables. Declaración. Asignación.}

\begin{itemize}
    \item Un parámetro tiene un valor que no se puede cambiar (PARAMETER).
    \item Una variable puede cambiar su valor cuantas veces sea necesario. 
    \item Por defecto, todas las variables que empiecen por \texttt{i,j,k,l,m} o \texttt{n} son entreas y las demas reales. Es muy recomendable declarar las variables que se utilicen (la sentencia {\tt IMPLICIT NONE} obliga a declarar todas las variables).
\end{itemize}

\subsection{Arrays, subíndices, substrings}

\begin{itemize}
    \item Un array se define mediante su nombre y dimensiones (cantidad y límites).
    \item Por defecto el primer índice es 1. En otro caso hay que indicar el rango {\tt i1:i2}.
    \item Los elementos del array se acceden por sus índices entre paréntesis.
\end{itemize}

\section{Operadores y expresiones}

\subsection{Aritméticas}

\begin{itemize}
    \item Los operadores aritméticos son {\tt +, -, *, /, **}.
    \item El orden de prioridades es el mismo que en el álgebra.
    \item No puede ver operadores seguidos (incorrecto {\tt a*-b}, correcto {\tt a*(-b)}).
\end{itemize}

\subsection{Relacion y expresiones lógicas}

\begin{itemize}
    \item Los operadores de expresiones son: 
    \begin{table}[h!] \centering
        \begin{tabular}{|c|c|c|c|c|c|}
            \hline 
            {\tt .EQ.} & {\tt .NE.} & {\tt .LT.} & {\tt .LE.} & {\tt .GT.} & {\tt .GE.} \\ \hline
            $==$ & $/=$ & $<$ & $<=$ & $>$ & $>=$ \\    \hline        
        \end{tabular}
    \end{table}

    \item Se pueden relacionar expresiones aritméticas con expresiones lógicas y expresiones de caracteres. 
    \item Es recomendable utilizar paréntesis y/ó sustituir las expresiones complicadas por combinaciones de expresiones más simples.
    \item Los operadores lógicos son:
    \begin{table}[h!] \centering
        \begin{tabular}{|c|c|c|c|c|c|}
            \hline  Operador &  {\tt .NOT.} & {\tt .AND.} & {\tt .OR.} & {\tt .EQV.} & {\tt .NEQV.}  \\ \hline Prioridad
             & 1 & 2 & 3 & 4 & 4 \\    \hline        
        \end{tabular}
    \end{table}
\end{itemize}