
\chapter*{Introducción}
\addcontentsline{toc}{chapter}{\protect\numberline{}Introducción}

Usaremos $N=500$ particulas y una densidad de 0.5 $N/V^3$. La variación máxima de energía permitida es 1/1000. \\

USaremos la aproximación de Lennard-Jones, hya que es suave, supone interacciones debiles ideales para los gases nobles. 

\begin{equation} 
    v_{ij} (r_{ij}) = 4 \epsilon \left[ (\sigma /r_{ij})^{12}-(\sigma /r_{ij})^6 \right]
\end{equation}


``Usar una suma doble para luego dividirlo por dos es para pegarle en la cara''. La parte de los sumatorios debe estar libre de polvo y paja para que corra veloz. 

$$ t_p =  \frac{1}{2} \sum \sum v_{ij} = \sum_{i=1}^{N-1} \sum_{j=i+1}^N v_{ij} $$