\chapter{El campo de Klein-Gordon}

\section{La ecuación de Klein-Gordo}

Una manera de tratar de combinar la mecánica cuántica y la relatividad especial es trabajar con las ecuaciones de orden pero abandonar la ecuación de Schödinger y tratar de construir una compatible con las simetrías relativistas. En la mecánica cuántica asociamos los operadores diferenciales con la energía y el momento:

\begin{equation}
    \text{Energía:} E \rightarrow i \hbar \parciales{}{t} \tquad \text{Momento:} \pn \rightarrow - i \hbar \nabla
\end{equation}
En una teoría no relativista la relación de dispersión es $E=\frac{\pn^2}{2m}$. Substituir los operadores momento y energia en esta relación de dispersión nos lleva a la ecuación de Schödinger $i \hbar \partial_t \psi = - \frac{h}{2m} \nabla^2 \psi$, donde $\nabla^2 = \delta^{ij} \partial_i \partial_j$. Veremos ver que pasa al usar la relación dispersión relativista: 

\begin{equation}
    E^2 = c^4 m^2 + c^2 \pn^2
\end{equation}
que pasa a ser

\begin{equation}
    - \hbar^2 \parciales{^2 \phi}{t^2} = c^4 m^2 \phi - \hbar^2 c^2 \nabla^2 \phi
\end{equation}
donde $\phi$ es la \textit{función de ondas}. Podemos expresar esta ecuación en términos relativistas si usamos $x^{\mu} = (ct,x^i)$ y el operador D'Alembert operador $\square = \partial_\mu \partial^\mu$ de tal modo que se convierte en:

\begin{equation}
    \parentesis{\square + \frac{c^2 m^2}{\hbar^2}} \phi = 0 \label{Ec:03-01-004-KleinGordon}
\end{equation}
esta es la {\bf ecuación de Klein-Gordon}. 

\subsection{Ondas planas como solución}

Vamos a ver ahora las soluciones mas simples de las ecuaciones de Klein-Gordon, que no son otras que las ondas planas. Podemos ver que:

\begin{equation}
    \phi (x) = N e^{-\frac{i}{\hbar} (Et - \pn \cdot \xn)}
\end{equation}
donde $N$ es el factor normalizador. Si substituimos esta ecuación en \ref{Ec:02-1-4-KleinGordon} tedremos que:

\begin{equation}
    \frac{-E^2 + c^2 \pn^2 + c^4 m^2}{c^2 \hbar^2}\phi = 0
\end{equation}
y por lo tanto que $E$ está relacionada con $p$ mediante la ecuación de dispersión relativista. El punto ahora es que tenemos dos tipos de soluciones para la energía:

\begin{equation}
    E = \pm w_{\pn} \tquad w_{\pn} = \sqrt{c^4 m^2 + c^2\pn^2}
\end{equation}
El hecho de que la relación de dispersión sea cuadrática procede de que la ecuación Klein-Gordon es una ecuación de segundo orden en la derivada temporal. Cuando constuimos el espacio de soluciones tenemos que incluir aquellas que corresponden a energías negativas.  \\

\subsection{Interpretación probabilistica: ¿Problemas?}

La presencia de soluciones para la ecuación de ondas que nos llevan a la existencia de una energía negativa debería chirriarnos bastante, sobretodo porque no parece que tenga un límite inferior, por lo que podríamos llegar a soluciones con menor energía {\it ad infinitum}.

La presencia de energía negativa también nos lleva a problemas cuando tratamos de realizar la interpretación probabilistica. Normalente, en la mecánica cuántica, la densidad de probabilidad viene dada por el módulo al cuadrado de la ecuación de ondas:

\begin{equation}
    \rho = |\psi|^2
\end{equation}
Y la {\it corriente de probabilidad} dada por 

\begin{equation}
    \Jn = \frac{\hbar}{2mi} \parentesis{\psi^* \nabla \psi - \psi \nabla \psi}
\end{equation}
de tal modo que la conservación de la probabilidad queda como

\begin{equation}
    \parciales{\rho}{t} + \nabla \cdot \Jn = 0
\end{equation}

¿Podemos construir algún tipo de cantidad conservada para el campo de Klein-Gordon? La respuesta es que si siempre que cojamos un $\phi$ complejo. De este modo de la ecuación Klein-Gordon deducimos que 

\begin{equation}
    \phi^* \parentesis{\square + \frac{c^2m^2}{\hbar^2}} \phi = 0 \Leftrightarrow 
    \phi \parentesis{\square + \frac{c^2m^2}{\hbar^2}} \phi^* = 0 
\end{equation}
lo que implica que
\begin{equation}
\phi^*\square \phi - \phi \square \phi = 0 \label{Ec:03-01-012}
\end{equation}
Si expresamos esto con índices relativistas tenemos que 

\begin{equation}
    \phi^* \square \phi = \phi^* \partial_\mu \partial^\mu \phi = \partial_\mu (\phi^* \partial^\mu \phi) - \partial_\mu \phi^* \partial^\mu \phi
\end{equation}
y de manera similiar el otro término (mismo procedimiento). Dado esto, podemos ver que la expresión \ref{Ec:03-01-012}, de tal modo que podemos escribir la ecuación 

\begin{equation}
    \partial_\mu \parentesis{\phi^* \partial^\mu \phi - \phi \partial^\mu \phi^*} = 0
\end{equation}