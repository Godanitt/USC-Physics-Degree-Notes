
\chapter*{Relatividad especial}
\addcontentsline{toc}{chapter}{\protect\numberline{}Introducción}


\section*{Axiomas fundamentales}

La teoría de la relatividad se busca crear un modelo matemático que sea capaz de compaginar la electrodinámica clásica y la dinámica del movimiento de los cuerpos. Para empezar la dinámica Newtoniana no impone ningún tipo de impedimento para que una partícula viaje más rápido que la velocidad de la luz. Además un estudio profundo de las transformaciones de Galileo nos revela que las leyes de Maxwell cambian de un sistema de referencia a otro. Fue Einstein el primero en estudiar cual deben ser las relaciones entre sistemas de referencia para que las leyes de la electrodinámica no cambiasen de un sistema a otro, obteniendo las relaciones de lo que hoy conocemos como relatividad especial. Lógicamente tuvo que asumir dos axiomas, que son los siguientes:

\begin{itemize}
\item \textbf{Principio de relatividad:} las leyes de la física deben adoptar la misma forma en todos os sistemas de referencia inerciales.

\item \textbf{Universalidad de la velocidad de la luz:} la velocidad de la luz $c=3\cdot 10^8$ (m/s) es una constante universal. Cualquier observador inercial mide la misma velocidad de la luz.
\end{itemize}


La primera consecuencia del segundo postulado es que podemos hacer una equivalente entre espacio tiempo, por lo que podemos medir el tiempo como una unidad especial, tal que $x^0=c \cdot t$. Otra consecuencia es que el tiempo no es absoluto, si no que cada sistema de referencia tendrá el suyo propio.

\section*{Transformaciones de Lorentz}
 \subsection*{Cuadrivectores}


Cuando hablamos de un \textit{suceso físico} (cualquier proceso, como puede ser caerse una caja, chocar dos partículas...) será necesario \textit{especificar la posición e instante}. Un \textbf{suceso} $x$ estará unequívocamente definido cuando conozcamos, en el SRI $\mathcal{O}$, el  \textbf{cuadrivector} (4-vector):

\begin{equation}
x = (x^0, \vec{x}) = (ct,x_1,x_2,x_3)
\end{equation}
Usaremos la notación de sub-índices y súper-índices, de tal modo que $x^\mu \ \mu=0,1,2,3$ denota el suceso $x$. En la relatividad especial podemos usar indistintamente ambos, en la relatividad general cobrará especial importancia, significando cosas diferentes $x^\mu$ que $x_\mu$. Llamaremos la \textbf{forma covariante} a $x_{\mu}$ y la \textbf{forma contravariante} a $x^{\mu}$. \\

\subsection*{Transformaciones de Lorentz}

En cualquier otro sistema de referencia $\mathcal{O}'$ tenemos que el cuadrivector del mismo suceso será diferente. La pregunta que debemos hacernos es si es posible relacionar las coordenadas de un suceso en $\mathcal{O}$ con el suceso en $\mathcal{O}'$. Lo primero que debe verificar es que, para sistemas de referencia que se muevan (entre sí) a una velocidad $v\ll c$, se recuperen las relaciones de Galileo. La transformación que nos permita relacionar un suceso en ambos sistemas de referencia inerciales vendrá determinada por la \textbf{matriz de Lorentz} $\Lambda_\mu^{\mu'}$, tal que 

\begin{equation}
x^{\mu'} = \Lambda_\mu^{\mu'} x^{\mu}  \tquad 
x^{\mu} = \Lambda_{\mu'}^{\mu} x^{\mu'} 
\end{equation}
donde claramente una es inversa de la otra. La forma de esta matriz es bastante complicada si, por ejemplo, estuvieran rotados (si, por ejemplo, los ejes $x$ de cada una de los SRI estuvieran separados por un ángulo $\phi$), ya que aparecería una matriz de rotación. \\

La matriz de Lorentz más sencilla se obtiene cuando ambos sistemas se mueven a una velocidad relativa $v$ respecto a un eje (por ejemplo el eje $x$ denotado por $x^1$), permaneciéndo estáticos respecto a los demás ejes. En ese caso tendremos que

\begin{equation}
\Lambda = \begin{pmatrix}
\gamma & - \gamma \beta & 0 & 0 \\
 - \beta \gamma &  \gamma  & 0 & 0 \\
0 & 0  & 1 & 0 \\
0 & 0  & 0 & 1
\end{pmatrix}
\end{equation}
donde hemos usado las siguientes relaciones (siendo $\gamma$ el \textbf{factor de Lorentz}):

\begin{equation}
\gamma = \frac{1}{\sqrt{1-v^2/c^2}} \tquad \beta = v/c
\end{equation}

\subsection*{Producto escalar}

Dado que estamos creando un nuevo espacio, nos hace falta crear un producto escalar, de tal modo que podamos decir que entre tal y cual suceso hay una distancia de tanto. Lógicamente esta distancia espacio-temporal debe ser invariante, no importa el sistema de referencia. Definimos el producto escalar de dos 4-vectores como

\begin{equation}
A \cdot B \equiv \eta_{\mu \nu} A^{\mu}  B^{\nu}
\end{equation}
donde $\eta_{\mu \nu}$ es la \textbf{métrica del sistema} (que cobrará especial relevancia en la relatividad general). La métrica del sistema en la relatividad especial es la \textbf{métrica de Minkowski}, tal que 


\begin{equation}
\eta = \begin{pmatrix}
1 & 0 & 0 & 0 \\
 0 &  -1  & 0 & 0 \\
0 & 0  & -1 & 0 \\
0 & 0  & 0 & -1
\end{pmatrix}
\end{equation}
de este modo se verifica que $\eta = \eta^{-1}$, o lo que es lo mismo, $\eta^{\mu \nu} = \eta_{\mu \nu}$. A partir de esto podemos definir las distancias en el espacio-tiempo, que llamaremos \textbf{intervalo} (aunque podemos llamarlas distancia entre dos sucesos, recordemos que no es la misma distancia que en la mecánica clásica). Sea $\Delta x^{\alpha} = x_A^{\alpha}-x_B^{\alpha}$, el intervalo será:

\begin{equation}
\D s_{AB}^2 =  (\Delta x^{0})^2 - (\Delta \xn)^2
\end{equation}
Cuando $x_{B}^\alpha = 0$ (origen del espacio-tiempo) tenemos que:

\begin{equation}
\D s_A^2 =  (x_A^0)^2 - (\xn_A)^2
\end{equation}
Es muy importante lo siguiente: \textit{el intervalo asociado a una pareja de sucesos es un invariante Lorentz}, de tal modo que cualquier observador medirá el mismos intervalo sea cual sea su sistema de referencia. Esta es la propiedad mas importante de los cuadrivectores: \textit{cualquier producto escalar de 2 cuadrivectores es un invariante Lorentz}. \\

\subsection*{Operadores en el espacio tiempo}

Definimos como el \textit{operador parcial} $\partial / \partial x^{\mu}$ aquel que verifica que:

\begin{equation}
\parciales{x^{\nu}}{x^{\mu}} = \delta_{\mu}^{\nu}
\end{equation}
También se puede denotar por $\partial_\mu$. De hecho de ves qeu el operador es un operador cuadrivector convariante. Si queremos derivar un operador covariante necesitaremos construir el operador parcial contravariante, tal que:

\begin{equation}
\parciales{x_{\nu}}{x_{\mu}} = \delta^{\mu}_{\nu}
\end{equation}
que se denota por $\partial^{\mu}$. Así podemos obtener el \textbf{opeador D'Alambertiano}, análogo relativista al laplaciano:

\begin{equation}
\square^2 \equiv \partial^\mu \partial_\mu \equiv  \parciales{^2}{(x^0)^2} -\parciales{^2}{(x^1)^2} - \parciales{^2}{(x^2)^2}  - \parciales{^2}{(x^3)^2} 
\end{equation}
Es común que en la literatura nos encontremos con las siguientes acotaciones:

\begin{itemize}
    \item Usaremos los términos $\dot{\phi}$ para tratar de describir la derivada respecto la coordenada temporal ($x^0$), de tal modo que:
    \begin{equation}
        \dot{\phi} = \partial_t \phi = \partial_0 \phi
    \end{equation}  
    \item Usaremos los el $\vec{\nabla}$ para tratar de describir la derivada respecto las coordenadas no temporales, de tal modo que:
    \begin{equation}
        \vnabla = \begin{pmatrix}
            \partial_1 \\
            \vdots \\ \partial_D
        \end{pmatrix}
    \end{equation}
\end{itemize}




\section*{Cinemática de la partícula}
\subsection*{Cuadrivector velocidad y momento}

Sería muy importante extender todos los observables a las formas 4-vectoriales, ya que nos permitiría relacionar los observables medibles en cualquier sistema de referencia inercial. Una de las mas interesantes sería el cuadri-vector velocidad (otro podría ser el cuadri-momento). Para esto necesitamos construir un escalar invariante con el que poder derivar la 4-posición. Si derivamos un 4-vector por un escalar que no es invariante Lorentz, el vector resultante no es un 4-vector. \\

Como hemos dicho debemos elegir un escalar invariante, que además esté relacionado de alguna manera con el escalar. Este no será otro que el \textbf{tiempo propio} $\tau $ ($\tau \equiv x^0$ en dicho SRI). El tiempo propio es aquel que mide un sistema de referencia que se mueve con el objeto. Matemáticamente esto esta invariancia es sencillamente demostrable, ya que en su sistema de referencia la posición del objeto que se mueve verifica en cualquier instante $\xn=0$, por lo que la posición en el espacio-tiempo en el SRI de la partícula:

\begin{equation}
\D s^2 = (\Delta x^{\alpha})^2 = - \tau^2= - (x^{0})^2
\end{equation}
Dado que $\D s^2$ es un invariante Lorentz, tenemos que $\D \tau^2$ (y por ende su raíz cuadrada) también lo será. En ese caso tendremos directamente que el cuadrivector velocidad:

\begin{equation}
    U^{\mu} = \frac{\D x^{\mu} (\tau)}{\tau}
\end{equation}
Además tendremos que el producto escalar de un cudrivector velocidad sobre sí mismo {\it siempre} será -1:

\begin{equation}
    U^{\mu} U_{\mu} = -1
\end{equation}
Lógicamente el cuadrivector momento será el cuadrivector velocidad por la masa de tal modo que: 

\begin{equation}
    P^{\mu} = m U^{\mu} = (E/c,\vn)
\end{equation}
