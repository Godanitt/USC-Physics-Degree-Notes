\chapter{Mecánica Clásica}

En este capítulo introducimos brevemente la formulación Lagrangiana de la mecánica clásica. La manifestación de las simetrías en esta formulación particular de la mecánica clásica y el teorema de Noether, que nos permite calcular las leyes de conservación se hace en la sección \ref{Sec:01-02}. Luego veremos tanto la formulación Hamiltoniana y su relación con la mecánica cuántica. finalmente veremos las reglas de la cinemática relativista así coomo la formulación covariante del electromagnetismo. 

\section{Formulación Lagrangiana}

\subsection{Ecuaciones de Euler-Lagrange}

Como sabemos el Lagrangiano de un sistema físico con $N$ grados de libertad viene dado por:

\begin{equation}
	L = T - V
\end{equation}
donde $T$ es la \textit{energía cinética} y $V$ la \textit{energía potencial}, ambas expresadas en las coordenadas generalizadas $q_i$ y sus respectivas velocidades $\dot{q}_i$. En un sistema con $N$ grados de libertad, las \textbf{ecuaciones de Euler-Lagrange} son: 

\begin{equation}
	\derivadas{}{t} \parciales{L}{\dot{q}_i} - \parciales{L}{q_i} \tquad \dot{q}_i = \derivadas{q}{t}
\end{equation} 
donde $i=1...N$. Decimos que dos lagrangianos son \textbf{equivalentes} si llevan a las mismas ecuaciones del movimiento. Dos lagrangianos $L'$ y $L$ que divergan entre sí tal que

\begin{equation}
	L' = aL  + \derivadas{F(\qn,t)}{t} + b
\end{equation} 
son equivalentes. Esto será importante más adelante. Definimos como fuerza generalizada asociada a la coordenada generalizada $q_i$ como:

\begin{equation}
	Q_i \equiv \sum_{a=1}^n \Fn_a^{\text{apl}} \parentesis{\parciales{\dot{\rn}_a}{q}_j}
\end{equation}
En general se suele verificar que $\Fn_a^{\textbf{apl}}=-\nabla_aV$ para $a=1,2...n$. 

\subsection{Principio de Hamilton}

El \textbf{principio de Hamilton} nos dice que la trayectoria que sigue un cuerpo entre dos puntos es aquella que hace estacionaria la acción\footnote{Se suele decir que <<minimiza>> la acción, aunque nunca se hace una exigencia para que esto sea así. El caso más general corresponde a decir que la acción se hace estacionaria, no necesariamente mínima.} respecto las diferentes variaciones del camino a seguir (es decir, respecto los diferentes caminos $\qn(t)$). La acción queda definida como el siguiente funcional:

\begin{equation}
	S[\qn] = \int_{t_i}^{t_f} \D t L(\qn,\dot{\qn},t)
\end{equation} 
Entonces el principio de Hamilton nos dice que las ecuaciones del movimiento en el sistema clásico corresponde aquellas que verifiquen 

\begin{equation}
	\frac{\delta S[\qn]}{\delta q_j (t)} =  0 \tquad \text{para} \ j=1,2,...,N
\end{equation}
Esta condición se puede demostrar que lleva directamente a las ecuaciones de Euler-Lagrange, tal que: 

\begin{equation}
	\frac{\delta S[\qn]}{\delta q_j (t)} = \derivadas{}{t} \parciales{L}{\dot{q}_i} - \parciales{L}{q_i} \tquad \dot{q}_i = 0 
\end{equation}


\section{Simetrías y Teorema de Noether} \label{Sec:01-02}

Supongamos que $\qn'$ y $\qn$ son dos posibles elecciones de las $N$ coordenadas generalizadas para un sistema. Entonces debe existir una función $\fn$ invertible que nos permite pasar de una a la otra, tal que $\qn'=\fn(\qn)$. Consideremos que existe un  parámetro $s$  \footnote{De manera general podría ser una familia de parámetros, aunque para lo que necesitamos uno será suficiente} tal que la transformación $\qn'=\fn_s(\qn)$ depende del mismo y cuando $s=0$ se verifica $\qn'=\qn$, esto es:

\begin{equation}
	\qn' (s,t) = \fn_s (\qn (t)) \quad \text{donde} \ s \in \mathbb{R} \ \text{y} \ \qn'(0,t) = \fn_0 (\qn(t)) = \qn(t)
\end{equation}
La familia de transformaciones $\fn_s$ es defina como una \textit{simetría de la acción},
siempre que se verifique $S[\qn] = S[\qn'(s)] = S[\fn_s(\qn)]$ para cada $s$, para cada camino $\qn$ para cualquier intervalo temporal definido por $t_i$ y $t_f$. Que sea independiente del intervalo temporal implica que

\begin{equation}
	\eval[ \derivadas{}{s} L (\qn',\dot{\qn}',t) |_{s=0} = 0
\end{equation}
Definimos $\etan(\qn)$ como la derivada de la familia $f_s$ cuando $s=0$ tal que 

\begin{equation}
	\etan (\qn) \equiv \eval{ \derivadas{\qn'}{s} }_{s=0} = \eval{ \derivadas{\fn_s(\qn)}{s} }_{s=0}
\end{equation}

\begin{teorema}
A partir de las anteriores definiciones, el \textbf{teorema de Noether} asegura que si $\fn_s$ es efectivamente una simetría de la acción, la \textit{primera integral de Noether} 

\begin{equation}
	C(\qn,\dot{\qn},t) \equiv \sum_{j=1}^N \pparciales{L}{\dot{q}_j} \eta_j (\qn) 
\end{equation}
es una \textit{constante del movimiento}, esto es, $\D C / \D t = 0$. 
\end{teorema}

\section{Oscilaciones pequeñas y modos normales}

Cuando tenemos un sistema físico complejo, estamos interesados en variaciones pequeñas alrededor del equilibrio, que se puede describir en términos de los llamados ``modos normales''. Consideremos entonces un sistema con $N$ grados de libertad, con $N$ coordenadas generalizadas $\qn=(q_1,q_2,...,q_n)$. Supongamos entonces que estamos suficientemente cerca de un punto de equilibrio (mínimo local de $V(\qn)$)- 

\begin{definicion}
	Decimos que un sistema está en \textbf{equilibrio} cuando todas las fuerzas asociadas a las coordenadas generalizadas en el sistema se hacen nulas:
	\begin{equation}
		Q_i(t)=0 \quad \forall \ t  
	\end{equation}
	Un sistema se dice que está en \textit{equilibrio estático} cuando además las velocidades generalizadas verifican $\dot{qn}=0$.
\end{definicion}

Cuando estamos en un sistema en equilibrio alrededor del punto $\qn_0$ siempre podemos expandir nuestro potencial por Taylor tal que

\begin{equation}
	V(\qn) = V(\qn_0) + \frac{1}{2}\sum_{i,j=1}^{N} \eval{\parciales{^2V}{q_i\partial q_j}}_{\qn=\qn_0} \eta_i \eta_j \equiv V_0 + \frac{1}{2}\sum_{i,j=1}^{N} K_{ij} \eta_i \eta_j
\end{equation}
donde $\etan = \qn - \qn_0$. En este caso $K$ es una matriz real y simétrica, y por tanto diagonalizable. La energía cinética por otro lado

\begin{equation}
	T=\frac{1}{2}\sum_{i,j}^N M_{ij}(\qn) \dot{\eta}_i \dot{\eta}_j
\end{equation}
donde $M$ también es real y diagonalizable, con valores positivas en el entorno del punto de equilibrio. Nuestro lagrangiano entonces se puede expresar como (lógicamente $V_0$ no nos interesa, siempre podemos reescalarlo)

\begin{equation}
	L=\frac{1}{2}\sum_{i,j}^N M_{ij}(\qn)^0 \dot{\eta}_i \dot{\eta}_j-+ \frac{1}{2}\sum_{i,j=1}^{N} K_{ij} \eta_i \eta_j
\end{equation}
Como podemos ver se parece mucho a un oscilador armónico, salvo por los términos acoplados. Siempre existe una transformación ortogonal sobre las coordenadas $\eta$ que nos permitan escribir el problema como un conjunto de osciladores armónicos linealmente independientes, que se expresan por la coordenada normal $\zeta_i$, cuyo lagrangiano es 

\begin{equation}
	L=\sum_{i=1}^{N} \frac{1}{2}\parentesis{\dot{\zeta}_i^2 - \omega_i^2 \zeta_i^2}
\end{equation}
y sus ecuaciones son 

\begin{equation}
	\ddot{\zeta_i}+\omega_i^2 \zeta_i^2=0
\end{equation}
donde $\omega_i$ son las diferentes frecuencias de oscilación del sistema y se llaman \textbf{modos normales} del sistema. Cada modo normal corresponde a una de las posibles soluciones que describen la oscilación del sistema entorno al equilibrio.

\section{Formulación Hamiltoniana}

La mecánica Lagrangiana es usa coordenadas generalizadas y velocidades para describir el comportamiento del sistema. Existe un método alternativo, en el que se usan las coordenadas generlizadas y sus respectivos momentos canónicos conjugados. Los dos formalismos están relacionados por una tranformación de Legendre.

\subsection{Hamiltoniano y Ecuaciones Hamiltonianas}

El \textbf{Hamiltoniano} $H$ asociado con el Lagrangiano $L$ se define a través de la transformada de Legendre multidimensional:

\begin{equation}
	H(\qn,\pn,t) \equiv \parentesis{\sum_{i=1}^N p_i \dot{q}_i }  - L(\qn,\dot{\qn},t) = \pn \cdot \dot{\qn} - L 
\end{equation}
donde $p_i$ es el momento canónico conjugado de la coordenada generalizada $q_i$ definido como

\begin{equation}
	p_i \equiv \parciales{L}{\dot{q}_i}
\end{equation}
Las \textit{ecuaciones del movimiento} ahora son:

\begin{equation}
	\dot{q}_i = \parciales{H}{p_i} \qquad \dot{q}_i = \parciales{H}{q_i} \qquad \derivadas{H}{t} = - \parciales{L}{t}
\end{equation}

\subsection{Corchetes de Poisson}

Supongamos dos funciones en el espacio de fases $F(\pn,\qn,t)$ y $G(\pn,\qn,t)$, cuya única condición es que el orden de las derivadas no altera el resultado. Si se verifican estas condiciones decimos que $F$ y $G$ son \textit{variables dinámicas}. El \textbf{corchete de Poisson} de estas dos funciones se define como

\begin{equation}
	\{ F,G \} = \sum_{j=1}^{N} \parentesis{\parciales{F}{q_j}\parciales{G}{p_j} -\parciales{F}{p_j} \parciales{G}{q_j}     } 
\end{equation}
Si $A$, $B$ y $C$ son variables dinámicas y $a$ y $b$ constantes reales, entonces las propiedades de los corchetes de Poisson pueden ser expresadas como


\begin{enumerate}
	\item Cierre: $\{A,B\}$ es también una función del espacio de fases.
	\item Antisimetría:  $\{A,B\}=-\{B,A\}$.
	\item Bilinearidad: $\{aA+bB,C\}=a\{A,C\}+b\{B,C\}$.
	\item Regla del Producto: $\{A,BC\}=\{A,B\}C+B\{A,C\}$.
	\item Identidad de Jacobi: $\{A,\{B,C\}\}+\{B,\{C,A\}\}+\{C,\{A,B\}\}=0$.
\end{enumerate}
Los corchetes de Poisson encierran gran parte de las propiedades de las variables dinámicas. Por ejemplo, el corchete de Poisson entre $A(\qn,\pn)$ con las coordenadas generalizas y sus momentos:

\begin{equation}
	\{q_i,A\} = \parciales{A}{p_i} \qquad 
	\{p_i,A\} = - \parciales{A}{q_i}
\end{equation}
Mientras que con el Hamiltoniano la variable dinámica $A(\qn,\pn,t)$:
\begin{equation}
	 \derivadas{F}{t} = \{ F,H\} + \parciales{F}{t}
\end{equation}
Otros corchetes muy importantes son los \textit{corchetes fundamentales}:

\begin{equation}
	\{q_i,p_j\} = \delta_{ij}  \qquad \{q_i,q_j\}=\{p_i,p_j\}=0
\end{equation}



\section{Cinemática Relativista}

Definimos el cuadrivector posición para un sistema de referencia:

\begin{equation}
	x^\mu = (ct,\xn)
\end{equation}
como todo cuadrivector, cuando queremos describirlo en otros sistema de referencia inercial tendremos que usar las transformaciones de Loretnz a través de la matriz $\Gamma$. Las distancias las calculamos a través del tensor de Minkowski 

\begin{equation}
	x^2 = x^\mu x_\mu = g^{\mu \nu} x^{\mu} x^{\nu}
\end{equation}
y es un \textbf{invariante Lorentz}\footnote{Cuando decimos que un escalar invariante Lorentz, covariante Lorentz, invariante relativista... nos referimos a que dicho valor no depende del sistema de referencia. Hay varias maneras de cambiar el sistema de referencia: rotarlo, trasladarlo, o darle un \textit{boost} (es decir, considerar que estamos en un sistema de referencia que se mueve con una velocidad $\vn$ respecto el anterior). Cuando decimos que un vector o tensor es covariante significa que cuando hacemos un cambio de sistema de referencia este objeto (vector, tensor) se le aplica la matriz de transformación de Lorentz, y por tanto el escalar generado por el vector o tensor es invariante Lorentz.}, ya que no importa en que sistema de referencia nos encontremos, la distancia (que no es equivalente a la distancia espacial $\xn^2$) entre dos puntos es siempre la misma. Para definir la velocidad de un objeto tal que la distancia $V^2$ sea invariante Lorentz debemos usar un 4-vector velocidad adecuado. Por eso debemos introducir el \textbf{factor de Lorentz} se define como
\begin{equation}
	\gamma= \frac{1}{\sqrt{1-\frac{v^2}{c^2}}}
\end{equation}
donde $v$ es la velocidad del sistema de referencia y/o objeto respecto al sistema de referencia de interés. Si $v$ es la velocidad de un objeto respecto a nosotros, definimos como la cuadrivelocidad como

\begin{equation}
	V^\mu  \gamma \parentesis{c,\vn} 
\end{equation}
que verifica siempre 

\begin{equation}
	V^2 = V_\mu V^\mu = 1
\end{equation}
Definimos como momento a

\begin{equation}
	p^\mu = (p^0,\pn) = m V^\mu \qquad p^2 = m^2 c^2 
\end{equation}
Definimos como energía relativista al término $E\equiv cp^0$ tal que:

\begin{equation}
	E^2 = m^2 c^4 + p^2 c^2
\end{equation}
donde $mc^2$ es la energía en reposo. La energía cinética se define como 

\begin{equation}
	T=E-mc^2 = \parentesis{\gamma-1}mc^2
\end{equation}



\section{Electromagnetismo}

\subsection{Ecuaciones de Maxwell}

Las ecuaciones de Maxwell describen el electromagnetismo y sus interacciones con la materia a partir de las densidades de corriente.  Las ecuaciones en unidades del sistema internacional son:

\begin{equation}
	\begin{aligned}
		\text{Ley de Gauss para el campo eléctrico:} \quad & \nabla \cdot \mathbf{E} = \frac{\rho}{\varepsilon_0} \\
		\text{Ley de Gauss para el magnetismo:} \quad & \nabla \cdot \mathbf{B} = 0 \\
		\text{Ley de Faraday:} \quad & \nabla \times \mathbf{E} = -\frac{\partial \mathbf{B}}{\partial t} \\
		\text{Ley de Ampère-Maxwell:} \quad & \nabla \times \mathbf{B} = \mu_0 \mathbf{J} + \mu_0 \varepsilon_0 \frac{\partial \mathbf{E}}{\partial t}
	\end{aligned}
\end{equation}
donde $\varepsilon_0$ es la \textit{permitividad eléctrica} y $\mu_0$ la permeabilidad magnética. Definimos como el cuadrivector de corriente eléctrica $j^\mu$ a

\begin{equation}
	j^\mu (x) = \parentesis{c\rho(x),\jn(x)}
\end{equation}
donde $\rho$ es la densidad de carga eléctrica y $\jn$ la corriente de densidad de carga clásica. Bajo transformaciones de Lorentz los vectores $\En$ y $\Bn$ se mezclan, al igual que hacen $\rho$ y $\jn$. La invariancia Lorentz de las ecuaciones de Maxwell no es inmediata, por lo que trataremos de expresarlas a través del \textbf{cuadrivector potencial}

\begin{equation}
	A^\mu \equiv \parentesis{A^0 (x),\An(x)} \equiv \parentesis{\Phi(x)/c,\An(x)}
\end{equation}
donde $\phi$ es el potencial eléctrico y $\An$ el potencial vectorial, tal que

\begin{equation}
	\Bn = \curl \An \tquad \En = - \nabla \Phi - \parciales{\An}{t}
\end{equation}
El \textbf{tensor de campo electromagnético} viene definido

\begin{equation}
	F^{\mu \nu} (x) \equiv \partial^\mu A^\nu (x) - \partial^\nu A^\mu (x)
\end{equation}
Tal que $F^{\mu \nu} = - F^{\nu \mu}$. Podemos ver que:

\begin{equation}
	F^{0i} = \partial^0 A^i - \partial^i A^0 = \frac{1}{c} \parentesis{\parciales{A^i}{t} + \partial_i \Phi} = - \frac{E^i}{c}
\end{equation}
\begin{equation}
	F^{ij} = \partial^i A^j - \partial^j A^i = - \epsilon^{ijk}\epsilon^{mnk} \partial_m A^n =  - \epsilon^{ijk} B^k
\end{equation}
En resumen, nosotros podemos escribir las matrices de elementos $F^{\mu \nu}$ tal que

\begin{equation}
	F^{\mu \nu} = 
	\begin{pmatrix}
		0 & -E_x & -E_y & -E_z \\
		E_x & 0 & -B_z & B_y \\
		E_y & B_z & 0 & -B_x \\
		E_z & -B_y & B_x & 0
	\end{pmatrix}
\end{equation}
Las ecuaciones de Maxwell en este contexto son:

\begin{equation}
	\partial_\mu \tilde{F}^{\mu \nu} = 0 \tquad \partial_\mu F^{\mu \nu} = \mu_0 j^\nu
\end{equation}
donde $\tilde{F}=^{\mu \nu} $ se llama el \textit{dual del tensor de campo} y es un \textit{pseudotensor}:
\begin{equation}
	\tilde{F}^{\mu \nu} \equiv \frac{1}{2} \epsilon^{\mu \nu \rho \sigma} F_{\rho \sigma} \qquad 	{F}^{\mu \nu} \equiv - \frac{1}{2} \epsilon^{\mu \nu \rho \sigma} \tilde{F}_{\rho \sigma}
\end{equation}
tal que la primera parte encierra las leyes de Faraday y la ley magnética de Gauss, mientras que la segunda encierra la ley de Gauss eléctrica y la ley de Ampere-Maxwell. El dual viene dado por:
\begin{equation}
	\tilde{F}^{\mu \nu} = 
	\begin{pmatrix}
		0 & -B_x & -B_y & -B_z \\
		B_x & 0 & E_z & -E_y \\
		B_y & -E_z & 0 & E_x \\
		B_z & E_y & -E_x & 0
	\end{pmatrix}
\end{equation}
Además se verifica que:

\begin{equation}
	F_{\mu \nu} F^{\mu \nu} = - \tilde{F}_{\mu \nu} \tilde{F}^{\mu \nu} = 2 \ccorchetes{\Bn^2 - (\En^2/c^2)} \qquad  \tilde{F}_{\mu \nu} {F}^{\mu \nu} = (-4/c) \Bn \cdot \Bn
\end{equation}
Además recordemos que \textit{vector de Poyting} viene dado por

\begin{equation}
	\Sn = \frac{1}{\mu_0} \En \times \Bn
\end{equation}
Y la energía por densidad de volumen:

\begin{equation}
	U = \frac{1}{2} \ccorchetes{\epsilon_0 \En^2 + \frac{\Bn^2}{\mu_0}}
\end{equation}


\subsection{Transformaciones Gauge}