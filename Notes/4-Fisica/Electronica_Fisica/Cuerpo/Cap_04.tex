
\chapter{El transistor bipolar de unión}


\section{Introducción}


\section{Transistor bipolar ideal}

El \textbf{transistor bipolar ideal} es aquel transistor en el cual no sucede recombinación/generación ni en las regiones de vaciamiento EB y BC ni en la región Base. Además, consdieraremos que estamos en bajo nivel de inyección (BNI, $n_p, p_n \ll p_p,n_n$), que no hay campo eléctrico en las regiones masivas y el flujo es en una dimensión. Así pues, definimos los siguientes términos: 


\subsection{Corrientes}

Para calcular las corrientes usamos los mismos conceptos que en el tema anterior, suponiendo que la corriente de difusión será...

\subsection{Regiones de operacion}

\subsection{Recombinación}

\subsection{Caracterísitcas ideales IV en la zona activa}

\subsection{Ecuaciones de Ebers-Moll}


\section{Desviaciones respecto al transistor ideal}
