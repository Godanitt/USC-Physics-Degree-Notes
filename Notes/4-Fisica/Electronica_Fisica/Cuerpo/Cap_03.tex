\chapter{La unión PN}

Definimos como unión $pn$ a un semiconductor al cual se le aplica en una región un dopado tipo $p$ y en otra región un dopado tipo $n$, de tal modo que $N_A \gg N_D$ en $x<0$ y $N_D \gg N_A$ en $x>0$. El estudio y caracterización de la unión $pn$ es fundamental para entender la electrónica moderna. Existen 3 situaciones en las que podemos estudiar la unión pn, que son:

\begin{itemize}
    \item \textbf{Equlibrio termodinámico}, en el que no aplicamos una diferencia de potencial $V_A$ entre la  zona $p$ y zona $n$ externa.
    \item \textbf{Polarización directa} en el que aplicamos una diferencia de potencial $V_A>0$, es decir, polarizamos $p$ respecto $n$.
    \item \textbf{Polarización inversa} en el que aplicamos una diferencia de potencial $V_A<0$, es decir, polarizamos $n$ respecto $p$.
\end{itemize} 

\section{Equilibrio termodinámico}

\subsection{Introducción}

Entendemos por \textit{equilibrio termodinámico} a la situación en la que no hay corriente externa palicada, no hay iluminación y tampoco hay gradientes de la temperatura. Además consideramos las siguientes suposiciones:

\begin{itemize}
    \item Dispositivo unidimensional.
    \item Unión metalúrgica en $x_j=0$. 
    \item Contactos óhmicos perfectos.
\end{itemize} 
Debido a la aparición de esta diferencia de dopado aparece una región alrededor de $x=0$ en el que $n$ y $p$ se difunden, esto es, $n\neq N_D$ u $p\neq N_A$, tal que $n$ y $p$ dependen de $x$. A esta región la llamamos \textbf{zona de vaciamiento}, y en ella aparece un campo eléctrico no nulo en virtud de la ecuación de Maxwell-Poisson:

\begin{equation}
    \div \Encal = \frac{\rho}{K_S \varepsilon_0}
\end{equation}
siendo $\rho=-qn(x)+qp(x)$, y como $V(x)=-\int \Ecal \D x$, también aparece un potencial no cpnstate (siempre en la zona de vaciamiento). Como sabemos, existe una relación entre $V$ y las energías $E_i,E_v$ y $E_c$. 

\begin{equation*}
    q \parciales{V}{x} = - \parciales{E_i}{x}
\end{equation*}
y como $E_F$ es contante en el equlibrio. Definimos como \textbf{unión pn escalon} a aquella unión $pn$ en la que con regiones $p$ y $n$ dopados uniformemente, con un salto abrupto en $x=0$.. Es el caso más sencillo y más usado. El esquema de bandas es:


Cabe destacar que la caracterización de cualquier unión $pn$ (sea escalón, gradual...) es muy similar, presentando esquemas de bandas similares, solo cambiando la forma en la región de vaciamiento.


\subsection{Caracterización de la unión pn escalón}
Caracterizar la unión $pn$ es equivalente a conocer la densidad de carga, el campo eléctrico y el potencial a lo largo del dispositivo, a sí como la anchura de la región de vaciamiento. Muchos de estos valores están relacionados entre sí, y todos ellos dependen de el nivel de dopado $(N_D,N_A)$ del material ($E_g,n_i$) y de la temperatura ($T$). La mayor parte de estos los calcularemos a partir de aproximaciones. Lo que si podemos calcular sin hacer (casi) ninguna aproximación es $V_{bi}$. Hay dos maneras de calcular $V_{bi}$, a saber, por energías y por corrientes. El cálculo por energías es el más sencillo, y por el que nosotros optamos. Se define $V_{bi}$ como:



\begin{equation}
    V_{bi}= 
\end{equation}

\subsection{Caracterización de la unión pn gradual}

%%%%%%%%%%%%%%%%%%%%%%%%%%%%%%%%%%%%%%%%%%%%%%%%%%%%%%%%%%%%%%%%%%%%%%%
%%%%%%%%%%%%%%%%%%%%%%%%%%%%%%%%%%%%%%%%%%%%%%%%%%%%%%%%%%%%%%%%%%%%%%%
%%%%%%%%%%%%%%%%%%%%%%%%%%%%%%%%%%%%%%%%%%%%%%%%%%%%%%%%%%%%%%%%%%%%%%%
%%%%%%%%%%%%%%%%%%%%%%%%  EJERCICIOS %%%%%%%%%%%%%%%%%%%%%%%%%%%%%%%%%%
%%%%%%%%%%%%%%%%%%%%%%%%%%%%%%%%%%%%%%%%%%%%%%%%%%%%%%%%%%%%%%%%%%%%%%%
%%%%%%%%%%%%%%%%%%%%%%%%%%%%%%%%%%%%%%%%%%%%%%%%%%%%%%%%%%%%%%%%%%%%%%%
%%%%%%%%%%%%%%%%%%%%%%%%%%%%%%%%%%%%%%%%%%%%%%%%%%%%%%%%%%%%%%%%%%%%%%%


\section{Ejercicios}


\tcbstartrecording

\begin{texercise}
    Hola chicos
    \tcblower
    Solución. 
\end{texercise}


\tcbstoprecording

\section{Soluciones}

\tcbinputrecords







