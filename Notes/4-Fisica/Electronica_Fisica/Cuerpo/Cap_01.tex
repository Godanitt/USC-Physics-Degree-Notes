
\chapter{Física de semiconductores}


\section{Ejercicios}

\begin{texercise}
	Se dopa silicio con boro en una proporción de 2 ppm (partes por millón)
	\begin{enumerate}[label=\alph*)]
		\item Calcula la concentración intrínseca del Si y obten una expresión de dicha concentración en función de la temperatura.
		\item Indica el tipo de conducción de este material y calcula la concentración de impurezas y la de electrones y huecos (n y p) a temperatura ambiente si todos los dopantes están ionizados.
		\item Calcula la posición del nivel de Fermi y dibuja el diagrama de bandas completo correspondiente.
		\item ¿Qué pasa si la concentración de impurezas igualase el valor de \( N_V \)? Representa gráficamente las bandas de energía frente a la concentración de impurezas usando todas las aproximaciones que conozcas.
	\end{enumerate}
	(DATO: Constante de red del Si \( a_0 = 5.431 \) Å).

	\tcblower
	\begin{enumerate}[label=\alph*)]
		\item La concentración intrínseca del Silicio en un semiconductor es el número de portadores $n_i$ en el semiconductor, si no estuviera dopado. No hay que confundir la concentración intrínseca $n_i$ con la concentración $n$, en la que si se tendrá en cuenta que el material está dopado. La concentración intrínseca es:
		
		\begin{equation}
			n_i = \sqrt{N_cN_v} e^{-E_G/2kT}
		\end{equation}
		donde $E_G=E_c-E-v$, y además

		\begin{equation}
			N_{C,V} = 2 \parentesis{\frac{m_{e,p}^* kT}{2\pi\hbar^2}}^{-3/2}
		\end{equation}

		\item Están dopando con boro, que es del grupo III, y por tanto es un dador. Esto significa que será un conductor tipo $p$. Para calcular la concentración de impurezas, primero tenemos que obtener la densidad de Boro en nuestro silicio. La densidad del silicio se calcular a partir de la constante de red y sabiendo que posee una red diamante. Así pues:
		\begin{equation}
			N_{Si} = \frac{8}{a} = 
		\end{equation}
		de lo que se puede deducir entonces que:
		\begin{equation}
			N_B = 2\cdot 10^{-6} \cdot N_{Si}
		\end{equation}
		Nos dicen que todos los dopantes están ionizados, es decir, que estamos en el régimen extrínseco. En este régimen todos los átomos de Boro son impurezas, tal que $N_A^-=N_A=N_B$. Suponiendo que $N_A^- \gg N_D^+$, tenemos que la ecuación de neutralidad de carga:

		\begin{equation}
			n\cdot p = n_i^2 \tquad p-n-N_A = 0 
		\end{equation}
		usando estas ecuaciones para despejar el valor de $n$ y $p$, tenemos que:	
		\begin{equation}
			p = \frac{N_A}{2} + \ccorchetes{\parentesis{\frac{N_A}{2}}^2 + n_i^2}^{1/2} 
		\end{equation}
		y luego calculamos 

		\begin{equation}
			n = \frac{n_i^2}{p}
		\end{equation}

		\item La posición del nivel de Fermi de un semiconductor dopado se calcula a partir del nivel de Fermi intrínseco. Así pues
		\begin{equation}
			E_{Fi} = E_i = \frac{E_c+E_v}{2} + \frac{3 kT}{4} \ln \frac{m_p^*}{m_e^*} 
		\end{equation}
		tal que la energía de Fermi Es
		\begin{equation}
			E_F = E_i + kT \ln \parentesis{\frac{p}{n_i}}
		\end{equation}
		
		\item Cuando la concentración de impurezas es igual al valor de $N_V$, dado que $p=N_V e^{(E_v-E_F)/kT}$, esto implicaría que $E_v = E_F$, y que por tanto la condición de \textit{semiconductor no degnerado} $E_F>E_v + 3kT$ no se cumpliría. Consecuentemente, estamos ante un semiconductor degenerado. *Introducir imagen para las bandas*.
	\end{enumerate}
\end{texercise}

\begin{mybox}

	\begin{enumerate}[label=\alph*)]
		\item La concentración intrínseca del Silicio en un semiconductor es el número de portadores $n_i$ en el semiconductor, si no estuviera dopado. No hay que confundir la concentración intrínseca $n_i$ con la concentración $n$, en la que si se tendrá en cuenta que el material está dopado. La concentración intrínseca es:
		
		\begin{equation}
			n_i = \sqrt{N_cN_v} e^{-E_G/2kT}
		\end{equation}
		donde $E_G=E_c-E-v$, y además

		\begin{equation}
			N_{C,V} = 2 \parentesis{\frac{m_{e,p}^* kT}{2\pi\hbar^2}}^{-3/2}
		\end{equation}

		\item Están dopando con boro, que es del grupo III, y por tanto es un dador. Esto significa que será un conductor tipo $p$. Para calcular la concentración de impurezas, primero tenemos que obtener la densidad de Boro en nuestro silicio. La densidad del silicio se calcular a partir de la constante de red y sabiendo que posee una red diamante. Así pues:
		\begin{equation}
			N_{Si} = \frac{8}{a} = 
		\end{equation}
		de lo que se puede deducir entonces que:
		\begin{equation}
			N_B = 2\cdot 10^{-6} \cdot N_{Si}
		\end{equation}
		Nos dicen que todos los dopantes están ionizados, es decir, que estamos en el régimen extrínseco. En este régimen todos los átomos de Boro son impurezas, tal que $N_A^-=N_A=N_B$. Suponiendo que $N_A^- \gg N_D^+$, tenemos que la ecuación de neutralidad de carga:

		\begin{equation}
			n\cdot p = n_i^2 \tquad p-n-N_A = 0 
		\end{equation}
		usando estas ecuaciones para despejar el valor de $n$ y $p$, tenemos que:	
		\begin{equation}
			p = \frac{N_A}{2} + \ccorchetes{\parentesis{\frac{N_A}{2}}^2 + n_i^2}^{1/2} 
		\end{equation}
		y luego calculamos 

		\begin{equation}
			n = \frac{n_i^2}{p}
		\end{equation}

		\item La posición del nivel de Fermi de un semiconductor dopado se calcula a partir del nivel de Fermi intrínseco. Así pues
		\begin{equation}
			E_{Fi} = E_i = \frac{E_c+E_v}{2} + \frac{3 kT}{4} \ln \frac{m_p^*}{m_e^*} 
		\end{equation}
		tal que la energía de Fermi Es
		\begin{equation}
			E_F = E_i + kT \ln \parentesis{\frac{p}{n_i}}
		\end{equation}
		
		\item Cuando la concentración de impurezas es igual al valor de $N_V$, dado que $p=N_V e^{(E_v-E_F)/kT}$, esto implicaría que $E_v = E_F$, y que por tanto la condición de \textit{semiconductor no degnerado} $E_F>E_v + 3kT$ no se cumpliría. Consecuentemente, estamos ante un semiconductor degenerado.
	\end{enumerate}
\end{mybox}


\begin{texercise}
	Una muestra de Si está dopada con \( 6 \times 10^{15} \) átomos de As por cm$^3$
	\begin{enumerate}[label=\alph*)]
		\item ¿Cuál es la concentración de portadores en la muestra de Si a 300 K?
		\item ¿Cuál es la concentración de portadores a 470 K?
		\item Para cada una de las dos temperaturas anteriores determinar la posición de \( E_i \), calcular \( E_F - E_i \) y dibujar a escala el diagrama de bandas de energía para la muestra.
		\item Si dopamos el Si con \( 10^{16} \) átomos donadores y \( 5 \times 10^{15} \) átomos aceptores por cm$^3$. ¿Cuál es la concentración de portadores a 300 K?+
		\item Partimos de una muestra de Si puro y lo dopamos exclusivamente con \( 10^{14} \) átomos donadores y \( 10^{14} \) átomos aceptores por cm³. Calcula la concentración de portadores y explica el resultado obtenido.
	\end{enumerate}
	Tener en cuenta que \( E_G = 1.08 \) eV a 470 K y suponer que \( m_e^*/m_h^* = 0.69 \) es independiente de la temperatura.

	\tcblower
	Hola
\end{texercise}


\begin{texercise}
	Cuestiones sobre el nivel de Fermi:

	\begin{enumerate}
		\item Calcular la temperatura \( T \) para que el nivel de Fermi de un cristal de Silicio tipo N con \( N_D = 10^{16} \) cm\(^{-3}\) quedara a una energía \( E_G/3 \) por debajo de la banda de conducción. Suponer que \( N_C \) y \( E_G \) son constantes con la temperatura e iguales a sus valores a temperatura ambiente. Y repetir para el caso de dopar con \( N_D = 10^{18} \) cm\(^{-3}\).

		\item En un semiconductor determinado, la probabilidad de que los electrones ocupen un estado de energía \( kT \), por encima del extremo inferior de la banda de conducción es \( e^{-10} \). Determinar la posición del nivel de Fermi en dicho material.

		\item ¿Cuál es la probabilidad de que un estado de energía \( kT \) por debajo del nivel de Fermi esté ocupado por un hueco?
	\end{enumerate}

	\tcblower
	Hola
\end{texercise}

\begin{texercise}
	Cuestiones sobre la resistividad y movilidad:

\begin{enumerate}
	\item La resistividad de un material tipo N es por lo regular más pequeña que la resistividad de un material tipo P de dopado comparable, explica por qué suele ocurrir esto. Calcula la resistividad del Si si se dopa con fósforo con una concentración de \( 10^{17} \) cm\(^{-3}\). Repite el cálculo para el caso en que dopemos con aluminio con la misma concentración y calcular la corriente de arrastre en ambos casos considerando un campo eléctrico de \( 10^5 \) V/cm.

	\item Calcula la densidad de impurezas necesarias para tener un cristal de Si tipo P con resistividad 0.1 \(\Omega\cdot\)cm. ¿Qué proporción hay de átomos de impureza sobre el número de átomos de Si? (DATO: Constante de red del Si \( a_0 = 5.431 \) Å). Si suponemos que el semiconductor es no degenerado, ¿cuánto vale \( D_p \)?
\end{enumerate}
\tcblower
Hola
\end{texercise}






