\chapter{Física de semiconductores}

\section{Banda de valencia y conducción}

Un semiconductor es un material sólido que presenta dos bandas (en realidad presenta más, solo accesibles\footnote{Cuando decimos accesibles nos referimos a que existe una posibilidad no nula de que estén ocupadas.} a energías térmicas o electromagnéticas muy altas, y por tanto innecsarias para nuestro estudio). Para definir una banda primero tenemos que entender que los electrones en los sólidos se pueden describir mediante una suma de ondas planas, y por tanto la energía de estos se puede describir como una función del momento dse onda $E(k)$ (aunque no tiene porque ser necesariamente de la forma $E=\hbar^2 k^2/2m$). Así, debido al caracter fermiónico de los electrones y otros potenciales aparecen energías inaccesibles para nuestros electrones, i.e. para ningún $k$ existen esas energías. Así aparecen diferentes \textit{bandas}, limitadas por una \textit{energía superior} y una \textit{energía inferior}.

\begin{figure}[h!] \centering
	\includegraphics[width=0.7\textwidth]{Cuerpo/Ch_01/01_01.png}
	\caption{Bandas de conducción y valencia para algunos semicdonductores.}
\end{figure}

Como dijimos, un semiconductor posee dos bandas, separadas por una energía $E_g$. A la banda con energía superior se le llama \textbf{banda de conducción} (B.C) y a la banda inferior se le llama \textbf{banda de valencia} (B.V). Las bandas, a su vez, poseen diferentes líneas de dispersión, esto es diferentes relaciones $E(k)$. El número y forma de estas dependerá del tipo de material y la dirección de la onda, por esa misma razón solemos ver en la parte inferior de las bandas $\langle 111\rangle $ o $\langle 100\rangle$. Está denotando la dirección de la onda. En general suelen ser materiales muy simétricos, y por tanto con pocas direcciones representamos todas las posibles relacioens de dispersión.

La \textit{energía mínima de la banda de conducción} se denota como $E_c$, la \textit{energía máxima de la banda de valencia} se denota por $E_v$, y la diferencia entre el máximo de la BV y el mínimo de BC se denota por $E_g$:

\begin{equation}
	E_g = E_c - E_v
\end{equation}
En general se suele definir $E_v=0$ como referencia. Así la banda de valencia posee energías negativas, y la banda de conducción energías positivas. Veamos las características de las bandas:

\begin{itemize}
	\item \textbf{Banda de valencia:} el \textit{máximo siempre aparece en $k=0$}. Está dividida en 3 subbandas (3 relaciones de dispersión), 2 de ellas degeneradas en $k=0$ (son indistiguibles en $k=0$).
	\item \textbf{Banda de conducción:} está dividida en subbandas (aunque el número depende del material), y el valor de $k_{\min}$ tal que $E_{\min}=E(k_{\min})$ del dependerá del material.
\end{itemize}


\subsection{Semiconductores directos e indirectos}

Como hemos dicho, $E_c$ es el mínimo de la banda de conducción y $E_v$ es el máximo de la banda de valencia. En función del valor de $k_{\min}$ distinguimos dos tipos de semiconductores:
\begin{itemize}
	\item Definimos un \textbf{semiconductor indirecto} a aquel que verifica que $k_{\min}\neq0$. Es decir, el gap de energía sucede entre diferentes momentos (lo que hará que cuando se excite a un electrón de la BV tenga que darsele un momento).
	\item Definimos un \textbf{semiconductor directo} a aquel que verifica que $k_{\min}=0$. Es decir, el gap de energía sucede a $\Delta k =0$ (lo que hará que cuando se excite a un electrón de la BV no se pueda intercambiar momento).
\end{itemize}


\begin{figure}[h!] \centering
	\includegraphics[width=0.8\textwidth]{Cuerpo/Ch_01/01_02.png}
	\caption{A la izquierda semicondutor de gap indirecto y a la derecha de gap directo.}
\end{figure}


\subsection{Forma de los mínimos y máximos de banda: masa equivalente}

Cerca de los extremos de las bandas $E_0=E(k_0)$ tenemos que se puede aproximar la enerǵia por una función parabólica, tal que:

\begin{equation}
	E = E_0 + \frac{1}{2} \parentesis{\parciales{^2E}{k^2}}_{k_0} (k-k_0)^2
\end{equation}
Como podemos ver, esto se parece mucho a la expresión $E=\hbar^2k^2/2m$ que se usa para \textit{partículas libres}. Es decir, cerca de los extremos de las bandas, los \textit{portadores actúan como partículas libres con masa efectiva $m^*$} definida como

\begin{equation}
	(m^*)^{-1} = \frac{1}{\hbar^2} \parciales{^2E}{k^2}
\end{equation}

\begin{Anotacion}
	\textcolor{red}{Imagen de los mínimos/máximos.}
\end{Anotacion}

\subsection{Ecuación del movimiento}

La ecuación del movimiento para los electrones es una generalización de la ecuación de Newton usando el momento de onda del electrón $\kn$. Sabiendo que

\begin{equation}
	\Fn = m^* \an = m^* \dot{\vn} = \dot{\pn} = \hbar\dot{\kn}
\end{equation}
de lo que se deduce que

\begin{equation}
	\hbar \derivadas{\kn}{t} = \Fn
\end{equation}
pudiendo ser la fuerza, por ejemplo, la fuerza de Lorentz $\Fn=-e(\En+\vn \times \Bn)$, o cualquier otra.

%%%%%%%%%%%%%%%%%%%%%%%%%%%%%%%%%%%%%%%%%%%%%%%%%%%%%%%%%%%%%%%%%%%%%%%
%%%%%%%%%%%%%%%%%%%%%%%%  SECCIÓN 2 %%%%%%%%%%%%%%%%%%%%%%%%%%%%%%%%%%%
%%%%%%%%%%%%%%%%%%%%%%%%%%%%%%%%%%%%%%%%%%%%%%%%%%%%%%%%%%%%%%%%%%%%%%%
\section{Portadores: conecpto de hueco y electrón}

En los semiconductores hay 2 tipos de portadores, los portadores tipo hueco (o tipo $p$\footnote{Por el hecho de que se pueden describir como partículas con carga positiva.}) y tipo electrón (tipo $n$). La pregunta que nos deberíamos plantear en este momento es: ¿Qué sentido tiene que existan portadores tipo electrón/hueco si solo tenemos electrones en el semiconductor?¿Como que ``tipo electrón'', no deberían ser simplemente electrones? La respuesta es un tanto complicada. Como hemos dicho, cerca de los extremos las partículas, los electrones se comportan como electrones libres con masa efectiva $m^*$ (de hay viene \textit{tipo electrón} se comportan casi como electrones). Sin embargo esta ``masa efectiva'' tiene un problema en el máximo de la banda de valencia: la masa efectiva es negativa (recordemos que en un máximo la curvatura $\partial^2 E/ \partial k^2<0$). 

Para solucionarlo trabajemos en la siguiente idea. Supongamos que tenemos la capa de valencia llena salvo por un electrón, que se ha excitado y ha subido a la capa de conducción. En la banda de valencia habrá entonces $N$ electrones menos uno. La suma de los momentos de todos los electrones de la banda será entonces:

\begin{equation}
	\kn = \sum_{i=1}^{4N-1} \kn_i
\end{equation}
o lo que es lo mismo:

\begin{equation}
	\kn = \sum_{i=1}^{4N} (\kn_i) - \kn_e
\end{equation}
denotándolo por $\kn_e$ ya que es un electrón cualquiera (son indistiguibles). Como sabemos, la suma del momento de los electrones en una banda tiene que ser cero, ya que $k$ tiene tanto valores negativos y positivas, y están todos ocupados. Es decir, tenemos que el momento total de la banda será:

\begin{equation}
	\kn \equiv \kn_h = - \kn_e
\end{equation}
si a este momento total lo llamamos $\kn_p$ (momento de portador $p$ o hueco), tenemos que \textit{el movimiento efectivo de una capa sin un electrón es en el sentido opuesto al que tendría un electŕon individual en la misma}. A este artificio matemático lo llamamos hueco, y no es más que la manera de describir el comportamiento de una capa entera (capa de valencia) a través de unas pocas partículas. La energía también tendrá el signo opuesto, ya que:

\begin{equation}
	E_h = \sum_{i} E_i - E_e
\end{equation}
y como $\sum_{i}E_i$ es una constante, la podemos ignorar. Así pues, definimos la \textit{energía del hueco}

\begin{equation}
	E_h \equiv - E_e
\end{equation}
Y por tanto, aunque la masa efectiva de un electrón en el máximo de la banda de valencia sea negativa (en un máximo $\partial^2 E/ \partial k^2<0$), para el objeto matemático así definido tenemos que la masa efectiva será positiva, y la carga será positiva. Que la carga sea positiva no es tan obvio. Para ello tenemos que ver que la ecuación del movimiento

\begin{equation}
	\hbar \derivadas{\kn_h}{t} = -e(\En + \vn_h \times \Bn)
\end{equation}
hace que se comporte como un electrón con carga positiva $(\kn_h \equiv - \kn_e)$. 

La pregunta ahora es: ¿Tiene sentido físico? La respuesta es que sí. El sentido físico está en que cuando excitamos un electrón a la BC desde la BV, queda un estado sin ocupar, un hueco, en la banda de valencia. Este hueco podría ser rellenado por los electrones vecinos, que desde fuera lo veríamos como un movimiento efectivo del hueco, que además tendría el sentido contrario al del electrón (si los electrones saltan por ejemplo de izquierda a derecha por culpa de un campo eléctrico, veremos al hueco saltando de la derecha a la izquierda).

Consecuentemente los electrones en la banda de conducción se comportarán como electrones normales libres por la salvedad de que la masa efectiva será diferente; mientras que la falta de electrones en la banda de valencia (por la excitación de estos a la banda de conducción) se describirá a través de una serie de partículas con masa efectiva positiva diferente a la masa del electrón, carga positiva, con momento y energía efectiva de signo contrario al que tendría un electrón libre en la banda de valencia.


%%%%%%%%%%%%%%%%%%%%%%%%%%%%%%%%%%%%%%%%%%%%%%%%%%%%%%%%%%%%%%%%%%%%%%%
%%%%%%%%%%%%%%%%%%%%%%%%  SECCIÓN 3 %%%%%%%%%%%%%%%%%%%%%%%%%%%%%%%%%%%
%%%%%%%%%%%%%%%%%%%%%%%%%%%%%%%%%%%%%%%%%%%%%%%%%%%%%%%%%%%%%%%%%%%%%%%

\section{Densidad de portadores y clasificación de semiconductores}

En esta sección vamos a tratar de expresar la densidad de portadores hueco, denotado por $p$, y la densidad de portadores electrones, denotado por $n$. Primero estudiaremos el caso mas general posible, en función de la densidad de estados $g(E)$ y de la función de Fermi-Dirac $f(E)$. Luego iremos haciendo ciertas aproximaciones para simplificar los resultados.

\subsection{Densidad de portadores en el caso más general posible}

El caso más general, como ya hemos dicho, estudia el número de portadores $n$ y $p$ a partir de las integrales sobre las densidades de energía y función de Fermi. La \textit{densidad de estados} $g(E)$ es la distribución de los estados de energía a una energía dada, mientras que la \textit{función de Fermi} indica, en condiciones de equlibrio, la probabilidad de que un estado permitido de enerǵia $E$ esté ocupado por un electrón. La función de densidad de estados depende de la dimensión del sistema, tal que:

\begin{equation}
	g_{3D} (E) = \frac{\sqrt{2}m^{3/2}E^{1/2}}{\pi^2 \hbar^3} \quad g_{2D} (E) = \frac{m}{\pi \hbar^2} \quad g_{1D} (E) = \frac{\sqrt{2}m^{1/2}}{\pi \hbar \sqrt{E}}
\end{equation}
La dedución de estas densidades no es exageradamente complicada, véase apéndice \ref{Sec:A-01}. Por otra parte, la función de Fermi nos dice que la probabilidad de que cierto estado esté ocupado es:


\begin{equation}
	f(E) = \frac{1}{1+e^{(E-E_F)/kT}}
\end{equation}
donde $E_F$ es la enerǵia de Fermi, de la cual hablaremos más adelante. Naifmente uno podría pensar que la densidad $n/p$ sería simplemente la integral $g(E)f(E)$ entre diferentes intervalos de energía, ya que como sabemos los electrones se encuentran siempre en la banda de conducción, y por tanto entre las energías $E_{\min}$ y $E_c$, mientras que los huecos se encuentran en la banda de valencia, y por tanto entre las energías $E_{\max}$ y $E_v$.

Sin embargo esto no es correcto, por dos razones. La primera de ellas es que por culpa de la forma de parábola cerca del máximo y el mínimo de la banda de valencia y conducción respectivamente, es necesario redefinir el cero en el máximo local/mínimo local. La razón no es evidente a primera vista, pero uno lo puede entender cuando piensa que en un extremo local solo caben 2 electrones (uno con espín arriba y otro con espín abajo). Así pues, las densidades en la banda de valencia $g_v(E)$ y en la banda de conducción $g_c(E)$ son:

\begin{equation}
	g_c(E) = \frac{(m_n^*)^{3/2} \sqrt{2(E-E_c)}}{\pi^2 \hbar^3} \quad E \geq E_c \qquad 	g_v(E) = \frac{(m_p^*)^{3/2} \sqrt{2(E_v-E)}}{\pi^2 \hbar^3} \quad E \leq E_v
\end{equation}
La segunda razón es que si $f(E)$ es la probabilidad de que esté un estado ocupado con energía $E$ en el equilibrio, entonces $1-f(E)$ será la probabilidad de que no esté ocupado, y por tanto la probabilidad de que haya un hueco. Así pues, \textbf{las densidades de los portadores son}

\begin{equation}
	n=\int_{E_c}^{E_{\max}} g_c(E)f(E)\D E \qquad p = \int_{E_{\min}}^{E_v} g_v(E) (1-f(E))\D E
\end{equation}
que se puede aproximar para obtener las densidades en función de la llamada \textit{integral de Fermi-Dirac de orden 1/2} $F_{1/2}(\eta_c)$. La aproximación consiste básicamente en decir que $E_{\max} \rightarrow \infty$, y que por tanto

\begin{equation}
	n = \frac{1}{2\pi^2} \parentesis{\frac{2m_e^*}{\hbar^2}}^{3/2} (kT)^{3/2} F_{1/2}(\eta_c) \qquad F_{1/2}(\eta_c) = \int_0^\infty \frac{\eta^{1/2}}{1+e^{\eta-\eta_c}}\D \eta
\end{equation}
\begin{equation}
	\eta=\frac{E-E_c}{kT} \qquad \eta_c= \frac{E_F-E_c}{kT}
\end{equation}
(no hemos hecho todos los pasos para deducir la forma ya que no se va a usar en nigún momento).
Esta ecuación no tiene solución analítica, por tanto es necesario hacer ciertas aproximaciones si queremos trabajar con ella, lo cual trataremos en el siguiente apartado.

\subsection{Semiconductores no degenerados y degenerados}

La aproximación más interesante (y útil) es la llamada \textit{aproximación de Bolztmann}:

\begin{equation}
	f(E) = \frac{1}{1+e^{(E-E_F)/kT}} \approx e^{-(E-E_F)/kT}
\end{equation}
que nos permite obtener una solución muy sencilla a $F_{1/2}$:

\begin{equation}
	F_{1/2} (\eta_c) = \int_{0}^{\infty} \frac{\eta^{1/2}}{1+ e^{\eta-\eta_c}} \D \eta \simeq \frac{\sqrt{\pi}}{2} e^{\eta_c}
\end{equation}
Esta aproximación solo es válida cuando $E_c-3kT>E_F>E_v+3kT$, y por tanto cuanta más alta la temperatura más restringida será su aplicación. Cuando un semiconductor verifica estas condiciones para una temperatura dada decimos que se encuentra en el \textit{régimen no degenerado}, mientras que cuando \textit{no} verifica dichas condiciones decimos que está en el \textit{régimen degenerado}.

\begin{figure}[h!] \centering
	\includegraphics[width=0.85\textwidth]{Cuerpo/Ch_01/01_03.png}
	\caption{Régimen degenerado y no degenerado.}
\end{figure}


En general nosotros trabajaremos únicamente con semiconductores no degenerados, y por tanto en el rango de validez de la aproximación de Bolztmann, la que da como resultado las siguientes densidades de portadores:

\begin{equation}
	p=N_v e^{(E_v-E_F)/kT}  \qquad n = N_c e^{(E_F-E_c)/kT}
\end{equation}
donde $N_c$ y $N_v$ son las llamadas \textbf{densidades equivalentes de los estados de la banda de valencia y conducción}, con la siguiente forma:

\begin{equation}
	N_c = 2 \parentesis{\frac{m_n^* kT}{2\pi\hbar^2}}^{3/2} \tquad
	N_v = 2 \parentesis{\frac{m_p^* kT}{2\pi\hbar^2}}^{3/2}
\end{equation}
Para una temperatura aproximada de $300$K tenemos que

\begin{equation}
	N_{C,V} = (2.509\times10^{19} \cm^{-3}) \parentesis{\frac{m_{n,p}^*}{m_0}}^{3/2}
\end{equation}
siendo $m_0$ la masa del electrón.
\subsection{Semiconductores intrínsecos caso no degenerado}

Definimos como \textbf{semiconductor intrínseco} a un material semiconductor extremadamente puro, sin dopantes, cuyas propiedades solo dependan del material. En este tipo de materiales el número de electrones es igual al número de huecos (en virtud de la neutralidad electrónica: el número de electrones en la banda de conducción será el mismo el número que electrones faltan en la banda de valencia, i.e. el número de huecos). Matemáticamente se expresa como

\begin{equation}
	n = p = n_i
\end{equation}
y se llama \textit{condición intrínseca}. Siempre que estudiemos semiconductores intrínsecos lo haremos a través de los semiconductores no degenerados, ya que de cualquier otra forma no podemos tener expresiones analíticas. Denotamos $E_i$ al \textbf{nivel de fermi intrínseco}. Usando la ecuación anterior, obtenemos que:

\begin{equation}
	n_i = \sqrt{np} = \sqrt{N_CN_V} e^{-E_g/2kT} 
\end{equation}
donde $E_g=E_c-E_v$ y se le llama \textit{energía de gap}. El valor de $n_i$ para un semicdonductor dado es muy importante, incluso cuando está dopado. La razón es que siempre podemos expresar $n$ y $p$  en función de $n_i$ (a la misma temperatura), ya que si $n=p=n_i$:

\begin{equation*}
	n_i = N_C e^{(E_i-E_c)/kT} = N_V e^{(E_v-E_i)/kT}  \Rightarrow
\end{equation*}
\begin{equation}
	N_C=n_i e^{(E_c-E_i)/kT} \quad N_V = n_i e^{(E_i-E_v)/kT} \label{Ec:01-3-13}
\end{equation}
tal que
\begin{equation}
	n=n_i e^{(E_F-E_i)/kT} \qquad p = n_i e^{(E_i-E_F)/kT} \label{Ec:01-3-14}
\end{equation}
lo cual nos está dando en realidad una información muy relevante: en función del nivel de fermi, i.e., si $E_F>E_i$ o $E_F<E_i$, podremos saber si el conductor es de tipo $n$ o tipo $p$ (solo cuando $E_F=E_i$ tenemos $n=p$, precisamente la condición intrínseca). Además tenemos que de la expresión anterior podemos deducir la llamada \textbf{ley de acción de masas} (que se verifique siempre que estemos en el rango no degenerado):

\begin{equation}
	np=n_i^2
\end{equation}
Si nos damos cuenta a partir de las ecuaciones \ref{Ec:01-3-13} podemos deducir una \textit{expresión para la posición del nivel de Fermi intrínseco $E_i$}. Para esto partimos de las ecuaciones \ref{Ec:01-3-14}, de lo que se deduce que:

\begin{equation}
	E_i = \frac{E_c+E_v}{2} + \frac{kT}{2} \ln \parentesis{\frac{N_C}{N_V}} = \frac{E_c+E_v}{2} + \frac{3}{4} kT \ln \parentesis{\frac{m_p^*}{m_n^*}}
\end{equation}
Incluso podemos obtener la \textit{expresión para la posición del nivel de Fermi $E_F$} para el caso más general. Para esto partimos de las ecuaciones

\begin{equation}
	\ln \parentesis{\frac{n}{n_i}} = \frac{1}{kT} \parentesis{E_F-E_i} \Rightarrow E_F = E_i + kT \ln \parentesis{\frac{n}{n_i}}
\end{equation}
\begin{equation}
	\ln \parentesis{\frac{p}{n_i}} = \frac{1}{kT} \parentesis{E_i-E_F} \Rightarrow E_F = E_i - kT \ln \parentesis{\frac{p}{n_i}}
\end{equation}
Siendo expresiones completamente compatibles (si se verifica una se verifica la otra) además de que mantienen la relación citada antes entre el nivel de Fermi y el número de portadores huecos/electrón.

\subsection{Semiconductores extrínsecos caso no degenerado}

Definimos como \textbf{semiconductor extrínseco} o \textbf{semiconductor dopado} a un material semiconductor al que se le han insertado átomos de otro grupo. Pero antes es importante preguntarse por qué se incluyen estos átomos en nuestro semiconductor, y cuáles son sus ventajas. La respuesta todavía no podemos darla de manera muy profunda, sin embargo si podemos decir lo siguiente: a temperaturas ambiente, la cantidad de portadores tipo $n$ y tipo $p$ intrínsecas son muy bajas, y por tanto habrá una conductividad muy baja. Cuando dopamos un semiconductor no solo estamos aumentando el número de portadores de un tipo, estamos aumentando la conductividad. De hecho, al ser capaces de controlar el nivel de dopado, podemos elegir la conductividad arbitrariamente, pudiendo optimizar y controlar totalmente las propiedades eléctricas.

Así pues, tenemos dos tipos de dopantes, que además definiran el tipo de portador mayoritario que tendremos. Tenemos pues:

\begin{itemize}
	\item \textbf{Dopante dador}. Los dopantes dadores o dadores aportan electrones a la banda de conducción (por lo general son elementos del grupo V, aportando un electrón), lo que hará que el portador mayoritario sea el portador $n$. A la concentración de dadores la denotamos por $N_D$. Entre ellos encontramos el fósforo (P), el arsénico (As) y el antimonio (Sb).
	\item \textbf{Dopante aceptor}. Los dopantes aceptores o aceptores aportan heucos a la banda de valencia (por lo general son elementos del grupo III, aportando un hueco), lo que hará que el portador mayoritario sea el portador $p$. A la concentración de dadores la denotamos por $N_A$. Entre ellos encontramos el boro (B), el aluminio (Al) y el galio (Ga).
\end{itemize}

\begin{figure}[h!] \centering
	\includegraphics[width=0.9\textwidth]{Cuerpo/Ch_01/01_04.png}
	\caption{Funcionamiento de los conductores tipo $n$.}
\end{figure}

\begin{figure}[h!] \centering
	\includegraphics[width=0.9\textwidth]{Cuerpo/Ch_01/01_05.png}
	\caption{Funcionamiento de los conductores tipo $p$.}
\end{figure}

Sin embargo no es igual el número de impurezas $N_D$ y $N_A$ al número de portadores que aportan $N_D^+$ y $N_A^-$. Pensemos por ejemplo el caso de los elementos del grupo V. Para que puedan aportar el quinto electrón es necesario romper el enlace que lo une con dicho átomo, es decir, hace falta ionizarlo, lo que se hace a través de la aportación de energía, por ejemplo energía térmica. A partir de cierta temperatura todas las impurezas están ionizadas (temperatura ambiente). Sin embargo no siempre será así, y si la energía térmica no es suficiente para ionizar todas las impurezas, debemos usar las siguientes expresiones:

\begin{equation}
	N_D^+ = \frac{N_D}{1+g_D e^{(E_F-E_D)/kT}} \tquad
	N_A^- = \frac{N_A}{1+g_D e^{(E_A-E_F)/kT}}
\end{equation}
donde $E_D$ y $E_A$ son las correspondientes energías de ionización. Al igual que antes tenemos la condición de electroneutralidad, aunque ahora va a cambiar un poco: tenemos que considerar que $N_D^+$ y $N_A^-$ aportan carga. Así pues, la \textbf{condición de electroneutralidad para extrínsecos} es:

\begin{equation}
	p + N_D^+ = n + N_A^-
\end{equation}
lo cual tiene todo el sentido del mundo: si tenemos $N_D^+>N_A^-$ (mas dadores que aceptores) lógicamente habrá más portadores tipo $n$ que tipo $p$. Dado que la ley de acción de masas $np=n_i^2$ se cumple \textit{para cualquier semiconductor no degenerado}, para cualquier conductor no degenerado extrínseco podemos conocer $n$ y $p$ en función de $n_i,N_D^+$ y $N_A^-$ (que surge tras despejar una ecuación de segundo grado):

\begin{equation}
	n = \frac{N_D-N_A}{2} + \ccorchetes{\parentesis{\frac{N_D-N_A}{2}}^2 + n_i^2}^{1/2} \tquad p = \frac{N_A-N_D}{2} + \ccorchetes{\parentesis{\frac{N_A-N_D}{2}}^2 + n_i^2}^{1/2}
\end{equation}
Los 3 casos más sencillos que nos pueden plantear son los siguientes:

\begin{itemize}
	\item Cuando $N_D^+=N_A^-$ tenemos que
	      \begin{equation}
		      n=p=n_i
	      \end{equation}
	\item Cuando $N_D^+ \gg N_A^-,n_i$. En este caso tenemos las siguientes ecuaciones:
	      \begin{equation}
		      n=N_D^+ \tquad p = \frac{n_i^2}{N_D^+}
	      \end{equation}
	\item Cuando $N_A^-\gg N_D^+,n_i$. En este caso tenemos las siguientes ecuaciones:
	      \begin{equation}
		      p=N_A^- \tquad n = \frac{n_i^2}{N_A^-}
	      \end{equation}
\end{itemize}
El resto de casos habrá que calcularlos aparte.

\subsection{Semiconductores extrínsecos: régimen intrínseco y extríseco}

Definimos como \textbf{régimen extrínseco} de un semiconductor extrínseco\footnote{Muchas veces, cuando se dice que está en el semicdonductor está en el régimen extrínseco ya se asume que está dopado, y por tanto se obvia.} aquel intervalo de temperaturas (aunque puede ser otra variable) en el que todas las impurezas están ionizadas y se verifica que $N_D^+$ o $N_A^-$ es mucho mayor que $n_i$. Definimos como \textbf{régimen intrínseco} aquella región de temperaturas en la cual el nivel de impurezas excitadas es comparable o menor al número de portadores excitados por fluctuaciones térmicas $n_i$. Cuando la temperatura es baja y no están excitados todas las impurezas, decimos que estamos en el régimen de \textit{freeze out}.

Definimos como \textbf{temperatura intríseca} a la temperatura que separa el régimen extrínseco e intrínseco, y se define como aquella temperatura para la cual $n(T_i)=2N_D$ o $p(T_i)=2N_A$ en función de si es dador o aceptor el dopante.


\begin{figure}[h!] \centering
	\includegraphics[width=0.85\textwidth]{Cuerpo/Ch_01/01_06.png}
	\caption{Régimenes intrínsecos y extrínsecos en función de la temperatura.}
\end{figure}


%%%%%%%%%%%%%%%%%%%%%%%%%%%%%%%%%%%%%%%%%%%%%%%%%%%%%%%%%%%%%%%%%%%%%%%
%%%%%%%%%%%%%%%%%%%%%%%%  SECCIÓN 4 %%%%%%%%%%%%%%%%%%%%%%%%%%%%%%%%%%%
%%%%%%%%%%%%%%%%%%%%%%%%%%%%%%%%%%%%%%%%%%%%%%%%%%%%%%%%%%%%%%%%%%%%%%%

\section{Datos interesantes}

Un semiconductor de Germanio y uno de Silicio tiennen una estructura formado por una \textbf{red de diamante}, mientra que el fosfuro de indio (InP) y el arsenuro de galio (GaAs) tienen una \textbf{blenda de zinc}.







