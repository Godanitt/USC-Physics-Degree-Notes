    Vamos a calcular las masas efectivas en función de las movilidades y las vidas medias:

    \begin{equation}
        \mu_n = D_n \frac{q}{kT} = \frac{q}{kT} \tau_n v^2_{th} = \frac{q}{kT} \tau_n \frac{3kT}{m_n^*} = \frac{3q\tau_n}{m_n^*}
    \end{equation}
    Entonces:
    \begin{equation}
        m_n^* = \frac{3q\tau_n}{\mu_n} \qquad m_p^* = \frac{3q\tau_p}{\mu_p^*}
    \end{equation}
    \begin{enumerate}[label=\alph*)]
        \item Tenemos que calcular, en la situación de
        Vamos a calcular las masas efectivas en función de las movilidades y las vidas medias:
        \begin{equation}
            \mu_n = D_n \frac{q}{kT} = \frac{q}{kT} \tau_n v^2_{th} = \frac{q}{kT} \tau_n \frac{3kT}{m_n^*} = \frac{3q\tau_n}{m_n^*}
        \end{equation}
        Entonces:
        \begin{equation}
            m_n^* = \frac{3q\tau_n}{\mu_n} \qquad m_p^* = \frac{3q\tau_p}{m_p^*}
        \end{equation}
        equlibrio, la anchura de todas las regiones del dispositivo y las bandas de energía (banda de conducción, banda de valencia, nivel de Fermi y nivel de Fermi intrínseco). También tenemos que calcular las distancias relativas entre los niveles (lo cual es obvio dado lo anterior).

        Primero tenemos que calcular las distancias, lo cual es simplemente aplicar las fórmulas para la situación de equilibrio
        \begin{equation}
            x_p = \ccorchetes{\frac{2K_S\varepsilon_0}{q} \frac{N_D}{N_A(N_A+N_D)}  V_{bi}}   \qquad
            x_n = \ccorchetes{\frac{2K_S\varepsilon_0}{q} \frac{N_A}{N_D(N_A+N_D)}  V_{bi}}
        \end{equation}
        Así pues, los valores numéricos son:

        \begin{equation}
            x_p =  \ [\cm] \tquad x_n = \ [\cm]
        \end{equation}
        Y luego tenemos que calcular los valores de todas y cada una de las bandas. Para conocer las banda, teniendo en cuenta que $E_F=0 \ [\eV]$  \textit{a lo largo de todo el dispositivo pn}. Por el resto simplemente aplicar las ecuaciones de la sección 2, tal que en la zona $p$ los valores son los que típicamente esperaríamos para un semiconductor $N_A$, mientras que en la zona $n$ será los que esperaríamos en un conductor $N_A$ menos $V_{bi}$. En la \textit{zona de vaciamiento} los valores de las bandas simplemente valdrán su valor en $p$ menos el valor $V(x)$:
        \begin{equation*}
            V_{bi} = \frac{kT}{q} \ln \parentesis{\frac{N_AN_D}{n_i^2}}
        \end{equation*}
        \begin{equation*}
            V(x) = \left\lbrace \begin{array}{ll}
                - \frac{qN_A}{2K_S\varepsilon_0} \parentesis{x_p - x}^2  & \ - x_p \leq x \leq 0 \\
                \frac{qN_D}{2K_S\varepsilon_0} \parentesis{x_n - x}^2 + V_{bi}  & \ 0 \leq x \leq x_n \\
            \end{array} \right.
        \end{equation*}
        Por el resto de situaciones, tenemos que en la zona $p$ las ecuaciones son:

        \begin{equation*}
            E_i = - kT \ln \parentesis{\frac{n}{n_i}} = kT \ln \parentesis{\frac{N_A}{n_i}} \qquad E_c  = E_i  + kT \ln \parentesis{\frac{N_c}{n_i} } \qquad E_v  =E_c-E_g
        \end{equation*}
        tal que

        \begin{equation*}
            E_g = 1.12 \ [\eV] \qquad N_C = 2 \parentesis{\frac{m_n^* kT}{2\pi \hbar^2}}^{3/2}  \ [\cm^{-3}] \qquad  N_V = 2 \parentesis{\frac{m_p^* kT}{2\pi \hbar^2}}^{3/2} \ [\cm^{-3}]
        \end{equation*}
        \begin{equation*}
            n_i = \sqrt{N_CN_V} e^{-E_g/kT}
        \end{equation*}
        Así pues obtenemos los siguientes resultados numéricos:
    \end{enumerate}
