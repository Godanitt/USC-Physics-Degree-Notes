	Primero tenemos que evaluar si son dadores/aceptores, calcular $n$ y $p$, luego evaluar las fórmulas para portadores mayoritarios y minoritarios, y finalmente la resistividad. Para calcular $n$ y $p$:
	\begin{enumerate}[label=\alph*)]
		\item El Boro es un átomo aceptor, por lo que tendremos como dato $N_A=10^{15} \ $átomos/$\cm^3$. Ahora calculamos, usando que $n_i=1.18\times 10^{10}$:
		\begin{equation}
			p = N_A = 10^{15} \qquad p = \frac{n_i^2}{p} =
		\end{equation}
		Calculamos las movilidades, usando la mayoritaria para los huecos y la minoritaria para los electrones:
		\begin{equation}
			\mu_n = \qquad \mu_p =
		\end{equation}
		La resistividad $\rho$ se calcula ahora fácilmente:
		\begin{equation}
			\rho = \frac{1}{e(\mu_n n + \mu_pp)}
		\end{equation}

		\item El Arsénico es un átomo dador, por lo que tendremos como dato $N_{D\text{eff}}=5 \times 10^{15} \ $átomos/$\cm^3$, que se deduce de:
		\begin{equation}
			N_{D\text{eff}} = N_{\text{As}} - N_{\text{B}} = 5 \times 10^{15}
		\end{equation}
		Ahora calculamos, usando que $n_i=1.18\times 10^{10}$:
		\begin{equation}
			n = N_{D\text{eff}} = 10^{15} \qquad p = \frac{n_i^2}{n} =
		\end{equation}
		Calculamos las movilidades, usando la mayoritaria para los electrones y la minoritaria para los huecos:
		\begin{equation}
			\mu_n = \qquad \mu_p =
		\end{equation}
		La resistividad $\rho$ se calcula ahora fácilmente:
		\begin{equation}
			\rho = \frac{1}{e(\mu_n n + \mu_pp)}
		\end{equation}

		\item El Galio es un átomo aceptor, por lo que tendremos como dato $N_{A\text{eff}}=5 \times 10^{15} \ $átomos/$\cm^3$, que se deduce de:
		\begin{equation}
			N_{A\text{eff}} = N_{\text{Ga}} + N_{\text{B}} - N_{\text{As}} = 5 \times 10^{15}
		\end{equation}
		Ahora calculamos, usando que $n_i=1.18\times 10^{10}$:
		\begin{equation}
			p = N_{A\text{eff}} = 5 \times 10^{15} \qquad p = \frac{n_i^2}{n} =
		\end{equation}
		Calculamos las movilidades, usando la mayoritaria para los huecos y la minoritaria para los electrones:
		\begin{equation}
			\mu_n = \qquad \mu_p =
		\end{equation}
		La resistividad $\rho$ se calcula ahora fácilmente:
		\begin{equation}
			\rho = \frac{1}{e(\mu_n n + \mu_pp)}
		\end{equation}
	\end{enumerate}
