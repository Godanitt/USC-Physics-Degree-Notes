
	Nos dan $E_c-E_T=0.530\eV$, nos dan $N_T$, $\rho$, $\tau_n$ y $\tau_p$. Son bastantes datos, por lo que es normal liarse, así que tenemos que tener muy claro que nos están preguntado y las aproximaciones que podemos hacer. No nos dan la temperatura, por lo que asumimos temperatura ambiente $300K$. Por ejemplo, al darnos $E_T$ respecto $E_c$, podemos deducir $E_T-E_i$, y ver si podemos usar la aproximación a niveles profundos. Veamos que si $E_c=E_g$ y $E_v=0$:
	
	\begin{equation}
		E_i = \frac{E_g}{2} + \frac{3}{4} kT \ln \parentesis{\frac{m_n^*}{m_p^*}} = 0.55 \ \eV
	\end{equation}
	donde $E_g=1.12$ eV, $m_n=1.18m_e$ y $m_p=0.81m_e$. Así $E_i=$ eV y por tanto:

	\begin{equation}
		E_T - E_i = E_g - 0.530 - E_i =0.037 \ \eV		
	\end{equation}
	que es comparable a $kT$ y por tanto no suficiente como para hacer la aproximación a niveles profundos $n_1=p_1=n_i$, y hay que calcularlos con las siguientes fórmulas:

	\begin{equation}
		n_1 = n_i e^{(E_T'-E_i)/kT} 	\qquad
		p_1 = p_i e^{(E_i-E_T')/kT}
	\end{equation}

	\begin{enumerate}[label=\alph*)]
		\item Queremos calcular la tasa de recombinación, es decir, $R$, en una zona sin portadores. Esto significa que la \textit{aproximación a  semiconductor vacío de portadores es válido}. Es decir, tenemos que:
		\begin{equation}
			R = \frac{np-n_i}{\tau_p (n+n_1)+\tau_n (p+p_1)} \approx -\frac{n_i^2}{\tau_p n_1 + \tau_n p_1}
		\end{equation}
		Entonces tenemos que
		\begin{equation}
			p_1 = 4.99\cdot 10^{10} \ \cm^{-3} \qquad n_1= 2.79\cdot 10^{9} \ \cm^{-3}
		\end{equation}
		\textcolor{red}{A lucia le da diferente, $n1=4.23 10^10$ y $p1=2.36 10^9$. También puso $E_i-E_T=-0.0373 \eV$ y $E_i-E_c=-0.5673\eV$}.
		\begin{equation}
			R = -1.96\cdot 10^{17} \ s^{-1}\cm^{-3}
		\end{equation}
		\textcolor{red}{A lucia le da $-7.38\cdot 10^{16}$, aunqeu la fórmula es igual}. Domina la generación al tener $R<0$. No existen portadores móviles, por lo que no peude haber destrucción, solo generación.
		\item Suponemos que solo los portadores minoritarios han desaparecido ($p\approx 0$), mientras que $n$ es similiar al equlibrio $n\approx n_0$. Tenemos ahora que:
		\begin{equation}
			R = - \frac{n_i^2}{\tau_p(n_0+n_1)+\tau_n(p_1)}
		\end{equation}
		Calculando $n$ a partir de $\rho$ usando que $n = 1/q\mu_n\rho$. Como sabemos:

		\begin{equation}
			\mu_n = D_n  \frac{q}{kT} = \frac{q}{kT}  \tau_n {v_{th}^2} =  \frac{q}{kT}  \tau_n \frac{3kT}{m_n^*} = \frac{3q\tau_n}{ m_n^*}
		\end{equation}
		Tal que

		\begin{equation}
			\mu_n=5.59 \cdot 10^3  \ \cm^2 /  \text{V s}  \qquad n_0 = \frac{m_n^*}{q^2\rho \tau_n} = 2.23 \cdot 10^{15} \ \cm^{-3}
		\end{equation}
		Y por tanto
		\begin{equation}
			R = -4.99 \cdot 10^{12} \ s^{-1}\cm^{-3}
		\end{equation}
		\textcolor{red}{Tenemso que $R=-3.4803\cdot 10^{11}$}. Domina generación: ¿Tiene sentido? La respuesta es que sí: no hay portadores minoritarios, por lo que tienen que se generados por los procesos RG, haciendo que domina la tasa de generación.
	\end{enumerate}	

