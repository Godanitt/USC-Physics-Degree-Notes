	\begin{enumerate}[label=\alph*)]
		\item Nos dicen que $N_D=N_0 \exp(-ax)$, y queremos calcular $E(x)$. Dependerá del nivel de profundidad que quieres darle al ejercicio. Suponemos, al no dar datos de los procesos recombinación-generación, que estamos en el \textit{modelo de arrastre-difusión}. Así pues, tenemos que:
		\begin{equation}
			\parciales{n}{t} = \div \Jn = 0 \rightarrow \div (\Jn_{\arr}+\Jn_{\diff}) = 0 \rightarrow J_{\arr}+J_{\diff} = 0
		\end{equation}
		tal que (pasamos de vectorial a escalar)
		\begin{equation}
			J_{\arr}+J_{\diff} = 0 \rightarrow q \mu_n n \Ecal + q D_n \derivadas{n}{x} = 0 \rightarrow \Ecal = - \frac{D_n}{\mu_n} \frac{1}{n} \derivadas{n}{x}
		\end{equation}
		donde usaremos que $n\sim N_D$ y que $D_n / \mu_n = kT/q$ es la \textit{relación de Einstein}. Así pues:

		\begin{equation}
			\Ecal = \frac{kT}{q} \frac{1}{n} \derivadas{n}{x} = \frac{kT}{q} \frac{1}{a}
		\end{equation}
		De lo que se deduce que:

		\begin{equation}
			\Ecal(x) = 2.59 \cdot 10^{4} \ V/m
		\end{equation}
		\item Considerando que estamos en equlibrio térmico $T=\cte$ y en equilibrio eléctricio $\Jn_T=0$. Igual que antes:
		\begin{equation}
			\Ecal = - \frac{kT}{q} \derivadas{(\ln(n))}{x}
		\end{equation}
		y ahora podemos hallar la diferencia de potencial total:

		\begin{equation}
			\Delta V = - \int_{0}^L \Ecal \D x
		\end{equation}
		Esto es:
		\begin{equation}
			\Delta V = \frac{kT}{q} \ccorchetes{\ln(n(L))-\ln(n(0))} = \frac{kT}{q} \ln\parentesis{\frac{n(L)}{n(0)}}
		\end{equation}
	\end{enumerate}
