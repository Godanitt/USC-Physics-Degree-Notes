
	\begin{enumerate}[label=\alph*)]
		\item	Tenemos que calcular la masa efectiva de los huecos para dos semiconductores diferentes, dado su densidad efectiva de estados en la banda de valencia. Esto significa que necesitamos invertir la fórmula típica, tal que
		\begin{equation}
			m_p^* = N_V  \frac{1}{2} \parentesis{\frac{2 \pi \hbar^2}{kT}}^{3/2}
		\end{equation}
		De lo que se deduce que para el Si y el GsAs:

		\begin{equation}
			\text{Si:} m_p^* = 7.38\cdot10^{-31} \text{kg} = 0.81 \ {m}_e
		\end{equation}
		\begin{equation}
			\text{GaSi:}m_p^* = 4.59 \cdot10^{-31} \text{kg} = 0.51 \ {m}_e
		\end{equation}
		\item Tenemos que calcular la posición del nivel intrínseco $E_i$ a la temperatura ambiente (300 K) y a 1000 $^\circ C$, asumiendo que $E_g$ es constante. Veamos que solo es aplicar una fórmula:
		\begin{equation}
			E_i = \frac{E_c+E_v}{2} + \frac{3}{4} kT \ln\parentesis{\frac{m_p^*}{m_n^*}}
		\end{equation}
		donde hemos considerado que $E_g=1.12$ en el silicio, y que $E_v=0$, ergo $E_c=1.12$. Hacemos la representación gráficamente
		\begin{center}
			\includegraphics[width=0.6\textwidth]{Cuerpo/Ch_01/Ejercicio_01_5.pdf}
		\end{center}
		Considerando esta imagen parece razonable, con ciertas puntualizaciones, considerar que $E_i$ está en medio de la banda prohibida.
	\end{enumerate}
	
