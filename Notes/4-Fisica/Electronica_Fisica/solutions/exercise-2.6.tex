	\begin{enumerate}[label=\alph*)]
		\item Tenemos un semiconductor tipo $N$ a $N_D$ dado, $G_L$ y tiempos de vida medios, y queremos calcular $n$ y $p$ en el estado estacionario, esto es, queremos calcualr $n$ cuando $R=G$. Como sabemos, $G=G_{th}+G_L$, y que $n=N_D$ (estamos a 300K, podemos considerar que todas las impurezas están ionizadas) y por tanto que $n_{n0}\gg p_{n0}$ (recordemos que $n_{n0}$ y $p_{n0}$ denotan concetración de portadores cuanod no hay luz), ya que:
		\begin{equation}
			n_{n0} = N_D \tquad p_{n0} = \frac{n_i^2}{n_{n0}} = 1.39 \cdot 10^5 \ \cm^{-3}
		\end{equation}
		donde $n_i=1.18\times 10^{10} \cm^{-3}$ para el Si a 300K. Entonces solo tenemos que usar la ecuación para $p_{n0}\ll n_{n0}$:

		\begin{equation}
			G_L = U = \frac{p_n-p_{n0}}{\tau_p}
		\end{equation}
		De lo qeu se deduce que \textit{la concetración de portadores} $p_n$ es:

		\begin{equation}
			p_n = G_L \tau_p + p_{n0} = 1.00 \cdot 10^{13} \ \cm^{13}
		\end{equation}
		y usando que $\Delta n = \Delta p$, tenemos que:
		\begin{equation}
			n_n = n_{n0} + \Delta p  = n_{n0} + G_L\tau_p = 1.01  \cdot 10^{15} \ \cm^{13}
		\end{equation}
		\item Para dibujar el diagrama de Bandas solo tenemos que calcular en nivel de fermi $E_F$ para las concetraciones nuevas. Así pues, usamos que:
		\begin{equation}
			E_F = \frac{E_c+E_v}{2} + \frac{3}{4} kT \ln \parentesis{\frac{m_n^*}{m_p^*}} - kT \ln \parentesis{\frac{p}{n_i}}
		\end{equation}
		Usando que $E_i= \ \eV$, los valores de la energía de Fermi:

		\begin{equation}
			\text{Sin iluminación:} \ E_F = 0.846 \ \eV \tquad
			\text{Con iluminación:} \ E_F = 0.378 \ \eV
		\end{equation}
	\end{enumerate}
