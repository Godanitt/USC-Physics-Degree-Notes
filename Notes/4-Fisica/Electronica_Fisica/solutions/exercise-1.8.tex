	La solución del ejercicio pasa por calcular los valores de los niveles de Fermi usando la ecuación
	\begin{equation}
		E_F = E_i + kT \ln \parentesis{\frac{N_D}{n_i}}
	\end{equation}
	ya que estamos ante un dador que tiene todos los átomos excitados $N_D$ tal que $N_D>>n_i,N_A$. Dado que consideramos esto a tempeartura ambiente, tenemos que $n_i=$ y por tanto que para estos $N_D$:
	\begin{equation}
		N_D=10^{15} \ \cm^{-3} \Rightarrow E_F =
	\end{equation}
	\begin{equation}
		N_D=10^{17} \ \cm^{-3} \Rightarrow E_F =
	\end{equation}
	\begin{equation}
		N_D=10^{19} \ \cm^{-3} \Rightarrow E_F =
	\end{equation}
	Una vez tenemos estos valores de $E_F$, veamos si es válido asumir que todosl os átomos donadores están ionizados, usando que

	\begin{equation}
		N_D^+ = \frac{N_D}{1+2\exp\ccorchetes{(E_F-E_D)kT}}
	\end{equation}
	Tal que para las energías dadas:

	\begin{equation}
		E_F=  \Rightarrow N_D^+ =
	\end{equation}
	\begin{equation}
		E_F = \Rightarrow N_D^+ =
	\end{equation}
	\begin{equation}
		E_F = \Rightarrow N_D^+ =
	\end{equation}
