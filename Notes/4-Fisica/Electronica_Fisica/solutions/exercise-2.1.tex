	\begin{enumerate}[label=\alph*)]
		\item La diferentencia radica en la masa efectiva, que se expresa en la movilidad:
		      \begin{equation}
			      \rho = \frac{1}{q(n\mu_n + p \mu_)}
		      \end{equation}
		      Cuando $\mu_n>\mu_p \Rightarrow \rho_n < \rho_p$. Y esto siempre ocurre. Las movilidades dependen de la temperatura y la cantidad que esté dopado, por lo que puede ser diferente. Para un semiconductor dopado tipo $N$:

		      \begin{equation}
			      \rho_N = \frac{1}{qn{\mu_n}} = \frac{1}{1.6\cdot 10^{19} \cdot 10^{17}\cdot 1350}= 0.0463 \Omega \cm
		      \end{equation}
		      Y para un tipo $P$:
		      \begin{equation}
			      \rho_N = \frac{1}{1.6\cdot 10^{19} \cdot 10^{17}\cdot 480} = 0.13 \Omega \cm
		      \end{equation}
		      (valores de movilidad sacados de la Wikipedia). Ahora podemos calucular la corriente de arrastre (usamos que $J=qnN\epsilon$, donde $\epsilon = 10^5$ eV)
		      \begin{equation}
			      J_N =  2.16 \cdot 10^6  A\cm^{-2}  \tquad J_p = 7.68 \cdot 10^3  A\cm^{-2}
		      \end{equation}
		\item Ahora lo que hacemos es considerar que el número de impurezas es igual al número de huecos (están todas completamente ionizadas). Lo que nos queda entonces es:
		      \begin{equation}
			      N_A = \frac{1}{q\rho \mu_p } = \frac{1}{1.6\cdot 10^{19} \cdot 0.1 \cdot 480} = 1.3 \cdot 10^{17} \cm^{-3}
		      \end{equation}
		      Podemos calcular con $a_0$ el númerode átomos de silicio por unidad de volumen:

		      \begin{equation}
			      N_{Si} = 5 \cdot 10^{22} \text{at} \cm^{-3}
		      \end{equation}
		      Y solo tenemos, para calcular la proporción:

		      \begin{equation}
			      \frac{N_A}{N_{Si}} = 2.6 \cdot 10^{-6} = 2.6 \ \text{ppm}
		      \end{equation}
		      Para acabar necesitamos calcular la relación de Eistein (solo usable en semicdonductores no degenerados). Así tenemos que

		      \begin{equation}
			      D_p = \frac{kT}{q} \mu_p = 12.4 \cm^2 / s
		      \end{equation}
	\end{enumerate}

