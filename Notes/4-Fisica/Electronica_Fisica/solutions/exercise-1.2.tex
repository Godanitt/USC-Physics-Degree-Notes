	\begin{enumerate}[label=\alph*)]	
		\item Tenemos primero que ver si está degenerado, sin embargo sabemos que para esta temperatura y el nivel de dopamiento no debería estar degenerado, y por tanto podríamos usar la ley de acción de masas junto con la condición de electroneutralidad para despejar $n$ en función de $N_D,N_A$ y $n_i$. Se puede obtener, dado que $N_D \gg n_i,N_A$, tenemos que

		\begin{equation}
			n\approx N_D = 6\cdot 10^{15} \cm^{-3} \tquad n\cdot p = n_i^2 \Rightarrow p = 1.67 \cdot 10^4 \cm^{-3}
		\end{equation}
		\item Nos dicen que a $T=470$K y que $E_G=1.08$eV. Lo único que no cambio es $N_D$. Lógicamente el número de portadores intrínsecos $n_i$ cambia al aumentar la temperatura: a mayor energía térmica promocionan más electrones, mas electrones van a poder excitarse desde la banda de valencia. Calculamos $n_i$ a partir de
		\begin{equation}
			n_i = \sqrt{N_cN_v} e^{-E_g/2kT}
		\end{equation}
		Luego solo tenemos que hacer

		\begin{equation}
			N_{c,v} = 4.829\cdot 10^{15} T^{3/2} \parentesis{\frac{m_{n,p}^*}{m_e}}
		\end{equation}
		A esta temperatura tenemos entonces que:

		\begin{equation}
			N_c = 6.3 \cdot 10^{19} \cm^{-3} \quad N_v = 3.6 \cdot 10^{19} \cm^{-3}
		\end{equation}
		Y por tanto

		\begin{equation}
			n_i = 7.74 \cdot 10^{13} \cm^{-3}
		\end{equation}
		De lo que se deduce, de nuevo, aplicnado la ley de acción de masas:

		\begin{equation}
			n=6.001 \cdot 10^{15} \cm^{-3} \quad p = 9.98 \cdot 10^{11} \cm^{-3}
		\end{equation}
		\item Determina la posción del nivel de Fermi intrínseco. Es secillo que:
		\begin{equation}
			E_i = \frac{E_c + E_v}{2} + \frac{3}{4} kT \ln \parentesis{\frac{m_p^*}{m_n^*}}
		\end{equation}
		Una vez tenemos $E_i$ para cada una de las temperaturas, calculamos la temperatura final
		\begin{equation}
			E_F = E_i + kT \ln \parentesis{\frac{n}{n_i}}
		\end{equation}
		*Introducir imagen*
		\item
		\item
\end{enumerate}


	
