\documentclass[11pt]{book}
\usepackage[utf8]{inputenc}
\usepackage[spanish,es-tabla,es-nodecimaldot]{babel}

% Paquetes

\usepackage{amsmath}
\usepackage{amsthm}
\usepackage{amsfonts}
\usepackage{amssymb}
\usepackage{makeidx}
\usepackage{graphicx}
\usepackage{lmodern}
\usepackage{xcolor} 
\usepackage{fancyhdr}
\usepackage{geometry}
\usepackage{lastpage}		
\usepackage{array}			 % Para fjar tamaño de columnas
\usepackage{tikz}
\usepackage{subcaption}
\usepackage{pgfplots} % Para controlar la perspectiva
\RequirePackage{siunitx}
\usepackage{extramarks} % Para poder usar firstleftmarks
\usepackage[version=4]{mhchem} % Para poder usar formulas de reacciones nucleares
\usepackage{xcolor}
\usepackage{newtxtext,newtxmath} % Cambia la fuente (pero mola)

%##############################################################################
%######### Ponemos el decimal con . ###########################################
%##############################################################################

\sisetup{output-decimal-marker={.},
	% exponentes ------------------------
	exponent-mode=threshold,
	exponent-thresholds=-3:2, % non usar exponentes 10^{-2,-1, 0, 1}
	% redondear -------------------------
	% round-mode=figures, % cifras sig
	% round-mode=places, % cantos decimales
	round-mode=uncertainty, % cifras sig da incerteza (necesario usar erro)
	round-precision=2,
	uncertainty-mode = separate,
	print-unity-mantissa=false,
	% unidades --------------------------
	inter-unit-product = \ensuremath{{}\cdot{}}, % separacion entre unidades
	% per-mode=power-positive-first, % so furrula con metodo interpretado puro
	inline-per-mode=single-symbol,
	display-per-mode=fraction,
}

%##############################################################################
%######### Para codigo python #################################################
%##############################################################################

\definecolor{codegreen}{rgb}{0,0.6,0}
\definecolor{codegray}{rgb}{0.5,0.5,0.5}
\definecolor{codepurple}{rgb}{0.58,0,0.82}
\definecolor{backcolour}{rgb}{0.95,0.95,0.92}

\usepackage{listings}


%\lstdefinestyle{mystyle}{	backgroundcolor=\color{backcolour},   	commentstyle=\color{codegreen},	keywordstyle=\color{magenta},	numberstyle=\tiny\color{codegray},	stringstyle=\color{codepurple},	basicstyle=\ttfamily\footnotesize,	breakatwhitespace=false,         	breaklines=true,                 	captionpos=b,                    	keepspaces=true,                 	numbers=left,                    	numbersep=5pt,                  	showspaces=false,                	showstringspaces=false,	showtabs=false,                  	tabsize=2}

%\lstset{style=mystyle}
%\usepackage{background}     % Para manejar el fondo


%##############################################################################
%######### Tipo de fuente #################################################
%##############################################################################

%\usepackage{kpfonts}

%\usepackage{helvet} 
%\renewcommand{\familydefault}{\sfdefault}.

%\usepackage{fontspec} % Paquete necesario para seleccionar fuentes
%\setmainfont{Verdana} % Cambia la fuente principal a Verdana


%##############################################################################
%######### Geometría #################################################
%##############################################################################

\geometry{a4paper, total={152mm,237mm}, left=31mm, top=30mm}



%##############################################################################
%######### Formatos capítulo #################################################
%##############################################################################

%\usepackage[lmodern]{quotchap}
%\usepackage[options]{fncychap}



%##############################################################################
%######### Hiperreferenias #################################################
%##############################################################################


\usepackage[colorlinks=true,allcolors=blue]{hyperref} % Crea las


%##############################################################################
%######### Formato de pagina #################################################
%##############################################################################

\renewcommand{\chaptermark}[1]{\markboth{\chaptername\ \thechapter.\ #1}{}}
\renewcommand{\sectionmark}[1]{\markright{\thesection.\ #1}}

\setlength{\headsep}{27pt} % Distancia entre la cabezera y el texto
\setlength{\footskip}{30pt} % Distancia entre el pie de pagina y el texto
\pagestyle{fancy}
\fancyhf{}
\fancyhead[LE]{\leftmark} % L,R,C-> left, right, center [LE,RO]
\fancyhead[RO]{\rightmark} % E,O -> even (par), odd (impar)
\fancyhead[LO,RE]{Daniel Vázquez Lago}
\fancyfoot[CE,CO]{\thepage}
\renewcommand{\headrulewidth}{1pt} % Cambiamos el grosor de la linea de arriba
\renewcommand{\footrulewidth}{0pt}



%##############################################################################
%#########  Modificar caption #################################################
%##############################################################################

\usepackage[font=small, justification=centering]{caption}  % Configura las captions



%##############################################################################
%######### Comandos propios #################################################
%##############################################################################


\newcommand{\parentesis}[1]{\left( #1  \right)}
\newcommand{\parciales}[2]{\frac{\partial #1}{\partial #2}}
\newcommand{\pparciales}[2]{\parentesis{\parciales{#1}{#2}}}
\newcommand{\ccorchetes}[1]{\left[ #1  \right]}
\newcommand{\D}{\mathrm{d}}
\newcommand{\derivadas}[2]{\frac{\D #1}{\D #2}}

\newcommand{\tquad}{\quad \quad \quad}
\newcommand{\vnabla}{\vec{\nabla}}

\newcommand{\Ocal}{\mathcal{O}}
\newcommand{\Fcal}{\mathcal{F}}
\newcommand{\Hcal}{\mathcal{H}}
\newcommand{\Ecal}{\mathcal{E}}

\newcommand{\cmm}{\text{cm}^{-1}}
\newcommand{\fcc}{\textit{fcc}}
\newcommand{\bcc}{\textit{bcc}}
\renewcommand{\sc}{\textit{sc}}
\newcommand{\hcp}{\textit{hcp}}



\newcommand{\Namas}{\text{Na}^+}
\newcommand{\Clmenos}{\text{Cl}^-}

\newcommand{\er}{$^{\text{er}}$}
\newcommand{\cte}{\text{cte}}


% Comandos vectoriales

\newcommand{\an}{\mathbf{a}}
\newcommand{\bn}{\mathbf{b}}
\newcommand{\dn}{\mathbf{d}}
\newcommand{\fn}{\mathbf{f}}
\newcommand{\jn}{\mathbf{j}}
\newcommand{\kn}{\mathbf{k}}
\newcommand{\pn}{\mathbf{p}}
\newcommand{\qn}{\mathbf{q}}
\newcommand{\rn}{\mathbf{r}}
\newcommand{\un}{\mathbf{u}}
\newcommand{\vn}{\mathbf{v}}
\newcommand{\xn}{\mathbf{x}}
\newcommand{\wn}{\mathbf{w}}
\newcommand{\yn}{\mathbf{y}}
\newcommand{\qndot}{\dot{\qn}}

\newcommand{\alphan}{\boldsymbol{\alpha}}
\newcommand{\sigman}{\boldsymbol{\sigma}}
\newcommand{\pin}{\boldsymbol{\pi}}
\newcommand{\rhon}{\boldsymbol{\rho}}
\newcommand{\epsilonn}{\boldsymbol{\epsilon}}
\newcommand{\omegan}{\boldsymbol{\omega}}


\newcommand{\An}{\mathbf{A}}
\newcommand{\Bn}{\mathbf{B}}
\newcommand{\En}{\mathbf{E}}
\newcommand{\Fn}{\mathbf{F}}
\newcommand{\Jn}{\mathbf{J}}
\newcommand{\Gn}{\mathbf{G}}
\newcommand{\Kn}{\mathbf{K}}
\newcommand{\Ln}{\mathbf{L}}
\newcommand{\Pn}{\mathbf{P}}
\newcommand{\Rn}{\mathbf{R}}
\newcommand{\Sn}{\mathbf{S}}
\newcommand{\Tn}{\mathbf{T}}
\newcommand{\In}{\mathbf{1}}
\newcommand{\Encal}{\boldsymbol{\mathcal{E}}}

\newcommand{\hnn}{\hat{\mathbf{n}}}
\newcommand{\hnr}{\hat{\mathbf{r}}}
\newcommand{\hnz}{\hat{\mathbf{z}}}
\newcommand{\hnv}{\hat{\mathbf{v}}}
\newcommand{\hnx}{\hat{\mathbf{x}}}
\newcommand{\hny}{\hat{\mathbf{y}}}
\newcommand{\hnu}{\hat{\mathbf{u}}}
\newcommand{\hnR}{\hat{\mathbf{R}}}
\newcommand{\hnp}{\hat{\mathbf{p}}}
\newcommand{\hnk}{\hat{\mathbf{k}}}
\newcommand{\hni}{\hat{\mathbf{i}}}
\newcommand{\hnj}{\hat{\mathbf{j}}}
\renewcommand{\hnk}{\hat{\mathbf{k}}}

% Definicion de definicion


\theoremstyle{theorem}
\newtheorem{definition}{Definición}[chapter]
\newtheorem{definition_equivalente}{Definición equivalente}[definition]
\theoremstyle{theorem}
\newtheorem{theorem}{Teorema}[chapter]

\numberwithin{equation}{section}
\numberwithin{figure}{chapter}

\definecolor{myblue}{RGB}{0, 102, 204}


% Autor y título

\author{Daniel Vazquez Lago}
\title{Notas Física del estado sólido}


\begin{document}

\maketitle

\newpage
\tableofcontents
\newpage



\chapter*{Introducción}
\addcontentsline{toc}{chapter}{\protect\numberline{}Introducción}

Usaremos $N=500$ particulas y una densidad de 0.5 $N/V^3$. La variación máxima de energía permitida es 1/1000. \\

USaremos la aproximación de Lennard-Jones, hya que es suave, supone interacciones debiles ideales para los gases nobles. 

\begin{equation} 
    v_{ij} (r_{ij}) = 4 \epsilon \left[ (\sigma /r_{ij})^{12}-(\sigma /r_{ij})^6 \right]
\end{equation}


``Usar una suma doble para luego dividirlo por dos es para pegarle en la cara''. La parte de los sumatorios debe estar libre de polvo y paja para que corra veloz. 

$$ t_p =  \frac{1}{2} \sum \sum v_{ij} = \sum_{i=1}^{N-1} \sum_{j=i+1}^N v_{ij} $$

\chapter{Introducción a Fortran}

En este capítulo vamos a introducir al lector el lenguaje de fortran, la prinicpal sintaxis y algunos ejemplos siempre que lo veamos adecuado. Lógicamente la mejor manera de aprender fortran es picando código, por lo que más que un manual para aprender este capítulo debería ser usado como manual de referencia en el caso de no conocer la sintaxis. Esta introdución está es un calco del Curso de Fortran impartido por Ángel Felipe Ortega del la UCM. Lógicamente hemos reducido este curso en algunos aspectos, y lo hemos ampliado en otro, a fin de que se adecue más al nivel requerido por esta asignatura.

\section{Primer contacto con fortran}


\section{Estructura del programa. Codigo fuente.}

\subsection{Formato código fuente}

El formato de condigo guente puede ser libre o fijo, y no deben mezclarse ambos en un fichero de código. El código fijo se considera obsoleto en Fortran95. En cualquier caso existen ciertas normas básicas y típicas de fortran, ataño obligatorias, que todavía se mantienen, por lo que es importante mencionarlas. Estas son:

\begin{itemize}
	\item Las sentencias de un programa se escribem en diferentes líneas.
	\item La posición de los caracterres dentro de las líneas es significativa.
	\item Columnas:
	      \begin{itemize}
		      \item 1-5. Número de etiqueta (de 1 a 5 dígitos, se usan números usualmente).
		      \item 6. Carácter de continuación de línea.
		      \item Resto. Sentencia.
	      \end{itemize}
	\item Comentarios:
	      \begin{itemize}
		      \item Las líneas en blanco se ignoran. Hacen más legible el programa.
		      \item Si el primer carácter de una línea es *, c o C la línea es de comentario.
		      \item Si aparece el carácter ! en una línea (salvo en la columna 6) lo que sigue es un comentario.
	      \end{itemize}
	\item Una línea puede contener varias sentencias separadas por punto y coma (;), el cual no puede estar en la columna 6. Sólo la primera de estas sentencias podría llevar etiqueta.
	\item Los espacios en blanco son significativos: {\tt IMPLICIT NONE, DO WHILE} (obsoleto), {\tt CASE DEFAULT}. Son opcionales en:
	      \begin{itemize}
		      \item Palabras clave dobles qeu comienzan por {\tt END} o {\tt ELSE}.
		      \item {\tt DOUBLE PRECISION, GO TO, IN OUT, SELECT CASE.}
	      \end{itemize}
	\item  El indicador de continuación de una línea es el carácter \&.
\end{itemize}

\subsection{Tipos intrínsecos de datos}

Fortran tiene los siguientes tipos de datos:

\begin{itemize}
	\item Enteros ({\tt INTEGER})
	\item Reales ({\tt REAL, DOUBLE PRECISION})
	\item Complejos ({\tt COMPLEX})
	\item Lógicos ({\tt LOGICAL})
	\item Caracteres ({\tt CHARACTER, CHARACTER(LEN=n), CHARACTER*n})
\end{itemize}

\subsubsection{Parámetros. Variables. Declaración. Asignación.}

\begin{itemize}
	\item Un parámetro tiene un valor que no se puede cambiar (PARAMETER).
	\item Una variable puede cambiar su valor cuantas veces sea necesario.
	\item Por defecto, todas las variables que empiecen por \texttt{i,j,k,l,m} o \texttt{n} son entreas y las demas reales. Es muy recomendable declarar las variables que se utilicen (la sentencia {\tt IMPLICIT NONE} obliga a declarar todas las variables).
\end{itemize}

\begin{lstlisting}[language=Fortran]
	INTEGER:: a, x, n
	DOUBLE PRECISION doble
	COMPLEX:: c, d
	CHARACTER(LEN=10) nombre, vocales*5, comput
	LOGICAL:: logico, zz
	PARAMETER (n=5, r=89.34, pi=3.141592, lac=-40, zz=.FALSE., &
		c=(2.45e2,-1.17), comput='ordenador', vocales='aeiou')

	! daria error poner n=8 porque no se puede cambiar un parametro

	x = 215
	doble = 6345.700234512846d-125
	logico = .TRUE.
	d = (2,4.5)
	a = 1200
	x = -1  				! se puede cambiar el valor de una variable
	nombre = 'Tarzan' 		! atencion al acento en algunos compiladores
	PRINT*, pi, x, doble, logico, d, a
	PRINT*, nombre, vocales
	END
\end{lstlisting}

\subsubsection{Arrays, subíndices, substrings}

\begin{itemize}
	\item Un array se define mediante su nombre y dimensiones (cantidad y límites).
	\item Por defecto el primer índice es 1. En otro caso hay que indicar el rango {\tt i1:i2}.
	\item Los elementos del array se acceden por sus índices entre paréntesis.
\end{itemize}

\begin{lstlisting}[language=Fortran]
	INTEGER n(10), n1(3,5), c(4,4,4,4), hh(0:4), bb(-6:4,2:5,75:99)
	REAL r1(5), r2(-2:4,6)
	CHARACTER(LEN=15):: mes(12), dia(0:6)

	n(3) = 4563
	n1(2,4) = n1(1,3) + 89
	hh(0) = -1
	bb(-3,4,80) = bb(-5,2,90) + 2.3*c(3,3,3,1)
	r2(-1,6) = r1(3) + r2(-2,3)
	mes(2) = 'FEBRERO'
	dia(0) = 'DOMINGO'
	PRINT*, n(3), n1(2,4), hh(0), mes(2)
	END
\end{lstlisting}



\subsection{Operadores y expresiones}

\subsubsection{Aritméticas}

\begin{itemize}
	\item Los operadores aritméticos son {\tt +, -, *, /, **}.
	\item El orden de prioridades es el mismo que en el álgebra.
	\item No puede ver operadores seguidos (incorrecto {\tt a*-b}, correcto {\tt a*(-b)}).
\end{itemize}

\subsubsection{Relacion y expresiones lógicas}

\begin{itemize}
	\item Los operadores de expresiones son:
	      \begin{table}[h!] \centering
		      \begin{tabular}{|c|c|c|c|c|c|}
			      \hline
			      {\tt .EQ.} & {\tt .NE.} & {\tt .LT.} & {\tt .LE.} & {\tt .GT.} & {\tt .GE.} \\ \hline
			      $==$       & $/=$       & $<$        & $<=$       & $>$        & $>=$       \\    \hline
		      \end{tabular}
	      \end{table}

	\item Se pueden relacionar expresiones aritméticas con expresiones lógicas y expresiones de caracteres.
	\item Es recomendable utilizar paréntesis y/ó sustituir las expresiones complicadas por combinaciones de expresiones más simples.
	\item Los operadores lógicos son:
	      \begin{table}[h!] \centering
		      \begin{tabular}{|c|c|c|c|c|c|}
			      \hline  Operador & {\tt .NOT.} & {\tt .AND.} & {\tt .OR.} & {\tt .EQV.} & {\tt .NEQV.} \\ \hline Prioridad
			                       & 1           & 2           & 3          & 4           & 4            \\    \hline
		      \end{tabular}
	      \end{table}
\end{itemize}

\subsection{Entrada y salida estándar sin formato}


Los dispositivos estándar (por defecto) de entrada y salida de datos son el teclado y la pantalla:

\begin{itemize}
	\item Lectura de datos de teclado. Son equivalentes las siguientes sentencias:
	      \begin{itemize}
		      \item {\tt READ (*,*), listavar}
		      \item {\tt READ* , listavar}
	      \end{itemize}

	\item Escritura de datos en pantalla. Son equivalentes las siguientes sentencias:
	      \begin{itemize}
		      \item {\tt WRITE (*,*), listavar}
		      \item {\tt PRINT* , listavar}
	      \end{itemize}
\end{itemize}
Donde {\tt listavar} es una lista de variables o elementos de arrays separados por comas.

\subsection{Sentencias, {\tt PROGRAM, END}}

\begin{itemize}
	\item Un programa puede comenzar con la sentencia {\tt PROGRAM nombprog.}
	\item Un programa debe terminar con la sentencia {\tt END  [PROGRAM [nombreprog]]}.
\end{itemize}
Donde \texttt{ nombreprog} es el nombre del programa, que debe empezar por una letra y admite hasta 31 letras, dígitos y guiones underscore.

\section{Setencias de control}

Las sentencias de control sirven para alterar la ejecución de las sentencias de un programa.



\subsection{Setencia {\tt CONTINUE}}

La sentencia \texttt{CONTINUE} es ejecutable, pero no realiza acción alguna. Es útil para rupturas de secuencia y manejo de errores en lectura de datos. Su número de etiqueta puede ser referenciado en sentencia \texttt{DO}.

\subsection{Setencia {\tt STOP}}

La sentencia {\tt STOP} detiene la ejecución del programa. Tiene dos variantes {\tt STOP [`mensaje'], STOP[n]}. Si está presente el literal {\tt `mensaje'} ó el número {\tt  n} (que ha de tener de 1 a 5 dígitos), se visualizan en pantalla. Puede llevar etiqueta y formar parte de una sentencia {\tt IF}, y sirve principalmente para detener la ejecución a causa de un error y con el literal `{\tt mensaje}' ó el número {\tt n}.

\subsection{Setencia {\tt GOTO}}

La sintaxis {\tt GOTO e} transfiere el control a la secuencia ejecutable con etiqueta {\tt e} que se encuentra en la misma unidad de programa que la setencia {\tt GOTO}. No se puede entrar en bloques {\tt DO, IF, CASE} desde fuera de ellos con las sentencias {\tt GOTO}. \\

Es una sentencia cuyo uso genera mucha polémica. Un programa con gran cantidad de {\tt GOTO} es difíciol de comprender, sobre todo si hay muchas transferencias a sentencias anteriores. Con una adecuada programación pueden sustituirse, con facilidad, la mayoría de las sentencias {\tt GOTO} por otras estructuras de control. Sin embargo, hay ocasiones cuya sustitución complica enormemente la lógica del programa. Es especialmente útil para tratar condiciones de error o de terminación de un bloque. Conviene NO abusar de esta sentencia. \\

Si la sentencia siguiente a la sentencia {\tt GOTO} no lleva etiqueta no se ejecutará nunca (código muerto). Es un síntoma de error de programación.

\subsection{Setencia {\tt IF}}

La sintaxis es {\tt IF (expres) sentec}. La expresión {\tt expres} debe ser escalar lógica, si es verdadera se ejecuta {\tt sentenc}; si es falsa no se ejecuta {\tt sentenc} y se continua a la sentencia siguiente. La sentencia {\tt sentenc} debe ser ejecutable, no puede ser la sentencia {\tt END} ni la sentencia inicial o final de bloques {\tt DO, IF, SELECT, CASE}. \\

Esta sentencia se suele utilizar para realizar, en función de la condición, una única asignación, una sencilla escritura de datos, una parada del programa, una ramificación del flujo del programa.

\subsection{Bloque {\tt IF-THEN-ENDIF}}

La sitáxis es la siguiente:

\begin{lstlisting}[language=Fortran]
	[nomb:] IF (expres) THEN 
		bloq....
	ENDIF[nomb]
\end{lstlisting}

La expresión \texttt{expres} debe ser escalar lógica. Si es verdadera se ejectutan las sentencias del bloque \texttt{bloq} entre \texttt{THEN} y \texttt{ENDIF}. Si es falsa se continúa en la siguiente sentencia a \texttt{ENDIF}. Si lleva nombre (opcional), debe ser un nombre válido en Fortran distinto de otras nombres en la unidad de alcance en la que está el bloque \texttt{IF}.


\subsection{Bloque {\tt IF-THEN-ELSE-ENDIF}}
La sitáxis es la siguiente:

\begin{lstlisting}[language=Fortran]
	[nomb:] IF (expres) THEN 
		bloq1 
	ELSE [nomb] 
		bloq2 
	EDNDIF[nomb] 
\end{lstlisting}

La expresión \texttt{expres} debe ser escalar lógica. Si es verdadera se ejectutan las sentencias del bloque \texttt{bloq} entre \texttt{THEN} y \texttt{ENDIF}. Si es falsa se continúa en la siguiente sentencia a \texttt{ENDIF}. Si lleva nombre (opcional), debe ser un nombre válido en Fortran distinto de otras nombres en la unidad de alcance en la que está el bloque \texttt{IF}.


\subsection{Bloque {\tt ELSE-IF}}

La sitáxis es la siguiente:

\begin{lstlisting}[language=Fortran]
	[nomb:] IF (expres) THEN 
		bloq1 
	[ElSEIF (expres_i) THEN [nomb] b
		bloq_i]...
	[ELSE [nomb]
		bloq2]
	EDNDIF[nomb] 
\end{lstlisting}

Cada expresión {\tt expres, expres\_i} debe ser escalar lógica. Si {\tt expres} es verdadera se ejecutan las sentencias del bloque {\tt bloq1} entre {\tt THEN} y el primer {\tt ELSEIF} y se pasa a la siguiente sentencia a \texttt{ENDIF}. Si \texttt{expres\_i} es falsa se inspeccionan en orden las expresioens \texttt{expres\_i} hasta que una sea verdadera, en cuyo caso se ejecutan las sentencias del bloque {\tt bloq\_i} correspondiente y el control pasa a la siguiente sentencia \texttt{ENDIF}. Si la expresión {\tt expres} y todas las expresiones {\tt expres\_i} son falsas, se ejecutan las sentencias del bloque \texttt{bloq2} entre {\tt ELSE} y {\tt ENDIF}. \\

Las sentencias {\tt ELSEIF} y {\tt ELSE} pueden llevar nombre sólo si sus sentencias {\tt IF} y {\tt ENDIF} llevan y, en estec aso debe ser el mismo. No se puede entrar en un bloque {\tt IF, THEN, ELSEIF} ni {\tt ELSE} desde fuera de él con sentencias {\tt GOTO}, aunque si se peude salir en cualquier lugar con la misma. En el siguiente código podemos ver un ejemplo del uso de este bloque para resolver una función definida a trozos.

\begin{lstlisting}[language=Fortran]
	REAL x, f
	PRINT*,'valor de x =' ; READ*, x
	IF (0<=x .AND. x<=1) THEN
		f = 3*x**2 - 1
	ELSEIF (5<=x .AND. x<=10) THEN
		f = 6*x + 4
	ELSEIF (20<=x .AND. x<=40) THEN
		f = -7*x + 1
	ELSE
		f = 0
	ENDIF
	PRINT*, ' x = ', x, ' f = ', f
	END
\end{lstlisting}


\subsection{Selector {\tt SELECT CASE}}

La sintaxis es la siguiente:

\begin{lstlisting}[language=Fortran]
	[nomb:] SELECT CASE (expres) 
	CASE (selector) [nomb]
		bloq... 
	CASE DEFAULT [nomb]
		bloq0... 
	ENDSELECT [nomb]
\end{lstlisting}

La expresión {\tt expres} debe ser escalar de tipo entera, lógica ({\it poco interesante}) ó carácter y los valores dados deben ser del mismo tipo (en el caso carácter las longitudes pueden ser diferentes pero no la clase, en los casos entero o lógico pueden ser diferentes). Si el valor {\tt expres} pertenece a un selector se ejecuta su bloque de sentencias y se continúa en la siguiente sentencia a  {\tt END SELECT}. Si no pertenece a ningún selector se ejecutan las sentencias del bloque \texttt{CASE DEFAULT}, si está presente y si no lo está se continúa en la siguiente sentencia a {\tt END SELECT}. \\

Si lleva nombre (opcional) debe ser un nombre válido en Fortran y distinto de otros nombres en la unidad de alcance en qeu se encuentra el bloque {\tt SELECT}. Las sentencias \texttt{CASE} y \texttt{CASE DEFAULT} pueden llevar nombre sólo si las sentencias {\tt SELECT CASE} y {\tt CASE} correspondientes lo llevan y, en este caso, debe ser el mismo. Los valores de los selectores han de ser disjuntos. Se separan por comas y puede especificarse un rango de valores, también disjuntos en cada selector. No se puede entrar en un bloque {\tt SELECT} ó {\tt CASE} desde fuera de él con sentencias {\tt GOTO}. Se puede salir en cualquier lugar con sentencias {\tt GOTO}. Los bloques {\tt SELECT CASE} pueden anidarse. \\

La diferencia principal entre los bloques {\tt IF} y {\tt CASE} es que en {\tt CASE} sólo se evalúa una expresión cuyo valor deb estar en un conjunto predefinido de valores, mientras que en {\tt IF} se pueden evaluar varias expresiones de naturaleza distinta. Veamos un ejemplo para entender mejor el problema:

\begin{lstlisting}[language=Fortran]
	CHARACTER car! equivale a CHARACTER(LEN=1) car
	INTEGER indice
	PRINT*, ' Introducir un caracter'
	READ*, car
	SELECT CASE (car)
	CASE ('a', 'e', 'i', 'o', 'u')
		PRINT*, ' Vocal minuscula : ', car
	CASE ('A', 'E', 'I', 'O', 'U')
		PRINT*, ' Vocal MAYUSCULA : ', car
	CASE ('b':'d', 'f':'h', 'j':'n', 'p':'t', 'v':'z')
		PRINT*, ' Consonante minuscula : ', car
	CASE ('B':'D', 'F':'H', 'J':'N', 'P':'T', 'V':'Z')
		PRINT*, ' Consonante MAYUSCULA : ', car
	CASE ('0':'9')
		PRINT*, ' Cifra del 0 al 9 : ', car
	CASE DEFAULT
		PRINT*, ' El caracter no es ni letra ni numero : ', car
	ENDSELECT

	PRINT*, ' Introducir un numero'
	READ*, indice

	SELECT CASE (indice)
	CASE (2, 3, 5, 7, 11, 13, 17, 19)
		PRINT*, ' Numero primo menor que 20 : ', indice
	CASE (20:29, 40:49, 60:69, 80:89)
		PRINT*, ' Numero menor que 100 con decena par : ', indice
	CASE (100:999)
		PRINT*, ' Numero de 3 cifras : ', indice
	CASE DEFAULT
		PRINT*, ' Resto de casos : ', indice
	ENDSELECT
	END
	
\end{lstlisting}

\subsection{Interacciones {\tt DO}}

La sintaxis es:

\begin{lstlisting}[language=Fortran]
	[nomb:] Do [,] var = expres1, expres2, expres3 
		bloq
	ENDDO [nomb]
\end{lstlisting}

La variable {\tt var} y las expresiones {\tt expres1, expres2, expres3} (esta última opcional) deben ser escalares enteras. La variable {\tt var} toma el valor inicial de {\tt expres1}, se ejecutan las sentencias {\tt bloq} del bloque {\tt DO}; {\tt var} se incrementa en {\tt expres3}, se ejecutan las sentencias del bloque {\tt DO}, y así sucesivamente hasta que {\tt var$>$expres2} (si {\tt expres3$>$0} ó  {\tt var$<$expres2} (si {\tt expres3$<$0}) en cuyo caso se continúa en la siguiente sentencia a {\tt ENDDO}. \\

El número de interacciones que se realizan en el bloque {\tt Do} (si no se sale de él antes de terminar) es: {\tt MAX\{ (expres2-expres1+expres3)/expres3,0 \}}. Cuando \{ {\tt expres1$>$expres2} y {\tt expres3$>$0} \}  ó \{ {\tt expres1$<$expres2} y {\tt expres3$<$0} \}  no se ejecuta el bloque {\tt DO}. Si {\tt expres1} y/ó {\tt expres2} y/ó {\tt expres3} son expresiones que incluyen variables, el valor de éstas puede cambiarse dentro del bloque  {\tt DO} y esto no altera el número de interaciones (calculado con sus valores iniciales). Ejemplo:

\begin{lstlisting}[language=Fortran]
	INTEGER, PARAMETER:: n=100000
	REAL x(n), suma1, suma2, suma3

	DO i = 1, n
		x(i) = 1.0/i
	ENDDO
	
	suma1 = 0.0
	DO i = 1, n
		suma1 = suma1 + x(i)
	ENDDO
	suma2 = 0.0
	DO i = n, 1, -1
		suma2 = suma2 + x(i)
	ENDDO
	PRINT*, ' suma1 = ', suma1, ' suma2 = ', suma2

	suma3 = 0
	DO i = 1, 100000, 3 ! i toma los valores 1, 4, 7,...
		IF (suma3 <= 5) THEN
			suma3 = suma3 + x(i)
		PRINT*, ' i =', i, ' suma3 = ', suma3
		ELSE
			PRINT*, ' suma3 > 5'
		STOP
		ENDIF
	ENDDO
	
	END
\end{lstlisting}




\subsection{{\tt DO} ilimitado, {\tt EXIT} y {\tt CYCLE}}

El {\tt DO} ilimitado tiene la siguiente sintaxis:

\begin{lstlisting}[language=Fortran]
	[nomb:] Do 
		bloq
	ENDDO [nomb]
\end{lstlisting}
Y la acción del mismo es que se repitan las sentencias {\tt bloq} indefinidamente, pudiendo salir del mismo con las sentencias {\tt EXIT} y {\tt GOTO}. \\


La sentencia {\tt EXIT[nomb]} dentro de un bloque {\tt Do} transfiere el control a la primera sentencia ejecutable después del {\tt ENDDO} a que se refiere, sino se indica el nombre {\tt nomb}, transfiere el control al {\tt ENDDO} del bloque más interior en el que está contenida. \\ 


La sentencia {\tt CYCLE[nomb]} dentro de un bloque {\tt Do} transfiere el control a la sentencia {\tt ENDDO} a que se refiere, sino se indica el nombre {\tt nomb}, transfiere el control al {\tt ENDDO} del bloque más interior en el que está contenida. 


\section{Utilidades de programa. Procedimientos.}

\subsection{Programa principal}

Un programa completo debe tener exactamente un programa principal. La forma es la siguiente:

\begin{lstlisting}[language=Fortran]
	PROGRAM [nombreprog]
		[sentencias de especificacion]
		[sentencias ejecutables]
	CONTAINS
		[subprogramas internos	]
	END PROGRAM [nombreprog]	
\end{lstlisting}

\subsection{Subprogramas externos}

Son llamados desde el programa principal o desde otros subprogramas. Pueden ser funciones o subrutinas. Ambas pueden ser recursivas, esto es, llamarse a sí mismas. No obstante esto es muy complejo, y su uso implica peor eficacia computacional. 

\subsection{Uso no recursivo de subprogramas {\tt FUNCTION}}

La sitaxis es la siguiente:

\begin{lstlisting}[language=Fortran]
	[tipo] FUNCTION nombfun ([argumentos ficticios])
		[sentencias de especificacion]
		[sentencias ejecutables]
	END [FUNCTION [nombreprog]]
\end{lstlisting}

Con esto estaríamos definiendo el subprograma {\tt FUNCTION nombfun}, invocándose con {\tt nombfun([argumentos actuales])}, y substituyendose los argumentos actuales en los ficticios y se evalúa la función. El valor asignado a {\tt nombfun} es el valor devuelto a la función. Téngase en cuenta que:

\begin{itemize}
	\item El término {\tt tipo} es opcional. Si se omite se toma el tipo por defecto o el que haya sido establecido por sentencias {\tt IMPLICIT}. 
	\item La llamada puede formar parte de una expresión o sentencia más larga. Un subprograma {\tt FUNCTION} puede contener cualquier sentencia excepto: {\tt PROGRAM, FUNCTION, SUBROUTINE} y {\tt BLOCK DATA}.
	\item La última sentencia tiene que ser {\tt END}. 
	\item Las variables y etiquetas en un subprograma {\tt FUNCTION} son locales, esto es, independientes del programa principal y las de otros subprogramas. 
	\item Los argumentos actuales deben coincidir en cantidad, orden, tipo y longitud con los argumentos ficticios. Puede no haber argumentos.
	\item Los argumentos actuales pueden modificarse: sin embargo, esta opción es especialmente desaconsejable. 
	\item Una función no recursiva no puede llamarse a sí misma ni directa ni indirectamente, pero sí puede llamar a otros subprogramas.
\end{itemize}

\subsubsection{Sentencia {\tt RETURN} en Subprogramas {\tt FUNCTION}}

{\tt RETURN} termina la ejecución de la función y devuelve el control a la unidad de programa que llamó a la función. Si la función no tiene sentencias {\tt RETURN} su ejecución termina al llegar a la sentencia {\tt END}. Puede ser una setntencia con etiqueta, y puede formar parte de una sentencia {\tt IF}.

\subsection{Uso no recursivo de Subprogramas {\tt SUBROUTINE}}

La sintaxis escalar

\begin{lstlisting}[language=Fortran]
	SUBROUTINE nombsubr ([argumentos ficticios])
		[sentencias de especificacion]
		[sentencias ejecutables]
	END [SUBROUTINE [nombresubr]]
\end{lstlisting}

La llamada a la subrutina se realiza mediante la sentencia {\tt CALL nombsubr [(argumentos actuales)]}, substituyendose los argumentos actuales en los ficticios. Las normas son las siguientes:

\begin{itemize}
	\item La llamada puede formar parte de una expresión o sentencia más larga. Un subprograma {\tt FUNCTION} puede contener cualquier sentencia excepto: {\tt PROGRAM, FUNCTION, SUBROUTINE} y {\tt BLOCK DATA}.
	\item La última sentencia tiene que ser {\tt END}. 
	\item Las variables y etiquetas en un subprograma {\tt FUNCTION} son locales, esto es, independientes del programa principal y las de otros subprogramas. 
	\item Los argumentos actuales deben coincidir en cantidad, orden, tipo y longitud con los argumentos ficticios. Puede no haber argumentos.
	\item Una función no recursiva no puede llamarse a sí misma ni directa ni indirectamente, pero sí puede llamar a otros subprogramas.
\end{itemize}

\subsubsection{Sentencia {\tt RETURN} en Subprogramas {\tt SUBROUTINE}}

{\tt RETURN} termina la ejecución de la subrutina y devuelve el control a la unidad de programa que llamó a la función. Si la subrutina no tiene sentencias {\tt RETURN} su ejecución termina al llegar a la sentencia {\tt END}. Puede ser una setntencia con etiqueta, y puede formar parte de una sentencia {\tt IF}.

\subsection{Argumentos de subprogramsas externos}

Los argumentos de un subprograma {\tt FUNCTION} O {\tt SUBROUTINE} pueden ser de naturaleza muy diversa: constantes o variables escalares, arrays o elementos de arrays, nombres de otros subprogramas, etc. Es necesario suministrar al compilador la información adecuada para identificar correctamente la naturaleza del argumento. 

\subsubsection{Propósito de los argumentos}

Los argumentos ficticios pueden tener una declaración de propósito de entrada salida o entrada/salida. El propósito se declara con el atributo {\tt INTENT}. 

\begin{itemize}
	\item {\tt INTENT (IN)}: declara un argumento de entrada. No debe cambiarse su valor dentro del subprograma. 
	\item {\tt INTENT (OUT)}: declara un argumento de salida. El argumento actual debe ser una variable y se vuelve indefinida en entrada.
	\item {\tt INTENT (IN/OUT)}: declara un argumento de entrada o salida. El argumento actual debe ser una variable.
\end{itemize}

Es recomendable declarar el propósito de los argumentos ficticios, lo cual ayuda a la documentación del programa y a las verificaciones durante la compilación.



\subsection{Sentencia {\tt EXTERNAL}}

Se escribe como {\tt EXTERNAL lista}, e identifica los nombres de {\tt lista} como subprogramas (funciones o subrutinas) externos definidos por el usuario. Al ser una sentencia de especificación, debe preceder a las  ejecutables y a las declaraciones de funciones. Cuando un argumento de un subprograma es el nombre de otro subprogarama, se debe declarar {\tt EXTERNAl} en su unidad de llamada. Si una función intrínseca se declara {\tt EXTERNAL} pierde su definición intrínseca en la unidad de programa asociada y se usa el subprograma del usuario. 


\begin{lstlisting}[language=Fortran]
	EXTERNAL fun1, fun2, sin
	x=1.5; n=3
	CALL ameba (x, n, y1, fun1)		! y1 = x**(5-n) = 2.25
	CALL ameba (x, n, y2, fun2)	 	! y2 = 3*x*(5-n) = 9
	s = sin (y1, y2, 6.0, x)			! s = y2/y1 + 6/x = 8
	PRINT*, y1, y2, s
	END

	SUBROUTINE ameba (x, n, y, f)
		y = f(x, 5, n)
	END

	FUNCTION fun1 (x, i, j)
		fun1 = x**(i-j)
	END

	FUNCTION fun2 (x, i, j)
		fun2 = 3*x*(i-j)
	END

	FUNCTION sin (a, b, c, d)
		sin = b/a + c/d
	END
\end{lstlisting}

\subsection{Sentencia {\tt INTRINSIC}}

La sintaxis {\tt INTRINSIC lista} declara los nombres de {\tt lista} como funciones intrínsecas. Las normas son
\begin{itemize}
	\item Los nombres de la {\tt lista} deben ser funciones intrínsecas.
	\item Si un argumento de un subprograma es una función intrínseca se debe declarar {\tt INTRINSIC} en la unidad de llamada. 
	\item Si un nombre está en una sentencia {\tt INTRINSIC} no puede estar en una sentencia {\tt EXTERNAL}.
\end{itemize}

\begin{lstlisting}[language=Fortran]
	INTRINSIC sin, cos, exp
	a=3.141592; b=-a
	r = fun (sin, cos, exp, a, b, 4)
	PRINT*, r
	END

	FUNCTION fun (f1, f2, f3, a, b, n) ! fun = (sin(a)+cos(b)+
	fun = (f1(a) + f2(b) + f3(a+b)) ** n ! exp(a+b))**n	
	END
\end{lstlisting}

\subsection{Subprogramas internos}

Son subprogramas contenidos en el programa principal, en un subprograma externo o en un módulo. Su uso es adecuado, a efectos de organización, para subprogramas cortos (del orden de unas 20 líneas), que sólo se necesitan en un único programa, subprograma o módulo.  La sintáxis es la siguiente:

\begin{lstlisting}[language=Fortran]
	CONTAINS
	subprogramas internos
\end{lstlisting}
Y las normas son:

\begin{itemize}
	\item Los subprogramas internos deben aparecer entre la sentencia {\tt CONTAINS} y la sentencia {\tt END} de la unidad de programa a la que pertenezcan. 
	\item Un subprograma interno no puede contener a otro subprograma interno.
	\item Un subprograma interno sólo puede llamarse desde su host.
	\item Un host conoce todo acrca de la interface con sus subprogramas internos, por tanto no hace falta declarar el tipo en el host para una función interna.
	\item Un subprograma interno tiene acceso a las variables del host.
	\item El host no tiene acceso a las variables locales de los subprogramas internos.
	\item La sentencia {\tt IMPLICIT NONE} en un host afecta al host y también a sus subprogramas internos. 
	\item Las etiquetas son locales. Si una sentencia tiene etiqueta, ésta debe estar en la misma unidad de alcance que la sentencia que la referencia.
\end{itemize}

\begin{lstlisting}[language=Fortran]
	PROGRAM interno
		CALL coefbinomial ! invoca una subrutina interna

	CONTAINS

		SUBROUTINE coefbinomial
			INTEGER n, k
			CALL leer(n) ! invoca una subrutina interna
			DO k = 0, n
				PRINT*, k, nsobrek(n,k) ! invoca una funcion interna
			ENDDO
		ENDSUBROUTINE coefbinomial

		SUBROUTINE leer(n)
			INTEGER n
			PRINT*, ' Introducir el valor de n'
			READ*, n
		ENDSUBROUTINE leer

		FUNCTION nsobrek(n,k)
			INTEGER nsobrek, n, k
			nsobrek = fact(n) / (fact(k)*fact(n-k)) ! invoca una funcion
		ENDFUNCTION nsobrek ! interna 3 veces

		FUNCTION fact(m)
			REAL fact
			INTEGER m, i
			fact = 1
			DO i = 2, m
				fact = i*fact
		ENDDO
		ENDFUNCTION fact

	ENDPROGRAM interno
\end{lstlisting}

Su principal utilidad es organizar mejor el código y permitir un mayor control sobre el ámbito de las variables, ya que los subprogramas internos solo pueden ser llamados desde el subprograma en el que están definidos. Las principales ventajas de los subprogramas internos son:

\begin{itemize}

	\item Encapsulamiento: Permiten encapsular la lógica auxiliar o funciones específicas que solo tienen sentido dentro del contexto del subprograma principal. Esto ayuda a mantener el código más legible y modular.

	\item Ámbito de variables: Las variables locales del subprograma externo pueden ser utilizadas directamente dentro del subprograma interno, lo que evita la necesidad de pasarlas como argumentos. Esto simplifica el manejo de variables cuando son comunes a ambos subprogramas.

	\item Modularidad: Facilita la descomposición de tareas complejas en tareas más simples, dividiendo la lógica en partes más manejables, lo que mejora la mantenibilidad del código.

\end{itemize}


\subsection{Módulos}

Un módulo permite empaquetar definiciones de datos y compartir datos entre diferentes unidades de programas que pueden incluso compilarse por separado. Sirve, especialmente, para crear grandes librerías de software. En su uso sencillo:

\begin{itemize}
	\item Ofrece posibilidades similares a {\tt INCLUDE}.
	\item Permite compartir datos en ejecución.
	\item Sirve para inicializar variables.
\end{itemize}

La sitátxis es la siguiente:

\begin{lstlisting}[language=Fortran]
	MODULE nombmod
		[sentencias de especificacion]
	ENDMODULE nombmod
\end{lstlisting}
Y las normas

\begin{itemize}
	\item Se puede acceder a un módulo desde el programa principal, un subprograma u otro módulo. Se accede a las especificaciones y variables del módulo con los valores asignados (si los tienen). Las variables y datos de un módulo con los valores asignados si los tienen. Las variables y datos de un módulo tienen, por defecto, alcance {\bf global}, en todas las unidades desde las que se acceden con {\tt USE}.
	\item Desde un módulo se tiene acceso a las otras entidades del módulo incluyendo subprogramas. 
	\item Puede contener sentencias {\tt USE} para acceder a otros módulos.
	\item No debe acceder a sí mismo directamente o indirectamente a través de {\tt USE}.
	\item El módulo debe compilarse antes que el programa que lo usa. En la sentencia compilación se crea un fichero {\tt *.mod} que es el que lee la sentencia {\tt USE}. Se recomienda que un módulo solo acceda a módulos anteriores a él.
\end{itemize}


\subsection{Orden de las sentencias}

Las diferentes sentencias que puede contener un programa de Fortran deben escribirse en el orden siguiente (tabla \ref{Tab:01-01}).

\begin{table}[h!] \centering
	\begin{tabular}{|l|l|l|} \hline
		\multicolumn{3}{|c|}{{\tt PROGRAM, FUNCTION, SUBROUTINE, MODULE}} \\ \hline
		\multicolumn{3}{|c|}{\texttt{USE}} \\ \hline
		\multirow{7}{*}{\texttt{FORMAT}} & \multicolumn{2}{l|}{\texttt{IMPLICIT NONE}}\\  \cline{2-3}
		 & \texttt{PARAMETER} & \texttt{IMPLICIT} \\  \cline{2-3}
		 & \multirow{4}{*}{\texttt{PARAMETER, DATA}} & Tipos derivados \\
		 &  & Bloques \texttt{INTERFACE} \\
		 &  & Declaración de tipos\\ 
		 &  & Sentencias de especificación \\  \cline{2-3}
		 &  \multicolumn{2}{l|}{Sentencias ejecutables}  \\ \hline
		\multicolumn{3}{|c|}{\texttt{CONTAINS}} \\ \hline
		\multicolumn{3}{|c|}{Subprogramas inteernos o subprogramas modulo} \\ \hline
		\multicolumn{3}{|c|}{\texttt{END}} \\ \hline
	\end{tabular}
	\caption{orden de las sentencias.}
	\label{Tab:01-01}
\end{table}

\section{Procedimientos intrínsecos}

Los procedimientos intrínsecos son funciones y subrutinas que forman parte del lenguaje Fortran estándar, suministradas con el compilador. En fortar 95 hay 109 funciones y 6 subrutinas que pueden clasificarse en cuatro categorías de procedimientos intrínsecos:

\begin{itemize}
	\item \textbf{Procedimientos elementales}: sus argumentos son escalares o arrayas. Si una función elemental se aplica a un array la función se aplica a cada elemento del array.
	\item \textbf{Funciones de interrogación}: devuelven propiedades de sus argumentos que no dependen de sus valores.
	\item \textbf{Funciones transformacionales}: suelen tener argumentos de arrays y resultados de arrays cuyos elementos dependen de muchos elementos del argumento.
	\item \textbf{Subrutinas no elementales}
\end{itemize}

Cada función devuelve un valor entero, real, complejo, lógico... de modo que tendremos que abreviar de algún modo el tipo de valor devuelto. En este caso usaremos que:


\begin{table}[h!] \centering
	\begin{tabular}{|c|c|}
		\hline 
		I & Entero \\ 
		\hline 
		R & Real \\
		\hline 
		N & Numerico  \\
		\hline 
		L & Logico \\
		\hline 
		CH & Carácter\\
		\hline 
	\end{tabular}
	\caption{tipo de variable.}
	\label{Tab:01-02}
\end{table}
\newpage

\subsection{Funciones elementales que convierten tipos}

\begin{table}[h!] \centering
	\begin{tabular}{l|l|l|l} 
	Nombre & Definición &  Tipo argumentos & Tipo función \\ \hline
	ABS(x) & Valor absoluto & I & I \\
	ABS(x) & Valor absoluto & R & R  \\
	ABS(z) & Módulo complejo & C & R \\
	AIMAG(z) & Parte imaginaria & C & R \\
	AINT(x) & Quita decimales &  R & R \\
	ANINT(x) & Redondeo & R & R \\
	CEILING(x) & Redondeo  (por arriba) &  R & I \\
	CMPLX(x[,y]) & Pasa a complejo & N  & C \\
	FLOOR(x) & Redondeo (por abajo) & R  & I \\
	INT(x) & Pasa a entero & N & I \\
	NINT(x) & Redondeo entero  & R  & I \\
	DNINT(x) & Redondeo entero (doble precisión) & R  & I \\
	REAL(x) & Pasa a real & N & R \\
	\end{tabular}
	\caption{funciones elementales que pueden convertir tipos.}
	\label{Tab:01-03}
\end{table}

\subsection{Funciones elementales que no convierten tipos}


El resultado de las funciones elementales \ref{Tab:01-03} es del tipo de su primer argumento.

\begin{table}[h!] \centering
	\begin{tabular}{l|l|l|l}
		Nombre & Definición &  Tipo argumentos & Tipo función \\ \hline
		CONJG(z) & Conjugado complejo & C & C \\
		DIM(x,y) & Diferencia positiva & (I,I)  ó (R,R) & I ó R \\
		MAX(x1,x2,[,x3,...]) & Máximo & (I,I,…) ó (R,R,…)  & I ó R \\
		MIN(x1,x2,[,x3,...]) & Mínimo & (I,I,…) ó (R,R,…) & I ó R \\
		MOD(x,y) & Resto de x módulo y  & (I,I) ó (R,R) & I ó R \\
		MODULO(x,y)  & x módulo y & (I,I) ó (R,R) & I ó R \\ 
		SIGN(x,y) & Transferencia de signo & (I,I) ó (R,R) & I ó R \\
	\end{tabular}
	\caption{funciones elementales que no pueden convertir tipos.}
	\label{Tab:01-03}
\end{table}

\subsection{Funciones matemáticas elementales}

El resultado de las funciones elementales \ref{Tab:01-04} es del tipo de su primer argumento.

\begin{table}[h!] \centering
	\begin{tabular}{l|l|l|l}
		Nombre & Definición & Tipo argumentos & Tipo función \\ \hline
		ACOS(x) & Arco Coseno & R / $|x|\leq $1 & R en [0,$\pi$]  \\
		ASIN(x) & Arco Seno & R /  $|x|\leq $1 & R en [$-\pi/2$,$\pi/2$]  \\
		ATAN(x) & Arco Tangente & R & R en  [$-\pi/2$,$\pi/2$] \\
		ATAN2(y,x) & Argumento número complejo & (R,R) & R en  ($-\pi$,$\pi$] \\
		COS(x) & Coseno & R ó C & R ó C \\
		COSH(x) & Coseno hiperbólico & R & R \\
		EXP(x) & Exponencial & R ó C & R ó C \\
		LOG(x) & Logaritmo neperiano & R ó C & R ó C \\
		LOG10(x) & Logaritmo decimal  & R x$>$0 & R \\ 
		SIN(x) & Seno &  R ó C & R ó C \\
		SINH(x) & Seno hiperbólico & R & R \\
		SQRT(x) & Raíz cuadrada & R ó C & R ó C \\
		TAN(x)  & Tangente & R & R \\ 
		TANH(x) & Tangente hiperbólica &  R & R \\
	\end{tabular}
	\caption{funciones matemáticas elementales.}
	\label{Tab:01-04}
\end{table}


\subsection{Operaciones con matrices y vectores}

La función {\tt DOT\_PRODUCT(x,y)} requiere que x e y tengan una dimensión y el mismo tamaño. Si $x$ es entero o real devuelve $\sum x_i y_i$; si $x$ es complejo devuelve $\sum \overline{x}_i y_i$. Véase tabla \ref{Tab:01-05}.  \\

\begin{table}[h!] \centering
	\begin{tabular}{l|l|l|l}
		Nombre & Definición  & Tipo argumentos & Tipo función \\ \hline
		DOT\_PRODUCT(x,y) & Producto escalar real & (I ó R, I ó R) & I ó R \\
		DOT\_PRODUCT(z,y) & Producto escalar complejo & (C,I ó R) & C \\
		MATMUL(a,b) & Producto matricial & (N,N) & N \\
		TRANSPOSE(a) & Matriz traspuesta & N & N 
	\end{tabular}
	\caption{}
	\label{Tab:01-05}
\end{table}

La función  {\tt MATMUL(a,b)} devuelve un array de dos dimensiones en función de la forma de los dos arrays según la tabla \ref{Tab:01-06}.


\begin{table}[h!] \centering
	\begin{tabular}{l|l|l|l}
		Operación & Forma de a & Forma de b & Forma de MATMUL(a,b) \\ \hline
		Matriz x Matriz & (n,m) & (m,k) & (n,k) \\
 		Vector x Matriz & (m) & (m,k) & (k) \\
		Matriz x Vector & (n,m) & (m) & (n) \\
	\end{tabular}
	\caption{}
	\label{Tab:01-06}
\end{table}

\begin{table}[h!] \centering
	\begin{tabular}{l|l|l|l}
		Nombre & Definición &  Tipo argumentos & Tipo función \\  \hline
		MAXVAL(x) & Máximo elemento & I ó R & I ó R \\
		MINVAL(x) & Mínimo elemento & I ó R & I ó R \\
		PRODUCT(x) & Producto de los elementos & I ó R & I ó R \\
		SUM(a) & Suma de los elementos & I ó R & I ó R
	\end{tabular}
	\caption{}
	\label{Tab:01-07}
\end{table}
\newpage

\subsection{Números aleatorios}

La sintaxis para llamar a números aleatorios es la siguiente {\tt CALL RANDOM\_NUMBER ([HARVEST=]aleat)}, donde {\tt aleat} es un argumento real (escalar o array), devolviendo números seudoaleatorios en {\tt aleat} en el rango [0,1). \\

El otro tipo de forma de obtener un número aleatorio es usar {\tt CALL RANDOM\_SEED ([SIZE],[PUT],[GET])}, donde los argumentos son:

\begin{itemize}
	\item {\tt SIZE}: variable escalar {\tt INTEGER}. Variable de salida que contiene el tamaño {\tt N} del array semilla.
	\item {\tt PUT}: array {\tt INTEGER} de dimensión ({\tt N}). Variable de entrada utilizada para establecer la semilla.
	\item {\tt GET}: array {\tt INTEGER} de dimensión ({\tt N}). Variable de salida que contiene el valor actual de la semilla.
\end{itemize}
Si no se especifica una semmilla se establece una semilla que depende del procesador. Veamos un ejemplo de donde usamos los números aleatorios y el tiempo de cálculo:

\begin{lstlisting}[language=Fortran]
	PROGRAM aleatorio 
	INTEGER t1(8), t2(8)
	INTEGER i, numrep, semilla(1)
	REAL x, sx, tdif
	CHARACTER(LEN=8) date1, date2
	CHARACTER(LEN=10) time1, time2
	CHARACTER(LEN=5) zona

	PRINT*, ' numero de repeticiones'
	READ*, numrep
	PRINT*, ' semilla inicial'
	READ*, semilla
	CALL RANDOM_SEED (PUT=semilla)

	CALL DATE_AND_TIME (VALUES=t1, DATE=date1, ZONE=zona, TIME=time1)
	DO i = 1, numrep
		CALL RANDOM_NUMBER (x)
		sx = SIN(x)
	ENDDO

	CALL DATE_AND_TIME (VALUES=t2, TIME=time2, DATE=date2)

	PRINT*, ' zona=', zona
	PRINT*, ' date1=', date1, ' date2=', date2
	PRINT*, ' time1=', time1, ' time2=', time2

	tdif = 0.001*(t2(8)-t1(8)) + (t2(7)-t1(7)) + 60.*(t2(6)-t1(6)) + &
		3600.*(t2(5)-t1(5))
	PRINT*, ' tdif =', tdif

	ENDPROGRAM aleatorio
\end{lstlisting}



\section{Entrada y salida de datos. Ficheros. Formatos.}


\subsection{Elementos y clases de ficheros}

Los conceptos fundamentales a considerar son: campos, registro y fichero. 

\begin{itemize}
	\item \textbf{Campo:} unidad de información que consta de varios caracteres que se tratan en conjunto.
	\item \textbf{Registro:} conjuto de campos, no necesariamente del mismo tipo.
	\item \textbf{Fichero:} conjunto de registros, no necesariamente con igual estructura.
\end{itemize}
Como ejemplo un fichero de personas podría contener un registro por cada persona y los campos podrían ser: nombre, DNI, dirección, edad, teléfono\dots En algunos casos, por ejemplo en las bases de datos, los registros de un fichero tienen la misma estrucutra, esto es, el mismo número y forma de los campos. Los tipos de {\bf acceso} a un fichero en Fortran son secuencial o directo:

\begin{itemize}
	\item \textbf{Secuencial:} para acceder a un registro hay que recorrer todo el fichero desde el principio hasta llegar a él.
	\item \textbf{Directo:} conociendo el número de orden de un registro en el dichero se puede acceder a él sin tener que recorrer los registros anteriores.
\end{itemize}

Los datos pueden almacenarse en \textbf{forma} formateada o no formateada:

\begin{itemize}
	\item \textbf{Formateada}: la información se guarda como caracteres ASCII, legibles con la mayoría de los procesadores de texto.
	\item \textbf{No formateada}: un fichero es una serie de registros formados por ``bloques físicos''.
\end{itemize}

\subsection{Lectura y escritura de datos}

Internamente el ordenador representa los números y caracteres con cierta codificación. Para poder interpretar unos datos de entrada o mostrar unos datos de salida de forma legible se hacen conversiones entre la representación interna y la externa mediante especificaciones de formato. \\

Las entidades a leer o escribir se llaman listas de entrada/salida (lista I/O). En entrada se deben leer variables, en salida pueden escribirse expresiones. Si un array está en una lista I/O, se consideran todos los elementos del array en el orden de un almacenamiento del array. Una lista puede contener un DO implícito de variables.  \\

Existen 3 formas de indicar el formato de los datos a leer o escribir:

\begin{itemize}
	\item Sentencia {\tt FORMAT} con etiqueta.
	\begin{itemize}
		\item Sintaxis: {\tt e FORMAT (codform)} 
		\item Acción: {\tt codform} especifica los códigos de formato de lectura o escritura. 
		\item Normas: {\tt e} es un número de etiqueta. Es una sentencia no ejecutable.
	\end{itemize}
	\item Una expresión carácter que contiene el formato entre paréntesis.
	\item Un asterisco * que indica formato libre (lista directa de entrada-salida).
\end{itemize}

Cada fichero {\bf externo} (terminal, impresora, fichero en disco ó en cinta...) del qeu se lee o en el qeu se escribe lleva asociado un número de unidad no negativo, generalmente en el rango 1 a 99. Un número de unidad {\tt u} asociado a un fichero externo puede ser:

\begin{itemize}
	\item Una expresión entera con valor admisible (generalmente $1\leq u \leq 99$).
	\item Un asterisco: entrada/salida estándar por defecto (generalmente teclado y pantalla).
\end{itemize}

Toda la sentencia de lectura o escritura en un fichero externo debe referirse explícitamente a su número de unidad asociado. Hay dos excepciones:

\begin{itemize}
	\item Sentencia {\tt READ} sin número de unidad.
	\begin{itemize}
		\item Sintaxis: {\tt READ} sin número de unidad.
		\item Acción: lee datos del teclado (en modo interactivo). {\tt fmt} indica el formato.
	\end{itemize}
	\item Sentencia {\t PRINT}
	\begin{itemize}
		\item Sintaxis: {\tt PRINT fmt [,listavar]}
		\item Escribe los datos en la pantalla (en modo interactivo) con el formato {\tt fmt}.
	\end{itemize}
\end{itemize}

\subsection{Acceso a ficheros externos}

\subsubsection{Sentencia \texttt{OPEN}}

La sintáxis elemental es 
\begin{lstlisting}[language=Fortran]
	OPEN ([UNIT=]u,FILE=nbf)
\end{lstlisting}
Conecta la unidad {\tt u} al fichero \texttt{nbf} (``abre'' la unidad {\tt u}). La palabra clave {\tt UNIT=} es opcional; {\tt u} es una expresión entera. Por defecto el fichero se considera secuencial formateado.  {\tt nbf} es una expresión carácter que proporciona el nombre del fichero.

\begin{lstlisting}[language=Fortran]
	CHARACTER(LEN=30) nomb
	nfile = 3; nf = 3; nomb = 'cilindro.sal'
	OPEN (UNIT=3, FILE='cilindro.sal') ! estas cuatro sentencias
	OPEN (3, FILE='cilindro.sal') ! son equivalentes
	OPEN (nfile, FILE='cilindro.sal') ! si nfile=3
	OPEN (nf, FILE=nomb) ! si nf=3, nomb='cilindro.sal'
	WRITE (3,'(A)') ' El fichero 3 ha sido abierto'
	END
\end{lstlisting}

En raras ocasiones se leen y escriben datos en un mismo fichero. Normalmente, los datos de entrada se leen de ficheros que no se desean modificar y los resultados se escriben en ficheros nuevos o se añaden a ficheros ya existentes o sustituyen la información previa en ficheros existentes. Además de {\tt UNIT=u} y {\tt FILE=nbf} la sentencia {\tt OPEN} admite numerosos especificadores.

\subsubsection{Sentencia \texttt{CLOSE}}


La sintáxis elemental es 

\begin{lstlisting}[language=Fortran]
	CLOSE ([UNIT=]u)
\end{lstlisting}
Desconecta la unidad {\tt u} (``cierra'' la unidad {\tt u}). La palabra clave {\tt UNIT=} es opcional; {\tt u} es una expresión entera. Por defecto, si el programa termina normalmente se cierran todos los ficheros. Esta sentencia es útil si hay que tener abiertos A LA VEZ bastantes ficheros, ya que el número máximo de ficheros abiertos simultáneamente peude ser muy limitado.


\subsection{Entrada y salida formateada}

La entrada y salida sin número de unidad ya se ha descrito. 

\subsubsection{\texttt{READ} con número de unidad}


La sintáxis elemental es 

\begin{lstlisting}[language=Fortran]
	READ ([UNIT=]u,  [FMT=]fmt [,IOSTAT=ios] [,ERR=e1] [,END=e2]) listavar
\end{lstlisting}
Lee los datos del fichero asociado a la unidad {\tt u} según el formato {\tt fmt} y los asigna a los itmes de {\tt listavar}.

\begin{itemize}
\item {\tt u} puede ser una expresión entera, un asterisco o el nombre de un fichero interno.
\item {\tt fmt} puede ser un número de etiqueta de una sentencia {\tt FORMAT}, una lista de códigos de formato entre apóstrofos (si un carácter de esta lista es un apóstrofo hay que duplicarlo).
\item {\tt IOSTAT=ios} es opcional. {\tt ios} es una variable escalar entera que tomará un valor negativo si ocurre un fin de registro, otro valor negativo distinto si ocurre un fin de fichero y un valor positivo si ocurre alguna condición de error. Valdrá 0 si no hay error en la lectura.
\item {\tt ERR=e1} (opcional); si se produce un error de lectura se continúa en la
sentencia con etiqueta {\tt e1}, si no está {\tt ERR=e1} se para la ejecución.
\item {\tt END=e2} (opcional); si se intenta leer después del fin del fichero se continúa en la sentencia con etiqueta {\tt e2}, si no está {\tt END=e2} se para la ejecución.
\item {\tt listavar} es una lista de variables separadas por comas
\end{itemize}


\subsubsection{\texttt{WRITE} con número de unidad}


La sintáxis elemental es 

\begin{lstlisting}[language=Fortran]
	WRITE ([UNIT=]u,  [FMT=]fmt [,IOSTAT=ios] [,ERR=e1]) listavar
\end{lstlisting}
Escribe los datos de {\tt listavar} en el fichero asociado a la unidad {\tt u} según el formato {\tt fmt}. El significado de los parámetros es el mismo que en la sentencia {\tt READ}.


\subsection{Códgigos de formato}

Los códigos de formato, también llamados descriptores de edición, indican como se realiza la conversión entre las representaciones interna y externa de los datos mediante las sentencias {\tt READ, WRITE} Y {\tt PRINT}. Se clasifican en tres grupos:

\begin{itemize}
	\item Para datos:
	\begin{table}[h!] \centering
	\begin{tabular}{|l|l|l|l|l|}
		\hline
		Enteros & Reales & Lógicos & Caractéres & Generales \\ \hline
		\texttt{I, B, O, Z} & \texttt{F, E, EN, ES, D} &  \texttt{L} & \texttt{A} & \texttt{G} \\ \hline
	\end{tabular}
	\end{table}
	\item Para literales: uso de comillas.
	\item Para control:
	\begin{table}[h!] \centering
	\begin{tabular}{|l|l|l|l|l|}
		\hline
		Posicion & Espacios & Signos & Escala & Fin de formato \\ \hline
		\texttt{X,T,TR,TL} & \texttt{BN, BZ} &  \texttt{S, SP, SS} & \texttt{P} & \texttt{:} \\ \hline
	\end{tabular}
	\end{table}
\end{itemize}

Significado de los valores {\tt w, m, d, e, k, n, r}:

\begin{itemize}
	\item {\tt w}: establece la anchura del campo.
	\item {\tt m}: indica la menos $m$ cifras en el campo.
	\item {\tt d}: indica el número de cifras decimales en el campo.
	\item {\tt e}: indica el número de cifras del exponente.
	\item {\tt k}: es el factor de escala.
	\item {\tt n}: indica la posición en el registro desde su principio (para el descriptor  {\tt T}).
	\item {\tt n}: número de espacios a mover (para los descriptores {\tt X, TR, TL}).
	\item {\tt r}: factor opcional de repetición, por defecto vale 1.
\end{itemize}
Con las restricciones: \texttt{w>0, e>0, 0$\leq$m$\leq$w, 0$\leq$d$\leq$w, 0$\leq$e$\leq$w, n$\geq$1, r$\geq$1, k$\geq$0.}
 
\subsubsection{Normas de edición de códigos de formato}

\begin{itemize}
	\item En fortran 95, \texttt{w} puede ser 0 para salida en los códigos \texttt{I, B, O, Z, F} y entonces la salida tendrá la anchura mínima necesaria para contener el dato asociado.
	\item Los descriptores de edición se separan por comas, las cuales pueden omitirse en los siguientes casos:
	\begin{itemize}
		\item Entre el fator de escala {\tt P} y los códigos \texttt{F, E, EN, ES, D, G}.
		\item Antes de / (si no lleva factor de repetición).
		\item Después de /
		\item Antes o después de :		
	\end{itemize}
	\item Pueden ponerse espacios en cualquier lugar del formato.
	\item Los descriptores de edición pueden anidarse entre paréntesis.
	\item El factor de opcional de repetición {\tt r} es una constante entera positiva opcional que puede preceder a los siguientes códigos de formato: los da datos, los espacios, {\tt X}, la barra /, y grupos de código entre paréntesis. 
\end{itemize}
\subsubsection{Normas de trasferencia de códigos de formato}

Cuando se alcanza el útilmo paréntesis derecho de una especificación completa de formato se procede de la siguiente forma

\begin{itemize}
	\item Si no hay más items en la lista I/O, termina la trasferencia de datos.
	\item Si la lista I/O tiene más items que cantidad de códigos de formato contando las repeticionesw:
	\begin{itemize}
		\item Si hay paréntesis interiores con código de formato, el control de formato vuelve al principio del paréntesis izquierdo correspondiente al último paréntesis derecho precedente con su factr de repetición si lo tiene y se salta al siguiente registro.
		\item Si no hay paréntesis interiores con códigos de formato el control vuelve al principio de formato y se salta al siguiente registro.
	\end{itemize}
\end{itemize}

\subsubsection{Codigo para datos enteros}

Sintaxis:

\begin{lstlisting}[language=Fortran]
	[r]Iw[.m] 
\end{lstlisting}
donde {\tt w} es la anchura del campo. Se añaden ceros iniciales, si son necesarios, hasta completar {\tt m} caracteres. Ha de ser {\tt m$\leq$n}. Si en escritura faltan posiciones se imprimen asteriscos; si sobran se rellenan con espacios por la izquierda. Los códigos {\tt B, O, Z} se usan de manéra análoga.

\begin{lstlisting}[language=Fortran]
	OPEN (11, FILE='EJ7-10.dat')
	! OPEN (11, FILE='EJ7-10.dat', BLANK='ZERO') ! Opcional
	OPEN (12, FILE='EJ7-10.sal')
	READ (11, '(I6,3I2,I3,I4)') i, j, k, l, m, n
	WRITE (12, '(2I5,I4.2,I2,I2,I6.5)') i, j, k, l, m, n
	PRINT*, i, j, k, l, m, n
	END
	
! registro leido: b-2345b1b23b9871bb5 (por defecto los espacios se ignoran)
! asignaciones: i=-2345, j=1, k=2, l=3, m=987, n=15
! registro escrito: -2345bbbb1bb02b3**b00015
! salida en pantalla: b-2345b1b2b3b987b15
\end{lstlisting}	


\subsubsection{Codigo para datos reales}

Los códigos para datos reales dependen del códigoa a usar. Por ello debemos distinguir los 4 tipos de formas:

\begin{itemize}
	\item \textbf{Código} \texttt{F} \textbf{(datos reales sin exponente)}. En este caso la sintaxis es
	 
	\begin{lstlisting}[language=Fortran]
		[r]Fw.d
	\end{lstlisting}

	donde {\tt w} es la anchura del campo y {\tt d} número de decimales detrás del punto decimal. La entrada debe ser una constante entera o real; si lleva el punto decimal éste prevalece sobre la especificación {\tt d}. La variable de salida debe ser REAL o COMPLEJA. En salida, se debe reservar un espacio para el signo, el punto, la parte entera y los {\tt d} decimales. Si faltan posiciones se imprimen asteriscos; si sobran, se rellenan con espacios por la izquierda. Un dato complejo requiere dos códigos de formato para reales.

	\item \textbf{Código} \texttt{E} \textbf{(datos reales con exponente)}. En este caso la sintaxis es

	\begin{lstlisting}[language=Fortran]
		[r]Ew.d
	\end{lstlisting}

	donde {\tt w} es la anchura del campo y {\tt d} número de decimales detrás del punto decimal. La entrada se usa igual que en el código {\tt F}. Si faltan posiciones se imprimen asteriscos; si sobran, se rellenan con espacios por la izquierda. Se recomienda que {\tt n}$\geq${\tt d+7}.
	
	\item \textbf{Código} \texttt{D} \textbf{(datos reales con exponente d)}. En este caso la sintaxis es
	 
	\begin{lstlisting}[language=Fortran]
		[r]FD.d
	\end{lstlisting}

	donde {\tt w} es la anchura del campo y {\tt d} número de decimales detrás del punto decimal. La entrada se usa igual que en el código {\tt E}. En lectura sirve para leer datos en doble precisión. En escritura sirve para escribir datos con exponente {\tt d} que posteriormente sean leídos en doble precisión.

	\item \textbf{Código} \texttt{G} \textbf{(datos reales sin exponente)}. En este caso la sintaxis es
	
	\begin{lstlisting}[language=Fortran]
		[r]Gw.d
	\end{lstlisting}

	En entrada se usa igual que en el código {\tt E}. La salida es de tipo {\tt F} o {\tt E} dependiendo de la magnitud del dato.
\end{itemize}

\subsubsection{Codigo para datos lógicos}
En este caso la sintaxis es
	
\begin{lstlisting}[language=Fortran]
	[r]Lw
\end{lstlisting}

En entrada si el primer carácter no espacio es {\tt T} o {\tt .T} se asigna {\tt .TRUE.} al dato leído, si es {\tt F} o {\tt .F} se asigna {\tt .FALSE.}. En salida se escribe {\tt T} o {\tt F} precedida de {\tt w-1} espacios.

\subsubsection{Codigo para datos carácter}
En este caso la sintaxis es
	
\begin{lstlisting}[language=Fortran]
	[r]A[w]
\end{lstlisting}

Si se usa la {\tt A} forma sin especificar, {\tt w}, se leen o escriben un número de caracteres igual a la longitud del item correspondiente de la lista I/O. En entrada sea {\tt lon} la longitud del dato carácter a leer con formato {\tt Aw}. 

\begin{itemize}
	\item Si {\tt w} $\geq$ {\tt lon} se leen los {\tt lon} caracteres más a la derecha del campo.
	\item Si {\tt w} $<$ {\tt lon} se leen los {\tt w} caracteres, se asignan justificados por la izquierda al dato carácter y se completa con {\tt lon-w}.
\end{itemize}
En salidad sea {\tt lon} la longitud del dato carácter a escribir con formato {\tt Aw.}

\begin{itemize}
	\item Si {\tt w} $>$ {\tt lon} se escriben {\tt w-lon} espacios seguidos de {\tt lon} caracteres del dato carácter.
	\item Si {\tt w} $\leq$ {\tt lon} se escriben los {\tt w} caracteres iniciales del dato carácter.
\end{itemize}

\subsection{Codigos de control}

Existen diferentes tipos de códigos de control:

\begin{itemize}
	\item \textbf{Espacio:} {\tt nX}. En entrada salta {\tt n} caractéres sin cambiar de registro. En salida deja {\tt n} espacios en blanco antes de escribir el próxmo item.
	\item \textbf{Tabulación absoluta:} {\tt Tn}. Sitúal a posición de lectura/escritura justo antes de a columna {\tt n} del registro actual, repsecto al límtie de tabulación permitida.
	\item \textbf{Tabulación a la derecha:} {\tt TRn}. Sitúa la posición de lectura/escritura {\tt n} columnas a la derecha a partir de la actual en el registro actual.
	\item \textbf{Tabulación a la izquierda:} {\tt RLn}. Sitúa la posición de lectura/escritura {\tt n} columnas a la izquierda a partir de la actual en el registro actual, con el límite de la tabulación izquierda.
	\item \textbf{Salto de registro:} {\tt /}. Indica el fin de trasferencia de datos al (del) registro actual, esto es, se salta al principio del siguiente registro para leer o escribir. Si hay n barras / consecutivas al principio o al final de la sentencia {\tt FORMAT} se saltan {\tt n} registros; si están en el interior sólo {\tt n-1}.
	\item \textbf{Fin de formato:} {\tt :}. Si al llegar a los : no quedan más items en la lista I/O acaba el formato, si quedan más se ignoran los :. Es útil en salida.
	\item \textbf{Control de carro:} {\tt 'b', '0', '1', '+'}. Las sentencias de salida formateada fueron diseñadas en su origen para impresoras de líneas, con el concepto de línea y página. Cuando la sentencia {\tt WRITE} envía datos a una impresora, el primer carácter de cada registro se interpreta como control y no se imprime. El efecto del primer carácter es:
	
	\begin{itemize}
		\item {\tt b}: empieza en una nueva línea.
		\item {\tt 0}: salta una línea.
		\item {\tt 1}: avanza hasta el principio de la página siguiente.
		\item {\tt +}: no avanza, permanece en la misma línea.
	\end{itemize}
\end{itemize}
Es una buena práctica de programación insertar un blanco como primer carácter de cada registro cuando se envía a la pantalla o a una impresora. Puede hacerse comenzando los formatos por {\tt (1X,...} o por {\tt (T2,...}

\subsection{Posicionamiento de ficheros}

\subsubsection{Sentencia {\tt BACKSPACE}}

La sintaxis 

\begin{lstlisting}[language=Fortran]
	BACKSPACE ([UNIT=]u,  [FMT=]fmt [,IOSTAT=ios] [,ERR=e1] [,END=e2])
\end{lstlisting}
si el fichero conectado a la unidad {\tt u} está posicionado dentro de un registro se vuelve al principio del registro actual; si está posicionado entre registros se vuelve al principio del registro precedente. {\tt IOSTAT, ERR} tienen el mismo signifiado que en {\tt READ}. {\bf Sirve para releer registros y para remplazar registros escritos.}

\subsubsection{Sentencia {\tt REWIND}}
Sitaxis:
\begin{lstlisting}[language=Fortran]
	BACKSPACE ([UNIT=]u,  [FMT=]fmt [,IOSTAT=ios] [,ERR=e1] [,END=e2])
\end{lstlisting}
Posiciona el fichero conectado a la unidad {\tt u} al principio de su primer registro. {\tt IOSTAT, ERR} tienen el mismo signifiado que en {\tt READ}. 

\subsubsection{Sentencia {\tt ENDFILE}}
La sintaxis elemental

\begin{lstlisting}[language=Fortran]
	ENDFILE ([UNIT=]u,  [FMT=]fmt [,IOSTAT=ios] [,ERR=e1] [,END=e2])
\end{lstlisting}
\subsection{Ficheros internos}
Escribe un registro de fin de fichero conectado a la unidad {\tt u}. Se posiciona después del registro de fin de fichero. {\tt IOSTAT, ERR} tienen el mismo signifiado que en {\tt READ}. Se escribe automáticamente un registro de fin de fichero si:

\begin{itemize}
	\item Se ejecuta \texttt{BACKSPACE} o  \texttt{REWIND} después de {\tt WRITE} en la unidad {\tt u}.
	\item Se cierra el fichero con {\tt CLOSE}.
	\item Se ejecuta {\tt OPEN} con la unidad {\tt u}.
	\item El programa termina normalmente.
\end{itemize}

\section{Elaboración de programas}

Es importante cuidar la elaboración del programa fuente y procurar satisfacer varios objetivos, entre otros: que sea claro y legible tanto para el autor del programa como para otros potenciales usuarios, que sea fácil de detectar errores, que sea eficiente en tiempo, precisión o memoria, que permita introducir cambios con facilidad, etc.

\subsection{Estilo de programación}

Algunos detalles que favorecen el estilo de programación son:

\begin{itemize}
	\item Amplio uso de comentarios: incluir una breve descripción de algoritmos o procedimientos al principio de cada unidad de programa, en secciones de código diferencias, en límties de arrays, en sentencias que deberían cambiarse para ejecución con otros datos, etc.
	\item Descripción del significado de cada variable.
	\item Declaración organizada de variables (alfabética, por tipos, agrupada por similaridades, etc.).
	\item Líneas en blanco de separación entre secciones de código (blucles, bloques, {\tt IF},...) y entre subprogramas.
	\item Desplazamiento (``Identación'') de las sentencais de estructuras (blucles, bloques, {\tt IF, CASE,...}) unos espacios (6 espacios).

\end{itemize}
\subsection{Depuración de errores}

Los errores que pueden cometerse en la elaboración de un programa Fortran son de clase muy diversa: sintaxis, diseño de programa, programación, algorítmicos, instalación del software, errores de tamaño de memoria, etc. Una vez realizada una correcta instalción del software, los otros errores son imputables al usuario ó a limitaciones del software o hardware. La ley de MURPHY no falla cuando se aplica en programación. Algunos detalles que favorecen la detección y correción de errores son:

\begin{itemize}
	\item Una redaccion clara con suficientes comentarios.
	\item Evitar estructuras de control, formatos y expresiones complicados.
	\item Si un programa es muy largo, conviene partirlo en subprogramas. Es difícil corregir un subprograma de más de unasd 300 líneas ejecutables.
	\item En primeras versiones de un programa, conviene incluir sentencias de escritura (a pantalla ó fichero) después de secciones diferenciadas de código con objeto de aislar posibles errores o comprobar el buen funcionamiento de partes de código.
\end{itemize}
 

\subsection{Optimización de programas}

Algunos detalles que afectan a la eficiencia de un programa son:

\begin{itemize}
	\item Uso de la opción de compilación para optimizar la velocidad de ejecución.
	\item Potenciación {\tt a**b}:
	      \begin{itemize}
		      \item Si {\tt b }es entero, tenemos que hasta {\tt b}$>$5 la exponenciación se obtiene con {\it multiplicaciones}.
		      \item Si {\tt b }es real, se calcula como {\tt EXP(b*LOG(a))}.
	      \end{itemize}
	\item La raíz cuadrada es una operación rápida.
	\item Siempre que se pueda conviene ahorrar operaciones y simplificar fórmulas aunque las expresiones pueden ser numéricamente distintas.
	\item Si no hay peligro conviene reutilizar variables, vectores, y matrices. Si los elementos de un vector o matriz son conocidos, se pueden prescidir de ellos. El acceso a elementos de arrays consume tiempo.
\end{itemize}
\chapter{Solido}

\section{Práctica 1: }
\section{Práctica 2: }
\section{Práctica 3: }
\section{Práctica 4: }
\section{Práctica 5: }
\section{Práctica 6: }
\section{Práctica 7: }
\section{Práctica 8: }



\newpage

\section*{\textit{Examenes resueltos}}
\addcontentsline{toc}{section}{\textit{Examenes resueltos}}

\subsection*{\textit{2024}}\begin{Enunciado}
	\addcontentsline{toc}{subsection}{\textit{2024}}

	\subsubsection{Difracción de rayos X (Elisa Casal)}
	Preguntas
	\begin{enumerate}[label=\alph*)]
		\item Se tiene una muestra policristalina en polvo de un material con estructura bcc de parámetro de red 4 Å. Si se realiza un difractograma de rayos X con una longitud de onda \( \lambda = 1.540\ \text{\AA} \), ¿a qué ángulos \( \theta \) aparecen los tres primeros picos de difracción?
		\item Si los ángulos de todos los picos de difracción estuviesen sobreestimados en aprox. \( 1^\circ \), ¿las constantes de red que se obtienen estarán sobreestimadas o subestimadas?
	\end{enumerate}
\end{Enunciado}


Solución
\begin{enumerate}[label=\alph*)]
	\item La ecuación que tenemos que usar es la siguiente:

	      \begin{equation}
		      \sin (\theta)=\frac{\lambda}{2a}\sqrt{h^2+k^2+l^2}
	      \end{equation}

	      No obstante, es necesario particularizarla según el criterio de selección de (h,k,l) para una FCC que es que h+k+l ha de ser par. Teniendo esto en cuenta para las tres primeras ternas que verifican el criterio de selección tenemos:

	      \begin{itemize}
		      \item (1,1,0) $\longrightarrow$ $\theta_1$=15.80º
		      \item (2,0,0) $\longrightarrow$ $\theta_2$=22.64º
		      \item (2,1,1) $\longrightarrow$ $\theta_3$=32.98º
	      \end{itemize}

	\item Si todos los ángulos medidos están sobreestimados tenemos $\theta_{\text{medida}}>\theta_{\text{real}}$. Para los ángulos de arriba ello implica: $\sin(\theta_{\text{medida}})<\sin(\theta_{\text{real}})$. Considerando que $a \propto 1/\sin(\theta)$ obtendríamos $a_{\text{medida}}<a_{\text{real}}$, es decir, obtenemos un valor infraestimado para la constante de red.
\end{enumerate}

\vspace*{2em}

\begin{Enunciado}
	\subsubsection{Fotoconductividad}
	Preguntas
	\begin{enumerate}[label=\alph*)]
		\item Confirma o refuta la siguiente afirmación: ``La conductividad de exceso, que se define como la conductividad por encima de la ambiental, resulta seguir una ley no lineal (en concreto, potencial) con la irradiancia recibida por el fotorreceptor, con un exponente ligeramente mayor que la unidad''.
		\item ¿Cómo mediste la conductancia intrínseca $G_0$? ¿Es mayor o menor que la que se obtiene con la luz de la bombilla incidiendo sobre la fotocélula?
	\end{enumerate}
\end{Enunciado}

Solución:

\begin{enumerate}[label=\alph*)]
	\item Aunque efectivamente la conductividad de exceso de define como la conductividad por encima de la ambiental, es decir:
	      \begin{equation*}
		      \Delta G (\Phi) = G(\Phi)  - G_0
	      \end{equation*}
	      siendo $\Phi$ la irradiancia, la relación no es lineal, es efectivamente potencial (con la raíz cuadrada), tal y como vamos a demostrar aquí. La realción con el número de portadores $n$ del exceso de conductividad es tal que
	      \begin{equation*}
		      \Delta G \propto n(\Phi) - n_0
	      \end{equation*}
	      y, el número de portadores $n(\Phi)$, a su vez, una relación no lineal con la irradiancia
	      \begin{equation*}
		      \gamma n(\Phi)^2 = \beta  \Phi
	      \end{equation*}
	      Es decir, la relación de la conductividad de exceso y la irradiancia se relaciónan a través de la raiz cuadrada:

	      \begin{equation*}
		      \Delta G \propto \sqrt{\Phi}
	      \end{equation*}
	      en general, al no tener una fuente monocromática, en realidad

	      \begin{equation*}
		      \Delta G \propto \Phi^{m/2}
	      \end{equation*}
	      siendo $m$ un valor que debería rondar sobre 1. Experimentalmente se obtuvo (Daniel Vázquez Lago) un valor $m\approx 1.06$, mientras que en otras prácticas (Raquel Alfonso) se obtuvo $m\approx 1.3-1.6$.

	\item La conductancia intrísneca que es la conductancia que tiene el el fotodiodo cuando está aislado (no recibe ningún tipo de irradiancia ambiental o externa). Para medirla estudiamos $I$ frente a $V$, tal que la ley de Ohm y una regresión lineal nos dará el valor de $G_0$:
	      \begin{equation*}
		      V = G_0 I
	      \end{equation*}
	      Es menor que la que se obtiene con la luz de la bombilla incidiendo sobre la fotocélula, ya que la luz genera un exceso de portadores al excitar electrones en la banda de valencia del semiconductor.
\end{enumerate}

\vspace*{2em}

\begin{Enunciado}
	\subsubsection{Conductividad eléctrica de placa dentada}
	Preguntas
	\begin{enumerate}[label=\alph*)]
		\item En la imposición de condiciones de contorno sobre la corriente de entrada, \textit{Agros2D} pide la densidad de corriente. ¿Cómo se obtiene la sección para determinar dicha densidad de corriente a partir de la intensidad?
		\item Indica y explica brevemente la regla de proporcionalidad que se aplica para la determinación de $\sigma$ a partir de los resultados de la simulación.
	\end{enumerate}
\end{Enunciado}

Solución:

\begin{enumerate}[label=\alph*)]
	\item La densidad de corriente de entrada $J$ y la intensidad que medimos en el laboratorio se relacionan de la siguiente manera:
	      \begin{equation*}
		      J = \frac{I}{A}
	      \end{equation*}
	      siendo este área el área que está alrededor del tornillo, véase

	      \begin{equation}
		      A = 2 \pi r \cdot d
	      \end{equation}
	      siendo $r$ el radio del borne y $d$ el espesor de la placa.
	      \begin{figure}[H] \centering
		      \includegraphics[width=0.6\linewidth]{Cuerpo/Ch_02/Examen_24_3.png}
	      \end{figure}
	\item La relación entre $\sigma_{\exp}$ y $\sigma_{sim}$ basicamente nos dice que:

	      \begin{equation*}
		      V_{\exp} \sigma_{\exp} = V_{sim} \cdot \sigma_{sim}
	      \end{equation*}
	      donde $\sigma$ es la conductividad y $\rho=1/\sigma$ sería la resistividad. El funcional de resistencia $R(\rho)$ depende de la resisitividad del metal y de la forma geométrica, que por definición dependerá linealmente de la resistividad $R(\rho) \propto \rho$ (piensa en las unidades, debe ser necesariemente lineal ya que $R$ va con $\Omega$ y $\rho$ va con $\Omega \cdot m$),y por tanto entre 2 metales con la misma forma:

	      \begin{equation}
		      \frac{R_{sim}}{R_{\exp}} = \frac{\rho_{sim}}{\rho_{\exp}}
	      \end{equation}
	      Y si ambos tienen la misma intensidad entonces la ley de Ohm (que en un metal se cumple siempre)

	      \begin{equation*}
		      V = R(\rho) I \longrightarrow I = \frac{V}{R(\rho)}
	      \end{equation*}
	      Finalmente, si $\rho = 1/\sigma$, efectivamente:

	      \begin{equation*}
		      V_{\exp} \sigma_{\exp} = V_{sim} \cdot \sigma_{sim}
	      \end{equation*}
	      \textbf{Resumen}: en el caso de que la geometría de dos objetos sea igual, ambos sean óhmicos y que les atraviese la misma corriente por los bordes, entonces la relacion voltaje x conductividad será igual en ambos.
\end{enumerate}

\vspace*{2em}
\begin{Enunciado}
    \subsubsection{Conductividad eléctrica de placa perforada (Elisa Casal)}
    Preguntas
    \begin{enumerate}[label=\alph*)]
    \item ¿A qué puntos se les impuso la condición de contorno de voltaje nulo? Discute brevemente por qué.
    \item En la simulación por elementos finitos, ¿de qué modo y dónde entraba el espesor de la placa metálica?
    \end{enumerate}
    \end{Enunciado}
    
Solución
\begin{enumerate}[label=\alph*)]
	\item El voltaje nulo se le puso como condición de contorno al terminal de salida.

	\item El espesor entra en juego para establecer la $J_0$ en la condición de contorno sobre el terminal de entrada.

	      Nosotros lo que hemos medido previamente a la simulación es $I_0$, que es la corriente que establecimos en las medidas experimentales y el $V_{exp}$ al que da lugar esta corriente. Como para introducir en el \textit{Agros2d} lo que necesitamos es una denisdad de corriente habrá que dividir $I_0$ entre el área tranversal que atraviesa. Este es un método que puede introducir mucha incertidumbre. El área tranversal que atraviesa la corriente será la longitud, \textit{L}, del terminal multiplicada por el espesor, \textit{d}, de la placa. Esa longitud, en principio, debería de ser la de los tres lados que forman el terminal. De esta manera la denidad de corriente para introducir como condición de contorno en el terminal de entrada es:

	      \begin{equation}
		      J_0=\frac{I_0}{L \cdot d}
	      \end{equation}
\end{enumerate}

\vspace*{2em}

\begin{Enunciado}
	\subsubsection{Temperatura de Debye}
	Preguntas
	\begin{enumerate}[label=\alph*)]
		\item Si tenemos en cuenta el intercambio de calor entre nuestro sistema (vaso con agua más muestra) con el ambiente durante el proceso de enfriamiento que experimenta el agua al introducir la muestra de cobre, ¿la temperatura final del agua es mayor o menor que la medida experimentalmente? ¿Y qué sucede con $\Theta_D$? ¿Aumenta o disminuye?
		\item Teniendo en cuenta el procedimiento experimental que seguimos en la práctica, ¿con qué tipo de material sería más “fácil” (en el sentido de precisa) la determinación de $\Theta_D$? ¿Con uno de $\Theta_D$ muy baja (digamos $\Theta_D \approx 50\ \text{K}$) o con otro de $\Theta_D$ muy alta (digamos $\Theta_D \approx 600\ \text{K}$)? ¿Y con respecto a la masa de la muestra? ¿Es mejor que sea más ligera o más pesada?
	\end{enumerate}
\end{Enunciado}

Solucion:

\begin{enumerate}[label=\alph*)]
	\item Si consideramos el intercambio de calor entre nuestro sistema (vaso con agua + muestra) y el ambiente, la temperatura final del agua \textit{medida experimentalmente será mayor que la real}. Esto ocurre porque, durante el proceso, el agua gana energía del ambiente al estar mas frío que este. Por tanto, la medida de $T_f$ será algo mayor que si el sistema estuviese perfectamente aislado, de tal modo que $\Delta T_{agua}$ es menor. consecuentemente, el valor $\Delta U_{\exp}$ será menor también que en el caso aislado.

	      Ahora, veamos que la siguiente gráfica en la que representamos $C_V$ frenet $T$ y $T/\theta_D$ respectivamente.

	      \begin{minipage}{0.47\linewidth}
		      \includegraphics[width=1\linewidth]{Cuerpo/Ch_02/Examen_24_5-2.png}
	      \end{minipage} \hfill
	      \begin{minipage}{0.47\linewidth}
		      \includegraphics[width=1\linewidth]{Cuerpo/Ch_02/Examen_24_5-3.png}
	      \end{minipage}

	      Dado que $\Delta U$ entre entre dos valores de $T/\Theta_D$ es la integral debajo de la curva de $C_V$:

	      \[
		      \Delta U_{\text{teo}} = \int_{T_i}^{T_f} C_V(T)_{\text{cobre}} \D T
	      \]

	      Y por tanto tenemos que efectivamente cuando mas alto $T/\theta_D$ aumenta el valor de $\Delta U$. Si aumentamos $T_f \uparrow$ (recordamos que $T_i=73$ K es fijo, el valor del nitrógen líquido), tenemos que $\Delta U_{\text{teo}} \uparrow$, sin embargo tenemos experimentalmente que $\Delta U_{\exp} \downarrow$. Así pues, la única forma de obtener un $\Delta U_{\text{teo}} \downarrow$ cuando $T_f \uparrow$ es aumentar $\theta_D \uparrow$. Es decir, $\theta_D$ experimental aumentará respecto al valor aislado. De hecho es lo que ocurre experimentalmente, ya que $\theta_D=343$ K (cobre) y experimentalmente en el laboratorio de mide $\theta_D \approx 360$ K (Elisa Casal).

	\item Si tenemos un material $\theta_D\approx 600$ K estaremos en el límite cuántico ($T<\Theta_D$), mientras que $\theta_D \approx 50$ K estaremos en el límite clásico. Si le echamos un vistazo a a laimagen .

	Como podemos ver, en el límite clásico, una variación de temperatura pequeña puede alterar bastante el valor de la energía interna, mientras que en el límite cuántico necesitaremos mayor temperatura para obtener la misma $\Delta U$. Necesitar menos temperatura, o que con la misma diferencia de temperatura inicial y final pueden mejorar bastante los resultados: mas diferencia habrá entre la temperatura inicial y final del agua, y por tanto mayor sensibilidad. Hablamos de temperatura de la muestra. Es decir, \textit{preferimos una temperatura de Debye baja}.

	Sin embargo es posible que alguién quiera barajar la idea de que para $\Theta_D$ baja $C_V$ no depende realmente de $\Theta_D$, y por tanto variar $\Theta_D$ cuando calculamos el valor $\Delta U_{\exp}$ para que se verifique $\Delta U_{\exp}\approx \Delta U_{\text{teo}}$. El autor no es particularmente seguidor de esta idea, ya que aunque esto es verdad uno va a seguir usando las ecauciones de $D_3(T)$ igual (por lo que el error debido a usar la aproximación será "el mismo"), además de ser una explicación bastante más compleja y que habría que ver si realmente sucede numéricamente. En cualquier caso habría que considerarla como un argumento en contra de $\Theta_D$ baja. Si a alguien se le ocurre otra idea que contacte conmigo.


	Respecto a la masa de la muestra, una muestra más pesada intercambia más calor con el agua (mayor $\Delta U$), ya que es capaz de ``almacenar más energía en su interior''. Es decir, para el mismo material cobre, más masa implica más moles, y más moles de cobre implican que $\Delta U$ será mas grande para el mismo $\Delta T$, y por tanto mayor sensibilidad.
\end{enumerate}

\vspace*{2em}

\begin{Enunciado}
	\subsubsection{Efecto Hall en semiconductores}
	Preguntas
	\begin{enumerate}[label=\alph*)]
		\item Supón que, debido por ejemplo a un efecto termoeléctrico, en toda la experiencia las lecturas de caída de voltaje entre los terminales transversales del semiconductor tienen una contribución extra de \(-0.1\,\text{V}\), y que no la corregimos. ¿Cómo afectaría ello al valor obtenido para el coeficiente Hall \( R_H \)?
		\item En un montaje experimental dado, con polaridades fijas de las conexiones entre la muestra, fuentes y demás equipamiento, cuando se consideran casos de conductores por electrones o por huecos, ¿las fuerzas de Lorentz tienen igual signo o signos opuestos en un caso frente al otro?
	\end{enumerate}
\end{Enunciado}

\begin{enumerate}[label=\alph*)]
	\item No, no nos afecta, ya que lo que nosotros usamos para calcular el efectp Hall es usar la pendiente de la curva, la cual no se ve afectada al aumentar $I$ uniformemente. Aumentar $I$ uniformemente solo modifica el término independiente.

	\item Las fuerzas de Lorentz actúan en el mismo sentido. Ingénuamente uno podría pensar que cambian ya que

	      \[
		      \Fn = q (\vn \times \Bn)
	      \]

	      tal que para huecos $q>0$ y para electrones $q<0$. Sin embargo, no estamos teniendo en cuenta que la intensidad se relaciona con la velocidad de la siguiente forma:

	      \[
		      \Jn = q \rho \langle \vn \rangle
	      \]

	      donde $\Jn$ es la densidad de carga y $\rho$ la densidad de portadores. Es decir, tenemos que

	      \[
		      \Fn = \frac{1}{\rho} (\Jn \times \Bn)
	      \]

	      viendo que efectivamente es independiente de la carga, y como nos dicen que las polarizaciones son iguales (mismo sentido de campo magnético y mismo sentido de intensidad) tenemos que $\Fn$ no depende del signo de la carga. En la imagen siguiente viene representado:

	      \begin{figure}[H] \centering
		      \includegraphics[width=0.8\linewidth]{Cuerpo/Ch_02/Examen_24_6.png}
	      \end{figure}


\end{enumerate}
\vspace*{2em}

\begin{Enunciado}
	\subsubsection{Gap de energía en Ge}
	\begin{minipage}{0.5\linewidth}
		Preguntas
		\begin{enumerate}[label=\alph*)]
			\item Describe el montaje utilizado para la realización de la práctica.
			\item ¿Por qué se utiliza una resistencia de carga (resistencia tampón) en el montaje de la práctica?
			\item Sistemáticamente, se observa una pequeña diferencia entre las curvas conductividad-temperatura a la subida y a la bajada de temperatura. En la gráfica adjunta, ¿qué símbolos corresponden a la bajada y cuáles a la subida?
		\end{enumerate}
	\end{minipage}\hfill
	\begin{minipage}{0.45\linewidth}
		\includegraphics[width=0.9\linewidth]{Cuerpo/Ch_02/Examen_24_7.png}
	\end{minipage}
\end{Enunciado}

Solución:

\begin{enumerate}[label=\alph*)]

	\item El montaje experimental consiste en una placa de Ge-n dopada (20×10×1 mm³), colocada sobre una baquelita portamuestras con calefactor y termómetro incorporados. La corriente que atraviesa el Ge se regula desde el conversor digital-analógico DAC1 del LabJack U6-PRO, y se mantiene en torno a 5 mA mediante una resistencia tampón de 332 Ω colocada en serie.

	      El sistema mide tres voltajes clave:
	      \begin{itemize}
		      \item \( V_{\text{Rt}} \): caída de tensión en la resistencia tampón, que permite calcular la corriente \( I_{\text{Ge}} \)
		      \item \( V_{\text{Ge}} \): caída de tensión en el propio germanio
		      \item \( V_{\text{Pt}} \): voltaje del termómetro de resistencia Pt-100, proporcional a la temperatura de la muestra
	      \end{itemize}

	      La temperatura se regula aplicando corriente al calefactor (resistencia \( R_{\text{cal}} \approx 3.5\ \Omega \)), controlado manualmente. El software registra los datos automáticamente y permite observar en tiempo real la gráfica \( G(T) \), donde \( G = I_{\text{Ge}} / V_{\text{Ge}} \).

	\item La resistencia de carga (o resistencia tampón) de 332 Ω se coloca en serie con la muestra de germanio por dos razones fundamentales:

	      \begin{itemize}
		      \item \textbf{Proteger el germanio}: limita la corriente máxima que puede circular, evitando que se supere el valor recomendado (\( I_{\text{max}} \approx 10\ \text{mA} \)) y previniendo daños térmicos por sobrecorriente.
		      \item \textbf{Medir la corriente indirectamente}: la caída de tensión \( V_{\text{Rt}} \) sobre esta resistencia permite calcular fácilmente la corriente usando la ley de Ohm:
		            \[
			            I_{\text{Ge}} = \frac{V_{\text{Rt}}}{R_{\text{t}}}
		            \]
		            Esto permite obtener la conductancia del germanio sin necesidad de insertar un amperímetro, manteniendo el circuito simple y preciso.
	      \end{itemize}

	\item En el guión se nos indica, que debemos graficar $\ln G$ vs $1/T$, subiendo y bajando para verificar si hay error asociado a la incercia  térmica o al sensor, como podemos ver. La ``incercia  térmica'' basicamente es el proceso por el cual durante la subida de temperatura, el sensor va retrasado respecto la muestra (es decir, se calienta más rápido de lo que el sensor muestra), y durante la bajada, se enfría más lento de lo que el sensor lo indica. Consecuentemente la bajada respecto la curva real está desplazada hacia la derecha y la subia está desplazada hacia la izquierda del valor real: gris circulo subida, negro triángulo bajada. 
\end{enumerate}

\vspace*{2em}

\begin{Enunciado}
	\subsubsection{Efecto Hall en metales}
	Preguntas
	\begin{enumerate}[label=\alph*)]
		\begin{minipage}{0.5\linewidth}
			\item Supongamos que la placa del metal no estuviese a escuadra con el sistema de bobinas, sino que el campo \( \vec{H} \) formase un ángulo de 5° con la normal a la placa. ¿Induciría esto que se obtuviese un valor absoluto del coeficiente Hall \( R_H \) mayor o menor que el real?

		\end{minipage}\hfill\begin{minipage}{0.45\linewidth}
			\includegraphics[width=0.9\linewidth]{Cuerpo/Ch_02/Examen_24_8.png}
		\end{minipage}
		\item Una muestra se somete a un campo magnético de \( 0.1\,\text{T} \) y una corriente eléctrica de \( 1\,\text{A} \) como se indica en la figura. A partir del voltaje resultante, determina el signo de los portadores. Suponer que los contactos de voltaje están perfectamente alineados.
	\end{enumerate}
\end{Enunciado}

Solución:

\begin{enumerate}[label=\alph*)]
	\item Como sabemos el coeficiente Hall es el la relación que hay entre el voltaje transversal que se produce por el efecto del campo magnético sobre una corriente, tal que:
	      \[
		      \frac{1}{R_H}\Encal  = \Jn \times \Bn
	      \]
	      tal que, si efectivamente $\Jn = J_x \hnx $ y $\Bn = B_z \hnz$, siempre y cuando midamos  tenemos que:

	      \[
		      R_H = \frac{E_y}{J_x B_z}
	      \]
	      podemos usar el voltaje que aparece entre las terminales $E_y = - V_H d$ donde $d$ es el tamaño de la placa en la dirección. Como podemos ver, esto es deducible a parir de las ecuacioens de Lorentz, en las cuales:

	      \[
		      R_H = \frac{1}{qn}
	      \]
	      Como podemos ver no depende del ángulo de $J_x$ o $B_z$, es una propiedad intrínseca del metal. Si cambiaramos el ángulo tal que $\Jn = J (\cos (\theta) \hnx + \sin (\theta) \hny)$  tendríamos que simplemente mediríamos un $\Encal$ más pequeño, es decir, tendríamos un voltaje más pequeño, no un $R_H$ más pequeño.


	\item Como en el apartado anterior, definimos
	      \[
		      R_H = \frac{E_y}{J_x B_z}
	      \]
	      donde $\hny = \hnx \times \hnz$. Usando las ecuaciones anteriores, tenemos que $E_y = - \Delta V d$, siendo $\Delta V = V_H(d)-V_H(0)$ la difernecia de voltaje entre la placa en $0$ y en $d$. Como podemos ver, en este caso $V_H(0)=0$ y $V_H(d)=-5$ mV, es decir, $R_H$ es positivo y por tanto, por definición esto implica que el signo de los portadores es positivos.


\end{enumerate}

\newpage

\subsection*{\textit{Mayo 2015}}
\addcontentsline{toc}{subsection}{\textit{Mayo 2015}}

\begin{Enunciado}
	\subsection*{Difracción de rayos X}
	\lipsum[1]
\end{Enunciado}

\begin{enumerate}
	\item La ecuación que tenemos que usar es la siguiente:

	      \begin{equation}
		      \sin(\theta)=\frac{\lambda}{2a}\sqrt{h^2+k^2+l^2}
	      \end{equation}

	      No obstante, es necesario particularizarla según el criterio de selección de (h,k,l) para una FCC que es que (h,k,l) deben tener la misma paridad. Teniendo esto en cuenta para las tres primeras ternas que verifican el criterio de selección tenemos:

	      \begin{itemize}
		      \item (1,1,1) $\longrightarrow$ 2$\theta_1$=30.94º
		      \item (2,0,0) $\longrightarrow$ 2$\theta_2$=35.88º
		      \item (2,2,0) $\longrightarrow$ 2$\theta_3$=51.64º
	      \end{itemize}

	\item La no monocromaticidad da lugar a la dispersión de los datos en torno ángulo dado por la $\lambda$ "que más peso tiene".
\end{enumerate}

\chapter{Reacciones Nucleares}

Las reacciones nuclares más comunes tienen lugar cuando una partícula enegética incide sobre un núcleo y éste se transforma: exictándose, rompiéndose o simplemente absorbiendo la partícula incidente. Estas partículas incidentes son generalmente neutrones, protones, partículas $\alpha$ o fotones $\gamma$. Para que penetren dentro de del núcleo sobre el que inciden es necesario que lleven cierta energía. En la Tierra esa energía se puede conseguir con aceleradores o reactores nucleares, y también a partri de fuentes naturales radiactivas. Las reacciones nucleares permiten el estudio de las interacciones que gobiernan el mundo subnuclear y, por otro lado, proporcionan la mayor parte de los datos tabulados sobre las propiedades nucleares. Estas dos cosas, están obviamente relacionadas, porque sólo es posible entender las propiedades de los núcleos, si se posee al mismo tiempo una buena comprensión de las interacciones nucleares.

% Faltan cosas

\section{Tipos de reacciones}

Representamos una reacción nuclear típica de las siguientes dos maneras equivalentes:

\begin{equation}
    a + A  \longrightarrow B + b \tquad A(a,b)B \label{Ec:03-01-01}
\end{equation}
donde $a$ es el proyectil o partícula acelerada que se hace incidir sobre el núcleo blanco, $A$, en reposo en el sistema laboratorio. De las partículas en el segundo miembro de la reacción, $B$ suele ser un núcleo pesado que no abandona el material del blanco, y $b$ una partícula para referirse a un conjunto de reacciones del mismo tipo. Así diríamos reacciones ($\alpha,n$) o ($n,\gamma$), por ejemplo. Hay muchas maneras de clasificar las reacciones nucleares. He aquí unas cuantas:

\begin{itemize}
    \item Se suele hablar de una \textbf{reacción de dispersión} (\textit{scattering process}) cuando las partículas iniciales y finales son las mismas\footnote{En física de partículas de altas energías se usa el término \textit{scattering} de un modo más general, no sólo para referirse a reaccioens en las que las partículas iniciales coinciden con las finales. En el \textit{deep inelastic scattering} por ejemplo, la energía del proyectil es tan alta que se producen muchas partículas en el estado final}. La dispersión puede ser \textbf{elástica} si las patículas o núcleos $B$ y $b$ se encuentran en su estado fundamental, o \textbf{inelástica} si alguna de las dos queda en un estado excitado, que posteriormente se suele desexcitar por emisión gamma. En una dispersión elástica la energía cinética se conserva ($Q=0$) y simplemente se redistribuye entre las partículas interaccionantes.
    \item Si las partículas $a$ y $b$ son la misma, y además tenemos otro nucleón en el estado final (3 partículas como productos) se suele denominar una \textbf{reacción knockout}.
    \item Tenemos una \textbf{reacciones de transferencia} (\textit{transfer reaction}) cuando se transfieren uno o varios nucleones entre el proyectil y el blanco.
\end{itemize}

% Faltan cosas


\section{Leyes de conservación}

\subsection{Conservación de la carga eléctrica y el número bariónico}

Aunque veremos en un capítulo posterior estas leyes de conservación con algo más de detalle, conviene mencionarlas ya aquí. La carga eléctrica total de las partículas iniciales de la reacción es siempre igual a la de las partículas finales. La \textbf{conservación de la carga eléctrica} es una ley \textit{muy fundamental} en nuestro entendimiento actual de la física, al mismo nivel que la ley de conservación de la energía.

% Insertar foto tikz

Si asginamos a cada a cada barión\footnote{Un barión es un hadrón (partícula sensible a la interacción fuerte) con espín semientero. Los bariones más ligeros son los familiares protón y neutrón.} una unidad positiva de \textit{número bariónico} y a cada antibarión una unidad negativa, podemos formular una ley de \textbf{conservación del número de bariónico} en cualqueir reacción diciendo que la suma de números bariónicos para las partículas iniciales debe coincidir con la misma suma para las partículas finales.  % Falta texto

\subsection{Conservación de la energía y del momento lineal}

Las interacciones nucleares tienen lugar a distancia mucho más pequeñas que la separación típica entre los núcleos de un material ordinario, por eso se puede considerar a las partículas interaccionantes en una reacción nuclear como un sistema aislado y aplicar la ley de conservación de la energía total y del momento lineal total. De acuerdo con la notación expresada en (\ref{Ec:03-01-01}) escribimos la \textbf{conservación de la energía}

\begin{equation}
    T_a + m_a c^2 +T_A+m_A c^2 = T_b +m_bc^2 + T_b + m_Bc^2
\end{equation}
tal que $E_A = T_A + m_Ac^2$... El valor $Q$ del proceso o de la reacción se define como la difernecia entre la energía cinética inicial y final 

\begin{equation}
    Q \equiv T_B + Tb - T_A - T_a = (m_A + m_a - m_B - m_b) c^2
\end{equation}

Si, como es habitual, estamos analizando un experimento en el que el núcleo blanco se encuentra en reposo en el sistema laboratorio ($T_A=0$), entonces tenemos que $Q = T_B + T_b - T_a$. La energía cinética del núcleo $T_B$ es dfifícil de medir, y son las energías del proyectil y la de la partícula emergente ($T_a$ y $T_b$) las que suelen medirse. La De todos modos veremos que podemos encontrar una expresión para $Q$ (ecuación ())  que no incluye la energía cinética.


% Falta texto

\subsection{Energía umbral de reacción}

\subsection{Conservación del moemnto angular y de la paridad}

\subsection{Isospín}

\section{Dispersión y secciones eficaces}

Lo que usualmente se mide en las reacciones nucleares es el momento de las partículas ligeras emitidas (y por lo tanto su energía cinética, supuesto que se conozca la identidad de la partícula) y su distribución angular. % Falta texto

\subsection{Atenuación de un haz al atravesar un blanco}


\subsection{Dispersión de Coulomb}

\subsection{Dispersión nuclear}


\section{Mecanismos de reacciones}

\section{Fisión}

\section{Fusión}

La mayor dificultad para lograr la fusión nuclear a gran escala consiste en mantener el material fusible confinado a altas temperaturas durante el tiempo suficiente. Hasta el momento se está investigando en dos métodos: el confinamiento magnético y el confinamiento inercial. En el primero se hace circular plasma caliente de núcleos $^2$H y $^3$H en una región confinada por campos electromagnéticos. En el segundo se inyecta luz láser en una pequeña región que contiene el material fusible. En cualquier caso, el aprovechamiento comercial de la energía de fusión parece todavía una posibilidad lejana.

\section{Apéndices}

\chapter{Dinámica de redes} \label{Ch:04}

En este capítulo se comienza el estudio de las vibraciones de los átomos alrededor de sus posiciones de equilibrio y sus efectos observables. Esta llamada \textit{dinámica de redes} es necesaria para explicar propiedaes como: i) la conductividad térmica de los aislatnes, ii) la dependencia en $T^3$ del calor específico a baja temperatura, iii) las energías de cohesión, iv) la dilatación térmica, v) la conductividad eléctrica \textit{finita} de los metales, vi) la reflectividad de los cristales iónicos, etc.
\chapter{El transistor MOSFET}

\section{Introducción}

\lipsum[1]

\section{Capacitor MOS}

\lipsum[1-2]

\begin{equation*}
	\phi_F \equiv \phi_B \equiv E_i(\text{sustrato}) - E_F 
    = \frac{kT}{q} \log \parentesis{\frac{N_A}{n_i}} = -\frac{kT}{q} \log \parentesis{\frac{N_D}{n_i}} 
\end{equation*}
\begin{equation*}
	\phi_S \equiv E_i(\text{sustrato}) - E_i (\text{interfaz}) 
\end{equation*}

\subsection{Capactior MOS ideal}

\lipsum[1-2]




\begin{equation*}
	\Ecal_S (x) = \left\lbrace \begin{array}{ll}
		\frac{qN_A}{K_S \epsilon_0} (W-x) & \quad \text{si} 0\leq W \leq x \\
		0 & \quad \text{si} \ W<x 
	\end{array} \right.
\end{equation*}



\begin{equation*}
	V_G = \phi_S + \frac{K_S}{K_o} x_o \sqrt{\frac{2qN_A}{K_S\epsilon_0} \phi_s}
\end{equation*}

\subsection{No idealidades}

\lipsum[1-2]

\chapter{Partículas elementales: interacciones y propiedades}
\chapter{Electrones en un potencial periódico: teoría de bandas} \label{Ch:07}


Para mejorar algunas de las predicciones del modelo de gas de electrones libres se introduce la interacción de los electrones con la red cristalina a través de un potencial periódico, despreciando las interacciones entre electrones. En este tema vamos a usar los temas 11 y 15 del Oxford Solid State \cite{Oxford_Solid_State}, en vez del tema 7 de \cite{Fisica_del_Estado_Solido}, ya que consideramos que, como muchos temas, está todo muy mal redactado.

Al principio vamos a tratar de estudiar una cadena de electrones unidimensional con los electrones fuertemente atados a los átomos. Esto nos permitirá ver de una manera sencilla el funcionamiento de las bandas, como se llenan, etc. Luego veremos el caso de la aproximación a red vacía, cuando los átomos están débilmente atados a los núcleos, que entraña una mayor dificultad al aparecer la teoría de perturbaciones. Una vez hayamos visto los casos más sencillos veremos la generalización de ambos resultados, manifestado en el teorema de Bloch, para luego estudiar mas a fondo los electrones fuertemente ligados pero ahora en más dimensiones. Finalmente acabaremos con el estudio de los estados ocupados a través de la superficie de Fermi y las zonas de Brilluoin.

\begin{Anotacion}
	\textcolor{red}{Revisar el Quinn \cite{Quiin} tema 4, parece que da lo mismo y bastante bien. Revisar también últimos puntos, para hacerlo compatible con el Oxford. En general, este tema debe ser rehecho de cero, en el libro original no me gusta y como está ahora mismo tampoco.}
\end{Anotacion}

\section{Cadena de electrones}

En los capítulos anteriores hemos considerado las propiedades de los fonones a través de un sistema unidimensional y luego lo hemos generalizado para varias dimensiones. En este punto vamos a hacer un tratamiento similar de los electrones, considerando ondas de electrones en vez de fonones (cabe destacar que en la mecánica cuántica y en la teoría cuantica de campos los electrones son ondas, por lo que en realidad esta asunción no está tan alejada de la realidad).

\subsection{Cadena de electrones en una dimensión}

Para tratar de describir un sistema de electrones en un átomo/molécula usamos la teoría LCAO\footnote{De sus siglas en inglés \textit{Linear Combination of Atomic Orbitales}, en español \textit{Combinación Lineal de Orbitales Atómicos} o CLOA}, que nos dice que podremos suponer que los orbitales moleculares son un conjunto de combinaciones lineales de orbitales atómicos (denotados por $n,l,m_l,m_s$). En esta imagen asumimos que solo tenemos un orbital en el átomo $n$ y lo denotamos por $|n \rangle$. Por conveniencia y facilidad aplicamos condiciones de contorno de tal manera que si hay $N$ sitios, el sitio $N$ y $0$ es el mismo. Además consideremos que los orbitales son ortogonales entre sí:

\begin{equation}
	\langle n | m \rangle = \delta_{n,m}
\end{equation}
Supongamos ahora la función más general del sistema, dada por una combinación de las diferentes funciones orbitales:

\begin{equation}
	|\Psi \rangle = \sum_n \phi_n |n\rangle
\end{equation}
Si la función de ondas es solución de la ecuación de Schödinger tenemos que:

\begin{equation}
	\Hcal |\Psi \rangle = E |\Psi \rangle 
\end{equation}
tal que:

\begin{equation}
	\sum_m \Hcal \phi_m |m\rangle =  \sum_m E \phi_m |m\rangle
\end{equation}
Si ahora aplicamos $\langle n |$, como $\langle n| \Hcal | m \rangle= H_{mn}$ no es necesariamente cero, tenemos que el problema de los autovalores se ha convertido en un conjunto de ecuaciones lineales (conocidos los valores $H_{nm}$) tal que:


\begin{equation}
	\sum_m H_{mn} \phi_m  =  E \phi_n \label{Ec:07-01-04}
\end{equation}
Consecuentemente, la solución mas general no es más que una combinación lineal de las soluciones orbitales que calculemos con el modelo. En realidad esta solución es una aproximación, llamada \textit{método variacional}, y será mejor cuantos más orbitales pongamos en el modelo. Si en vez de tener un solo orbital $|n\rangle$ por sitio, tuviéramos varios $|n,\alpha\rangle$ (recordemos que en este caso $n$ denota la posición del electrón, unido al átomo $n$-ésimo) siendo $\alpha$ un conjunto de números, nuestra solución sería cada vez más precisa.

Este método es el llamado LCAO pero aplicado para nuestra cadena de electrones. Sin embargo incluir más orbitales puede provocar que la condición de ortogonalidad ya no se verifique $\langle n ,\alpha | m,\beta \rangle \neq \delta_{nm} \delta_{\alpha \beta}$. En general asumiremos que existe un solo orbital por sitio, en pos de que se verifique la condición de ortogonalidad. Nuestro hamiltoniano vendrá dado por:

\begin{equation}
	\Hcal = \Hcal_0 + \sum_{j} V_j
\end{equation}
siendo $\Hcal_0 = \pn^2 /2m$ la energía cinética y $V_j$ la interacción de Coulomb en del electrón en la posición $\rn$ en el núcleo ubicado en $j$:

\begin{equation}
	V_j = V(\rn-\Rn_j)
\end{equation}
Con estas definiciones tenemos que la solución:

\begin{equation}
	\Hcal |m\rangle = (\Hcal_0 + V_m) |m\rangle + \sum_{j\neq m} V_j |m\rangle
\end{equation}
Lógicamente $\Hcal_0 +V_m |m\rangle = \varepsilon_{at}|m\rangle$ (siendo $\varepsilon_{at}$ la energía atómica). Así tenemos que:

\begin{equation}
	H_{nm} = \langle n | \Hcal | m \rangle = \varepsilon_{at} \delta_{nm}+\sum_{j\neq m} \langle n |V_j |m\rangle
\end{equation}
Si suponemos que la interacción $\langle n | V_j | m \rangle$ solo ocurre entre los primeros vecinos, tal que:

\begin{equation}
	\sum_{j\neq m} \langle n |V_j |m\rangle = \left\lbrace \begin{array}{lc}
		V_0 & \ n=m \\
		-t & \ n=m\pm 1 \\
		 0 & \text{cualquier otro n}
	\end{array} \right.
\end{equation}
La consecuencia $\langle n|V_j|n\pm1\rangle \neq 0$ es enorme. Como ahora la matriz hamiltoniana no está diagonalizada, los autoestados reales no serán $|n\rangle$ o $|m\rangle$, si no una combinación lineal de ellos, y por tanto es posible que aunque en el estado $t=0$ el electrón $n$ se encuentre en el estado $|n\rangle$, el electrón ya no se encuentre en el estado$|n\rangle$ en el instante $t\neq 0$. La matemática detrás de esto no es complicada, aunque si tediosa. Lo importante es que este potencial está permitiendo saltos de electrones entre los diferentes átomos, y cuanto más grande sea $t$ mas posibles serán. El hamiltoniano no diagonalizado es ahora:

\begin{equation}
	H_{nm} = \varepsilon_0 \delta_{nm} - t(\delta_{n,n+1}+\delta_{n,n-1})
\end{equation}

\subsection{Solución a la cadena de electrones unidimensional}

La solución a a la cadena de electrones monoatómica es bien sencilla: es una combinación de ondas planas. Supongamos entones que la solución es

\begin{equation}
	\phi_n = \frac{e^{ikna}}{\sqrt{N}}
\end{equation}
donde el $1/\sqrt{N}$ es el factor de normalización. Sustituyendo esto en la ecuación de Schrödinger independiente del tiempo (si quisiéramos resolver la ec. dependiente del tiempo solo tendríamos que añadir un término $e^{i\omega t}$). Podemos ver en nuestra solución que $k\rightarrow k + 2 \pi /a$ también es una solución. Además si aplicamos las condiciones de contorno obtenemos que nuestros momentos se cuantizan en unidades de $2\pi /L$. 

Sustituyendo la solución de ondas planas en nuestra ecuación (\ref{Ec:07-01-04}) tenemos que:

\begin{equation}
	\sum_m H_{nm} \phi_m = \varepsilon_0 \frac{e^{-ikna}}{\sqrt{N}} - t \parentesis{\frac{e^{-ik(n+1)a}}{\sqrt{N}}+\frac{e^{-ik(n-1)a}}{\sqrt{N}}} = E \frac{e^{-ikna}}{\sqrt{N}} = E\phi_n
\end{equation}
de lo que obtenemos el espectro de energías:

\begin{equation}
	E=\varepsilon_0 - 2 t \cos (ka)
\end{equation}
que se parece bastante al espectro de nuestro fonón unidimensional. La curva de dispersión, periódica en $k\rightarrow k+2\pi/a$, tiene una región donde la velocidad de grupo es cero en $k=n\pi/a$ para cualquier $n$ entero. A diferencia de los electrones libres, la ecuación de dispersión de los electrones nos da un valor máximo y mínimo de la energía, ahora está acotada. Consecuentemente los electrones solo tienen autovalores en una determinada \textbf{banda} de energías $E\in (\varepsilon_0-2t,\varepsilon_0+2t)$, tal y como podemos ver en \ref{Fig:07-01-00}. El término banda se usa entonces para describir el rango posible de valores energéticos en el cual se puede encontrar el electrón y para describir una rama conectada para una función de dispersión (en esta imagen solo hay un modo posible para cada $k$, pero en general tendremos más, por ejemplo al considerar mas orbitales, más vecinos o varias dimensiones). Además podemos ver que si las distancias interatómicas son suficientemente grandes el comportamiento de los electrones será exactamente igual al de los orbitales aislados: energías bien defindias en orbitales \ref{Fig:07-01-01}.


\begin{figure}[h!]\centering
	\begin{subfigure}{0.46\linewidth} \centering
		\includegraphics[scale=0.35]{Cuerpo/Ch_07/Oxford-05.png}
		\caption{Dependencia de la energía respecto $k$ \cite{Oxford_Solid_State}.}
		\label{Fig:07-01-00}
	\end{subfigure}
	\begin{subfigure}{0.46\linewidth} \centering
		\includegraphics[scale=0.3]{Cuerpo/Ch_07/Oxford-06.png}
		\caption{Anchura de la bandas de energía en función de la distancia interatómica $a$ \cite{Oxford_Solid_State}.}
		\label{Fig:07-01-01}
	\end{subfigure}
	\caption{Comportamiento de la energía y anchura de bandas en función de $k$ y $a$.}
\end{figure}

La diferencia energética entre el máximo y mínimo de la banda se le llama \textit{ancho de banda}. Estados con energías mas allá del ancho de banda (o con menos energías) no son posibles. El ancho de banda lo determina el término $t$ o ``término de hopping'', que depende, de varios términos, como la distancia entre átomos. El término $t$ nos da un valor de la energía de intercambio, es decir, la energía que cuesta transferir el electrón del átomo $m$ al $m\pm 1$. Para pequeños $k$ tenemos que la ecuación de dispersión nos queda como

\begin{equation}
	E(k) = \cte + ta^2 k^2
\end{equation}
El resultado del comportamiento parabólico es muy similar al de los electrones libres, por lo que si relacionamos ambos términos de dispersión salvo que ahora $m\rightarrow m^* $:

\begin{equation}
	\frac{\hbar^2 k^2}{2m^* } = t a^2 k^2
\end{equation}
tal que

\begin{equation}
	m^* = \parentesis{\hbar^2}{2ta^2}
\end{equation}
es la \textbf{masa efectiva}, que se define como aquella masa la cual debería tener el electrón para que a bajos $k$ la dispersión se comporte como un conjunto de masa $m^*$. Evidentemente la masa efectiva no tiene ninguna relación con la masa del electrón, pero si con el término de intercambio. 


\subsection{Introducción al llenado de bandas}

Ahora, al igual que en el caso de las vibraciones, debemos imponer condiciones de contorno a fin de representar materiales finitos. Dado que la solución de la función de ondas es la misma (onda plana) podemos pensar que exigir las condiciones de contorno periódicas nos lleva a resultados completamente análogos. Uno de estos resultados es que ahora el número de $k$-estados posibles está limitado al número de electrones libres $N$. Sin embargo hay 2 estados de espín, por lo que hay $2N$ estados posibles. 

Además hemos visto como la energía se encuentra en una banda. Entones, por ejemplo, si cada uno de los átomos donase un electrón a la banda tendríamos que la banda estaría medio llena, y por lo tanto aplicar una fuerza externa, como podría ser la fuerza coulombiana a través de un campo eléctrico pequeño, los estados ocupados podrían desplazarse hacia la derecha \ref{Fig:07-01-03} de tal modo que aparece un momento total no cero, y por tanto aparece una corriente. Esto explica porqué muchos cristales de átomos monovalentes son metales.

Por otro lado, si cada átomo de nuestro momento fuera divalente, donando 2 electrones a la banda (lo cual no es imcompatible con el modelo ya que podrían tener los mismos números cuánticos solo diferenciándose en el de espín), tendremos que la banda estaría llena: no existe campo eléctrico que pueda mover $k$ hacia un lado u otro, ya que todos los estados posibles de $k$ están ocupados. Es decir, \textit{las bandas ocupadas no pueden llevar corriente}, comportándose como aislantes. Lógicamente la realidad es mucho mas complicada, como vamos a ver.


\begin{figure}[h!] \centering
	\includegraphics[width=0.5\textwidth]{Cuerpo/Ch_07/Oxford-07.png}
	\caption{Dependencia de la energía respecto $k$ \cite{Oxford_Solid_State}.}
	\label{Fig:07-01-03}
\end{figure}



\subsection{Múltiples bandas}

En el modelo introducido, hemos considerado que solo hay un átomo en la celda unidad y un sólo orbital por átomo. Lógicamente para mejorar el modelo deberíamos considerar que hay mas de un orbital en cada celda unitaria. Veamos los dos posibles casos:

\begin{itemize}
	\item Una posibilidad es considerar que en la celda unitaria hay un átomo pero con varios orbitales atómicos. Cuando los átomos están muy lejos entre sí, cada átomo solo tiene los orbitales asignados, tal que los electrones se colocan en dichos orbitales de manera exacta. Sin embargo, cuando los átomos se acercan y las energías posibles para cada electrón comienzan a aumentar (apareciendo bandas) es posible que las bandas de los diferentes orbitales se superpongan, tal y como muestra la imagen \ref{Fig:07-01-05}.
	\item Una situación similar es cuando consdieramos dos átomos por celda unidad pero solo un orbital atómico por átomo. El resultado es un espectro como el de la imagen \ref{Fig:07-01-04}. Al igual que en el caso de las vibraciones, ahora tenemos dos posibles valores de la energía para cada momento. En el lenguaje de los electrones decimos que hay dos bandas (no hacemos la distinción entre banda acústica u óptica, aunque la idea es similar). Notemos que el gap entre todas bandas no hay estados $k$ que puedan producir dichas energías.
	
	Supongamos el caso en el que ambos átomos son divalentes (tienen dos electrones en la última capa de valencia). Ambas bandas estarían completamente llenas, tanto los estados con espín arriba $\uparrow$ y abajo $\downarrow$. Por otro lado, si fueran monovalentes solo estaría llena la banda con menos energía (con ambos espines), quedando los estados con la banda de más energía completamente vacía. En el esquema en zona extendida la superficie de Fermi y la primera zona de Brilluoin coincidirían. ¿Y si ahora introducimos un campo eléctrico externo no nulo? Pues ahora los electrones tendrían que saltar de una banda a otra, para que así aparezca un momento cristalino no nulo. Sin embargo el campo eléctrico debe ser lo suficientemente fuerte para que el electrón salte sobre las bandas (una temperatura alta también puede ayudar, ya que existen fluctuaciones energéticas térmicas). Sin embargo esto se ha hecho para el caso en el que ambos orbitales están separados, pero para distancias interatómicas suficientemente pequeñas podría ser que no existiera tal separación y se comportase como un metal en cualquiera de los casos. Esto se tratará de manera mas profunda en la sección final.
\end{itemize}


\begin{figure}[h!]\centering
	\begin{subfigure}{0.46\linewidth} \centering
		\includegraphics[scale=0.45]{Cuerpo/Ch_07/Oxford-08.png}
		\caption{Dependencia de la energía respecto $k$ \cite{Oxford_Solid_State}.}
		\label{Fig:07-01-04}
	\end{subfigure}
	\begin{subfigure}{0.46\linewidth} \centering
		\includegraphics[scale=0.25]{Cuerpo/Ch_07/Oxford-09.png}
		\caption{Anchura de la bandas de energía en función de la distancia interatómica $a$ \cite{Oxford_Solid_State}.}
		\label{Fig:07-01-05}
	\end{subfigure}
	\caption{Comportamiento de la energía y anchura de bandas en función de $k$ y $a$.}
\end{figure}


\section{Aproximación de red vacía}

En la sección anterior hemos descrito algunas de las propiedades de los electrones en un potencial periódico para el modelo unidimensional. Ahora haremos la expansión a 3 dimensiones y a un potencial muy débil. De hecho la sección anterior es el caso opuesto al que vamos a estudiar aquí, ya que suponía electrones muy fuertemente ligados a los átomos. Supongamos que nuestro hamiltoniano viene dado por:

\begin{equation}
	\Hcal = \Hcal_0 + V(\rn)
\end{equation}
tal que $V(\rn)$ es periódico, es decir que verifica 

\begin{equation}
	V(\rn+\Rn)=V(\rn)
\end{equation}
donde $\Rn$ es un \textbf{vector de red}. Como sabemos los autoestados de la ecuación de Schrödinger libre (sin potencial) son ondas planas denotadas por $|\kn\rangle$ tal que

\begin{equation}
	|\kn\rangle \equiv A e^{i\kn \rn}
\end{equation}
donde la energía viene dada por

\begin{equation}
	\varepsilon_0 = \frac{\hbar^2 |\kn|^2}{2m}
\end{equation}
El valor medio $\langle \kn'|V|\kn \rangle$ debe venir dado por:


\begin{equation}
	\langle \kn'|V|\kn \rangle  = \frac{1}{L^3} \int \D \rn e^{i(\kn-\kn')\cdot \rn} V(\rn) \equiv V_{\kn'-\kn}
\end{equation}
que es cero a no ser que $\kn'-\kn$ sea un \textit{vector de la red recíproca}. Consecuentemente cualquier solo ondas planas separadas por un vector de red recíproca $\Gn$ pueden interactuar. Aplicando teoría de perturbaciones tenemos que:

\begin{equation}
	\varepsilon(\kn)  =\varepsilon_0 (\kn) + \langle \kn |V|\kn \rangle = \varepsilon_0 (\kn) + V_0
\end{equation}
nos mueve un término $V_0$ todos los posibles autovalores (siendo este su único efecto). Lógicamente este término no va a afectar nada a la física de nuestro problema, por lo que asumir que es cero solo simplifica las ecuaciones. La teoría de perturbaciones de segundo orden nos dice que


\begin{equation}
	\varepsilon(\kn)  =\varepsilon_0 (\kn) + \sum_{\kn'=\kn+\Gn} \frac{\langle \kn' |V|\kn \rangle}{  \varepsilon_0 (\kn) - \varepsilon_0 (\kn')}
\end{equation}
donde el $'$ nos dice que $\Gn \neq 0$. Entonces existen 3 posibilidades:

\begin{itemize}
	\item Que $\varepsilon(\kn')\neq \varepsilon(\kn)$. En ese caso tenemos que $\varepsilon(\kn) \approx \varepsilon_0(\kn)$, ya que estamos suponiendo que el potencial periódico es muy débil y por  tanto $\langle \kn' |V|\kn \rangle \ll \varepsilon(\kn) - \varepsilon(\kn')$. 
	\item Que se verifique que $\varepsilon(\kn')=\varepsilon(\kn)$, es decir que el estado con dicha energía está \textit{degenerado}. Debido a que $\varepsilon(\kn')=\varepsilon(\kn)$ y que $\kn'=\kn+\Gn$ con $\Gn\neq 0$, tenemos que la única posibilidad es que 
	\begin{equation}
		k'=-k = \frac{n\pi}{a}
	\end{equation}
	donde $a$ es la distancia característica de la red.	Es decir, los únicos estados degenerados de energía son aquellos que están en las zonas de frontera de Brillouin.
	\item Que se verifique que $\varepsilon(\kn')\approx\varepsilon(\kn)$, es decir que el estado con dicha energía está casi degenerado. Debido a que $\varepsilon(\kn')\approx\varepsilon(\kn)$, cosa que solo pasa si la diferencia entre ambos es $\Gn$ y el vector de onda está muy próximo del plano de Bragg, tal que $k\approx n\pi/a$, tendremos que $\langle \kn' |V|\kn \rangle \sim \varepsilon(\kn) - \varepsilon(\kn')$ y por tanto no es despreciable.
	
\end{itemize}	

Como podemos ver esto nos lleva a que los vectores de onda que no estén en una de la zonas de frontera de Brilluoin (es decir, que caigan en un plano de Braggs) tienen una energía $\varepsilon(\kn)$, y por tanto cuanto mas lejos del plano de Bragg más válido será el comportamiento de electrones libres.

\subsection{Teoría de perturbaciones para estados degenerados}

Para estados degenerados la teoría de perturbaciones nos dice que los autoestados son una combinación lineal de los estados originales, y los autovalores una combinación de ambos también. Entones el problema es básicamente diagonalizar $\Hcal$, para lo cual tenemos que conocer los términos de la matriz, dados por:

\begin{eqnarray*}
	\langle \kn | \Hcal |\kn \rangle &= & \varepsilon_0(\kn) \\	
	\langle \kn' | \Hcal |\kn \rangle' &= & \varepsilon_0(\kn')=\varepsilon(\kn+\Gn) \\	
	\langle \kn' | \Hcal |\kn \rangle &= & V_{\kn-\kn'}=V_{\Gn}^* \\	
	\langle \kn | \Hcal' |\kn \rangle &= & V_{\kn'-\kn}=V_{\Gn}
\end{eqnarray*}
de tal modo que 
\begin{equation}
	\Hcal \equiv \begin{pmatrix}
		\varepsilon(\kn) & V_{\Gn}^* \\ 
		V_{\Gn} & \varepsilon(\kn+\Gn)
	\end{pmatrix}
\end{equation}
Usando el principio variacional, el autoestado será una combinación de $|\kn\rangle$ y $|\kn'\rangle$ que verifique:

\begin{equation}
|\Psi \rangle = \alpha |\kn \rangle + \beta |\kn'\rangle = \alpha |\kn \rangle + \beta |\kn+\Gn\rangle 
\end{equation}
tal que:

\begin{equation}\begin{pmatrix}
	\varepsilon(\kn) & V_{\Gn}^* \\ 
	V_{\Gn} & \varepsilon(\kn+\Gn)
\end{pmatrix} \begin{pmatrix}
	\alpha \\ \beta
\end{pmatrix} = \begin{pmatrix}
	\alpha \\ \beta
\end{pmatrix} 
\end{equation}
Sea $\varepsilon$ el autovalor de $\Hcal$, tenemos que la ecuación para calcularlo es:

\begin{equation}
	\parentesis{\varepsilon_0(\kn)-\varepsilon}\parentesis{\varepsilon(\kn+\Gn)-\varepsilon} - |V_{\Gn}|^2 = 0
\end{equation}
Entonces en función de cada caso tendremos diferentes comportamientos. Veamos cada caso por separado. C

\subsubsection{Vector de ondas en el plano de Bragg}

El caso más simple es en el que precisamente $\kn$ es un vector que cae en el plano de Bragg. En este caso la ecuación se convierte:

\begin{equation}
	\parentesis{\varepsilon_0 (\kn)-E}^2 =|V_\Gn|^2
\end{equation}
Con las dos soluciones:

\begin{equation}
	E_\pm = \varepsilon_0 (\kn) \pm |V_\Gn|
\end{equation}
Es decir, existe un gap en los bordes de la zona de Brillouin (planos de Bragg). Siempre que existan dos momentos $\kn$ y $\kn'$ con una $\varepsilon_0(\kn)$ en presencia de un potencial $V_\Gn$ periódico, los autoestados de energía estarán separados por $\pm |V_\Gn|$.

Es decir, habrá regiones de energía inaccesibles para cualquier estado $|\kn\rangle$, formando, al igual que en la sección anterior, bandas de energía permitidas y bandas de energías prohibidas. Las bandas de energías prohibidas estarán asociadas a los valores $\varepsilon(\kn)$ si $\kn \in $\{PZB,SZB,TZB...\}, tal que la banda es $E\in (\varepsilon(\kn)-V_\Gn,\varepsilon(\kn)+V_\Gn)$. 

\subsubsection{Caso unidimensional}

Para entender el mejor estos conceptos vamos a estudiar el caso unidimensional con un potencial concreto: $V(x)=V_0 \cos (2\pi x/a)$ con $V_0>0$. La frontera de la primera zona de Brillouin (en 1D son puntos, al igual que en 2D son lineas y en 3D planos) son $k=\pi/a$ y $k'=-k=-\pi/a$; de tal modo que se verifica que $k'-k=G$ y que $\varepsilon_0 (k')= \varepsilon_0(k)$. Las soluciones:  

\begin{equation}
	|\psi_\pm \rangle = \frac{1}{\sqrt{2}} \parentesis{|k\rangle \pm |k'\rangle}
\end{equation}
asignando a cada uno de estos los autovalores de energías $E_\pm$ respectivamente. Dado que podemos escribir estas autofunciones como exponenciales complejas, tenemos que:

\begin{equation*}
	\psi_+ \sim e^{ikx/a}+e^{-ikx/a} \propto \cos (x\pi/a)
\end{equation*} 
\begin{equation*}
	\psi_- \sim e^{ikx/a}-e^{-ikx/a} \propto \sin (x\pi/a)
\end{equation*} 
Si nos fijamos en las densidades de probabilidad $|\psi_\pm|^2$ tenemos que la densidad de $\psi_+$ es máxima cuando el potencial $V$ es máximo mientras que $\psi_-$ es máximo cuando el potencial es mínimo, tal y como podemos ver en la figura \ref{Fig:07-01}. Consecuentemente el principio general es que el potencial periódico hace que interfieran las ondas $|\kn\rangle $ y $|\kn+\Gn\rangle$, y cuando estas energías son las mismas, la mezcla de entre ellas es fuerte y ambas se combinan para generar dos estados, una de máxima energía y otra de mínima energía. 


\begin{figure}[h!] \centering
	\includegraphics[scale=0.35]{Cuerpo/Ch_07/Oxford-01.png}
	\caption{Forma de las funciones de onda cuando $k=n\pi k/a$.}
	\label{Fig:07-01}
\end{figure}    

\subsubsection{Vector de ondas cerca del plano de Bragg}

No es muy complicado de extender este razonamiento a puntos muy cerca del $\kn$ que esté en la zona de frontera cerca del $\kn$ degenerado. Por simplicidad, seguiremos considerando el caso unidimensional. Supongamos entonces las energías en los estados $k=n\pi/a+\delta$ y $k=-n\pi/a+\delta$. En este caso estamos estudiando vectores de onda en la proximidad del plano de Bragg, tal que

\begin{eqnarray}
	\varepsilon_0 (n\pi/a + \delta) & = & \frac{\hbar^2}{2m} \ccorchetes{\parentesis{n\pi/a}^2 + 2 n \pi \delta / a + \delta^2} \\
	\varepsilon_0 (-n\pi/a + \delta) & = & \frac{\hbar^2}{2m} \ccorchetes{\parentesis{n\pi/a}^2 - 2 n \pi \delta / a + \delta^2}
\end{eqnarray}	
Como podemos ver ahora la ecuación característica (aquella que nos calcula los autovalores de la energía) vendrá dada por:

\begin{equation}
	\parentesis{\frac{\hbar^2}{2m} \ccorchetes{(n\pi/a)}^2 - E + \frac{\hbar^2}{2m} 2 n \pi \delta / a } \parentesis{\frac{\hbar^2}{2m} \ccorchetes{(n\pi/a)}^2 - E - \frac{\hbar^2}{2m} 2 n \pi \delta / a } - |V_{\Gn}|^2 = 0
\end{equation}
que despejando:

\begin{equation}
	\parentesis{\frac{\hbar^2}{2m} \ccorchetes{(n\pi/a)}^2 - E }^2 = \parentesis{ \frac{\hbar^2}{2m} 2 n \pi \delta / a }^2+|V_{\Gn}|^2 
\end{equation}
de tal modo que la solución viene dada por: 

\begin{equation}
	E_\pm = \frac{\hbar^2}{2m} \ccorchetes{(n\pi/a)^2+\delta^2} \pm \sqrt{\parentesis{ \frac{\hbar^2}{2m} 2 n \pi \delta / a }^2+|V_{\Gn}|^2 }
\end{equation}
y si $\delta \ll 1$

\begin{equation}
	E_\pm = \frac{\hbar^2(n\pi/a)^2}{2m} \pm |V_G| + \frac{\hbar^2 \delta^2}{2m}\ccorchetes{1\pm \frac{\hbar^2 (n\pi/a)^2}{m} \frac{1}{|V_G|}}
\end{equation}
Como podemos ver cerca de la zona de Brilluoin el valor de $E_\pm$ depende de $\delta^2$, lo  que significa que cerca de las bandas prohibidas tenemos espectros parabólidos, como vemos en la imagen \ref{Fig:07-02}. Esto se parece mucho a la estructura de la primera sección, de tal forma que aparecen bandas energéticas donda existen autoestados válidos, llamadas \textbf{bandas permitidas}, y las bandas donde hay gaps de energía y no hay estados posibles para dichas energías, formado las \textbf{bandas prohibidas}.

\begin{figure}[h!] \centering
	\includegraphics[scale=0.4]{Cuerpo/Ch_07/Oxford-02.png}
	\caption{Dispersión del momento en la región de casi electrones libres, pudiéndose observar que los momentos cerca de la zona de Brilluoin se comportan como parábolas.}
	\label{Fig:07-02}
\end{figure}    
	
En la siguiente imagen podemos ver cual es la forma de la dispersión de energía en dos esquemas que tenemos que tener muy claro que aportan la misma información, solo que uno lo hace de una manera mucho más compacta. Aunque pueda parecer que la zona reducida (\ref{Fig:07-03b}) nos dice que para cada $k$ tenemos varios valores posibles de energía, eso no es verdad. Cuando pasamos de una línea inferior a una línea superior estamos sumando al $\kn$ un $\Gn$, en el caso 1D estamos sumando a $k$ un valor $\pi/a$, obteniendo así las diferentes líneas. 


Usando el concepto de la \textit{masa efectiva}, que es como recordamos la masa \textit{que tendría que tener nuestra carga} para que se comportará como un electrón libre, tenemos que los autovalores de la energía las podemos poner (aproximadamente):

\begin{eqnarray}
	E_+ (G+\delta) & = & C_+ + \frac{\hbar^2 \delta^2}{2m_+^*} \\
	E_- (G+\delta) & = & C_- - \frac{\hbar^2 \delta^2}{2m_+^*} 
\end{eqnarray}
donde la masa efectiva debe venir dada por

\begin{equation}
	m_{\pm}^* = \frac{m}{\left|1 \pm \frac{\hbar^2 (n\pi/a)^2}{m} \frac{1}{|V_G|} \right| }
\end{equation}

\begin{figure}[h!] \centering
	\begin{subfigure}{0.45\linewidth} \centering 
		\includegraphics[scale=0.29]{Cuerpo/Ch_07/Oxford-03.png}
		\caption{Dispersión en la aproximación de red vacía en el esquema extendido.}
		\label{Fig:07-03a}
	\end{subfigure} 
	\begin{subfigure}{0.45\linewidth} \centering
		\includegraphics[scale=0.4]{Cuerpo/Ch_07/04.png}
		\caption{Dispersión en la aproximación de red vacía en el esquema reducido.}
		\label{Fig:07-03b}
	\end{subfigure} 
	\caption{Dispersión de la energía en función del momento en el esquema ampliado y reducido.}
\end{figure}   


\subsection{Aproximación a red vacía en 2 y 3 dimensiones}

Los principios de la la aproximación a red vacía son muy similares en 2 y 3 dimensiones a los mencionados anteriormente: cerca de la zona de brillouin tenemos que aparece un gap debido a la interacción con un vector de onda separado por un vector  de la red recíproca. Esto generará dos estados, uno con más energía y otro con menos energía. Como podemos ver la generalización no es muy complicada, solo que ahora existirá que hay varios posibles $\kn'$ generados por la traslación de un vector de la red recíproca, por lo que la degneración es múltiple, y la matriz hamiltoniana a diagonalizar tendrá una extensión mayor. Este caso de varios estados con la misma energía lo podemos ver claramente en el caso 2 dimensional de la figura \ref{Fig:07-04}.

\begin{figure}[h!] \centering
	\includegraphics[scale=0.35]{Cuerpo/Ch_07/Fotos libro 2.pdf}
	\caption{(a) Ejemplo de estados doblemente degenerados sobre planos Bragg de una red cuadrada. (b) Estados cuádruplemente degenerados sobre las esquinas de la PZB de una red cuadrada.}
	\label{Fig:07-04}
\end{figure}    

\section{Teorema de Bloch}

En el modelo de casi electrones libres o aproximación a red vacía partimos de que las ondas planas son perturbadas débilmente por un potencial periódico. Pero en los materiales reales, la interacción entre átomos/electrones puede ser tan fuerte que la teoría de perturbaciones no sea válida. ¿Cómo hacemos entonces para describir el comportamiento de los electrones sin ondas planas? El hecho es que, en realidad, el momento $\kn$ usando en las ondas planas no es una cantidad conservada, mientras que el momento del cristal si lo es. No importa como de fuerte sea el potencial periódico, el momento cristalino es conservado. Este hecho descubierto por Felix Bloch se convirtió en el \textbf{teorema de Bloch}:

\begin{mybox}
	El \textbf{teorema de Bloch} nos dice que un electrón en un potencial periódico se puede describir en términos de las autofunciones de la fomra
	
	\begin{equation}
		\Psi_\kn^\alpha (\rn) = e^{i\kn \cdot \rn} u_\kn^\alpha (\rn)
	\end{equation}
	donde $u_\kn^\alpha$ es un potencial periódico en la celda unitaria y $\kn$ es el momento cristalino que podemos escogerlo de tal forma que caiga en la primera zona de Brilluoin.
\end{mybox}
En el esquema de zona reducida existirán varios estados posibles con el mismo $\kn$ cada uno denotado por $\alpha$. La función periódica $u$ es usualmente llamada la \textbf{función de Bloch} y $\Psi$ la \textbf{función de onda plana modificada}. Como $u$ es periódico, tenemos que se puede expresar como la suma de ondas planas sobre la red recíproca:

\begin{equation}
	u_\kn^\alpha (\rn) = \sum_{\Gn} \tilde{u}^{\alpha}_{\Gn,\kn} e^{i\Gn \cdot \rn}
\end{equation}
Esto garantiza que $u_\kn^\alpha (\rn) = u_\kn^{\alpha} (\rn + \Rn)$ para cualquier \textit{vector de red}. Esta es otra manera de expresar el teorema de Bloch, de tal manera que la suma de ondas planas $\kn$ pueden diferir por un conjunto de vectores recíprocos $\Gn$.

Una de las consecuencias del teorema de Bloch es que aquellas ondas que no estén separadas por un vector de ondas de la red recíproca no son capaces de interaccionar, esto es, que $\langle \kn'|V|\kn \rangle=0$ si $\kn' \neq \kn + \Gn$, de lo cual se deduce que en realidad la ecuación de Schrödinger 

\begin{equation}
	\ccorchetes{\frac{\pn^2}{2m} + V(\rn)} \Psi (\rn) = E \Psi (\rn)
\end{equation}
en el espacio de momentos 


\begin{equation}
	\sum_{\Gn} V_{\Gn} \Psi_{\kn - \Gn} =\ccorchetes{E-\frac{\pn^2}{2m}}\Psi_{\kn }
\end{equation}
donde $V_{\kn-\kn'}$ no es cero si $\kn-\kn'=\Gn$. Entonces es claro que tenemos una ecuación de Schrödinger para un conjunto de $\Psi_{\kn-\Gn}$ que verifique que $\Psi_\kn^\alpha (\rn) = e^{i\kn \cdot \rn} u_\kn^\alpha (\rn)$.

Aunque debe no debería ser sorprendendente que los electrones en un potencial periódico se comporten como una combinación lineal de ondas planas en función del momento cristalino, no debemos subestimar la importancia del teorema de Bloch. Este teorema nos dice que incluso cuando el potencial del electrón siente cada átomo de manera muy fuerte, los electrones siguen comportándose como si no sintieran a los átomos, siendo casi electrones libres. 



\section{Electrones fuertemente ligados}

En esta sección vamos a aproximar el cálculo de las funciones de onda y energías electrónicas en el caso de que que los átomos próximos perturben levemente los estados atómicos, enfoque particularmente útil para describir electrones de bandas internas $d$ de metales de transición y aislantes. Recomendamos pág. 112-114 John J. Quinn \cite{Quiin}.

Consideramos para simplificar una cristal monoatómico, y denotamos por $\epsilon_n$, $\phi_n$ las energías y autoestados atómicos: $\Hcal_{\text{at}} \phi_n = \epsilon_n \phi_n$. La aproximación lineal es probar como solución $\Psi_\kn (\rn) = \sum_\Rn C_\kn \Phi  (\rn - \Rn)$, donde la función $\Phi$ será una combinación lineal de orbitales atómicos degenerados $\Phi (\rn) = \sum_n b_n \phi_n (\rn)$. Debemos exigir que $\Psi_\kn$ sea de la forma de Bloch. Esto se cumple si $C_\kn (\Rn) = e^{i \kn \cdot \Rn}$, como es fácil de comprobar, con lo que 

\begin{equation}
	\Psi_\kn (\rn) = \sum_\Rn e^{i \kn \cdot \Rn} \Psi (\rn - \Rn) \label{Ec:07-05-01}
\end{equation} 
Por otro lado el hamiltoniano completo será 

\begin{equation}
	\Hcal = \Hcal_{\text{at}} + \Delta U(\rn)
\end{equation} 
donde $\Delta U (\rn)$ se supone sólo apreciable lejos de los iones (figura \ref{Fig:07-05}).

\begin{figure}[h!] \centering
	\includegraphics[scale=0.3]{Cuerpo/Ch_07/Fotos libro 5.pdf}
	\caption{Potencial periódico expresado como suma de un potencial atómico más una perturbación.}
	\label{Fig:07-05}
\end{figure}    

Utilizando la ecuación se Schrödinger podemos hallar que

\begin{equation}
	(\Hcal_\text{at} + \Delta U) |\Psi_\kn \rangle = \varepsilon(\kn) |\Psi_\kn \rangle
\end{equation}
que multiplicando por $\langle \psi_s |$   

\begin{equation}
	\varepsilon_s \langle \phi_s |\Psi_\kn \rangle + \langle \phi_s | \Delta U |\Psi_\kn \rangle = \varepsilon(\kn) \langle \phi_s |\Psi_\kn \rangle
\end{equation}
y despejando 

\begin{equation}
	\varepsilon(\kn) = \varepsilon_s + \frac{\langle \phi_s |\Delta U | \Psi_\kn \rangle}{\langle \phi_s |\Psi_\kn \rangle}
\end{equation}
utilizando ahora la ecuación (\ref{Ec:07-05-01}) e introduciendo la simplificación de considerar en $\phi$ sólo orbitales de tipo $s$:

\begin{equation}
	\begin{split}
		\varepsilon (\kn) \ = \ & \ \varepsilon_s  + \frac{\sum_\Rn e^{i \kn \cdot \Rn }\int \phi_s (\rn) \Delta U (\rn) \phi_s (\rn - \Rn) \D^3 \rn }{\sum_\Rn e^{i \kn \cdot \Rn }\int \phi_s (\rn)\phi_s (\rn - \Rn) \D^3 \rn }\\
		& \varepsilon_s -  \frac{\beta + \sum_{\Rn\neq 0} \gamma(\Rn) e^{i\kn \cdot \Rn}}{1+\sum_{\Rn\neq 0} \alpha (\Rn) e^{i\kn \cdot \Rn}}
	\end{split}
\end{equation}
donde hemos usado la siguiente notación: 

\begin{equation}
	\begin{split}
		\beta \ = \ &  \ - \int   \D^3 \rn \Delta U (\rn) |\phi_s (\rn) |^2 \\
		\alpha (\Rn) \ = \ & \ \int \D^3\rn \phi_s^* (\rn) \phi_s (\rn-\Rn) \\
		\gamma (\Rn) \ = \ & - \int \D^3 \rn \phi_s^* (\rn) \Delta U (\rn) \phi_s (\rn - \Rn)
	\end{split}
\end{equation}
Teniendo en cuenta que para un nivel s $\Psi (\rn) = \Psi (r)$, y que para un cristal monoatómico existe simetría de inversión ($\Rn \rightarrow - \Rn$) se tiene $\gamma (\Rn) = \gamma (R)$, $\alpha (\Rn) = \alpha (R)$ y $e^{i \kn \cdot \Rn} = \cos (\kn \cdot \Rn)$. Finalmente, despreciando $\gamma$ y $\alpha$ salvo para los vecinos más próximos (vmp), resulta:

\begin{equation}
	\varepsilon(\kn) = \varepsilon_s - \frac{\beta + \gamma_{\text{vmp}} \sum_{\text{vmp}} \cos (\kn \cdot \Rn)}{1+\alpha_{\text{vmp}} \sum_{\text{vmp}} \cos (\kn \cdot \Rn)}
\end{equation}
En ocasiones, puede también despreciarse la pequeña contribución de $\alpha_{\text{vmp}}$ en el denominador, quedando

\begin{equation}
	\varepsilon(\kn) = \varepsilon_s - \beta - \gamma_{\text{vmp}} \sum_{\text{vmp}} \cos (\kn\cdot \Rn)
\end{equation}
El resultado es que cada nivel atómico discreto da lugar a una banda posible de valores de la energía (dependientes de $\kn$), cuya anchura está controlada por $\gamma$. Es importante que este modelo puede dar lugar a \textit{solapamiento} entre bandas (que para cierto $\kn$ una banda tenga mayores valores de $\varepsilon$ que la banda inmediatamente superior) incluso en 1D (figura \ref{Fig:07-06}). 


\begin{figure}[h!] \centering
	\includegraphics[scale=0.35]{Cuerpo/Ch_07/Fotos libro 6.pdf}
	\caption{Bandas de energía a partir de la aproximación de electrones fuertemente ligados. La zona sombreada representa el solapamiento entre la 1ª y 2ª bandas.}
	\label{Fig:07-06}
\end{figure}    

\section{Superficie de Fermi y zonas de Briollouin}

Una vez conocida la relación de dispersión monoeléctria $\varepsilon(\kn)$, se estudia a continuación como se ocupan los distintos niveles energéticos con los electrones aportados por el cristal. El concepto de la superficie de Fermi (estudiada en el capítulo anterior) como la superficie en el espacio de fases que separa, a $T=0$ K, los estados electrónicos llenos de los vacío, sigue teniendo sentido. Como se verá, la relación topológica entre la superficie de Fermi y las zonas de Brillouin es determinante para muchas propiedades metálicas.

\begin{figure}[h!] \centering
	\includegraphics[scale=0.35]{Cuerpo/Ch_07/Fotos libro 7.pdf}
	\caption{1ª, 2ª, 3ª y 4ª zonas de Brillouin para una red cuadrada 2D, según los esquemas en zona extendida (a) y reducida (b). En gris se representan los estados ocupados.}
	\label{Fig:07-07}
\end{figure}    

En la figura \ref{Fig:07-07} (a) se encuentran las primeras zonas de Brillouin para la red cuadrada, y la superficie de Fermi en la aproximación de red vacía. En este ejemplo, la primera zona de Brillouin (PZB) está completamente llena, y la 2ª (SZB), 3ª (TZB) y 4ª (CZB) parcialmente llenas. Se puede hacer una transformación al esquema en zona reducida por las traslaciones en vectores de red y resulta lo que ilustra la figura \ref{Fig:07-07} (b) (obsérvese que todas las zonas tienen el mismo volumen fásico). Las regiones grises son las ocupadas por electrones y tienen una energía inferior a la de las regiones blancas. En la figura \ref{Fig:07-08} se muestran otros ejemplos de zonas de Brillouin, esta vez para la estructuras \bcc y \fcc. Finalmente, en la figura \ref{Fig:07-09} se muestra un ejemplo, correspondiente a una estructura \fcc, de cómo es la relación topológica entre las zonas de Brillouin y la superficie de Fermi para electrones libres, según la valencia de los átomos que componen la estructura. 

La forma real de la superficie de Fermi puede determinarse mediante experimentos de magnetorresistencia, efecto pelicular anómalo, resonancia ciclotrón, magnetoacústica y el efecto Haas-van Alphen (oscilación del momento magnético de un metal en función de la intensidad del campo mangético aplicado). Como ya se explicó anteriormente, las superficies de Fermi se ven más deformadas (respecto de las de electrones libres) cerca de las fronteras de zona, de modo que en general las cortas perpendicularmente.

\begin{figure}[h!] \centering
	\includegraphics[scale=0.35]{Cuerpo/Ch_07/Fotos libro 8.pdf}
	\caption{Primeras zonas de Brillouin para las estructuras \bcc y \fcc, según el esquema en zona reducida.} 
	\label{Fig:07-08}
\end{figure}    
\begin{figure}[h!] \centering
	\includegraphics[scale=0.45]{Cuerpo/Ch_07/Fotos libro 9.pdf}
	\caption{Relación entre las primeras zonas de Brillouin de la estructura \fcc y la superficie de Fermi de electrones libres (esférica), según la valencia atómica.}
	\label{Fig:07-09}
\end{figure}    


\section{Metales, aislantes y semiconductores}

La importancia del ``mapeo'' de la superficie de Fermi en las distintas zonas, del que hemos visto algunos ejemplos, viene de lo siguiente. Como se verá en detalle en el siguiente Capítulo \ref{Ch:08}, debido al gap de energía en las fronteras de zona, las bandas llenan son aislantes, las zonas casi llenas conducen por cargas positivas, y las casi vacías por cargas negativas. Hay que recordar también que, como hemos visto, el número de electrones posibles en una zona o banda es $2N$ siendo $N$ el número de celdas primitivas en el cristal. En base a esto caben varias posibilidades: una sustancia con un número \textit{impar} de electrones de valencia por celda primitiva será siempre un \textit{metal} (conductor) ya que no puede llenar bandas enteras. Si el número de electrones es \textit{par} es necesario saber si las bandas sucesivas se solapan en energía o no. Si no hay solapamiento será \textit{aislante} y si solapan sucesivas (o \textit{semimetal} si el solapamiento pequeño). Podría darse también el caso de materiales aislantes con un gap de energía tan pequeña ($<1$ eV) que para $T\neq 0$ K conduzcan por promoción térmica de electrones entre bandas (\textit{semiconductores}). Estas posibilidades se esquematizan en la figura \ref{Fig:07-10}.

\begin{figure}[h!] \centering
	\includegraphics[scale=0.5]{Cuerpo/Ch_07/Fotos libro 10.pdf}
	\caption{Clasificación de los sólidos según la relación que hay entre el nivel de Fermi y la estructura de bandas.}
	\label{Fig:07-10}
\end{figure}    


\begin{Anotacion}
	\textcolor{red}{Hacen falta muchos ejemplos, o bien en forma de ejercicios o de manera explícita. Además la relación superficie de fermi y zonas de brilluoin, así como la explicación de las zonas extendidas/reducidas deja mucho que desear (en todos los libros). Revisar el \cite{Quiin}, quizás haya algo de interés.}
\end{Anotacion}




\chapter{El modelo de quarks y el modelo estándar}
\chapter{Cristales semiconductores} \label{Ch:09}

Como ya se describió en capítulos anteriores, un semiconductor no es sino un aislantes (a $T=0$K) pero con un gap de energía suficientemente pequeño ($\varepsilon_g <1$ eV) como para que, digamos a temperatura ambiente, una proporción apreciable de la banda de valencia (B.V.) pasan a ocupar la banda de conducción (B.C.). Estos electrones térmicamente excitados y los huecos que dejan de la B.V. pueden transportar corriente eléctrica (figura \ref{Fig:09-01}).


\begin{figure}[h!] \centering
	\includegraphics[scale=0.52]{Cuerpo/Ch_09/Fotos libro 1.pdf}
	\caption{Concentración de portadores para algunos metales, semimetales y semiconductores.}
	\label{Fig:09-01}
\end{figure}

Una propiedad especial de los semiconductores, que no se encuentra en metales, es que su conductividad eléctrica se puede alterar en muchos órdenes de manitud añadiendo pequeñas cantidades de otros elementos, que se denominan genéricamente \textit{impurezas}. Éstas determinan el carácter electrón o hueco de la conducción. Toda la electrónica de estado sólido (transistores, conmutadores diodos, células fotovoltaicas, etc.) descansa en esta propiedad conocida como \textit{dopado}.

Los semiconductores son generalmente cristales de enlace covalente. Los más comunes son los del grupo IV y los compuestos de los grupos III-V. El gap de energía $\varepsilon_g$ de algunos materiales se muestra en la tabla \ref{Tab:09-01}. Obsérvese la suave dependencia de $\varepsilon_g$ con la temperatura, debida en parte a la variación del espacio atómico (dilatación).

\begin{table}[h!] \centering
\begin{tabular}{ccc}
	Cristal & $\varepsilon_g$ (0 K) & $\varepsilon_g$ (300 K) \\ \hline
	Si & 1.17 & 1.12 \\
	Ge & 0.75 & 0.6 \\
	InSb & 0.23 & 0.17 \\ 
	GaAs & 1.52 & 1.43 \\
	Te & 0.33 & - \\ 
	SiC & 43.0 & - \\
	C (dia) & 5.5 & 5.5
\end{tabular}
\caption{Gap de energía para algunos materiales semiconductores.}
\label{Tab:09-01}
\end{table}
Otro parámetro fundamental es la masa efectiva, algunos de cuyos valores numéricos se muestran en la tabla \ref{Tab:09-02}. Hay dos masas de huecos porque en estos semiconductores hay dos bandas de valencia degeneradas (los máximos coinciden), pero que tienen distinta curvatura y, por tanto, distintas masas efectivas:

\begin{table}[h!] \centering
	\begin{tabular}{ccccc}
		Cristal & $m_l$ & $m_t$ & $m_h$ & $m_h'$ \\ \hline
		Si & 0.98 & 0.19 & 0.52 & 0.16 \\
		Ge & 1.57 & 0.082 & 0.34 & 0.043 \\
		InSb & 0.015 & 0.015 & 0.39 & 0.021 \\
		GaAs & 0.066 & 0.12 & 0.5 & 0.082
	\end{tabular}
	\caption{Masas efectivas de algunos semicondctores en unidades de la asa electrónica. $l$ y $t$ se refieren a la componente longitudinal y transversal de los bolsillos electrónicos, respectivamente.}
	\label{Tab:09-02}
\end{table}

\section{Concetración de portadores en equilibrio térmico}

Una propiedad fundamental de un semiconductor a temperatura $T$ es la concetración electrónca $n$ en la B.C. y la de los huecos $p$ en la B.V. Si denotamos por $D_c(\varepsilon)$ y $D_v(\varepsilon)$ las correspondientes densidades de estados por unidad de volumen, tendremos:

\begin{equation}
\begin{split}
	n \ = \ & \int_{\varepsilon_c}^{\infty} D_c (\varepsilon) f_{FD} (\varepsilon,T) \D \varepsilon \\
	p \ = \ & \int_{-\infty}^{\varepsilon_v} D_v (\varepsilon) [1-f_{FD}(\varepsilon,T)]\D \varepsilon
\end{split}\label{Ec:09-01-01}
\end{equation}
En la aproximación parabólica ($m^* = \cte$) la densidad de estados es una generalización directa de la obtenida para electrones libres. En concreto se encuentra

\begin{equation}
	D_{c,v} (\varepsilon) = \sqrt{2|\varepsilon-\varepsilon_{c,v}} \frac{m_{c,v}^{*3/2}}{\hbar^3 \pi^2}
\end{equation}
siendo $m_{c,v}^*=(m_1^*m_2^*m_3^*)^{1/3}$ la masa efectiva media geométrica de los valores principales del tensor de masa efecgiva de la B.C. y B.V. respectivamente. En todo lo que sigue adoptareos como válidas las condiciones:

\begin{equation}
\begin{split}
\varepsilon_c  - \mu & \gg k_B T \\
\mu - \varepsilon_v & \gg k_B T
\end{split} \label{Ec:09-01-03}
\end{equation}
en cuyo caso se dice no degenerado y se puede aplicar la estadística de Maxwell-Boltzmann en vez de la de Fermi-Dirac. En estas econdiciones las ecuaciones (\ref{Ec:09-01-01}) se reducen a:

\begin{equation}
	\begin{split}
		n = & N_c (T) e^{- \frac{\varepsilon_c - \mu}{k_B T}} \\
		n = & N_v (T) e^{- \frac{\mu-\varepsilon_v}{k_B T}} 
	\end{split} \label{Ec:09-01-04}
\end{equation}
donde los prefactores $N_c$ y $N_v$, a veces denominadas desidades efectivas de estados de la B.C. y B.V. vienen dados por

\begin{equation}
	N_{c,v}= \parentesis{\frac{m_{c,v}^* k_B T}{2^{1/3} \pi \hbar^2}}^{3/2} = 2.5 \parentesis{\frac{m_{c,v}^*}{m_e}}^{3/2} \ccorchetes{\frac{T(\text{K})}{300}}^{3/2} \cdot \unit{10^{19} \cm^{-3}} \label{Ec:09-01-05}
\end{equation}
y marcan la concetración máxima de portadores en un semiconductor no degenerado: $\unit{10^{18}-10^{19} \cm^{-3}}$. Aunque no concemos $\mu$ en (\ref{Ec:09-01-04}) podemos sin embargo obtener con toda generalidad, por simple multiplicación 

\begin{equation} 
	n p = N_c N_v e^{-\varepsilon_g/k_BT}\label{Ec:09-01-06}
\end{equation}
que se conoce como \textit{ley de acción de masas}, y donde ya no interviene $\mu$. En el caso de un semiconductor \textit{intrínseco}, que es aquél donde las impurezas no tienen influencia relevante (por ejemplo porque su número es despreciable), deberá ocurrir que $n=p=n_i$, lo que permite obtener de (\ref{Ec:09-01-04}) y (\ref{Ec:09-01-06})

\begin{equation}
	n_i = \sqrt{N_cN_v}e^{-\varepsilon_g/2k_BT} \label{Ec:09-01-07}
\end{equation}
Por ejemplo, $n_i(300\unit{\K})\approx\unit{10^{10}\cm^{-3}}$ en el Si. Asimismo, se obtiene

\begin{equation}
	\mu = \varepsilon_v + \frac{1}{2} \varepsilon_g + \frac{3}{4} k_B T \ln \parentesis{\frac{m_v^+}{m_c^+}} \label{Ec:09-01-08}
\end{equation}
de modo que $\mu$ está situado aproximadamente en la mitad del gap. La figura \ref{Fig:09-02} ilustra algunas de las relaciones encontradas.
\begin{figure}[h!] \centering
	\includegraphics[scale=0.4]{Cuerpo/Ch_09/Fotos libro 2.pdf}
	\caption{Densidad de estados, ocupación de estados, y concetración de huecos y electrones para un semiconductor intrínseco.}
	\label{Fig:09-02}
\end{figure}

\section{Semiconductores dopados}

La manera de incrementar y controlar los portadores de carga en un semiconductor es dopándolo, esto es, añadiendo sustitucionalmente ciertas impurezas eléctricamente activas. Las impuras que aportan electrones (huecos) son normalmente elementos del grupo V (III) que se llaman dadoras (aceptores) y entonces el semiconductor se dice de tipo $n$ (tipo $p$). Supongamos por concretar un semiconductor del grupo IV, como  el Ge, dopado con impurezas dadoras repartidas aleatoriamente en el cristal. Cada una se puede consdierar u ncentro de carga +$e$ al que está ligado un electrón, ya que los otros cuatro están ya ocupados formando los cuatros enlaces covalentes de la estructura del diamante. La energía necesaria para ionizar la impureza, pasando entonces el electrón del estado ligado localizado al estado libre deslocalizado (es decir, a la B.C.), es relativamente baja. Esto se debe a que el electrón en exceso ve un potencial periódico además de un centro de carga atractor $+e$, lo que significaba pasar de $m$ a $m^*$. Pero además, el electrón está en un medio dieléctrico (polarizable) que es el propio semiconductor, lo que implica pasar de $e^2$ a $e^2/\kappa$ siendo $\kappa$ la permitividad eléctrica relativa del medio, por ejemplo $\kappa(\ce{Ge})=15.8$, $\kappa(\ce{Si})=11.7$. Cuantitativamente se procede utilizando las fórmulas del átomo de hidrógeno. Así el radio de la primera órbita de Bohr $a_0=\hbar^2 / 2me^2=0.53 \ \unit{\angstrom}$ pasa a ser
 
\begin{equation}
	a_d = \frac{m}{m^*} \kappa a_0
\end{equation}
y la energía de ionización $\varepsilon_0=me^4/2\hbar^2=13.6 \ \unit{\eV}$ queda en

\begin{equation}
	\varepsilon_d = \frac{m^*}{m} \frac{1}{\kappa^2} \varepsilon_0
\end{equation}
Para valores típicos de $m^*$ y $\kappa$, $a_d$ puede ser $10^2 \ \unit{\angstrom}$ o más, y $\varepsilon_d$ puede quedar en $0.01-0.1 \ \unit{\eV}$. Argumentos similares se pueden desarrollar para las impurezas aceptoras, que se ionizan fácilmente (energía de ionización $\varepsilon_a$) captando un electrón de la B.V. Este electrón que ahora está localizado deja un hueco ``libre'' en la B.V. que contribuye a la conducción. Las impurezas dadoras generan un nuevo nivel  energético ($\varepsilon_D$) situado una energía $\varepsilon_d$ por debajo de la banda de conducción, y las aceptoras generan otro nivel ($\varepsilon_A$) situado una energía $\varepsilon_a$ por encima de la banda de valencia (ver figura \ref{Fig:09-03}).

\begin{figure}[h!] \centering
	\includegraphics[scale=0.35]{Cuerpo/Ch_09/Fotos libro 3.pdf}
	\caption{Densidad de estados, ocupación de estados, y concentración de huecos y electrones para semiconductores con impurezas dadoras (arriba), o aceptoras (abajo).}
	\label{Fig:09-03}
\end{figure}

La información fundamental respecto a los niveles de asociados a las impurezas es su grado de ocupación a una temperatura determinada. En el caso de impurezas aceptoras esta coincidirá con la impurezas ionizadas (que han captado un electrón de la banda de valencia). Utilizando la mecánica estadística es fácil comprobar que a una temperatura determinada la concetración de impurezas aceptoras ionizadas viene dada por

\begin{equation}
	N_A^- = \frac{N_A}{1+2e^{(\varepsilon_A-\mu)/k_BT}} \label{Ec:09-02-03}
\end{equation}
donde $N_A$ es la concetración total de impurezas aceptoras. Nótese la similitud de esta expresión con la distribución de Fermi-Dirac (la diferencia es debida a que la repulsión entre $e^-$ localizados hace muy improbable la ocupación de la impureza con dos electrones simultáneamente). En el caso de impurezas dadoras, la proporción de impurezas ionizadas (que han donado un $e^-$ a la banda de conducción) coincidirá con el grado de \textit{desocupación} de los niveles asociados. De manera análoga, se encuentra que la concetración de dichas impurezas ionizadas a una temperatura dada viene dada por
 
\begin{equation}
	N_D^+ =  \frac{N_D}{1+2e^{(\mu-\varepsilon_D)/k_BT}} \label{Ec:09-02-04}
\end{equation}
donde ahora $N_D$ es la concentración total de impurezas dadoras.

\section{Concentración de portadores en Semiconductores dopados}

La cercanía de los niveles dadores y aceptoras a las bandas hace esperar en un semiconductor dopado, por ejemplo tipo $n$, sesa mucho más probable la promoción de electrones a la B.C. desde los niveles dadores que desde la B.V. Dicho de otro modo, la concentración de portadores estará controlada por las impurzas; el semiconductor se dice entonces en \textit{régimen de ionización}. Una vez todas las impurezas dadores están ionizadas (esto ocurre a partir de la llamada \textit{temperatura de saturación}, $T_s \leq 100 \ \unit{K}$) se entra en el \textit{régimen de saturación}, en que $n\approx N_D$ y por tanto independientes de la temperatura (numéricamente $10^{14}<N_D<10^{18} \unit{cm^{-3}}$). Los electrones se dicen en este caso portadores mayoritarios y correspondientes, por la ley de acción de masas, los huecos son minoritarios. Al aumentar la temperatura deberá ocurrir que el número de electrones procedentes de la B.V. supere el debido a las impurezas, de modo que $n\gg N_D$ entrándose entones en el \textit{régimen intrínseco}, en que la presencia de impurezas resulte irrelevante y $n\approx n_i$. La temperatura a partir de la cual se entra en este régimen se conoce como \textit{temperatura intrínseca}, $T_i$. Argumentos similares, aunque complementarios, se pueden aducir para semiconductores tipo $p$. La figura \ref{Fig:09-03}, ilustra el efecto de las impurezas.

El estudio Cuantitativamente parte de las expresiones (\ref{Ec:09-01-01}) para la concentración de portadore en las bandas que, por la generalidad de su obtención, siguen siendo válidas en presencia de impurezas. No obstante, desconocemos el potencial químico, pues la expresión (\ref{Ec:09-01-08}) vista no es ahora válida, ya que al haber aporte extra de electrones y/o huecosya no se verifica la condición $n=p$. La generalización de esta condición en presencia de impurezas es 

\begin{equation}
	n+N_A^- = p + N_D^+ \label{Ec:09-03-01}
\end{equation}
llamado condición de \textit{electroneutralidad}. La utilización consiste en sustituir (\ref{Ec:09-01-01}), (\ref{Ec:09-02-03}) y (\ref{Ec:09-02-04}) en (\ref{Ec:09-03-01}) con objeto de delimitar $\mu$ y llevarlo luego a (\ref{Ec:09-01-01}) para conocer la conetración de portadores a cualquier temperatura. Cualitativamente la evolución de $n$ en función de la temperatura para un semiconductor tipo $n$ es el que ilustra la figura \ref{Fig:09-04}.

\begin{figure}[h!] \centering
	\includegraphics[scale=0.35]{Cuerpo/Ch_09/Fotos libro 4.pdf}
	\caption{Dependencia con la temperatura de la concentración de portadores en un semiconductor dopado con impurezas dadoras.}
	\label{Fig:09-04}
\end{figure}

\section{Conductividad y movilidad}

Consideraremos sólo semiconductores no degradados, es decir, aquéllos para los que la aproximación de Boltzmann (\ref{Ec:09-01-03}) es válida. Las contribuciones a la conductividad eléctrica debidas a electrones en la B.C. y a huecos en la B.V. se expresan por 

\begin{equation}
\begin{split}
	\sigma_n = & \frac{ne^2 \bar{\tau}_e}{m_c^*} \\	
	\sigma_p = & \frac{pe^2 \bar{\tau}_h}{m_v^*}
\end{split} \label{Ec:09-04-01}
\end{equation}
y la conductividad eléctrica total vendrá dada por $\sigma=\sigma_n + \sigma_p$. Los tiempos de relajación de la ecuación anterior (\ref{Ec:09-04-01}) son medias para todos los electrones en la B.C. y huecos en la B.V., pues todos ellos participan en la conducción: la aproximación de Boltzmann hace que la probabilidad de ocupación de un estado en B.V. o B.C. sesa muy pequeña por lo que un electrón y/o hueco tiene acceso a los niveles próximos, condición necesario para que se pueda ``acelerar''. Las ecuaciones (\ref{Ec:09-04-01}) pueden expresarse en función de las correspondientes \textit{movilidades} $\mu_n$ y $\mu_p$ según


\begin{equation}
	\begin{split}
		\sigma_n = en\mu_n \quad & \parentesis{\mu_n \equiv \frac{e^2 \bar{\tau}_e}{m_c^*}} \\	
		\sigma_p = ep\mu_p \quad & \parentesis{\mu_p \equiv \frac{e^2 \bar{\tau}_h}{m_v^*} }
	\end{split} \label{Ec:09-04-02}
\end{equation}
Los mecanismos básicos de dispersión de los portadores que afectan a la movilidad en semiconductores son: los fonones, como en metales, y las impurezas ionizadas a través de la dispersión de Rutherford (salvo a muy bajas temperaturas en que permanecen neutras). Sin entrar en detalles digamos que las dependencias respectivas con la temperatura son:

\begin{equation}
\begin{split}
\mu_{\text{fon}} & \propto T^{-3/2} \\
\mu_{\text{imp}} & \propto T^{3/2} \\
\end{split}
\end{equation}
Obsérvese que la movilidad por interacción con fonones disminuye al aumentar la temperatura por aumentar el número de éstos, pero sin embargo, la movilidad aumenta para la interacción con impurezas. La razón es que al aumentar la temperatura aumenta la velocidad de los portadores de modo que están menos tiempo en el radio de influencia de cada interacción. La figura \ref{Fig:09-05} muestra los dos regímenes comentados (el cambio de régimen depende de la concentración de impurezas, pero en general ocurre para $T<100\ \unit{K}$).

\begin{figure}[h!] \centering
	\includegraphics[scale=0.35]{Cuerpo/Ch_09/Fotos libro 5.pdf}
	\caption{Dependencia con la temperatura de la movilidad en un semiconductor con impurezas.}
	\label{Fig:09-05}
\end{figure}

Es interesante particularizar para un semiconductor intrínseco a \textit{alta temperatura}. A partir de (\ref{Ec:09-01-05}), (\ref{Ec:09-01-07}) y (\ref{Ec:09-04-02}) se encuentra para la conductividad eléctrica

\begin{equation}
	\sigma = en\mu_n + ep\mu_p \propto \parentesis{T^{3/2}e^{-\varepsilon_g/2k_BT}}T^{-3/2}  = e^{-\varepsilon_g/2k_BT}
\end{equation}
Este resultado proporciona un método sencillo para determinar el gap de energía de un semiconductor a partir de la conductividad en función de la temperatura.

\section{Semiconductores inhomogeneos: la unión p-n.}

Se estudia el comportamiento de una unión de un semiconductor tipo $p$ con otro tipo $n$, que es la base de buena parte de la microelectrónica. Cualitativamente, los electrones tienden a difundirse del lado $n$ donde son mayoritarios al $p$ donde son minoritarios, y viceversa para los huecos. Esta difusión origina un campo eléctrico en una región de dimensiones $d_n+d_p$ aldrededor de la superficie de separación que frena la difusión, llegándose así al equilibrio. Además, como los portadores son muy móviles, la concentración de los mismos debe ser muy baja allí donde el campo eléctrico es apreciable. Por tanto la ausencia de portadores libres en la región de campo hace que ésta aparezca cargada, tal como ilustra la figura \ref{Fig:09-06} justificando además el nombre de \textit{región de carga espacial} que recive. Típicamente $d_n+d_p\sim10^2 - 10^4 \unit{\angstrom}$ y el campo en la región de carga es $10^5 - 10^7$ V/m.


\begin{figure}[h!] \centering
	\includegraphics[scale=0.35]{Cuerpo/Ch_09/Fotos libro 6.pdf}
	\caption{Densidad de carga, densidad de protones, y potencial eléctrico en la región de caga especial de unión $p-n$.}
	\label{Fig:09-06}
\end{figure}

Vamos a considerar ahora qué ocurre se aplica un potencial externo $V$ (tomaremos $V>0$ si es más alto de lado $p$ que el lado $n$, ver \ref{Fig:09-07}). Puesto que muy cerca de la unión, en la capa de carga espacial, la concentración de portadores es casi nula, la resistencia eléctrica que ésta presenta será mucho mayor que en las regiones homogéneas. Esto implica que cuando se aplica un potencial $V$ a la unión de mayor caída se dará en la carga espacial. Queremos deducir aproximadamente la corriente eléctrica a través de la unión si $V\neq 0$. Para empezar, consideremos en detalle la corriente debida a huecos. Hay dos contribuciones:

\begin{itemize}
	\item La corriente de generación $j_h^{\text{gen}}$ del lado $n$ al $p$, que aparece por los huecos generados térmicamente del lado $n$ donde son minoritarios al lado $p$ donde son mayoritarios. Aunque la concentración es pequeño, la corriente a t ravés de la unión puedde ser alta ya que los huecos se ven lanzados por el campo eléctrico en la unión, y será poco sensible al valor de $V$ si $eV\gg \varepsilon_g$.
	\item La corriente de recombinación $j_h^{\text{rec}}$, del lado $p$ al lado $n$, debida a los huecos que, llegados a la unión del lado $p$, tengan energía suficiente para vencer el campo eléctrica en la misma. Esta corriente debida a hueos será:
	\begin{equation}
		j_h^{\text{rec}} \propto \exp \parentesis{-e\frac{\Delta \phi_0-V}{k_BT}} = j_h^{\text{rec}}|_{V=0} \exp \parentesis{\frac{eV}{k_BT}}
	\end{equation}
\end{itemize}
Se debe cumplir además que

\begin{equation}
	\parentesis{j_h^{\text{rec}}}_{V=0} = \parentesis{j_h^{\text{gen}}}_{V=0} \approx\parentesis{j_h^{\text{gen}}}_V
\end{equation}
por la pequeña dependencia de $J_h^{\text{rec}}$ al potencial externo; con esto

\begin{equation}
	j_h^{\text{rec}} = j_h^{\text{gen}} \exp \parentesis{\frac{eV}{k_BT}}
\end{equation}
de donde 

\begin{equation}
	j_h = j_h^{\text{rec}}  - j_h^{\text{gen}}  = j_h^{\text{gen}} \ccorchetes{\exp\parentesis{\frac{eV}{k_BT}}-1} 
\end{equation}
\begin{figure}[h!] \centering
	\includegraphics[scale=0.25]{Cuerpo/Ch_09/Fotos libro 7.pdf}
	\caption{Efecto de un potencial externo sobre la densidad de carga en la región de carga especial de unión $p-n$.}
	\label{Fig:09-07}
\end{figure}
Análogo análisis se aplica a los electrones, de forma que para la densidad de corriente eléctrica \textit{total} se tiene 
 
\begin{equation}
	j=\parentesis{j_h^{\text{rec}}-j_e^{\text{gen}}} \ccorchetes{\exp \parentesis{\frac{eV}{k_BT}}-1} \label{Ec:09-05-05}
\end{equation}
Se peude probar que el término $j_h^{\text{gen}}+j_e^{\text{gen}}$ varía con la temperatura según $\exp\parentesis{-\varepsilon_g/k_BT}$. La alta simetría de la unión respecto de la polarización en voltaje tal como resume la ecuación (\ref{Ec:09-05-05}) y se representa en la , constituye el principio \textit{rectificador} de la unión.

\begin{figure}[h!] \centering
	\includegraphics[scale=0.25]{Cuerpo/Ch_09/Fotos libro 8.pdf}
	\caption{Relación $j-V$ para la unión $p-n$.}
	\label{Fig:09-08}
\end{figure}
\chapter{Magnetismo de sólidos} \label{Ch:10}

En este Capítulo se estudian algunas de las contribuciones más importantes al magnetismo de los sólidos. Como se verá, alguna de ellas (como la ferromagnética) constituye un fenómeno cooperativo que lleva asociado una transición de fase. Es también destacable que el magnetismo de sólidos es en buena medida un efecto cuántico dado que muchas de sus causas (el momento mangnético de espín, la interacción de intercambio, etc.) no tienen análogo clásico.

\section{Relaciones básicas}

La \textbf{magnetización} (o imanación) $\Mn$ se define como el momento magnético por unidad de volumen 

\begin{equation}
	\Mn = \frac{1}{V} \sum_V \mun	
\end{equation}
donde los $\mun$ son los momentos magnéticos atómicos o iónicos en el ``volumen de control'' $V$. Recordemos que para partículas sin espín $\mun =  \sum_i \frac{1}{2} \rn_i \times q_i \vn_i$ y que para \textit{lazos} de corriente $\mun  = I \An$ siendo $I$ la intensidad y  $\An$ el área encerrada. El magnetismo de los sólidos se clasifica de acuerdo a la interrelación de su magnetización con el campo magnético aplicado $\Hn$. Así, se introduce la \textit{susceptibilidad magnética} $\chi$ por 

\begin{equation}
	\Mn = \chi \Hn
\end{equation}
En general $\chi$ será un tensor, pero no consideraremos esta complicación aquí. Si un medio tiene $\chi$ negativa se el sólido se dice \textit{diamagnético} y si por contra $\chi$ es positiva el sólido se dice \textit{paramagnético}. Además, existe un grupo de materiales que pueden poseer magnetización aun en ausencia de campo aplicado y que son \textit{ferromangéticos} (observar que $\chi \rightarrow \infty$).

\section{Diamagnetismo atómico}

La naturaleza del diamagnetismo atómico se puede comprender a partir de un modelo clásico en que cada órbita electrónica es considerada una espira de corriente. De acuerdo con la ley de Lenz, al variar el flujo magnético sobre el circuito surge una fuerza electromotriz de inducción que hace variar la corriente y por tanto genera un momento magnético adicional. Así, si $A$ es el área de la órbita, $\omega$ la velocidad angular e $I$ la intensidad de corriente, el momento magnético será $\mu=IA=-(e\omega / 2\pi)A$, y bajo la aplicación de un campo $B$ perpendicular a la órbita, aquél se incrementa en

\begin{equation}
	\Delta \mu = - \frac{eA}{2\pi} \Delta \Omega \label{Ec:10-02-01}
\end{equation}
La variación de velocidad angular $\Delta \omega$ se determina del balance de fuerzas sobre la órbita, que consideramos circular de radio $r$ (ver figura \ref{Fig:10-01}): antes de la aplicación del campo de fuerza nuclear se equipara a la fuerza centrífuga, $F_\text{núcleo} = m\omega^2 r$; tras la aplicación del campo aparece adicionalmente la fuerza de Lorentz $F_L = ev B$ dirigida hacia el núcleo, que se compensa con un aumento de la velocidad angular 

\begin{equation}
	m(\omega + \Delta \omega)^2 r = F_\text{núcleo} + e v B
\end{equation}
Como para los campos magnéticos aplicables en un laboratorio $evB\ll m \omega^2 r$, se tiene que $\Delta \omega \ll \omega$. Así pues, despreciando términos en $\Delta \omega^2$ se deduce 

\begin{equation}
	\Delta \omega = \frac{evB}{2mr\omega} = \frac{eB}{2m}
\end{equation}
que se denomina \textit{frecuencia de Larmor}. Combinando con (\ref{Ec:10-02-01}) y generalizando a un átomo de $Z$ electrones:

\begin{equation}
	\Delta \mu = - Z \frac{e^2A}{4\pi m} B = - Z \frac{e^2 \langle \rho^2 \rangle}{4m}B
\end{equation}
siendo $\langle \rho^2 \rangle = \langle x^2 + y^2 \rangle = \langle x^2 \rangle + \langle y^2 \rangle$ el valor cuadrático medio de la distancia de los electrones al eje que pasa por el núcleo paralelamente al campo (eje $z$). Si admitimos simetría esférica, $\langle x^2 \rangle=\langle y^2 \rangle=\langle z^2 \rangle$ y con ello el radio cuadrático medio atómico resulta $\langle r^2 \rangle\= \frac{3}{2} \langle \rho^2 \rangle$. Llamando $n$ al número de átomos por unidad de volumen la magnetización resultante es

\begin{equation}
	M = n \Delta \mu = - \frac{nZe^2 \langle r^2 \rangle}{6m}B
\end{equation}
y la susceptibilidad magnética 

\begin{equation}
	\chi = \frac{M}{H} \approx \frac{mu_0 M}{B} = - \frac{\mu_0 n Z e^2 \langle r^2 \rangle}{6m}
\end{equation}
que es el resultado clásico de \textit{Langevin}. Aunque en general $B=\mu_0 (H+M)$, aquí se ha aproximado $B\approx \mu_0 H$ dado que $|M|\ll H$ (o bien que $|\chi |\ll 1$). En efecto, para $n=5\times10^{28} \unit{m}^{-3}$ y $r=10^{-10}$ m, $\chi = -10^{-6}Z$. El diamagnetismo atómico es propio de \textit{todos los cuerpos sin excepción}, aunque a menudo está enmascarado por el paramagnetismo o el ferromagnetismo.



\begin{figure}[h!] \centering
	\includegraphics[scale=0.35]{Cuerpo/Ch_10/Fotos libro 1.pdf}
	\caption{Balance  de fuerzassobre un electrón orbitando en torno a un núcleo en ausencia (izquierda) o en presencia (derecha) de un campo mangético externo. El referencial usado es uno centrado en el núcleo y que gira con el electrón de modo qeu hay que considerar las fuerzas inerciales (aquí sólo hay fuerza centrífuga).}
	\label{Fig:10-01}
\end{figure}

\section{Paramagnetismo atómico}

\subsection{Origen del momento mangnético atómico}

\subsection[Dependencia de la magnetización respecto $\vec{\Bn}$ y $T$]{Dependencia de la magneteización paramagnética con la temperatura y el campo magnético}

\subsection{Ley de Curie}

\section{Paramagnetismo de los electrones de conducción}

\section{La interacción de intercambio}

\section{Ferromagnetismo}

\section{Dominios ferromagnéticos}

\section{Orden ferrimangnético}

\begin{figure}[h!] \centering
	\includegraphics[scale=0.35]{Cuerpo/Ch_10/Fotos libro 2.pdf}
	\caption{Dependencia con $x$ de la función de Brillouin y de la magnetización paramagnética para distintos valores de $J$.}
	\label{Fig:10-02}
\end{figure}
\begin{figure}[h!] \centering
	\includegraphics[scale=0.35]{Cuerpo/Ch_10/Fotos libro 3.pdf}
	\caption{Desplazamiento relativo de los niveles energéticos electrónicos correspondientes a espín $\uparrow$ y espín $\downarrow$.}
	\label{Fig:10-03}
\end{figure}
\begin{figure}[h!] \centering
	\includegraphics[scale=0.35]{Cuerpo/Ch_10/Fotos libro 4.pdf}
	\caption{Dependencia con la temperatura de la magnetización espontánea para el Ni cuando $T<T_C$, y compensación con las predicciones de campo medio.}
	\label{Fig:10-04}
\end{figure}
\begin{figure}[h!] \centering
	\includegraphics[scale=0.35]{Cuerpo/Ch_10/Fotos libro 5.pdf}
	\caption{En la izquierda la magnetización de una muestra de  Co para las direcciones paralela y perpendicular al plano basal de la estructura hexagonal. En la derecha de anisotropía del Co según el ángulo que forma la magnetización con el eje hexagonal.}
	\label{Fig:10-05}
\end{figure}
\begin{figure}[h!] \centering
	\includegraphics[scale=0.35]{Cuerpo/Ch_10/Fotos libro 6.pdf}
	\caption{Distribución electrónica para diferntes orientaciones del campo mangético aplicado.}
	\label{Fig:10-06}
\end{figure}
\begin{figure}[h!] \centering
	\includegraphics[scale=0.35]{Cuerpo/Ch_10/Fotos libro 7.pdf}
	\caption{Aparición de dominios en una muestra ferromagnética. Las flechas indican la orientación de los momentos magnéticos en cada dominio.}
	\label{Fig:10-07}
\end{figure}
\begin{figure}[h!] \centering
	\includegraphics[scale=0.35]{Cuerpo/Ch_10/Fotos libro 8.pdf}
	\caption{Desplazamiento de las fronteras entre dominios debido a una variación del campo magnético aplicado.}
	\label{Fig:10-08}
\end{figure}
\begin{figure}[h!] \centering
	\includegraphics[scale=0.35]{Cuerpo/Ch_10/Fotos libro 9.pdf}
	\caption{Histéresis de la dependencia $M(H)$ debida al anclado de las paredes Bloch en los defectos del material.}
	\label{Fig:10-09}
\end{figure}
\chapter{Superconductividad} \label{Ch:11}

\section{Fenomenología básica}

\section{Aspectos termodinámicos de los superconductores}

\section{Ecuaciones de London}

\section{La teoría de Ginzburg-Landau (GL)}

\section{Propiedades magnéticas de los superconductores de tipo II}

\section{Introducción a la teoría BCS}

\section{Compuestos superconductores}

\appendix 
\chapter{Apéndice} \label{Ch:Anex_A}

\section{Teoría de perturbaciones independiente del tiempo}

\subsection{Primer orden}

\subsection{Segundo orden}


\section{Momento angular y espín}

Las leyes de la naturaleza no deberían depender de como este orientado nuestro laboratorio. Se espera entonces que nuestras teorías sean invariante bajo rotaciones. En este apartado vamos a probar como la invariancia bajo rotaciones lleva a la existencia de la conservación del momento $\Jn$. Una rotación en un espacio tridimensional es una trasnformación lineal $x_i'=\sum_j R_{ij} x_j$ de las Coordenadas cartesianas  $x_i$ que deja invariante el producto escalar $\xn \cdot \yn$. De este modo tenemos que:

\subsection{Momento angular para $j=1/2,1,3/2$}

\subsection{Representaciones del operador rotación: matrices de rotación}


\subsection{Coeficientes de Clebsch-Gordan}

Dos sistemas con momentos angulares $\Jn_1$ y $\Jn_2$ pueden ser considerados juntos como un sistema global de momento angular total $\Jn_3 = \Jn_1 + \Jn_2$. Existen dos bases de autofunciones de este tercer sistema, representadas por $|j_1 j_2 j_3 m_3\rangle$ y $|j_1 j_2 m_1 m_2\rangle$. Lógicametne podremos cambiar de un estado a otro usando:

\begin{equation}
   |j_1 j_2 j_3 m_3 \rangle  = \sum_{m_1, m_2}  \langle j_1 j_2 m_1 m_2 | j_1 j_2 j_3 m_3 \rangle    |j_1 j_2 m_1 m_2\rangle
\end{equation}
A los elementos de la matriz $\langle j_1 j_2 m_1 m_2 | j_1 j_2 j_3 m_3 \rangle$ se le llaman \textbf{coeficientes de Clebsch-Gordan}. Una notación alternativa es:

\begin{equation}
    \begin{array}{c}
    \Psi^{m_3}_{j_1j_2j_3} = \sum_{m_1,m_2} C_{j_1j_2} (j_3 m_3; m_1 m_2) \Psi^{m_1 m_2}_{j_1 j_2} \\ \\
    \Psi^{m_1 m_2}_{j_1j_2} = \sum_{j_3,m_3} C_{j_1j_2} (j_3 m_3; m_1 m_2) \Psi^{m_1 m_2}_{j_1 j_2} 
    \end{array}
\end{equation}

\subsection{Armónicos esféricos}

Los armónicos esféricos $Y_l^m (\theta, \varphi)$ son las autofunciones del orbital momento angular orbital, y satisfacen las siguientes ecuaciones diferenciales:

\begin{equation}
    \ccorchetes{\frac{1}{\sin (\theta)} \parciales{}{\theta} \parentesis{\sin (\theta) \parciales{1}{\theta} + \frac{1}{\sin^2 (\theta)} \parciales{^2}{\varphi^2}}} Y_l^m + l(l+1) Y_l^m = 0
\end{equation}
Y vienen dadas explícitamente por:

\begin{equation}
    Y_l^m (\theta,\varphi) = (-1)^m \ccorchetes{\frac{2l+1}{4 \pi} \frac{(l-m)!}{(l+m)!}}^{1/2} P_l^m (\cos (\theta)) e^{im\varphi}
\end{equation}

\subsection{El teorema de Wigner-Eckart}

Sean los $|\Phi_j^m\rangle$ los autoestados del momento angular con autovalores $j(j+1 )\hbar^2$ y $m_j \hbar$ para $J^2$ y $J_3$ respectivamente. Recordar que

\begin{eqnarray}
    (J_1\pm iJ_2) |\Phi_j^m \rangle = \hbar \sqrt{j(j+1)-m(m\pm 1)} |\Phi_j^{m\pm 1}
\end{eqnarray}
Sea $|\Psi_j^m\rangle$ otros autoestados del momento angular. Podemos demostrar que

\begin{equation}
    \langle \Phi_j^{m+1} | \Psi_j^{m+1} \rangle = \langle \Phi_j^m |\Psi_j^m \rangle
\end{equation}
Esto demuestra que $\langle \Phi_j^m |\Psi_j^m \rangle$ es {\it independiente} de $m$. Cualquier otro elemenot de la matriz con valores de $j$ y $m$ diferentes se anulan:

\begin{equation}
    \langle \Psi_{j_3}^{m_3} | O_{j_2}^{m_2} \rangle = 0
\end{equation}
Definimos como un {\bf tensor irreducible} de rango $j$ como un conjunto de $2j+1$ operadores $O_{j}^m$  ($m=-j,-j+1,...,j$) que al aplicarle lso generadores de rotación

\begin{equation}
    [J_3,O_j^m] = \hbar m O_j^m \tquad [J_1\pm i J_2, O_j^m] = \hbar \sqrt{j(j+1)-m(m\pm 1)} O_{j}^{m\pm 1}
\end{equation}
Algunos ejemplos de tensores irreducibles son los {\it armónicos esféricos}. 

\begin{theorem}[{\bf Wigner-Eckart}]
    Sea $\langle j_i j_2 m_1 m_2 |j_1 j_2 j_3 m_3 \rangle$ es el coeficiente de Clebsch-Gordan asociado con el acoplamiento de los momentos angulares $\Jn_1$ y $\Jn_2$ que componen $\Jn_3$; y $\langle \Phi || O || \Psi\rangle$, llamada la {\it matriz irreducible elemental}, que puede depende de todo menos de las tres componentes $m_1,m_2$ y $m_3$; el teorema de Wigner-Eckart nos dice que:

    \begin{equation}
        \langle \Phi_{j_3}^{m_3} | O_{j_1}^{m_1} |\Psi_{j_2}^{m_2} \rangle = \frac{1}{2j_3+1} \langle  j_i j_2 m_1 m_2 | j_1 j_2 j_3 m_3 \rangle \langle \Phi || O || \Psi  \rangle
    \end{equation}
    El teorema de Wigner-Eckart se puede expresar de otra forma, la {\bf Formula de Landé}. Sea $\An$ un vector cualquiera y $\Jn$ un moemnto angular. Esta fórmula nos dice que:

    \begin{equation}
        \langle \Phi_{j}^{m} | \An |\Psi_{j}^{m'} \rangle = \frac{\langle \Phi_j^m |\An \cdot \Jn | \Psi_j^m\rangle }{j(j+1)\hbar^2} \langle \Phi_j^m | \Jn| \Psi_j^{m'}  \rangle
    \end{equation}
    
\end{theorem}

\bibliography{Bibliografia.bib}
\bibliographystyle{unsrt}




\end{document}




