
\chapter{Dinámica de redes} \label{Ch:04}

En este capítulo se comienza el estudio de las vibraciones de los átomos alrededor de sus posiciones de equilibrio y sus efectos observables. Esta llamada \textit{dinámica de redes} es necesaria para explicar propiedades como: i) la conductividad térmica de los aislantes, ii) la dependencia en $T^3$ del calor específico a baja temperatura, iii) las energías de cohesión, iv) la dilatación térmica, v) la conductividad eléctrica \textit{finita} de los metales, vi) la reflectividad de los cristales iónicos, etc.


\section{Vibraciones de los cristales con base atómica}

Aquí se utilizará la llamada \textit{aproximación armónica} que consiste en aceptar que las desviaciones de las posiciones atómicas (respecto del equilibrio) son mucho menores que las distancias interatómicas 

\begin{equation}
	\rn (t) = \Rn + \Delta  \rn (t)  \quad \textit{con} \quad |\Delta \rn (t) | \ll |\rn (t)| \sim |\Rn|
\end{equation}
Esta aproximación equivale a decir que la fuerza elástica sobre un átomo es \textit{lineal} con los desplazamientos relativos $\Fn \propto \Delta \rn (t)$.

\subsection{La cadena lineal monoatómica}

El modelo es una cadena unidimensional de átomo de masa $M$ e interdistancia $a$ que puede servir para representar algunas de las vibraciones de un cristal tridimensional. Así, las vibraciones elásticas de un cristal cuando el vector de onda $\kn$ apunta en las direcciones [100], [110]  ó [111] en un cristal cúbico pueden suponerse unidimensionales en el sentido de que se necesita una sola coordenada, pues se espera que los planos de átomos perpendiculares $\kn$ se muevan en fase. 

\begin{figure}[h!] \centering
    \includegraphics[scale=0.45]{Cuerpo/Ch_04/Fotos libro 1.pdf}
    \caption{Parámetro de red y posiciones de átomos de masa $M$ conectados por una fuerza de constante $C$ entre planos adyacentes. Los desplazamientos de los átomos se designan por $u_{s-1},u_{s},u_{s+1}$.}
    \label{Fig:04-01}
\end{figure}    

Denotando por $u_s$ la desviación del átomo (o plano) $s$ (ver figura \ref{Fig:04-01}), su ecuación del movimiento es

\begin{equation}
	M \derivadas{^2 u_s}{t^2} = F_s = - \parciales{U_{\text{arm}}}{u_s} \label{Ec:04-01-02}
\end{equation}
con 

\begin{equation}
	U_{\text{arm}} = \frac{1}{2} \sum_{s,l} \frac{C_l}{2} \parentesis{u_{s+l}-u_s}^2
\end{equation}
$C_l$ es la constante de fuerza entre dos átomos separados $la$. Esta expresión se justifica por la aproximación armónica citada. Si los desplazamientos son suficientemente pequeños el término dominante del desarrollo en serie de Taylor de la energía potencial es el cuadrático, pues el lineal no existe porque se exige $U_{\text{arm}}$ mínimo (derivada nula) en $u_s$. El factor $1/2$ que multiplica a la suma se introduce para no contar dos veces la misma interacción. 

Para simplificar se tendrían en cuenta aquí solo interacciones entre vecinos más próximos, es decir, $l=\pm 1$, con lo cual $C_\pm = C$ y 

\begin{equation}
	U_{\text{arm}} = \frac{C}{2} \sum_s \ccorchetes{\parentesis{u_{s+1}+u_s}^2+\parentesis{u_{s-1}-u_s}^2}
\end{equation}
y por tanto la fuerza total sobre $s$ 

\begin{equation}
	F_s = C (u_{s+1}-u_s) + C(u_{s-1}-u_s)
\end{equation}
Por la anisotropía de los sólidos, en general $C$ será diferente según la dirección en la que se producen las oscilaciones. 

Con esta aproximación, la ecuación del movimiento (\ref{Ec:04-01-02}) admite la solución armónica

\begin{equation}
	u_s = u_k e^{i(ksa-\omega t)}
\end{equation}
siempre que se verifique 

\begin{equation}
	\omega = \sqrt{\frac{4C}{M}} \left| \sin \parentesis{\frac{1}{2} k a} \right| \label{Ec:04-01-07}
\end{equation}
que es llamada la \textit{relación de dispersión} y cuya gráfica se muestra en la figura \ref{Fig:04-02}. Algunos puntos importantes son:

\begin{itemize}
	\item Hay simetría $k\rightarrow -k$.
	\item Hay periodicidad $k\rightarrow k+n2\pi/a=k+G$. La razón de que $k$ y $k+G$ sean equivalentes es que dan lugar a las mismas coordenadas atómicas, como se muestra con el ejemplo de la figura \ref{Fig:04-03}. Por la \textit{redundancia} existente en el espectro sólo consideramos los vectores de onda $k \in PZB$ (\textbf{primera zona de Brillouin}).
	\item La \textit{velocidad de grupo} $v_g = \D\omega / \D k$ se anula en $k=\pm \pi/a$. Observar que éstos precisamente los valores que satisfacen la condición de Bragg, por lo que parece como si la onda $k=\pi / a$ se reflejara internamente ($k \rightarrow -k$) para dar lugar a una \textit{onda estacionaria}.
	\item No pueden existir ondas viajeras con $\omega>2 (C/M)^{1/2}$.
	\item Para $ka\rightarrow 0 \approx a(C/M)^{1/2}k$, es decir, la \textit{la velocidad de fase} $c=\omega /k$ es constante, y es por lo que la relación de dispersión (\ref{Ec:04-01-07}) se denomina \textit{rama acústica}. $c=a \omega_{\max} /2$ es la velocidad del sonido en el cristal.
\end{itemize}


\begin{figure}[h!] \centering
    \includegraphics[scale=0.45]{Cuerpo/Ch_04/Fotos libro 2.pdf}
    \caption{Representación de la relación de dispersión de una cadena monoatómica de átomos de masa $M$ y constante de acoplamiento $C$. Observar que para $k\rightarrow 0$, $\omega \rightarrow k$}
    \label{Fig:04-02}
\end{figure}    

\begin{figure}[h!] \centering
    \includegraphics[scale=0.45]{Cuerpo/Ch_04/Fotos libro 3.pdf}
    \caption{Ejemplo de ondas con distancia longitud de onda que sin embargo representan el mismo estado de movimiento de los átomos.}
    \label{Fig:04-03}
\end{figure}    

\subsection{Cristales monoatómicoos tridimensionales}

En este caso, los átomos ocupan posiciones $\rn(t) = \Rn + \un (\Rn,t)$. Si llamamos a $\phi$ al potencial de interacción entre dos átomos, la \textit{aproximación armónica} nos permite aproximar la energía total por un desarrollo en serie a segundo orden:

\begin{equation}
	U_{\text{arm}} = \frac{1}{4} \sum_{\Rn,\Rn'} \ccorchetes{\un(\Rn)-\un(\Rn')} \Phi \ccorchetes{\un(\Rn)-\un(\Rn')} \label{Ec:04-01-08}
\end{equation}
con $\Phi_{ij}=\partial^2 \phi / \partial x_i \partial x_j$, siendo $x_i(t) = R_i + u_i (\Rn,t) \ (i,j=1,2,3)$. Equivalentemente (\ref{Ec:04-01-08}) se puede escribir como:

\begin{equation}
	U_{\text{arm}} = \frac{1}{4} \sum_{\Rn,\Rn'}\un(\Rn) D(\Rn-\Rn' ) \un(\Rn') \label{Ec:04-01-09}
\end{equation}
donde 

\begin{equation}
	 D(\Rn-\Rn' ) = \delta_{\Rn \Rn'} \sum_{\Rn''} \Phi (\Rn- \Rn'') - \Phi (\Rn - \Rn')
\end{equation}
es una matriz que contiene las interacciones entre pares de átomos. La {\bf ecuación dinámica} es

\begin{equation}
 	M \derivadas{\un (\Rn)}{t^2} =  - \parciales{U_\text{arm}}{\un (\Rn)} = \sum_{\Rn'} D (\Rn-\Rn') \un (\Rn') \label{Ec:04-01-11}
\end{equation}
que admite soluciones de la forma

\begin{equation}
	\un (\Rn,t) = \epsilonn (\Rn,t) e^{i(\kn \cdot \Rn - \omega t)}  \label{Ec:04-01-12}
\end{equation}
donde $\epsilonn(\kn)$ es el \textit{vector de polarización}. Al sustituir  (\ref{Ec:04-01-12}) en (\ref{Ec:04-01-11}) se ve que la condición de existencia de solución es 

\begin{equation}
	D(\kn) \epsilonn (\kn) = M \omega^2 (\kn) \epsilonn \kn \label{Ec:04-01-13}
\end{equation}
donde $D(\kn) = \sum_\Rn D(\Rn) e^{i \kn \cdot \Rn}$ es la llamada \textit{matriz dinámica}, que se puede ver que es real y simétrica. Esto garantiza la existencia de \textit{tres} soluciones a (\ref{Ec:04-01-13}) que verifican $\epsilonn_i (\kn) \cdot \epsilonn_j (\kn) = \delta_{ij}$, ($i,j=1,2,3$), y que se denominan \textit{ramas acústicas} por verificarse en ellas que $\omega_i (\kn) \approx c_i (\kn)k$ para $ka\ll 1$. Dado un $\kn$, los vectores $\epsilonn$ no tienen en general que ser paralelos o perpendiculares a $\kn$. Por eso sólo se puede hablar aproximadamente de \textit{polarización longitudinal} o \textit{transversal}. A pesar de ello, los tres modos posibles para cada $\kn$ se denominan \textit{acústica longitudinal} (LA) y \textit{acústicos transversales} (TA). El resto de las propiedades coinciden con las del caso unidimensional.

\section{Vibraciones de cristales con base diatómica}

\subsection{La cadena lineal diatómica}

El modelo es una cadena unidimensional con base diatómica. Los átomos de la base se distinguirán por tener masa distinta (véase \ref{Fig:04-04}), aunque sería equivalente distinguirlos por sus acoplamientos. Como antes, este modelo podría representar vibraciones de cristales reales en direcciones particulares. Las ecuaciones del movimiento, admitiendo interacción sólo a vecinos más próximos y una única constante de fuerza $C$ son 

\begin{equation*}
	M_1 \derivadas{^2 u_s}{t^2} = C(v_s - u_s) + C(v_{s-1}-u_s)
\end{equation*}
\begin{equation}
	M_2 \derivadas{^2 v_s}{t^2} = C(u_{s+1} - v_s) + C(u_{s}-v_s) \label{Ec:04-02-01}
\end{equation}

\begin{figure}[h!] \centering
    \includegraphics[scale=0.4]{Cuerpo/Ch_04/Fotos libro 4.pdf}
    \caption{Parámetro de red y posiciones de átomos 1 (masa $M_1$) y 2 (masa $M_2$) conectados por una fuerza de constante $C$ entre planos adyacentes. Los desplazamientos de los átomos 1 se designan por $u$ y los átomos 2 por $v$.}
    \label{Fig:04-04}
\end{figure}    

Una solución en modos normales, permitiendo amplitudes distintas para $M_1$ y para $M_2$ es 

\begin{equation}
	u_s = u_k e^{i(ksa-\omega  t)} \tquad v_s = v_k e^{i(ksa-\omega t)} \label{Ec:04-02-02}
\end{equation}
Al sustituir (\ref{Ec:04-02-01}) y (\ref{Ec:04-02-02}) se encuentra 

\begin{equation}
    -M_1 \omega^2 u_k = C(v_k - u_k) + C(v_k e^{-ika}-u_k)
\end{equation}
\begin{equation*}
    -M_2 \omega^2 v_k = C(u_k e^{ika} - b_k) + C(u_k - v_k)    
\end{equation*}
Para que exista una solución en $u_k,v_k$ debe verificarse que

\begin{eqnarray}
    \begin{vmatrix}
        2C - M_1\omega^2 & -C(1+e^{-ika})\\
        -C (1+e^{ika}) & 2C - M_2 \omega^2  
    \end{vmatrix}= 0 \Rightarrow \\
    \omega^2 = \frac{C(M_1+M_2)}{M_1M_2} \ccorchetes{1 \pm \sqrt{1-\frac{4M_1M_2}{(M_1+M_2)^2} \sen \parentesis{  ka/2}}}
\end{eqnarray}
El mayor cambio respecto a la cadenal ineal monoatómica es que ahora existen \textit{dos modos} ($\pm$) para cada valor de $k$, como se ilustra gráficamente en la imagen \ref{Fig:04-05}. Una de las soluciones tiene las características de una \textit{rama acústica}; la otra, por corresponder, como veremos, a frecuencias próximas a las ópticas, se denomia \textit{rama óptica.}


\begin{figure}[h!] \centering
    \includegraphics[scale=0.43]{Cuerpo/Ch_04/Fotos libro 5.pdf}
    \caption{Representación de la relación de dipsersión de una cadena diatómica de átomos de masas $M_1$ y $M_2$ y constante de acoplamiento $C$. Observar que para la rama óptica $\omega \rightarrow \text{cte} \neq 0$ cuando $k\rightarrow 0$.}
    \label{Fig:04-05}
\end{figure}    

Es interesante excitar el tipo de movimiento atómico asociado a cada una de las ramas. Esto es especialmente sencillo si nos situamos en $ka\ll 1$, pues entonces las ecuaciones (\ref{Ec:04-02-01}) y (\ref{Ec:04-02-02}) dan $u_k/v_k=1$ para la \textit{rama acústica} y $u_k/v_k = -M_2/M_1$ para la \textit{rama óptica}. Esto quiere decir que el modo acústico los átomso de la celda se mueven en fase, es decir, la celda vibra como un todo y el movimiento es sobre todo \textit{intercelda}. En cambio, en el modo óptico los átomos se mueven en oposición de fase, de modo que el centro de masas está inmóvil y el movimiento es \textit{intracelda}. Estas características además no dependen de la aproximación $ka\ll 1$. 


La figura \ref{Fig:04-06} ilustra la esencia del movimiento de los modos ótpicos y acústicos en el caso de un cristal diatómico. 

\begin{figure}[h!] \centering
    \includegraphics[scale=0.42]{Cuerpo/Ch_04/Fotos libro 6.pdf}
    \caption{Modos óptico y acústica en una cadena diatómica. El desplazamiento atómico respecto a la posición de equilibrio se representa verticalmente.}
    \label{Fig:04-06}
\end{figure}    

Obsérvese que si los dos átomos fueran iones de signo opuesto, como en un cristal iónico, cabría esperar que los modos ópticos fueran excitados por un campo eléctrico de la frecuencia adecuada. Veremos en la sección \ref{Sec:04-04} que, en efecto, existe una fuerte interacción de estos modos con las \textit{ondas e.m. infrarrojas}. 

Nótese finalmente, que si la cadena tuviese una base formada por $z$ átomos habría una \textit{rama acústica} y $z-1$ \textit{ramas ópticas}.

\subsection{Cristales tridimensionales poliatómicos}

Como generalización (sin demostración) natural de lo que precede, se tiene que en un crital con base de $z$ átomos existen, para cada valor de $k$, $3z$ modos normales. Estos modos se pueen agrupar en $3z$ ramas, de las que $3$ son acústicas y $3z-1$ son ótpticas. El factor 3 está asociado a las tres pola rizacioens posibles y se habla de modos LA, TA, LO, TO. La figura \ref{Fig:04-07} muestra espectros de vibración de algunos sólidos simples. El punto importante es que las frecuencias características de vibración son del orden de $10^{12}$ Hz que corresponden al \textit{infrarrojo}.

\begin{figure}[h!] \centering
    \includegraphics[scale=0.42]{Cuerpo/Ch_04/Fotos libro 7.pdf}
    \caption{Ejemplos de relación de dispersión experiental $\omega (\kn)$. En el eje horizontal se representa $\kn/\kn_{\max}$ para distintas direcciones, y en los ejes verticales la frecuencia $\omega/2\pi$ en undiades de $10^{12}$ Hz.}
    \label{Fig:04-07}
\end{figure}    

\section{Fonones}

\subsection{Cuantización de las ondas elásticas}

Para introducir el concepto de \textbf{fonón} vamos a ver una manera más formal de obtener la relación de dispersión dada por la ecuación \ref{Ec:04-01-07} haciendo uso de las \textit{variables normales}. En concreto, si se define 

\begin{eqnarray}
    \xi (k,t) & \equiv & \frac{1}{\sqrt{N}} \sum_{s=1}^N u_s (t) e^{-iksa} \\
    \Pi (k,t) & \equiv & \frac{1}{\sqrt{N}} \sum_{s=1}^N p_s (t) e^{iksa} \ \ccorchetes{\text{con} \ p_s (t) = M \parciales{u_s(t)}{t}}
\end{eqnarray}
multiplicando la ecuación del movimiento (\ref{Ec:04-01-02}) por $e^{iksa}$ y sumando a todos los valores de $s$, se encuentra que

\begin{eqnarray}
    M \parciales{}{t} \xi (k,t) & = & \Pi (k,t) \\
    \parciales{}{t} \Pi (k,t) & =  & - M \omega^2 (k) \xi (k,t)    
\end{eqnarray}c
con $\omega^2 (k) = \frac{2}{M} \sum_l C_l \sen^2 ( ka/2)$, que a primeros vecinos ($l=\pm 1$) se reduce a la ecuación \ref{Ec:04-01-07}.

Las funciones $\xi$ y $\Pi$ responden a las ecuaciones canónicas de un oscilador de masa $M$ y frecuencia $\omega$. Las soluciones ($\eta,\Pi$) se denominan \textit{modos normales} o propios del cristal. Cualquier movimiento de los átomos se pueden describir como suma de \textit{modos normales independientes} según

\begin{equation}
    u_s (t) = \frac{1}{\sqrt{N}} \sum_{k \in PZB} \xi (k,t) e^{iksa} \approx \frac{L}{2\pi \sqrt{N}} \int_{-\pi/a}^{\pi/a} \xi (k,t) e^{iksa} \D k
\end{equation}
Por tanto, un modo normal es una \textit{excitación colectiva} (involucra a todos los átoos) caracterizada por un vector de onda $\kn$ y una frecuencia $\omega$. 

La energía de cada uno de los modos posibles de vibración de un cristal está discretizada, siendo sus valores posibles los conocidos por el oscilador armónico, es decir,

\begin{equation}
    \epsilon (\kn) = \ccorchetes{n(\kn) + \frac{1}{2}} \hbar \omega (\kn)
\end{equation}
donde $n(\kn)=0,1,2,...$ es el llamado \textbf{número de ocupación}. La energía \textit{mínima} $\frac{1}{2} \hbar \omega (\kn)$ es la \textit{energía del punto cero} de cada modo. En el caso más general:
 
\begin{equation}
    \epsilon_p (\kn) = \ccorchetes{n_p(\kn) + \frac{1}{2}} \hbar \omega_p ( \kn)
\end{equation}
es decir, cada modo normal está caracterizado, además de por el vector de onda y la frecuencia, por el índice de rama, $p$. Recuérdese que para un cristal con base de $z$ átomos el número de modos por cada $\kn$ es $3z$.  

El \textit{cuanto} de energía de vibración $\hbar \omega_p (\kn)$ se denomina \textbf{fonón} y es del orden de 4 meV peus $\omega/2\pi\approx 10^{12} $ Hz.

El número de ocupación es una medida de \textit{cuantos fonones} de un determinado tipo hay excitados. En la \textit{aproximación semiclásica} ($n\gg 1$) se recupera el sentido de la amplitud de oscilación $u_k$ de un modo de vector de onda $\kn$ (de la rama $p$) y entonces se verifica 

\begin{equation}
    \frac{1}{2} N M \omega_p^2 (\kn) u_p^2 (\kn) = n_p (\kn) \hbar \omega_p (\kn)
\end{equation}
Desde el punto de vista clásico, más o menos energía se traduce en mayor o menor amplitud de oscilación, mientras que el punto de vista cuántico se traduce en mayor o menor número de ocupación. \\

No es difícil verificar que \textit{un modo normal no transporta impulso neto} (centro de masas del cristal inmóvil) , por lo que no cabe decir que $\hbar \kn$ es el impulso del fonón, pues entonces el impulso total del modo debería de ser $n \hbar \kn$. A pesar de esto $\hbar \kn$ del fonón verifica, en su interacción con otras partículas, \textit{leyes de conservación} similares a las del impulso habitual, como vamos a ver.

\subsection{Espectroscopía de fonones}

Supóngase un cristal monoatómico sobre le que incide una onda (neutrones, rayos x, etc.) de vector de onda $\Kn$ y de frecuencia $\Omega$. La amplitud dispersada, en la dirección $\Kn'$, por los átomos en posiciones de equilibrio $[\rn_j (t)= \Rn_j]$ es, como sabemos

\begin{equation}
    A_{\text{salida}} \propto e^{-i \Omega t } \sum_j e^{-i \Delta \Kn \cdot \rn_j} \label{Ec:04-03-09}
\end{equation}
donde $\Delta \Kn = \Kn ' - \Kn$. Supongamos ahora que existe un modo normal establecido en el cristal, de modo que las posiciones atómicas vienen dadas por $\rn_j(t) = \Rn_j + \un_j (t)$ donde $\un=\un_0  e^{\pm i \ccorchetes{\kn \Rn_j -  \omega (\kn)t}}$ . Al sustituir en (\ref{Ec:04-03-09}) se obtiene, en aproximación armónica ($u \ll R$) 

\begin{equation}
    A_{\text{salida}} \propto e^{-i\Omega t} \sum_j e^{-i\Delta \Kn \cdot \Rn_j} - i \Delta \Kn \cdot \un_0 \sum_j e^{- i (\Delta \Kn \pm \kn) \cdot \Rn_j} e^{- i \ccorchetes{\Omega \pm \omega (\kn)}t}
\end{equation}
de forma que, además de la dispersión en elástica (1\er término), tendremos \textit{dispersión inelástica} según las condiciones 

\begin{equation}
    \Omega' = \Omega \pm \omega (\kn) \tquad \Kn' = \Kn \pm \kn + \Gn
\end{equation}
Por estas relaciones de (cuasi)conservación $\hbar \kn$ recibe el nombre de \textit{cuasiimpulso}. Irradiando un cristal y examinando la radiación dispersada inelásticamente, según estas relaciones es posible conocer el espectro fonónico de un cristal $\omega (\kn)$. Entre las radiaciones más utilizadas están:

\begin{itemize}
    \item Radiación electromagnética en el rango del \textbf{\textit{infrarrojo}}: cubre $10\le k \le 10^4 \unit{cm^{-1}}$ y $10^{12} \leq \nu \leq 10^{14}$ Hz (4 meV $\leq \epsilon \leq 1$ eV). Aunque su energía es similar a la de los fonones, no lo es $\hbar \kn$ pues $k_{PZB} \approx 10^8  \unit{\cm^{-1}}$. Se utiliza por ejemplo en análisis químicos para identificar grupos funcionales dentro de las moléculas (\textit{modos ópticos}).
    \item Radiación electromagnética en el rango de la \textbf{\textit{luz visible}}: con $k\approx 10^5 \unit{\cm^{-1}}$ ($2\leq \epsilon \leq 4$ eV) se utiliza para la dispersión inelástica (\textit{espectroscopía Raman}). Aunque los cambios de energía de los fotones son muy pequeños, se pueden determinar por técnicas interferométricas. La \textit{espectroscopía Ramen} sirve para estudiar la dinámica de los electrones de conducción en metales, la naturaleza de los cristales, etc.
    \item \textbf{\textit{Neutrones}}: constituyen la sonda ideal por cuanto los \textit{neutrones térmicos} tienen tanto el ector de onda como la energía comparables a los de los fonones y de hecho la mayoría de las curvas de dispersión de fonoes en sólidos se han obtenido empleando neutrones.
\end{itemize}


El espectro fonónico $\omega (\kn)$ permite obtener información muy valiosa sobre las interacciones entre átomos (alcance, intensidad, etc.) en los sólidos. Como ilustración considérse vibracioens unidimensionales incluyendo interacción de un plano con los $l$ más próximos. Como ya se ha visto la relación de dispersión es entonces
\begin{equation}
    \omega^2 (k) = \frac{4}{M} \sum_{l>0} C_l \sin^2 \parentesis{\frac{1}{2} k la}
\end{equation}
multiplicando ahora por $\cos (ksa)$ e integrando (en la $PZB$) se encuentra

\begin{equation}
    C_s = \frac{-Ma}{2\pi} \int_{-\pi/a}^{\pi/a} \omega^2 (k) \cos (ska) \D k
\end{equation}
que nos da la constante de fuerza entre cualesqueira planos atómicos a partir de $\omega (k)$.

\section{Vibraciones de los cristales iónicos} \label{Sec:04-04}
\begin{figure}[h!] \centering
    \includegraphics[scale=0.5]{Cuerpo/Ch_04/Fotos libro 8.pdf}
    \caption{(a) Dependencia con la frecuencia de la permitividad eléctrica relativa en cristales iónicos. (b) reflectividad de algunos cristales iónicos para longitudes de odna en el rango infrarrojo.}
    \label{Fig:04-08}
\end{figure}    
