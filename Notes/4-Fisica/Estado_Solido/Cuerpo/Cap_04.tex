
\chapter{Dinámica de redes} \label{Ch:04}

En este capítulo se comienza el estudio de las vibraciones de los átomos alrededor de sus posiciones de equilibrio y sus efectos observables. Esta llamada \textit{dinámica de redes} es necesaria para explicar propiedaes como: i) la conductividad térmica de los aislatnes, ii) la dependencia en $T^3$ del calor específico a baja temperatura, iii) las energías de cohesión, iv) la dilatación térmica, v) la conductividad eléctrica \textit{finita} de los metales, vi) la reflectividad de los cristales iónicos, etc.