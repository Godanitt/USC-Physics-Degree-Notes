\chapter{Gas de fermi de electrones libres} \label{Ch:06}

En este capítulo se comienza el estudio de los metales con un primer modelo en el que los electrones de valencia de los átomos del metal se \textit{independizan} cosntituyéndose en electrones de conducción que se mueven de una forma casi completamente libre a través del metal. De manera más precisa se supone que la red de iones positivos en el metal está inmóvil (red \textit{fría}) y además se sustituye por un fondo positivo de carga (a veces llamado \textit{modelo jalea}) de modo que el potencial eléctrico a que están sometidos los electrones de conducción es una constante que puede tomarse como cero. Se admite además que los electrones no interaccionan entre sí, pero debido a su carácter fermiónico les aplicaremos el Principio de Exclusión de Pauli. Hablaremos entonces de \textit{Gas de Fermi de elctrones libres}.

\section{Estados fundamentales del gas de Fermi}

\subsection{Niveles de energía}


\begin{figure}[h!] \centering
    \includegraphics[scale=0.5]{Cuerpo/Ch_06/Fotos libro 1.pdf}
    \caption{Distribución de estados electrónicos ocupados en el espacio de fases.}
    \label{Fig:06-01}
\end{figure}    


\begin{figure}[h!] \centering
    \includegraphics[scale=0.5]{Cuerpo/Ch_06/Fotos libro 2.pdf}
    \caption{Cálculo de la densidad de estados electrónicos.}
    \label{Fig:06-02}
\end{figure}    



\subsection{Ocupación de estados a $T>0$}

\begin{figure}[h!] \centering
    \includegraphics[scale=0.5]{Cuerpo/Ch_06/Fotos libro 3.pdf}
    \caption{Distribución de Fermi-Dirac.}
    \label{Fig:06-03}
\end{figure}    

\subsection{Interacción electrón-electrón}

\begin{figure}[h!] \centering
    \includegraphics[scale=0.5]{Cuerpo/Ch_06/Fotos libro 4.pdf}
    \caption{Restricción a los procesos de colisión $e^- - e^-$ debido a las leyes de conservación de la energía (a) y del momento (b).}
    \label{Fig:06-04}
\end{figure}  

\section{Capacidad térmica electrónica}

\section{Conductividad eléctricca DC}

\subsection{Modelo cinético de Drude y ecuación dinámica}

\subsection{Ley de Ohm}
\begin{figure}[h!] \centering
    \includegraphics[scale=0.5]{Cuerpo/Ch_06/Fotos libro 5.pdf}
    \caption{Desplazamiento de la ``esfera de Fermi'' bajo la aplicación de un campo eléctrico. Las líneas indican algunos procesos de colisión permitidos.}
    \label{Fig:06-05}
\end{figure}  

\subsection{Dependencia con la temperatura de la conductividad eléctrica}
\begin{figure}[h!] \centering
    \includegraphics[scale=0.5]{Cuerpo/Ch_06/Fotos libro 6.pdf}
    \caption{Comprobación de la regla de Mathiessen con la resistividad de aleaciones de Pb-In. Al aumentar $x$ el desorden de la aleación y por tanto la contribución constante $\rho_{\text{def}}$ frente a $\rho_{\text{fon}} (T)$ que casi no cambia (nótese que la pendiente no varía). Es interesante que estas aleaciones son superconductores por debajo de $\sim 7$ K (veáse Capítulo \ref{Ch:11}).}
    \label{Fig:06-06}
\end{figure}  

\section{Conductivdad térmica electrónica}

\section{Ley de Wiedmann-Franz}

\section{Efecto Hall y magnetorresistividad}
\begin{figure}[h!] \centering
    \includegraphics[scale=0.5]{Cuerpo/Ch_06/Fotos libro 7.pdf}
    \caption{Configuración experimental para comprobar el efecto Hall.}
    \label{Fig:06-07}
\end{figure}  


\section{Conductividad AC y propiedades ópticas}
