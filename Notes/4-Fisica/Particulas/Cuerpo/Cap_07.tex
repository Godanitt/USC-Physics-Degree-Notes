\chapter{Interacción electrodébil}

\section{La violación de Paridad}

La ecuación de Dirac como ya hemos visto: 

\begin{equation}
    (\slashed{p} - m ) \psi = 0
\end{equation} 
Tiene soluciones libres bien conocidas por todos: 

\begin{equation}
    \psi(x) = u(p) e^{-ipx} \quad  \quad \psi(x) = v(p) e^{ipx}
\end{equation}
donde $u(p)$ y $v(p)$ son las funciones de onda de Dirac. En general tenemos varias soluciones en función del sistema de referencia en el que nos encontremos. Tenemos en total 4 soluciones: 2 para electrones (1 con espin +1/2 otra con espín -1/2) y 2 para positrones (1 con espin +1/2 y otra con espin -1/2): 

\begin{equation*}
    u^\uparrow = \mqty(\uparrow \\ 0)  \qquad 
    u^\downarrow = \mqty(\downarrow \\ 0)  \qquad 
    v^\uparrow = \mqty( 0 \\ \downarrow) \qquad 
    v^\downarrow = \mqty(0 \\ \uparrow)   
\end{equation*}
donde 

\begin{equation*}
    \uparrow = \mqty(1 \\ 0) \qquad 
    \downarrow = \mqty(0 \\ 1)
\end{equation*}
La solución más general de dicha ecuación es para un estado de partícula con un momento $p$ y energía $E=\sqrt{p^2 + m^2}$:

\begin{Resaltar}
    \begin{center}
    \textbf{Solución de Dirac más general:}
    \end{center}
    \begin{equation*}
        u^\uparrow(p) = N  \mqty(\uparrow \\ \frac{p}{E+m} \uparrow)  \qquad 
        u^\uparrow(p) = N  \mqty(\downarrow \\ -\frac{p}{E+m} \downarrow)  \qquad 
        V^\uparrow(p) = N  \mqty(-\frac{p}{E+m}\downarrow \\ \downarrow)  \qquad 
        V^\downarrow(p) = N  \mqty(\frac{p}{E+m}\uparrow \\ \uparrow)  \qquad 
    \end{equation*}
\end{Resaltar}
