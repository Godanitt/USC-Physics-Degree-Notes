\chapter{Electrodinámica Cuántica}

\section{La ecuación de Dirac}

En el mundo real, los quarks y los leptones llevan espín 1/2, y los bosones gauge 1. En un principio la mecánica cuántica que describre las partículas estaba definida por la \textbf{ecuación de Schorödinger}:

\begin{equation}
    - \frac{\hbar^2}{2 m} \nabla^2 \psi(\vec{r},t) + V(\vec{r}) \psi(\vec{r},t) = i \hbar \frac{\partial}{\partial t} \psi(\vec{r},t)
\end{equation}
siendo $\psi$ la función de onda. En la mecánica cuántica relativista las partículas de espín cero están descritas por la ecuación de Klein-Gordon, que sigue los pasos de la ecuación de Schrödinger pero para la relación de energía-momento relativita:


\begin{equation*}
    p^\mu \cdot p_\mu - m^2 c^2 = 0
\end{equation*}
llegando a 

\begin{equation*}
    \parentesis{-\frac{1}{c^2} \parciales{^2}{t^2} + \nabla^2} \phi = \frac{m^2 c^2}{\hbar^2} \phi
\end{equation*}
siendo $\phi$ un escalar relativista. La ecuación de Klein-Gordon es una ecuación de segundo orden en el tiempo, lo que implica que la función de onda $\phi$ no es única, sino que tiene dos soluciones independientes, además que habrá qeu especificar dos valores iniciales: el de la función de onda y el de la derivada. Ambos pueden ser ``arbitrarios'', por lo que no puede mantener su rol de ``determinar la densidad de probabilidad''. 

Como sabemos la \textbf{vector corriente densidad de probabilidad} en Schrödinger es:

\begin{equation*}
    \Jn = - \frac{i\hbar}{2m} \parentesis{\psi^* \nabla \psi - \psi \nabla \psi^*}
\end{equation*}
y la conservación de la probabilidad se expresa como 

\begin{equation*}
    \nabla \Jn + \parciales{\rho}{t} = 0
\end{equation*}
siendo $\rho$ positivo. En el caso de Klein-Gordon: 


\begin{equation*}
    \parciales{\rho}{t} = \frac{i\hbar}{2m} \parentesis{\phi^* \partial_t \phi - \phi \partial_t \phi^*}
\end{equation*}
los valores iniciales de $\phi$ y $\partial_t \phi$ pueden serescogidos libremente, por lo que la \textit{densidad peude ser negativa}, y por tanto no es un candidato a ser densdiad de probabilidad. 


Para describir las partículas de espín 1/2 como los electrones es necesario construir una función de onda relativista. Dirac resolvió este problema constuyendo una serie de matrices que transforman como un 4-vector: las \textbf{matrices de Dirac}. Estas matrices se constuyeron para que verificar la regla de anticonmutaciónLa acción de Dirac es invariante bajo transformaciones de Lorentz, pero no lo es bajo transformaciones de gauge.

\begin{equation*}
   \{ \gamma^\mu ,\gamma^\nu \} = \gamma^\mu \gamma^\nu + \gamma^\nu \gamma^\mu = 2 g^{\mu\nu} \mathbb{I}
\end{equation*}
donde $g^{\mu\nu}$ es el tensor métrico de Minkowski. La ecuación de Dirac se obtiene al aplicar la relación de energía-momento relativista a una función de onda $\psi$ que es un \textbf{espinor}:

\begin{equation*}
    H \psi = \parentesis{\alphan \cdot \Pn + \beta m } \psi
\end{equation*}
siendo $\Pn$ el opedador momento relativista ($\Pn = -i \hbar \nabla$) y $H$ el operador hamiltoniano relativista $H=i \hbar \partial_t$, $\alpha^i$ y $\beta$ matrices de Dirac, y $m$ la masa de la partícula. Estas matrices tienen la particularidad de que al elevar al cuadrado la ecuación de Dirac obtenemos 

\begin{equation*}
    H^2 \psi = \parentesis{\Pn^2 + m^2} \psi 
\end{equation*}
que es evidentemente la ecuación de la expresión de energía relativista. Las relaciones de Dirac exigen que:

\begin{itemize}
    \item Las matrices $\alpha^i$ y $\beta$ son matrices complejas que conmutan entre ellas. 
    \item El cuadrado de las matrices es la identidad. 
\end{itemize}
Estas relaciones de no-conmutación llevan a la directa deducción de que no peuden ser numeros: deben ser matrices, y por tanto la función de onda $\psi$ no es un escalar, sino un \textbf{espinor} de Dirac. 

Puede ser dmoestrado qeu $\alpha_i$ y $\beta$ son matrices hermíticas con traza nula de dimensión par con valores $\pm 1$. EL orden más pequeño que satisface estos requisitos exige una dimensión 4 $\times$ 4, y la elección de las matrices no es unívoco. La eleccción más frecuente es la representación de Dirac-Pauli:

\begin{equation}
    \alpha^i = \begin{pmatrix}
        0 & \sigma^i \\
        \sigma^i & 0
    \end{pmatrix} , \quad
    \beta = \begin{pmatrix}
        I & 0 \\
        0 & -I
    \end{pmatrix}
\end{equation}
siendo $I$ la matriz identidad y $\sigma^i$ las matrices de Pauli. Si multiplicamos la ecuación de dirac por $\beta$ obtenemos la forma más común de representar la ecuación de Dirac.

\begin{Resaltar}
    \begin{center}
        \textbf{Ecuación de Dirac en su forma covariante}
    \end{center}
    \begin{equation}
        \parentesis{i \gamma^\mu \partial_\mu - m } \psi = 0
    \end{equation}
\end{Resaltar}
donde $\gamma^\mu \equiv (\beta, \beta \alphan)$ .

\subsection{Corriente de densidad de probabilidad}


El \textbf{adjunto} de una función de ondas $\adjunto$ se define como 

\begin{equation}
    \adjunto \equiv \psi^\dagger \gamma^0
\end{equation}
puediendo expresar la ecuación de Dirac a partir de este adjunto. Por otro lado, la corriente de proabilidad es: 

\begin{equation}    
    j^\mu = \adjunto \gamma^\mu \psi
\end{equation}
y satisface la ecuación de continuidad donde $\jn^\mu=(\rho,\Jn)$. De hecho se puede identificar a $j^\mu$ por la \textbf{corriente de densidad de carga} cuando multiplicamos $j^\mu$ por la carga de la partícula. 


\section{Soluciones de la ecuación de Dirac: espinores.}
Las soluciones libres bien conocidas por todos: 

\begin{equation}
    \psi(x) = u(p) e^{-ipx} \quad  \quad \psi(x) = v(p) e^{ipx}
\end{equation}
donde $u(p)$ y $v(p)$ son las funciones de onda de Dirac. En general tenemos varias soluciones en función del sistema de referencia en el que nos encontremos. Tenemos en total 4 soluciones: 2 para electrones (1 con espin +1/2 otra con espín -1/2) y 2 para positrones (1 con espin +1/2 y otra con espin -1/2): 

\begin{equation*}
    u^\uparrow = \mqty(\uparrow \\ 0)  \qquad 
    u^\downarrow = \mqty(\downarrow \\ 0)  \qquad 
    v^\uparrow = \mqty( 0 \\ \downarrow) \qquad 
    v^\downarrow = \mqty(0 \\ \uparrow)   
\end{equation*}
donde 

\begin{equation*}
    \uparrow = \mqty(1 \\ 0) \qquad 
    \downarrow = \mqty(0 \\ 1)
\end{equation*}
La solución más general de dicha ecuación es para un estado de partícula con un momento $p$ y energía $E=\sqrt{p^2 + m^2}$:

\begin{Resaltar}
    \begin{center}
    \textbf{Solución de Dirac más general:}
    \end{center}
    \begin{equation*}
        u^\uparrow(p) = N  \mqty(\uparrow \\ \frac{p}{E+m} \uparrow)  \qquad 
        u^\uparrow(p) = N  \mqty(\downarrow \\ -\frac{p}{E+m} \downarrow)  \qquad 
        v^\uparrow(p) = N  \mqty(-\frac{p}{E+m}\downarrow \\ \downarrow)  \qquad 
        v^\downarrow(p) = N  \mqty(\frac{p}{E+m}\uparrow \\ \uparrow)  \qquad 
    \end{equation*}
\end{Resaltar}


\section{Acoplamiento del campo electromagnético}

La ecuación de Dirac se acopla al campo electromagnética de un modo tipico en la mecánica cuántica: introduciendo un el vector potencial electromangético en las derivadas. Esto implica que pasamso de la derivada parcial covariante:

\begin{equation}
    \partial_\mu \to D_\mu = \partial_\mu - i q A_\mu
\end{equation}
de tal modo que la ecuación de Dirac se convierte:

\begin{equation}
    \parentesis{i \gamma^\mu (\partial_\mu + i e A_\mu) - m} \psi = 0
\end{equation}
donde $A_\mu=\parentesis{\phi/c, \An}$ siendo $\phi$ el potencial escalar eléctrico y $\An$ el potencial vectorial magnético. 

\section{Covariantes bilineares}

Para construir los objetos-corrientes covariantes Lorentz (con objetos-corrientes nos refererimso a magnitudes físicas, véase densdiad de corriente eléctrica, energía...) se constuyen como bilineares, tal qeu
\begin{equation}
    \adjunto \Gamma \psi
\end{equation}
siendo $\Gamma$ uno de los siguientes objetos $\Gamma \in (1,\gamma^5, \gamma^\mu, \gamma^\mu \gamma^5, \sigma^{\mu\nu})$ siendo $\gamma^5 = i \gamma^0 \gamma^1 \gamma^2 \gamma^3$ y $\sigma^{\mu\nu} = \frac{i}{2} [\gamma^\mu, \gamma^\nu]$. En función de cual usameos podremos representar:

\begin{itemize}
    \item Escalar $\Gamma=1$.
    \item Escalar-Axial $\Gamma=\gamma^5$.
    \item Vector $\Gamma=\gamma^\mu$.
    \item Vector axial $\Gamma=\gamma^\mu \gamma^5$.
    \item Tensor $\Gamma=\sigma^{\mu\nu}$.
\end{itemize}

\section{Lagrangiano de Dirac y QED}

El lagrangiano de Dirac, que es capaz de obtener las ecuaciones de movimiento de Dirac (i.e. la ecuación de Dirac a través de las ecuaciones de Euler-Lagrange).

\begin{Resaltar}
    \begin{center}
        \textbf{Lagrangiano de Dirac}
    \end{center}
    \begin{equation}
        \mathcal{L} = \adjunto \parentesis{i \gamma^\mu \partial_\mu - m} \psi
    \end{equation}
\end{Resaltar}
Incluyendo el acoplamiento al campo electromagnético $\Lcal_{\text{Maxwell}}=-\frac{1}{4} F_{\mu\nu} F^{\mu\nu}$, donde $F_{\mu\nu} = \partial_\mu A_\nu - \partial_\nu A_\mu$ es el tensor de campo electromagnético y la derivada covariante para el acoplamiento al campo electromagnético $D_\mu = \partial_\mu - i q A_\mu$ obtenemos $\Lcal_{\text{QED}}$. 
\begin{Resaltar}
    \begin{center}
        \textbf{Lagrangiano de QED}
    \end{center}
    \begin{equation}
        \mathcal{L} = - \frac{1}{4} F_{\mu \nu} F^{\mu \nu} + \adjunto \parentesis{i \gamma^\mu (\partial_\mu + i q A_\mu) - m} \psi
    \end{equation}
\end{Resaltar}
Vemos claramente el significado de este lagrangiano: la primera parte describe la dinámica del campo elecromangético, la segunda la dinámica de los electrones, y además tendremos la interacción entre ambos controlado por $\adjunto \gamma^\mu A_\mu \psi$. Las ecuaciones de Mawell relativistas 

\begin{equation}
    \partial_\mu F^{\mu\nu}  = j^\nu  \qquad \partial_\mu \bar{F}^{\mu\nu} = 0
\end{equation}
donde $\bar{F}^{\mu\nu} = \frac{1}{2} \epsilon^{\mu\nu\rho\sigma} F_{\rho\sigma}$ es el dual del tensor electromagnético. Las ecuaciones de Maxwell clásicas no son invairantes bajo transformaciones de Lorentz, pero si las ecuaciones de maxwell covariantes. 

\section{Interacción entre electrones y fotones}
La interacción entre electrones y fotones se describe a través de la \textbf{acción de Dirac}:
\begin{equation}
    S = \int d^4 x \mathcal{L} = \int d^4 x \adjunto \parentesis{i \gamma^\mu (\partial_\mu + i q A_\mu) - m} \psi - \frac{1}{4} F_{\mu\nu} F^{\mu\nu}
\end{equation}
donde la acción es la integral del lagrangiano en el tiempo y el espacio. La acción es un escalar relativista, y por tanto debe ser invariante bajo transformaciones de Lorentz. 