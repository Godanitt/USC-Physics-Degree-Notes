
\newpage
\section*{\textit{Exámenes}}
\addcontentsline{toc}{section}{\textit{Exámenes}}

\subsection*{\textit{Examen 2024}}

\begin{Enunciado}

	\subsubsection*{Pregunta 1}

	¿Se conserva la paridad en la Electrodinámica Cuántica? ¿Y en la interacción fuerte descrita por la Cromodinámica Cuántica? ¿Y en la gravitación cuántica?

\end{Enunciado}

La paridad en la electrodinámica cuántica se conserva. En la interacción fuerte y en la gravitación cuántica también, solo se viola por la fuerza electrodébil. Esto se ve claramente en el carcter de las corrientes $J^{\mu}$ de todas las fuerzas, siendo la débil la única con un término $\frac{1}{\sqrt{2}}(1+\gamma^5)$.


\vspace*{2em}

\begin{Enunciado}
	\subsubsection*{Pregunta 2}

	¿Coinciden los valores de $G_F$ medidos en la desintegración $\beta$ con los medidos en la desintegración del muón? ¿Qué factor origina la diferencia? ¿Juega algún papel que los quarks tienen masa? Indica sólo la idea.

\end{Enunciado}

No coinciden, ya que $G_F^\beta = \SI{1.136e-5}{GeV}$ y $G_F^\mu = \SI{1.1663788e-5}{GeV}$. La diferencia fundamental está en el ángulo de cabbibo relacionado con las propiedades de los quarks, en particular con su masa (si no tuvieran masa el ángulo de cabbibo sería cero y coincidirían).

\vspace*{2em}

\begin{Enunciado}
	\subsubsection*{Pregunta 3}

	¿Cuál es la orientación preferente del espín del electrón emitido en la desintegración del $^{60}$Co (en reposo) respecto a su momento? ¿Y la del positrón en la emisión del $^{14}$O? Explica por qué.

\end{Enunciado}

La orientación del espín del electrón será preferentemente hacia el sentido contrario qeu su momento, y la del positrón a favor de su momento. La interacción débil viola paridad totalmente, de tal modo que solo deja electrones con quiralidad a izquerdas y positrones con quiralidad a derechas. La consecuencia es que en el caso ultrarrelativista el electrón tendrá helicidad -1 y el positrón helicidad +1 (i.e., el espín y el momento serán antiparalelos y espín y el momento serán paralelos). Sin embargo al no ser una partícula sin masa, la helicidad y la quiralidad no se igualarán, por lo que en realidad es posible encontrar un electrón con helicidad +1 y un positrón con helicidad -1, aunque será mucho menos probable.

\vspace*{2em}

\begin{Enunciado}
	\subsubsection*{Pregunta 4}

	¿Era necesaria una teoría relativista para comprender la desintegración $\beta$, siendo como son muy bajas las velocidades del protón y del neutrón? ¿Es relativista la teoría de Fermi? ¿Es cierta la teoría de Fermi? Explica lo esencial.

\end{Enunciado}

Aunque las velocidades sean bajas, la teoría relativista es fundamental ya que si no no se entendería el origen de $\gamma^\mu$ (que proceden de una teoría cuántica relativista). La teoría de Fermi es relativista, aunque no es cierta, pues no es capaz de predecir el carácter quiral a izquierdas de las desintegraciones beta (aunque si el tiempo de vida).

\vspace*{2em}

\begin{Enunciado}
	\subsubsection*{Pregunta 5}

	Di si es más probable la desintegración del $\pi^-$ en electrón que en muón, y explica por qué con un dibujo. ¿Sabes el factor aproximado?

\end{Enunciado}

La desintegración del $\pi^-$ en el muón mucho más probable (en un orden de 1.3$\times 10^{-4}$), debido a que la masa del muón $m_\mu\approx \SI{105.7}{MeV}$ y la del electrón $m_e\approx \SI{511}{keV}$. El $\pi^-$ al ser una partícula con espín 0 y emitir un neutrino a derechas $\bar{\mu}_R$, hace que la conservación del momento-momento angular  obligue al lepon a tener una helicidad positiva. La helicidad se iguala a la quiralidad en el caso ultrarrelativista, caso que se se da de manera mucho más intensa en el electrón, y como un electrón no puede tener quiralidad a derechas en una interacción débil, prácticamente no se observan.


\vspace*{2em}

\begin{Enunciado}
	\subsubsection*{Pregunta 6}

	¿Qué masa tienen, y cuánto viven aproximadamente en su sistema en reposo, los piones cargados?

\end{Enunciado}

Los $\pi^{\pm}$ tienen una masa aproximada de $m_{\pi^{\pm}}=139.6$ MeV/c$^2$ y viven entorno a $\tau_{\pi^{\pm}}=$26.0 ns.

\vspace*{2em}

\begin{Enunciado}
	\subsubsection*{Pregunta 7}

	Con $r_p = 0.84\,\mathrm{fm}$, deduce del principio de indeterminación qué sabores de antiquarks deben estar presentes en el protón. ¿Cuánto midieron los experimentos de neutrinos para $\bar{q}/(q + \bar{q})$, y con qué energía aproximada del neutrino?

\end{Enunciado}

El principio de indeterminación:

\begin{equation}
	\Delta p \Delta x \sim \hbar
\end{equation}
nos dice que:

\begin{equation}
	\Delta p \approx \frac{\hbar}{r_p} \approx \SI{2.35e+2}{MeV/c}
\end{equation}
que es suficiente para generar pares de quark-antiquark de tipo $u,d,s$, ya que $m_u=1.5$ MeV/c$^2$, $m_u=4$ MeV/c$^2$, $m_s=95$ MeV/c$^2$ (recordamos que la energía necesaria para generar un par quark-antiquark es, como mínimo, $2m_{q}$ siendo $q=u,d,s,c,b,t$).

\vspace*{2em}

\begin{Enunciado}
	\subsubsection*{Pregunta 8}

	¿En qué consiste la variable $y$ de Bjorken en un experimento de neutrinos sobre materia hadrónica, y cómo se determina en el laboratorio? Explícalo por separado para corrientes cargadas y para corrientes neutras.

\end{Enunciado}

La variable $y$ de Bjorken se define como

\begin{equation}
	y = \frac{pq}{pk}
\end{equation}
donde $p$ es el momento inicial del protón/neutrón, $k$ sería el momento del neutrino inicial, $k'$ sería el momento del leptón creado $q=k-k'$ sería la diferencia de los momentos. Podemos reescribir la variable $y$ como

\begin{eqnarray}
	y = 1 - \frac{1}{2} (1+\cos (\theta))
\end{eqnarray}
siendo $\theta$ el ángulo entre el haz incidente y el leptón (muón) saliente.

Para medirlo en el laboratorio lo calcularemos en función del bosón que interactue:

\begin{itemize}
	\item En el caso de \textit{corriente cargada} basta con caracterizar bien el haz incidente (energía y dirección) y el muón (ángulo saliente respecto la dirección, energía) y el punto de colisión.
	\item En el caso de \textit{corriente neutra} también será importante conocer la masa/energía intercambiada con los hadrones $X$, además de las variables mencionadas para las corrientes cargadas.
\end{itemize}

\vspace*{2em}

\begin{Enunciado}
	\subsubsection*{Pregunta 9}

	¿Por qué el $\nu_\mu$ penetra más que el $\mu^-$, ambos con $E = 1\,\mathrm{GeV}$? El punto clave.

\end{Enunciado}

El punto clave y fundamental es básicamente que la sección eficaz del neutrino es mucho más pequeña que la del muón, además de que no interacciona por bremsstrahlung. Las interacciones con la materia nos hablan de secciones eficaces:

\begin{equation}
	\sigma_{\mu} = \SI{1}{nb} \qquad \sigma_{\nu_{\mu}} = {E_{\nu}(\GeV)} \times 10^{-2}  \SI{}{factorb}
\end{equation}
\vspace*{2em}

\begin{Enunciado}
	\subsubsection*{Pregunta 10}

	Las corrientes neutras se descubrieron en el CERN en 4 reacciones clave, en el experimento Gargamelle. ¿Cuántas de ellas puedes escribir? Denota el núcleo de Freón como $N$. ¿En qué año se publicaron los resultados?

\end{Enunciado}

Las reacciones clave fueron las dispersiones de neutrinos sobre materia hadrónica y la dispersión de electron-neutrino muónico

\begin{equation}
	\bar{\nu}_\mu + N \rightarrow  \bar{\nu}_\mu + N  \qquad
	{\nu}_\mu + N \rightarrow   {\nu}_\mu + X
\end{equation}
\begin{equation}
	{\nu}_\mu + e^- \rightarrow   {\nu}_\mu + e^- \qquad
	{\nu}_\mu + e^- \rightarrow   {\nu}_\mu + e^-
\end{equation}
siendo $N$ un núlceo Frenón  (CF$_3$Br) en el experimento y $X$ un conjunto de hadrones. El año fue 1973.

\vspace*{2em}
\begin{Enunciado}
	\subsubsection*{Pregunta 11}

	Escribe una colisión elástica concreta entre un neutrino y un quark con un sabor único. ¿Tiene necesariamente el acoplo una parte $V + A$? Razona la respuesta.

\end{Enunciado}

Una colisión elástica concreta sería:

\begin{equation}
	\nu_\mu + d \rightarrow  \nu_\mu + d
\end{equation}
La respuesta es que sí tiene una parte de acoplo V+A, ya que la corriente neutra, la única responsable de dicha interacción, ya que no intercambia sabor y no intercambia carga. y el fotón no interacciona con neutrinos. La respeusta es evidente cuando tenemos en cuenta que la corriente neutra tiene un pequeño porcentaje  de V+A, a saber, $g_L=0.287$ y $g_R=0.042$.

\vspace*{2em}

\begin{Enunciado}
	\subsubsection*{Pregunta 12}

	Escribe los dobletes de isospín débil que forman los leptones conocidos, indicando su quiralidad y el valor de $T_3$ en cada caso.

\end{Enunciado}

Los dobletes de isospín débil serían:

\begin{equation}
	\binom{\nu_e}{e^-}_L \quad \binom{\nu_\mu}{\mu}_L \quad \binom{\nu_\tau}{\tau}_L
\end{equation}
el valor de $T_3$ es trivial, $+1/2$ para los neutrinos y $-1/2$ para los leptones cargados.

\vspace*{2em}

\begin{Enunciado}
	\subsubsection*{Pregunta 13}

	Nombra los 4 bosones físicos de espín 1 de la teoría de unificación electrodébil, utilizando sus símbolos estándar. También los símbolos que corresponden a los 4 bosones gauge de la teoría $SU(2)_L \times U(1)_Y$.

\end{Enunciado}


Los 4 bosones físicos son: $W^+,W^-,Z^0,\gamma$. Los símbolos en la teoŕia gauge $SU(2)_L \times U(1)_Y$ son, respectivamente, $W_1,W_2,W_3,B$.

\vspace*{2em}

\begin{Enunciado}
	\subsubsection*{Pregunta 14}

	¿Permanece la corriente $J_Y^\mu \equiv 2(J^\mu_{\text{em}} - J^\mu_3)$ invariante bajo rotaciones del grupo $SU(2)_L \times U(1)_Y$, en la teoría de unificación electrodébil? ¿Puedes explicitar cada término, en función de los espinores quirales $e_R$, $e_L$ y $\nu_{e,L}$?

\end{Enunciado}

Si, permanece invariante bajo rotaciones débiles de dicho grupo y en la teoría electrodébil, por cosntrucción. Tedríamos que:

\begin{equation}
	J^\mu_{\text{em}} =  \overline{e_R} \gamma^\mu e_R + \overline{e_L} \gamma^\mu e_L
\end{equation}
\begin{equation}
	J^\mu_{3} = \overline{\nu_{e,L}} \gamma^\mu \nu_{e,L}  -  \overline{e_L} \gamma^\mu e_L
\end{equation}
La explicación es sencilla:

\begin{equation*}
	J^\mu_3 = \overline{\chi_L} \gamma^\mu \sigma_3 \chi_L e^{i(p_f - p_i)} , \qquad J^\mu_{\text{em}} = - \overline{e} \gamma^\mu e
\end{equation*}

y como los neutrinos $T_3=1/2$ y los electrones $T=-1/2$, es trivial la expresión, mientras que para la parte electromagnética es el desglose de ambos términos por igual, ya que realmente no ``diferencia'' izquierdas y derechas.


\vspace*{2em}

\begin{Enunciado}
	\subsubsection*{Pregunta 15}

	¿Es válida la teoría $SU(2)_L \times U(1)_Y$ sólo para leptones, o también para los quarks? Escribe los 6 fermiones (3 leptones y 3 quarks) de la segunda generación que se acoplan en ella, indicando su multiplicidad y quiralidad.

\end{Enunciado}

La teoría $SU(2)_L \times U(1)_Y$ es válida para cualquier partícula con hipercarga, solo que, al igual que para los leptones, asume quarks sin masa (no son físicos). De hecho la teoría $SU(2)_L \times U(1)_Y$  no distingue los fermiones. Los 6 fermiones de segunda generación:

\begin{itemize}
	\item Fermiones. Tenemos un doblete a izquierdas $\binom{\nu_\mu}{\mu}_L$ y dos singlete a derechas $(\mu)_R$ y $(\nu_\mu)_R$. Aunque el neutrino a derechas no se acopla al tener hipercarga $Y=(2Q-T_3)$ nula.
	\item Quarks. Tenemos un doblete a izquierdas $\binom{c}{s}_L$, y dos singletes a derechas $(u)_R$ y $(d)_R$.
\end{itemize}

\vspace*{2em}

\begin{Enunciado}
	\subsubsection*{Pregunta 16}

	¿En qué máquina se midieron por primera vez $M_{Z^0}$ y $\sin^2\theta_W$ con precisión $10^{-4}$? Indica las partículas colisionantes, su energía, y el año en que se hizo.

\end{Enunciado}

Se midieron, por primera vez, en el experimento LEP (ALEPH,L3,DELPHI,OPAL) del CERN. Las particulas colisionantes eran $e^+ e^-$ con una energía de $\sqrt{s}=91$ GeV. Se midió en el 1991.

\vspace*{2em}

\begin{Enunciado}
	\subsubsection*{Pregunta 17}

	¿Sabes cuánto vale en GeV el valor esperado en el vacío $v$ del campo de Higgs? ¿Hay algún observable que permita determinarlo unívocamente, en la teoría de unificación electrodébil? ¿Conoces la relación entre ambos?

\end{Enunciado}

El GeV esperado es de 246 MeV. Si, hay un observable, que es la constante de Fermi $G_F$. La relación es:

\begin{equation}
	\frac{G_F}{\sqrt{2}} = \frac{1}{2 v^2}
\end{equation}



\vspace*{2em}

\begin{Enunciado}
	\subsubsection*{Pregunta 18}

	¿Es capaz de entender el Modelo Estándar las leyes físicas (no matemáticas) que subyacen en el potencial de autoacoplo del campo de Higgs? ¿Se parece a algún fenómeno que hayas estudiado en otras materias?

\end{Enunciado}

No, la teoría no nos informa sobre el mecanimso físico que da lugar al valor en el vacio $v$ del campo $H$. Se puede parecer al método que Landau y Ginzburg le dieron masa al fotón en el interior de un superconductor (1950).


\vspace*{2em}

\begin{Enunciado}
	\subsubsection*{Pregunta 19}

	¿Conoces las masas en GeV/$c^2$ de los bosones $W^\pm$ y $Z^0$? Indica la relación que tienen en la teoría de Weinberg con el valor esperado en el vacío del campo de Higgs $v$ y con las constantes de acoplo de la simetría gauge.

\end{Enunciado}

Si, las masas son $m_{W^{\pm}}=80.38$ GeV/c$^2$ y $m_Z=91.18$ GeV/c$^2$. Las diferentes relaciones que debemos consdirar son:

\begin{equation}
	v^2 = \frac{-\mu^2}{2\lambda} \qquad M_{W} = \frac{1}{2} v g \qquad M_{Z}= \sqrt{g^2 + (g')^2}\frac{v}{2}
\end{equation}
siendo $g$ y $g'$ las constantes de acoplo relacionables con $\theta_W$ y $e$, tal que $g=e/\sin \theta_W$ y $g'=e/\cos \theta_W$.


\vspace*{2em}

\begin{Enunciado}
	\subsubsection*{Pregunta 20}

	¿De qué único parámetro depende el cociente $M_{Z^0}/M_{W^\pm}$? ¿Qué simetría gauge hace posible su independencia de $v$ y cuáles son sus constantes de acoplo?

\end{Enunciado}

Depende únicamente de $\cos (\theta_W)$, es decir, del coseno del ángulo de Weinberg. La simetría que hace posible esta es efectivamente la simetría $SU(2)_L \times U(1)_Y$, que obliga a que la única relación entre las masas sea el ángulo de Weinberg. Las constantes de acoplo son $g$ y $g'$, relacionables con $e$ y $\theta_W$, tal que $g=e/\sin \theta_W$ y $g'=e/\cos \theta_W$.

\vspace*{2em}

\begin{Enunciado}
	\subsubsection*{Pregunta 21}

	Indica el tipo de relación que existe entre la constante de Fermi $G_F$, la constante de acoplo $g^2$ del grupo $SU(2)_L$ y la masa del $W^\pm$.

\end{Enunciado}

La relación entre la constante real $G_F$, $g$ y $M_W$ es trivial:

\begin{equation}
	\frac{G_F}{\sqrt{2}} = \frac{g^2}{8M_W^2}
\end{equation}
y su valor es $G_F=\SI{1.166378(7)e-5}{GeV^{-2}}$ y $M_W=80.43$ GeV/c$^2$.

\vspace*{2em}

\begin{Enunciado}
	\subsubsection*{Pregunta 22}

	¿Se pudo predecir históricamente la masa del bosón $W^\pm$ a partir de la constante de Fermi $G_F$ y la carga del electrón $e$? ¿Qué otro parámetro estuvo involucrado en la predicción? ¿Era conocido cuando se descubrió?

\end{Enunciado}

No se podría predecir, ya que la constante de acopolo $g=e/\cos(\theta_W)$ depende del ángulo de Weinberg, ya que la masa del bosón W:

\begin{equation}
	M_W = \left( \frac{\sqrt{2}g^2}{8G_F} \right)^{1/2}
\end{equation}
El bosón W fue descubierto y su masa del bosón medida antes (SPS CERN, 1983) que el cálculo con precisión de $\theta_W^2$ (LEP CERN, 1991).


\begin{Enunciado}
	\subsubsection*{Pregunta 23}

	¿Es suficiente la teoría $SU(2)_L \times U(1)_Y$ para predecir $M_{Z^0}$, aun conociendo de forma empírica las constantes $G_F$ (relacionada con $M_{W^\pm}$), $e$, y $\sin^2\theta_W$? Explica el contenido físico de la predicción sobre el parámetro $\rho$.

\end{Enunciado}

No, ya que la teoría $SU(2)_L \times U(1)_Y$ no es capaz de dar valor al parámetro $\rho$, es decir, no es capaz de dar una relación entre $M_Z$ y $M_W$ con ninguna constante.

Esta predicción impone una fuerte restricción sobre el tipo de ruptura espontánea de simetría permisible.

\begin{Enunciado}
	\subsubsection*{Pregunta 24}

	¿Siguen los módulos de la matriz CKM algún patrón aproximado en potencias del seno $\lambda$ del ángulo de Cabibbo? Indica el nombre del físico que postuló dicho patrón, y cuál fue el descubrimiento que le motivó a hacerlo.

\end{Enunciado}

Si, un patrón tal qeu:

\begin{equation}
	|V_{CKM}| \simeq \begin{pmatrix}
		1         & \lambda   & \lambda^3 \\
		\lambda   & 1         & \lambda^2 \\
		\lambda^3 & \lambda^2 & 1
	\end{pmatrix}
\end{equation}
donde $\lambda\equiv \sin \theta_c$. Los físicos que lo propusieron fueron Kobayashi y Maskawa en 1973. El motivo fue descrburir que los quarks tienen masa (SLAC, 1971-1973).



\begin{Enunciado}
	\subsubsection*{Pregunta 25}

	¿Es posible en el Modelo Estándar la desintegración $Z^0 \to \nu_\tau \bar{\nu}_\tau$? Dibuja el diagrama de Feynman. ¿Predice la teoría los valores de $C_V$ y $C_A$ para el $\nu_\tau$ y de $\Gamma(Z^0 \to \nu_\tau \bar{\nu}_\tau)$? ¿En qué acelerador se midió esta última, y en qué año?

\end{Enunciado}


Si, es posible dicho decaimiento. El diagrama de Feynman:

\begin{figure}[h]
	\centering
	\begin{tikzpicture}
		\begin{feynman}
			\diagram [horizontal=a to b] {
			a [particle=\(Z^0\)] -- [boson] b,
			b -- [fermion] f1 [particle=\(\nu_\tau\)],
			b -- [anti fermion] f2 [particle=\(\bar{\nu}_\tau\)],
			};
		\end{feynman}
	\end{tikzpicture}
\end{figure}

Fue medido en el LEP CERN en 1991. Si, la teoría predijo dichos valores, descrubriendo que había 3 sabores de neutrinos. 

\vspace*{2em}

\begin{Enunciado}
	\subsubsection*{Pregunta 26}

	¿Puede formarse un bosón $W^-$ en el choque entre un quark $d$ y un antiquark $\bar{u}$? Indica qué partículas harías colisionar para formarlo, y a qué energía mínima (supón $x \approx \frac{1}{3}$ y que los quarks arrastran $\approx \frac{1}{2}$ del momento del protón). Indica quién lo hizo y el año.

\end{Enunciado}


Dado que $d$ tiene carga -1/3 y $\bar{u}$ una carga -2/3, en principio no se viola carga. Logicamente debemos usar algún tipo de partícula cargada, como quarks, usando por ejemplo colisiones de protón-antiprotón. La energía más pequeña posible en el centro de masas de los quarks sería $\sqrt{s} = M_W c^2= 80.4$ GeV. Suponemos que todas las partículas son ultrarrelativistas tal qeu $E\approx p c$. Nos dicen que los quarks arrastran 1/2 del momento del protón, por lo que:

\begin{equation}
	\sqrt{s}_{{qq}} = (E_{q1}^2+E_{q2}^2)^{1/2} = \frac{\sqrt(2)}{3} E_p = \frac{1}{3} \sqrt{s}_{p\bar{p}}  
\end{equation}
Ahora si se llevan el momento de la mitad:

\begin{equation}
	\sqrt{s}_{{qq}}= \frac{1}{6} \sqrt{s}_{p\bar{p}}  \longrightarrow \sqrt{s}_{p\bar{p}}^{\min} \approx 6 M_W c^2  \approx \SI{482}{GeV}
\end{equation} 
Ellos usaron $\SI{540}{GeV}$. 

\vspace*{2em}

\begin{Enunciado}
	\subsubsection*{Pregunta 27}

	¿Cuántos parámetros reales, medibles e independientes tiene la matriz $V_{\text{CKM}}$ (módulos y fases)?

\end{Enunciado}

Tenemos en total $\frac{1}{2} (N-1)(N-2)$ parámetros, con $N$=3, lo que implica en total 1 fase real, medible e independeinte, que en la matriz CKM será $\lambda \equiv \sin \theta_c$. 

\vspace*{2em}

\begin{Enunciado}
	\subsubsection*{Pregunta 28}

	Realiza el contaje del número total de parámetros reales que tiene el Modelo Estándar, de forma no redundante. Esto incluye las masas de quarks y leptones, así como los parámetros medibles e independientes de la matriz CKM, y otros. Ten en cuenta que el modelo predice masa nula de los neutrinos. Para QCD, toma sólo $\alpha_s(M_{Z^0})$.

\end{Enunciado}

Tenemos en total la masa de todos los quarks (6), la masa de todos los leptones (3), la masa del bosón de Higgs (1), el valor del vacio $v$ (1), las constantes de acoplo (2) $g$ y $g'$, la constante de QCD (1), y los parámetros de la matriz CKM, 3 reales y 1 fase (4). En total contabilizamos 18.

Las masas de los bosones, la consatnte de Fermi y la carga del electrón se peuden calcular a partir de $g,g'$ y $v$:

\begin{equation}
	e = g \sin \theta_W \quad \cos \theta_W = \frac{g}{\sqrt{g^2 + (g')^2}} \quad G_F = \frac{1}{\sqrt{2} v^2 } \quad M_W = g v \quad M_Z  = \frac{M_W}{\cos (\theta_W)}
\end{equation}

\vspace*{2em}

\begin{Enunciado}
	\subsubsection*{Pregunta 29}

	¿Qué fracción (\%) de los parámetros anteriores corresponde a constantes de acoplo adimensionales de los grupos de simetría gauge, y qué fracción a física desconocida?

\end{Enunciado}

Pues serían las constantes $g,g'$ y $\alpha$, por lo que 2/18=12\% a física desconocida el resto 88\%.  
\vspace*{2em}

\begin{Enunciado}
	\subsubsection*{Pregunta 30}

	Sabiendo que $g \approx 0.629$ y $g' \approx 0.345$, ¿predice la teoría el valor de la carga del electrón sólo a partir de estos dos parámetros? Si es cierto, calcula $e$ (en unidades $\hbar = c = \varepsilon_0 = 1$), así como la constante adimensional $\alpha = \dfrac{e^2}{4\pi}$, y compara con su valor real.

\end{Enunciado}

La ecuación que los relaciona es sencilla:

\begin{equation}
	e = g \sin \theta_W  = g' \cos \theta_W \qquad \cos \theta_W = \frac{g}{\sqrt{g^2 + (g')^2}}
\end{equation}
de lo que se deduce qeu
\begin{equation}
	e = \frac{gg'}{\sqrt{g^2 + g'^2}} \approx 0.302
\end{equation}
y $\alpha\approx 0.00728$ tal qeu $1/137 \approx 0.00729$. 


\vspace*{2em}
