\documentclass[12pt,a4paper]{article}
\usepackage[utf8]{inputenc}
\usepackage[spanish]{babel}

% Paquetes

\usepackage{amsmath}
\usepackage{amsfonts}
\usepackage{amssymb}
\usepackage{graphicx}
\usepackage[colorlinks=true,allcolors=blue]{hyperref} % Crea las hiperreferencias
\graphicspath{ {Imagenes/} }

% Autor y titulo

\title{Apuntes Cuantica III}
\author{Daniel Vázquez Lago}

% Forma del  texto

\setlength{\parindent}{15px}
\usepackage[left=2.25cm,right=2cm,top=4cm,bottom=2cm]{geometry}

% Otros


\numberwithin{equation}{section}
\numberwithin{figure}{section}

% Comandos propios

\newcommand{\parentesis}[1]{\left( #1  \right)}
\newcommand{\parciales}[2]{\frac{\partial #1}{\partial #2}}
\newcommand{\pparciales}[2]{\parentesis{\parciales{#1}{#2}}}
\newcommand{\ccorchetes}[1]{\left[ #1  \right]}
\newcommand{\D}{\mathrm{d}}
\newcommand{\derivadas}[2]{\frac{\D #1}{\D #2}}

\newcommand{\tquad}{\quad \quad \quad}

% Comandos vectoriales

\newcommand{\un}{\mathbf{u}}
\newcommand{\vn}{\mathbf{v}}
\newcommand{\xn}{\mathbf{x}}

\newcommand{\Kn}{\mathbf{K}}
\newcommand{\Rn}{\mathbf{R}}
\newcommand{\Tn}{\mathbf{T}}

\begin{document}

\maketitle

\newpage

\tableofcontents

\newpage

\section{Estructura fina del hidrógeno}


\subsection{Ecuación de Dirac}

\subsection{Acoplo de un Campo Electromagnético a la ecuación de Dirac}

\subsection{El átomo de hidrógeno sin correcciones}

\end{document}


%\section{Estructura Cuantica del átomo}

%\subsection{Correciones relativistas}

%La teoría de Bohr fue un cambio radical en la compresión de la materia. Fundamentada principalmente en la cuantización de los orbitales, sus orígines fueron escabrosos, ya que los físicos de aquel momento estaban preocupados sobre como los electrones podrían orbitar (circularmente, tal y como decía el modelo de Bohr) antes de que fueran expulsados/tragados. \\

%Sin embargo la comunidad no tardo en darse cuenta de que las órbitas circulares no eran mas que una sobre simplificación del movimiento de los electrones. Por ejemplo Sommerfield creo una teoría mecánica del movimiento de los electrones que asumía un movimiento elíptico, consistente con la relatividad especial. Este nuevo modelo se baso en una regla general que decía \textit{la integral de momento asociado con una coordenada alrededor de un periodo de la partícula da como resultado un múltiplo de la constante de Plank}. Aplicando esta cuantización al momento alrededor de una órbita circular nos lleva a:

%\begin{equation}
%   m_e v \times 2 \pi r = n h
%\end{equation}
%Ademaś de esta cuantización de la coordenada $\theta$ Sommerfield consideró cuantizar la coordenada $r$. Él se encontro que para un potencial proporcional a $1/r$ existen estados estacionarios (con una alta excentricidad). El mucho esfuerzo realizado para cuantizar estas órbitas tuvo su recompensa, ya que este modelo fue capaz de predecir y explicar la forma de las lineas espetrales. 

%\subsection{Átomos alcalinos}

%El modelo del átomo de hidrógeno fue fundamental para el desarrollo de la teoería cuántica, y continua siendo esencial estudiarlo. 

%\subsection{Átomo de Helio}

%\subsection{Desplazamiento de Lamb}

%\section{El enlace molecular y la estructura cristalina}

%\section{Teoría cuántica de colisiones}
