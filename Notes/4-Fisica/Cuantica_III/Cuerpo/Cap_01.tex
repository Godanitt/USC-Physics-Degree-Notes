\chapter{Introducción}
%\addcontentsline{toc}{section}{\protect\numberline{}Introducción}

Los estados físicos están represenatdos como vectores de un espacio de Hilbert. Las cantidades observables están representadas por operadores hermíticos $(A^{\dagger})_{ij} = a_{ij}^*$ actuando sobre los estados del espacio de Hilbert. El valor de una propiedad representada por elobservable $A$ da como resulado diferente varios autovalores y, tras la medida, el estado del vector del sistema es el autoestado asociado al autovalor obtenido $\phi_a$. \\

La probabilidad de obtener un valor particular es:

\begin{equation}
    P(a)= \frac{|\langle \phi_a | \Psi \rangle |^2}{|\langle \phi_a | \phi_a \rangle ||\langle \Psi | \Psi \rangle |}
\end{equation}

Las simetrías en la mecánica cuántica se representan en operadores unitarios, ya que no pueden cambiar las probabilidades de transición.       