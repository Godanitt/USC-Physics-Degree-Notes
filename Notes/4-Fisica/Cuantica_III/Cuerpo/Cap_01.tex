\chapter{Estructura fina del hidrógeno}

El átomo de hidrógeno ejerció durante la primera mitad del siglo XX una poderosa influencia en la mecánica cuántica. Con la mejora de las técnicas de medida, se fueron descubriendo cada vez líneas mas finas en la medida del espectro del átomo de hidrógeno, hasta que llegado un punto muchas de estas ya no podían ser explicadas con una formulación de la teoría cuántica relativista (ecuación de Dirac), por lo que se necesito un desarrollo aún mayor de la teoría cuántica: la electrodinámica cuántica. 

Sin embargo para alcanzar a comprender del todo las diferentes líneas en el espectro atómico (dividas en finas e hiperfinas) del hidrógeno es necesario comenzar por lo más básico, y esto nos lleva a la ecuación de Schrödinger y a la solución de la misma para un potencial de Coulomb, ya que esta nos permite describir con precisión el comportamiento de un electrón alrededor de un núcleo, al menos aproximadamente, ya que como veremos es el término coulombiano el que más importancia tiene. Luego una vez resuelta la ecuación de Schrödinger podremos pasar a las correciones relativistas de la misma, que se obtienen a través de la ecuación de Dirac. 

En esta ecuación de Schrödinger corregida o ecuación de Dirac para el orden $v^2/c^2$ encontraremos todos los términos necesarios para explicar la estructura fina del hidrógeno (asociada a las primeras lineas del espectro del hidrógeno), la estructura hiperfina (las segundas líneas del espectro del hidrógeno) y el comportamiento del átomo de hidrógeno en presencia de campos eléctrico y magnético (que explican el efecto Zeeman, Stark...). La estructura fina la estudiaremos en este capítulo mientras que el acoplamiento a campos externos y la estructura hiperfina lo dejaremos para el capítulo \ref{Ch:02}. Además también veremos el desplazamiento Lamb, que es una correción de la electrodinámica cuántica con dimensiones suficientemente grandes para superar (en átomos de poca masa y orbitales $s$) la estructura hiperfina.

\section{Ecuación de Schrödinger}

La ecuación de Schrödinger independiente del tiempo nos permite obtener los autovalores de la energía y los autoestados asociados a un electrón libre orbitando alrededor de un núcleo formado por $Z$ protones. En este caso la ecuación se viene dada por:

\begin{equation}
    \Hcal \psi =\parentesis{\frac{p^2}{2m} + V(\rn)} \psi 
\end{equation}
donde $p^2/2m$ es la energía cinética y $V(r)=-e\phi$ es el potencial electrostático coulombiano. Dado que $\pn $ como $-i\hbar \nabla$, tenemos que la anterior ecuación nos queda como
\begin{equation}
    \Hcal \psi = \parentesis{- \frac{\hbar^2}{2m} \nabla^2 + V(\rn)} \psi = E \psi
\end{equation}
En coordenadas esféricas:
\begin{equation}
    r^2 \parentesis{\parciales{^2}{r^2} + \frac{2}{r} \parciales{}{r}} \psi + \frac{2mr^2}{\hbar^2} (E-V) \psi = \frac{1}{\hbar^2} L^2 \psi
\end{equation}
donde 

\begin{equation}
    L^2 = - \hbar^2 \ccorchetes{\frac{1}{\sin \theta} \parciales{}{\theta} \parentesis{\sin  \theta \parciales{}{\theta}} + \frac{1}{\sin^2 \theta} \parciales{^2}{\psi^2}}
\end{equation}
Asumiendo a un potencial radial, podemos hallar las soluciones por separación de variables

\begin{equation}
    \psi (r,\theta,\phi) =  \frac{1}{r} u(r) Y(\theta,\phi) = R(r) Y(\theta, \phi)
\end{equation}
Hallar la forma analítica para $R(r)$ y $Y(\theta,\phi)$ implicarían, por lo menos, un par de páginas. Dado que este es un resumen, y que existen decenas de manuales y apuntes donde se ve la solución paso por paso no escribimos aquí la misma. Las energías válidas para estos sistemas, en unidades del sistema internacional son: 
\begin{Anotacion}
	\textcolor{red}{Corregir la siguiente ecuación, poner de otra forma.}
\end{Anotacion}
\begin{equation}
    E_n = - \ccorchetes{\frac{m}{2\hbar^2} \parentesis{\frac{e^2}{4 \pi \varepsilon_0}}^2} \frac{Z^2}{n^2} = - \frac{1}{2} \alpha^2 m c^2 \frac{Z^2}{n^2 } = \frac{E_1}{n^2}
\end{equation}
donde $E_1$ es la energía del estado fundamental\footnote{El estado fundamental es aquel asociado al autovalor de la energía más pequeño. Para el átomo de hidrógeno este estado es $|100\rangle$, con una energía de $-13.6$ eV.}. Cabe destacar que son las condiciones de contorno las que obligan a que la energía se \textit{discretice}. Finalmente los estados posibles para la solución de la función de onda de un electrón en un átomo de tipo hidrogenoide vienen dados por

\begin{equation}
|n l m \rangle \equiv \psi_{rlm} (r,\theta,\varphi) = R_{nl} (r) Y_l^m (\theta,\varphi) = \frac{1}{r} u_{nl} (r) Y_l^ (\theta,\varphi)
\end{equation}
donde las funciones $Y_l^m (\theta,\varphi)$ son los \textbf{armónicos esféricos} (véase anexo \ref{Ch:Anex_A}), de tal modo que la función $u_{nl} (r)$ viene dada por

\begin{equation}
    u_{nl} (r) = \sqrt{\frac{(n-l-1)! Z}{n^2 [(n+l)!]^3a_0}} \parentesis{\frac{2Zr}{na_0}}^{l+1} e^{Zr/(na_0)} L_{n+l}^{2l+1} \parentesis{\frac{2Zr}{na_0}}
\end{equation}
Los valores $n,l,m$ tienen nombres y para un $n$ dado solo pueden existir un número determinado de $l$, así como para un $l$ dado solo un número de $m$ dados pueden existir. Además estos tienen nombre, los cuales son:

\begin{itemize}
    \item \textbf{Número cuántico principal}: el número cuántico principal $n$ tiene un rango de valores infinito $n=0,1,2,3...$. 
    \item \textbf{Número cuántico orbital angular:} el número cuántico $l$ tiene un rango de valores que depende de $n$. Para un $n$ dado puede tener valores tales que $l=0,1,2,...,n-1$. En espectroscopía se le ponen los nombres $s,p,d,f,g...$ \footnote{los nombres se deben a: s (sharp), p (principal), d (difussed), f (fundamental); y para los siguientes se decidió seguir el alfabeto.} en vez de $l=0,1,2,3...$.
    \item \textbf{Número cuántico orbital magnético:} el número cuántico $m$ tiene un rango de valores que depende de $l$. Para un $l$ dado puede tener valores de entre $(-l,...,0,...,l)$.
\end{itemize}
La degeneración paera un valor de energía $n$ dado es de $n^2$, ya que:

\begin{equation}    
    g_n = \sum_{l=0}^{n-1} (2l+1) = n^2
\end{equation}

Este fenómeno de que para un mismo estado de energía contemplemos diferentes funciones de ondas solo ocurre con los pontenciales proporcionales a $1/r$ y $r^2$. Es interesante ver algunos de las gráficas de $R_{nl} = \frac{1}{r} u_{nl} (r)$, ya que, por ejemplo, a un mayor valor de $n$ un mayor número de nodos aparecen para un valor dado de $l$. La probabilidad de encontrar a un electrón en un elemento de volumen $\D \tau$ es:

\begin{equation}
    \psi^* \psi \D \tau = \frac{1}{r^2} u_{nl}^2 (r) Y_l^{m*} (\theta , \varphi)  Y_l^m r^2 (\theta , \varphi) r^2 \sin \theta \D r \D \theta \D \varphi
\end{equation}


\subsection{Valores esperados}

\subsection{Funciones de onda de un electrón}

\begin{equation}
    \Psi = \begin{pmatrix}
    R_1 (r) Y_l^{m-\frac{1}{2}} (\theta,\varphi) \\
    R_1 (r) Y_l^{m+\frac{1}{2}} (\theta,\varphi) 
    \end{pmatrix}
\end{equation}



\section{Ecuación de Dirac: correciones a la ecuación de Schrödinger}


La ecuación de Dirac nos lleva a la siguiente ecuación:

\begin{multline}
     \Hcal \varphi = \left[ q \phi + \frac{1}{2m} \parentesis{\pn-\frac{q}{c}\An}^2 - \frac{q\hbar}{2mc} \sigman \cdot (\nabla \times \An)  \right. \\
    \left. - \frac{p^4}{8 m^3 c^2} + \frac{\hbar^2 q}{8 m^2 c^2} \nabla^2 \phi + \frac{ \hbar q}{4m^2 c^2} \sigman [(\nabla \phi)\times \pn] \right] \varphi
    \label{Ec:01-03-15}
\end{multline}
Esta es la ecuación dirac para una partícula cargada, y será suficiente como para estudiar las correciones relativistas al primer orden del hidrógeno (correciones de estructura fina). Correciones de mayor orden no son permitidas por correciones relativistas a las ecuaciones de ondas, y por tanto sería necesario un tratamiento desde la teoría cuántica de campos. 

Vamos a introducir entonces ahora el nivel de importancia de cada uno de los términos que aparecen en esta ecuación. En espectroscopía atómica, es común trabajar con la inversa de los centímetros como una medida de energía, debido a la relación entre la energía y la longitud de onda, segun la ecuación $1/\lambda = E/hc$. Para $E=1\eV$ tendríamos una energía asociada de $8065.5 \cmm$. Entonces tenemos, para un electrón $q=-e$:

\begin{itemize}
\item El potencial eléctrico $e\phi$ tiene un valor de $10\cmm$ o $\sim 12 \eV$.
\item El término $ \frac{1}{2m} \parentesis{\pn+\frac{e}{c}\An}^2$ tiene una contribución aproximada de $10^5 \cmm$, siendo responsable de procesos físicos importatísimos, como pueden ser la abosrción, emisión y dispersión de ondas electromagnéticas, el diamagnetismo y el efecto Zeeman, entre otras.
\item La interacción entre el momento mangnético de espín con un campo magnético $\Bn = \nabla \times \An$ dado por la contribución $\frac{e\hbar}{2mc} \sigman \cdot (\nabla \times \An) $ tiene una energía de entorno $1\cmm$ ($1.2 \cdot 10^{-4} \eV$). 
\item La correción relativista de la energía cinética $\frac{p^4}{8 m^3 c^2} $ aporta $1\cmm$ ($1.2 \cdot 10^{-4} \eV$).
\item El término de Darwin $\frac{\hbar^2 e}{8 m^2 c^2} \nabla^2 \phi$, que no tiene un análogo clásico y es responsable de la energía de intercambio de los estados s, tiene una contribución menor que $0.1\cmm$.
\item La interacción de espín-órbita viene del término $\frac{ \hbar e}{4m^2 c^2} \sigman [(\nabla \phi)\times \pn]$. En el hidrógeno supone una correción pequeña ($10^{-5} \eV$), aunque para átomos pesados puede llegar a ser considerablemente mayor, de 10 a $10^3 \cmm$ (0.0012 eV a 0.12 eV). Si por ejemplo $\phi$ solo dependiera de $r$, de tal modo que $\nabla \phi = \frac{\rn}{r} \derivadas{\phi}{r}$, usando que $\Ln = \rn \times \pn$ y que $\Sn = \frac{\hbar}{2} \sigman$, tenemos una expresión tal que:
\begin{equation}
    -\frac{ \hbar e}{4m^2 c^2} \sigman [(\nabla \phi)\times \pn] = - \frac{e}{2m^2 c^2} \frac{1}{r} \derivadas{\phi}{r}  \Sn (\rn \times \pn) = - \frac{e}{2m^2 c^2} \frac{1}{r} \derivadas{\phi}{r} \Ln \Sn \label{Ec:01-03-16}
\end{equation}
\end{itemize}
Finalmente, debemos recalcar que el factor $K$ usado debe ser cogido con pinzas, ya que cuando el potencial escalar se vuelve sigular (por ejemplo, para $r=0$) debemos resolver este problema por otro camino. La manera de resolverlo será vista en el capítulo \ref{Ch:02}.

\section{Estructura fina}


\subsection{Acoplamiento espín-órbita}

\subsection{Término de Darwin}

\subsection{Correción relativista al momento}

\subsection{Estructura fina}

\section{Desplazamiento Lamb}
