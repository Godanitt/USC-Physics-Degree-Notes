\appendix 
\chapter{Apéndice} \label{Ch:Anex_A}

\section{Sistemas de unidades}

\section{Teoría de perturbaciones independiente del tiempo}

Consideremos un hamiltoniano sin perturbar $\Hcal_0$, con su base de autoestados (ortonormales), $\phi_a$ y sus correspondientes autovalores $E_a$:

\begin{equation}
    \Hcal_0 \phi_a = E_a \phi_a \tquad \langle \phi_a | \phi_b \rangle = \delta_{ab}
    \label{Ec:A-02-01}
\end{equation}
Ahora supongamos que existe un término pequeño $\delta \Hcal$ del Hamiltoniano, proporcional a un parámetro pequeño $\varepsilon$. En ese caso los valores de la energía sufren una transformación $E_a \rightarrow E_a + \delta E_a$, con un cambio en su correspondiente autoestado $\phi_a \rightarrow \phi_a + \delta \phi_a$, donde $\delta E_a$ y $\delta \phi_a$ están dados por una serie de potencias de $\varepsilon$:

\begin{equation}
    \delta E_a = \delta_1 E_a + \delta_2 E_a + \ldots \tquad \delta \phi_a = \delta_1 \phi_a + \delta_2 + \ldots 
\end{equation}
con el valor de $\delta_n E_a$ y $\delta_n \phi_a$ proporcioanl a $ \varepsilon^n$. La ecuación de Schrödinger puede ser expresada como:

\begin{equation}
    \parentesis{\Hcal_0 + \delta \Hcal} \parentesis{\phi_a + \delta \phi_a} = \parentesis{E_a + \delta E_a} \parentesis{\phi_a + \delta \phi_a}
\end{equation}
de tal modo que:

\begin{equation}
    \Hcal_0  \phi_a + \Hcal_0 \phi_a + \delta H \phi_a = E_a \phi_a + E_a \delta \phi_a + \delta E_a \phi_a + \delta E_a \delta \phi_a
\end{equation}
Dado que $\Hcal_0 \phi_a = E_a \phi_a$ (\ref{Ec:A-02-01}) la eliminamos en ambos lados:
 
\begin{equation}
    \Hcal_0 \phi_a + \delta H \phi_a = E_a \phi_a + E_a \delta \phi_a + \delta E_a \phi_a 
\end{equation}

\subsection{Primer orden}

Para calcular la perturbación de primer orden nos tenemos que fijar cuales son los términos que van con $\varepsilon$. Aquellas que vayan con $\varepsilon^2$ pertenecerán a lo que llamamos perturbación de segundo orden (apartado \ref{Subsec:A-02-01}), de tal modo que $\delta\Hcal\delta\phi_a$ y $\delta E_a\delta\phi_a$ no vamos a tenerlas en cuenta. Así, para los términos de primer orden $\varepsilon$:

\begin{equation}
    \Hcal_0 \delta_1 \phi_a + \delta \Hcal \phi_a = E_a \delta_1 \phi_a + \delta_1 E_a \phi_a \label{Ec:A-02-06}
\end{equation}
Para encontrar $\delta_1 E_a$ tenemos que realizar el producto escalar en la ecuación anterior con $\phi_a$:

\begin{equation}
    \langle \phi_a | \Hcal_0 \delta_1 \phi_a \rangle + \langle \phi_a | \delta \Hcal \phi_a \rangle = E_a \langle \phi_a |\delta_1 \phi_a \rangle + \delta_1 E_a \langle \phi_a |\phi_a \rangle
\end{equation}
Dado que $\Hcal$ es hermítico, tenemos que $\langle \phi_a |\Hcal_0 \delta_1 \phi_a\rangle = \langle \Hcal_0 \phi_a |\delta_1 \phi_a\rangle = E_a \langle \phi_a |\delta_1 \phi_a \rangle $, con lo cual se cancelan los términos.

\begin{equation}
    \delta_1 E_a = \langle \phi_a |\delta \Hcal |\phi_a\rangle  \label{Ec:A-02-08}
\end{equation}
Sin embargo, este procedimiento es solo aplicable a estados no degenerados. Para ver el porqué, tendremos que calcular el cambio de una autofunción producido por una perturbación. Tomemos entonces el producto escalar de (\ref{Ec:A-02-06}) con el autoestado de la función del estado fundamental.

\begin{equation*}
    \langle |\Hcal_0 \delta_1 \phi_a \rangle + \langle \phi_b | \delta \Hcal \phi_a\rangle = E_a \langle \phi_b |\delta_1 \phi_a \rangle + \delta_1 E_a \langle \phi_a | \phi_n \rangle
\end{equation*}
\begin{equation}
\langle \phi_b |\delta \Hcal |\phi_a \rangle = (E_a - E_b) \langle |\delta_1 \phi_a\rangle
\end{equation}
Para $a=b$ tendríamos que esta ecuación es igual a (\ref{Ec:A-02-08}). Lo que ahora conocemos, a diferencia de antes, es que

\begin{equation}
    \langle \phi_b |\delta \Hcal |\phi_a \rangle = (E_a - E_b) \langle \phi_b |\delta_1 \phi_a \rangle, \tquad \text{para}  \ a \neq b
\end{equation}
Ahora el problema es claro: si existen dos estados con $\phi_b \neq \phi_a$ tal que $E_b = E_a$, entonces la ecuación anterior es inconsistente a menos que $\langle \phi_b | \delta \Hcal |\phi_a\rangle$ sea coer, lo cual no suele ser el caso.

\subsubsection{Estados no degenerados}

Primero vamos a estudiar el caso mas sencillo posible: el estado no degenerado. En otras palabras: cada estado tiene su propia energía, diferente a cualquier otro estado. En este caso podemos escribir:

\begin{equation}
    \langle \phi_b |\delta_1 \phi_a \rangle = \frac{\langle \phi_b |\delta \Hcal |\phi_a \rangle}{(E_a - E_b)}, \tquad \text{para}  \ a \neq b \label{Ec:A-02-11}
\end{equation}
Ahora bien, ¿Qué pasa con el componente de $\delta_1 \phi_a$ proyecto sobre $\delta_a$ dado por $\langle \phi_a |\delta_1 \phi_a \rangle$? Para encontar el valor necesitamos imponer la condición de que $\phi_a + \delta_1 \phi_a$ está normalizada, esto es:

\begin{equation}
    1 = \langle \phi_a + \delta_1 \phi_a | \phi_a + \delta_1 \phi_a \rangle = 1+ \langle \phi_a |\delta_1 \phi_a \rangle + \langle \phi_a | \phi_a \rangle + \Ocal (\varepsilon^2)
\end{equation}
De lo cual se deduce la condición de que 

\begin{equation}
     \Real \parentesis{\langle \phi_a |\delta_1 \phi_a \rangle }=0
\end{equation}
No hay que imponer nada a la parte imaginaria del producto $\langle \phi_a | \delta_1 \phi_a \rangle $, y dado que este valor, en realdiad, lo podemos elegir libremente, ya que solo representa la elección de fase del vector. En partircular elegimos que $\langle \phi_a | \delta_1 \phi_a \rangle $ sea real, de tal modo que la condición de noramlización es tan sencilla como

\begin{equation}
    \langle \phi_a |\delta_1 \phi_a \rangle = 0
\end{equation}
Volviendo a la ecuación (\ref{Ec:A-02-11}) y usamos que

\begin{equation}
    \delta_1 \phi_a = \sum_b |\phi_b \rangle \langle \phi_b | \delta_1 \phi_a \rangle = \sum_{b \neq a} \frac{\langle \phi_b |\delta \Hcal |\phi_a \rangle}{(E_a-E_b)} |\phi_b \rangle
\end{equation}
de tal modo que la \textit{perturbación de primer orden} para $\phi_a$ es:

\begin{equation}
    \delta_1 \phi_a = \sum_{b\neq a} \frac{\langle \phi_b |\delta \Hcal |\phi_b \rangle}{(E_a - E_b)}|\phi_b \rangle
\end{equation}

\subsubsection{Estados degenrados}

Supongamos que hay un número de estados $\phi_{a_1}, \phi_{a_2},\ldots, \phi_{a_n}$, todos con la misma energía $E_a$. Dado que $\Hcal$ es hermítico, las cantidades $\langle \phi_{a_s} |\Hcal |\phi_{a_r}\rangle$ forman una matriz Hermítica. Debido a un teorema de álgebra matricial, tenemos que  dicha matriz podrá ser estudiada como una matriz formada por los autovectores $u_1,u_2,\ldots,u_n$, de tal manera que
 
\begin{equation}
    \sum_{r=1}^n \langle \phi_{a_s} |\delta \Hcal | \phi_{a_r} \rangle = \Delta_q u_{qs}
\end{equation}
donde $\Delta_q$ eson los autovalores y $u_{qr}$ son los miembros de la matriz cuyos componentes son los vectores ortonormales $u_1,u_2,\ldots,u_n$ (esto es, $u_{qr}$ es el valor de la posición $r$ del vector $q$):

% falta texto

\subsection{Segundo orden} \label{Subsec:A-02-02}

\subsubsection{Caso no degenerado}

\subsubsection{Caso degenerado}

\section{Teoría de perturbaciones dependiente del tiempo. Regla de oro de Fermi}


\section{Métodos variacionales.}

\section{Momento angular y espín}

Las leyes de la naturaleza no deberían depender de como este orientado nuestro laboratorio. Se espera entonces que nuestras teorías sean invariante bajo rotaciones. En este apartado vamos a probar como la invariancia bajo rotaciones lleva a la existencia de la conservación del momento $\Jn$. Una rotación en un espacio tridimensional es una trasnformación lineal $x_i'=\sum_j R_{ij} x_j$ de las Coordenadas cartesianas  $x_i$ que deja invariante el producto escalar $\xn \cdot \yn$. De este modo tenemos que:

\subsection{Momento angular para $j=1/2,1,3/2$}

\subsection{Representaciones del operador rotación: matrices de rotación}


\subsection{Coeficientes de Clebsch-Gordan, símbolos $3j$ y $6j$}

Dos sistemas con momentos angulares $\Jn_1$ y $\Jn_2$ pueden ser considerados juntos como un sistema global de momento angular total $\Jn_3 = \Jn_1 + \Jn_2$. Existen dos bases de autofunciones de este tercer sistema, representadas por $|j_1 j_2 j_3 m_3\rangle$ y $|j_1 j_2 m_1 m_2\rangle$. Lógicametne podremos cambiar de un estado a otro usando:

\begin{equation}
   |j_1 j_2 j_3 m_3 \rangle  = \sum_{m_1, m_2}  \langle j_1 j_2 m_1 m_2 | j_1 j_2 j_3 m_3 \rangle    |j_1 j_2 m_1 m_2\rangle
\end{equation}
A los elementos de la matriz $\langle j_1 j_2 m_1 m_2 | j_1 j_2 j_3 m_3 \rangle$ se le llaman \textbf{coeficientes de Clebsch-Gordan}. Una notación alternativa es:

\begin{equation}
    \begin{array}{c}
    \Psi^{m_3}_{j_1j_2j_3} = \sum_{m_1,m_2} C_{j_1j_2} (j_3 m_3; m_1 m_2) \Psi^{m_1 m_2}_{j_1 j_2} \\ \\
    \Psi^{m_1 m_2}_{j_1j_2} = \sum_{j_3,m_3} C_{j_1j_2} (j_3 m_3; m_1 m_2) \Psi^{m_1 m_2}_{j_1 j_2} 
    \end{array}
\end{equation}

\subsection{Armónicos esféricos}

Los armónicos esféricos $Y_l^m (\theta, \varphi)$ son las autofunciones del orbital momento angular orbital, y satisfacen las siguientes ecuaciones diferenciales:

\begin{equation}
    \ccorchetes{\frac{1}{\sin (\theta)} \parciales{}{\theta} \parentesis{\sin (\theta) \parciales{1}{\theta} + \frac{1}{\sin^2 (\theta)} \parciales{^2}{\varphi^2}}} Y_l^m + l(l+1) Y_l^m = 0
\end{equation}
Y vienen dadas explícitamente por:

\begin{equation}
    Y_l^m (\theta,\varphi) = (-1)^m \ccorchetes{\frac{2l+1}{4 \pi} \frac{(l-m)!}{(l+m)!}}^{1/2} P_l^m (\cos (\theta)) e^{im\varphi}
\end{equation}

\subsection{El teorema de Wigner-Eckart}

Sean los $|\Phi_j^m\rangle$ los autoestados del momento angular con autovalores $j(j+1 )\hbar^2$ y $m_j \hbar$ para $J^2$ y $J_3$ respectivamente. Recordar que

\begin{eqnarray}
    (J_1\pm iJ_2) |\Phi_j^m \rangle = \hbar \sqrt{j(j+1)-m(m\pm 1)} |\Phi_j^{m\pm 1}
\end{eqnarray}
Sea $|\Psi_j^m\rangle$ otros autoestados del momento angular. Podemos demostrar que

\begin{equation}
    \langle \Phi_j^{m+1} | \Psi_j^{m+1} \rangle = \langle \Phi_j^m |\Psi_j^m \rangle
\end{equation}
Esto demuestra que $\langle \Phi_j^m |\Psi_j^m \rangle$ es {\it independiente} de $m$. Cualquier otro elemenot de la matriz con valores de $j$ y $m$ diferentes se anulan:

\begin{equation}
    \langle \Psi_{j_3}^{m_3} | O_{j_2}^{m_2} \rangle = 0
\end{equation}
Definimos como un {\bf tensor irreducible} de rango $j$ como un conjunto de $2j+1$ operadores $O_{j}^m$  ($m=-j,-j+1,...,j$) que al aplicarle lso generadores de rotación

\begin{equation}
    [J_3,O_j^m] = \hbar m O_j^m \tquad [J_1\pm i J_2, O_j^m] = \hbar \sqrt{j(j+1)-m(m\pm 1)} O_{j}^{m\pm 1}
\end{equation}
Algunos ejemplos de tensores irreducibles son los {\it armónicos esféricos}. 

\begin{theorem}[{\bf Wigner-Eckart}]
    Sea $\langle j_i j_2 m_1 m_2 |j_1 j_2 j_3 m_3 \rangle$ es el coeficiente de Clebsch-Gordan asociado con el acoplamiento de los momentos angulares $\Jn_1$ y $\Jn_2$ que componen $\Jn_3$; y $\langle \Phi || O || \Psi\rangle$, llamada la {\it matriz irreducible elemental}, que puede depende de todo menos de las tres componentes $m_1,m_2$ y $m_3$; el teorema de Wigner-Eckart nos dice que:

    \begin{equation}
        \langle \Phi_{j_3}^{m_3} | O_{j_1}^{m_1} |\Psi_{j_2}^{m_2} \rangle = \frac{1}{2j_3+1} \langle  j_i j_2 m_1 m_2 | j_1 j_2 j_3 m_3 \rangle \langle \Phi || O || \Psi  \rangle
    \end{equation}
    El teorema de Wigner-Eckart se puede expresar de otra forma, la {\bf Formula de Landé}. Sea $\An$ un vector cualquiera y $\Jn$ un moemnto angular. Esta fórmula nos dice que:

    \begin{equation}
        \langle \Phi_{j}^{m} | \An |\Psi_{j}^{m'} \rangle = \frac{\langle \Phi_j^m |\An \cdot \Jn | \Psi_j^m\rangle }{j(j+1)\hbar^2} \langle \Phi_j^m | \Jn| \Psi_j^{m'}  \rangle
    \end{equation}
    
\end{theorem}

\section{Operadores tensoriales irreducibles}

 
