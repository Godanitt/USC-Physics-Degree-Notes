\chapter{Átomos en campos estáticos e interacciones hiperfinas} \label{Ch:02}

\section{Campo magnético}

\subsection{Término lineal. Efecto Zeeman y efecto Paschen-Back.}

\subsection{Término cuadrático. Diamagnetismo.}

\section{Campo eléctrico}

\subsection{Efecto Stark Llineal}

\subsection{Efecto Star cuadrático}

\subsection{Ionización debida a un campo electrostático}

\section{Estructura hiperfina}

Un núcleo ideal sería un núcleo puntual sin ningún tipo de momento angular o espín. Un núcleo ideal generaría un potencial con la forma del potencial de Coulomb $-Ze/r$. Pero los núcleos son sistemas mucho más complicados, ni son puntuales, ni tienen una carga necesariamente esférica, ni tienen un espín/momento angular nulo. Todas las interacciones entre los electroens y los átomos que no vengan dadas por la interacción Coulombiana se llama \textit{perturbación hiperfina}. Las más importantes de estas perturbaciones son: el espín nuclear y el momento cuadrupolar nuclear. El espín nuclear nos lleva a un momento magnético nuclear no nulo, que interactura con el momento magnético del electrón asociado con su momento angular y/o su espín. Además, una distribución de la carga no esférica nos lleva a que exista un momento cuadrupolar eléctrico que interactua con el campo eléctrico de Coulomb.

\subsection{Interacción magnética hiperfina.}

\subsection{Interacción magnética en sistemas de un electrón.}

\subsubsection{Perturbación magnética hiperfina en presencia de un campo externo}

Un caso particular pero muy importante en los experimentos de resonancia magnética, donde necesitamos considerar la interacción hiperfina en presencia de un campo mangético externo. Vamos a estudiar la interacción para el hidrógeno en un estado $s$, donde solo el término de Fermi contribuye al hamiltoniano. Usando la ecuación (), de tal modo que $\Hcal_m = \frac{1}{\hbar} \mu_B \Bn (\Ln+2\Sn)$ con $\Ln=0$, añadimos el término de contacto e introducimos la innteracción con el espín nuclear $\Inn$, con un campo magnético externo $\Bn$. Así, el hamiltoniano:

\begin{equation}
    \Hcal = \frac{2\mu_B}{\hbar} \Bn \Sn + A_F \Inn \Sn - \gamma \Bn \Inn = \frac{g_e \mu_B}{\hbar} \Bn \Sn + A_F \Inn \Sn - \frac{g_N \mu_N}{\hbar} \Bn \Inn
\end{equation}


\subsection{Interacción cuadrupolar}

\subsection{IPerturbaciones de isótopas}



