\chapter{Atomos multielectrónicos y moléculas}

\section{Átomo de helio y molécula de hidrógeno}

\subsection{Estado fundamental de He usando el principio variacional}

\subsection{Catión de dihidrógeno H$_2^+$}

\section{Atomos con $N$ electrones}

\subsection{Simetrías del hamiltoniano. Acomplamiento $LS$ y $jj$}

\subsection{Aproximaciones de campo central}

\subsection{Interacción espín-órbita}

\subsection{El modelo Thomas-Fermi}

\subsection{Ecuaciones Hartree-Fock}

\section{Moléculas}

\subsection{Aproximación de Bohr-Oppenhaimer}

Para un sistema de electrones y nucleos podemos construir un hamiltoniano con la forma:

\begin{eqnarray}
    \Hcal = \sum_\alpha \frac{P_\alpha^2}{2M_\alpha } + \sum_i \frac{p^2_i}{2m} + V(Q_\alpha, q_i)
\end{eqnarray}
donde $P_\alpha/2M_\alpha$ es el operador energía cinética para un nucleón de masa $M_\alpha$, $P_i^2 / 2m$ es el operador energía cinética para el electrón $i$-ésimo de masa $m$, $Q_\alpha$ es una serie de coordenadas nucleares y $q_i$ de coordenadas electrónicas, y 

\begin{eqnarray}
    V(Q_\alpha,q_i) = \sum_{i<r} \frac{e^2}{r_{ij}} + \sum_{\alpha < \beta} \frac{(Z_\alpha e)(Z_\beta e)}{r_{\alpha \beta}} - \sum_{i,\alpha} \frac{(Z_\alpha e)e}{r_{i\alpha}}
\end{eqnarray}
es la energía potencial del sistema entero, esto es, del conjunto de electrones y nucleones. El primer término es la serie de Coulomb entre pares de electrones, el segundo término es la suma para todos lso pares de nucleones, y el tercero recoge las interacciones entre cada nucleón y cada electrón. 


\subsection{Orbitales moleculares y el método auto-consciente}

La aproximación de Bohr-Oppenhaimer 


\subsection{Orbitales moleculares}




