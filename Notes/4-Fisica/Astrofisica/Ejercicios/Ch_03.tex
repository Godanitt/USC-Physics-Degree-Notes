

\newpage
\section*{\textit{Ejercicios}}
\addcontentsline{toc}{section}{\textit{Ejercicios}}

\begin{Enunciado}
	\subsubsection{Pregunta 1}

	Define un parsec e estima a súa distancia en metros e anos luz.

\end{Enunciado}

Es bastante sencillo, ya que solo tenemos que aplicar la ecuación 

\begin{equation*}
    1 \text{pc} = \frac{1\ua}{\tan \delta} \simeq \frac{1 \ua}{1''} = 3.26 \ \text{años luz}
\end{equation*}
que en metors 1 pc serán $\SI{3.08e+13}{m}$.

\vspace*{2em}

\begin{Enunciado}
	\subsubsection{Pregunta 2}

	Calcula o número de fotóns emitidos por unha bombilla de 100 W con $\lambda = 500\,\text{nm}$.

\end{Enunciado}

El número de fotones emitidos es sencillo, ya que si la energía de cada fotón es:

\begin{equation*}
    E_\gamma = h \nu = \frac{h c}{\lambda}
\end{equation*}
y tenemos en total una potencia $P=100$ J/s (recordamos [W]=[J/s]) tenemso pues:

\begin{equation}
    n_{\gamma} = \frac{P}{E_\gamma}
\end{equation}
siendo $n_{\gamma}$ la tasa de emisión de fotones.


\vspace*{2em}

\begin{Enunciado}
	\subsubsection{Pregunta 3}

	Estima a radiancia da radiación solar a partir de $L_\odot = 3{,}8 \times 10^{26}\,\text{W}$, $R_\odot = 7{,}0 \times 10^8\,\text{m}$ e $d_\odot = 1{,}5 \times 10^{11}\,\text{m}$.

\end{Enunciado}

La radiancia se define como \textit{la potencia por unidad de area y ángulo sólido} 

\vspace*{2em}

\begin{Enunciado}
	\subsubsection{Pregunta 4}

	Calcula o brillo superficial do Sol visto desde a Terra e desde Xúpiter. Datos: constante solar $K_\odot = 1{,}4 \times 10^3\,\text{W}\,\text{m}^{-2}$ e diámetro angular do Sol $\alpha = 32'$.

\end{Enunciado}
Solución

\vspace*{2em}

\begin{Enunciado}
	\subsubsection{Pregunta 5}

	Calcula o fluxo total dun disco galáctico con $I(r) = I_0 \exp(-r/R)$.

\end{Enunciado}
Solución


\vspace*{2em}

\begin{Enunciado}
	\subsubsection{Pregunta 6}

	Compara o tamaño dunha xigante vermella (de temperatura superficial $T \sim 4000\,\text{K}$) e unha xigante azul ($T \sim 20000\,\text{K}$) que teñan a mesma luminosidade.

\end{Enunciado}
Solución

\vspace*{2em}

\begin{Enunciado}
	\subsubsection{Pregunta 7}

	Na estrela Vega, a segunda máis brillante do hemisferio norte, a súa liña espectral H$\alpha$, que debera aparecer con $\lambda = 656{,}28\,\text{nm}$, obsérvase na posición $\lambda = 656{,}25\,\text{nm}$. Estimar o movemento radial relativo con respecto a nós.

\end{Enunciado}
Solución

\vspace*{2em}

\begin{Enunciado}
	\subsubsection{Pregunta 8}

	Unha estrela presenta unha liña H$\alpha$ (de $\lambda = 656{,}28\,\text{nm}$) na posición $\lambda = 656{,}9\,\text{nm}$. A que velocidade se alonxa de nós a estrela? Nun quasar obsérvase a liña L$\alpha$ (de $\lambda = 121{,}6\,\text{nm}$) en $\lambda = 500\,\text{nm}$. A que velocidade se alonxa o quasar?

\end{Enunciado}
Solución

\vspace*{2em}

\begin{Enunciado}
	\subsubsection{Pregunta 9}

	Calcula o desprazamento Doppler da liña de $363{,}4\,\text{nm}$ do Ca II para un membro do cúmulo de galaxias de Hydra con velocidade radial de alonxamento de $60900\,\text{km/s}$. En que rexión do espectro se observa a liña?

\end{Enunciado}
Solución

\vspace*{2em}

\begin{Enunciado}
	\subsubsection{Pregunta 10}

	Mídese o espectro global dunha galaxia espiral e obsérvase que a liña H$\alpha$ ($\lambda = 656{,}3\,\text{nm}$) ten unha anchura típica de $0{,}5\,\text{nm}$. Estimar a orde de magnitude da súa velocidade de rotación.

\end{Enunciado}
Solución

\vspace*{2em}

\begin{Enunciado}
	\subsubsection{Pregunta 11}

	Determinar a lonxitude de onda do fotón emitido por un átomo de H nunha transición entre os niveis $n_i = 110$ e $n_f = 109$.

\end{Enunciado}
Solución

\vspace*{2em}

\begin{Enunciado}
	\subsubsection{Pregunta 12}

	Sirio B ten unha masa de $0{,}96\,M_\odot$, unha temperatura de $5000\,\text{K}$ na súa superficie e unha magnitude absoluta de $M = 11{,}54$. Calcula a súa densidade.

\end{Enunciado}
Solución

\vspace*{2em}

\begin{Enunciado}
	\subsubsection{Pregunta 13}

	Un sistema binario ten dúas estrelas de magnitudes $m_1 = 1$ e $m_2 = 2$. Determina a magnitude total do sistema binario.

\end{Enunciado}
Solución

\vspace*{2em}

\begin{Enunciado}
	\subsubsection{Pregunta 14}

	A magnitude absoluta dunha estrela é $M = -2$ e a súa magnitude aparente é $m = 8$. Calcula a distancia á que se atopa a estrela. Debido á materia interestelar prodúcese o fenómeno da extinción, cun valor medio de $\Delta m \approx 2\,\text{mag/kpc}$. Calcula a distancia á que se atopa a estrela tendo en conta esta extinción media.

\end{Enunciado}
Solución

\vspace*{2em}

\begin{Enunciado}
	\subsubsection{Pregunta 15}

	Un fotómetro fotoeléctrico determina para unha estrela unha magnitude visual $m = 2{,}5$. A súa curva da intensidade radiante ten un máximo para $\lambda = 400\,\text{nm}$. A súa paralaxe medida sobre o fondo do ceo é de $\pi = 0{,}174''$ e cun interferómetro mídese un radio angular de $0{,}0010''$. Calcúlese para a estrela: a temperatura, distancia, magnitude absoluta visual, radio, luminosidade, magnitude bolométrica absoluta e corrección bolométrica.

\end{Enunciado}
Solución

\vspace*{2em}

\begin{Enunciado}
	\subsubsection{Pregunta 16}

	A partir dos índices de cor medidos dunha estrela dedúcese que é do tipo espectral G2V. Supoñendo que a súa magnitude bolométrica aparente é $12{,}72$, calcula a distancia á estrela usando o método da paralaxe espectroscópica.

\end{Enunciado}
Solución

\vspace*{2em}

\begin{Enunciado}
	\subsubsection{Pregunta 17}

	O cúmulo globular M13 na constelación de Hércules ten un radio aparente de $12'$ e unha magnitude aparente visual de 6. As estrelas variables RR Lyrae ($V = 0{,}75$) do cúmulo teñen unha magnitude aparente de $14{,}4$. Determina a distancia e o radio do cúmulo. Estima a masa do cúmulo supoñendo que se compón de estrelas de tipo solar.

\end{Enunciado}
Solución

\vspace*{2em}

\begin{Enunciado}
	\subsubsection{Pregunta 18}

	A partir dunha táboa de estrelas cercanas cos seus datos, debuxa o diagrama HR. Determina a que clase de luminosidade corresponde cada unha de forma aproximada.

\end{Enunciado}
Solución

\vspace*{2em}

\begin{Enunciado}
	\subsubsection{Pregunta 19}

	As cefeidas da Gran Nube de Magallanes (LMC) cun período de 10 días teñen unha magnitude de $m_1 = 13{,}5$. As cefeidas de igual período atopadas en Andrómeda teñen magnitude $m_2 = 19{,}25$. Se a distancia á LMC é de $50\,\text{kpc}$, determina a distancia á galaxia de Andrómeda.

\end{Enunciado}
Solución

\vspace*{2em}

\begin{Enunciado}
	\subsubsection{Pregunta 20}

	Unha estrela ten o máximo da súa intensidade de radiación emitida nunha lonxitude de onda de $400\,\text{nm}$. Sabendo que as súas magnitudes bolométricas aparentes e absolutas son respectivamente 2 e $-4{,}5$, determina a súa distancia, luminosidade e radio.

\end{Enunciado}
Solución

\vspace*{2em}

\begin{Enunciado}
	\subsubsection{Pregunta 21}

	A nebulosa do Cangrexo é o resto dunha supernova. O seu tamaño angular é de $3'$. Analizando o seu espectro obsérvase unha velocidade de expansión duns $1500\,\text{km/s}$, e visualmente amplíase en $\alpha = 0{,}21''\,\text{ano}^{-1}$. Cando tivo lugar a explosión da supernova? A que distancia está?

\end{Enunciado}
Solución

\vspace*{2em}

\begin{Enunciado}
	\subsubsection{Pregunta 22}

	Atopar a relación entre a magnitude absoluta $M$ e a luminosidade $L$ para unha estrela a partir dos valores para o Sol. Nota: $M_{b\odot} = 4{,}72$ e $L_{b\odot} = 4 \times 10^{26}\,\text{W}$.

\end{Enunciado}
Solución

\vspace*{2em}

\begin{Enunciado}
	\subsubsection{Pregunta 23}

	A superficie do Sol ten aproximadamente $5 \cdot 10^5$ átomos de H por cada átomo de Ca. Calcula a intensidade relativa entre as liñas da serie de Balmer do H e as liñas H e K do Ca II. Asume unha temperatura $T = 5777\,\text{K}$ e unha presión electrónica constante de $P_e = n_e k_B T = 1{,}5\,\text{N/m}^2$.

\end{Enunciado}
Solución

\vspace*{2em}

\begin{Enunciado}
	\subsubsection{Pregunta 24}

	Un sistema binario ten dúas estrelas ananas con temperaturas de $5500\,\text{K}$ e $4800\,\text{K}$. Cal é a relación entre as intensidades da liña do Fe II que ten un potencial de excitación de $3\,\text{eV}$? O potencial de ionización do Fe I é de $7{,}9\,\text{eV}$ e asumiremos que a densidade electrónica é a mesma.

\end{Enunciado}
Solución

\vspace*{2em}

\begin{Enunciado}
	\subsubsection{Pregunta 25}

	Calcula a fracción de átomos $n_2/n_1$ para o ión de He II en función da temperatura. A que temperatura se produce a inversión de poboacións? De acordo cos resultados, pódese esperar ver as liñas espectrais do He II en estrelas das clases espectrais O e B?

\end{Enunciado}
Solución

\vspace*{2em}

\begin{Enunciado}
	\subsubsection{Pregunta 26}

	Calcula a relación de átomos H II / H I e de He II / He I en función da temperatura. Fai un esquema aproximado de ambos comportamentos.

\end{Enunciado}
Solución

\vspace*{2em}

\begin{Enunciado}
	\subsubsection{Pregunta 27}

	A partir dunha táboa de función de partición para os distintos elementos, compara as funcións de partición do He I e do Ne I para varias temperaturas. Tendo en conta os resultados, determina cal dos dous átomos é máis fácil de ionizar.

\end{Enunciado}
Solución

\vspace*{2em}

\begin{Enunciado}
	\subsubsection{Pregunta 28}

	Determinar a densidade media do Sol e de aí a súa presión e temperatura a unha profundidade que sexa a metade do seu radio.

\end{Enunciado}
Solución

\vspace*{2em}

\begin{Enunciado}
	\subsubsection{Pregunta 29}

	Supoñamos que nunha estrela podemos romper o equilibrio entre presión e forza gravitatoria nun factor 0{,}1 durante 15 minutos. Trata de determinar a orde da correspondente variación do radio da estrela.

\end{Enunciado}
Solución

\vspace*{2em}

\begin{Enunciado}
	\subsubsection{Pregunta 30}

	Estima a relación entre a densidade de enerxía en forma de materia e en forma de radiación no interior dunha estrela.

\end{Enunciado}
Solución

\vspace*{2em}
\begin{Enunciado}
	\subsubsection{Pregunta 31}

	Determina a masa molecular media do Sol, tomando para as fraccións relativas de masa os valores $X = 0{,}71$, $Y = 0{,}27$, $Z = 0{,}02$ (fraccións de H, He e outros elementos, respectivamente).

\end{Enunciado}
Solución

\vspace*{2em}

\begin{Enunciado}
	\subsubsection{Pregunta 32}

	Resolve as ecuacións de equilibrio dunha estrela baixo as condicións máis sinxelas de integración.

\end{Enunciado}
Solución

\vspace*{2em}

\begin{Enunciado}
	\subsubsection{Pregunta 33}

	Determina a distancia total que percorre un fotón producido no centro do Sol ata que alcanza a súa superficie, asumindo un coeficiente de absorción óptica $k_\lambda = 10\,\text{m}^2/\text{kg}$. Estima tamén o tempo que tarda en facelo.

\end{Enunciado}
Solución

\vspace*{2em}

\begin{Enunciado}
	\subsubsection{Pregunta 34}

	Canto se espera que sexa a vida media da estrela Vega (alpha-Lyrae)? Os datos son $V = 0{,}03$, $B - V = 0{,}0$, $r = 25{,}3$ anos luz.

\end{Enunciado}
Solución

\vspace*{2em}

\begin{Enunciado}
	\subsubsection{Pregunta 35}

	O Sol xira cun período medio de 24 días. Cal será o seu período cando se convirta nunha enana branca?

\end{Enunciado}
Solución

\vspace*{2em}

\begin{Enunciado}
	\subsubsection{Pregunta 36}

	Unha estrela A posúe unha magnitude aparente $m_A = 12$. A estrela B é $10^4$ veces máis brillante que A, e outra estrela C é á súa vez $10^4$ veces menos brillante que A. Determina as magnitudes aparentes de B e C.

\end{Enunciado}
Solución

\vspace*{2em}

\begin{Enunciado}
	\subsubsection{Pregunta 37}

	Un sistema dobre visual ten un semieixo maior de dimensión angular $\alpha = 11{,}99''$, un período de 430 anos e unha paralaxe de $0{,}17''$. Determina a masa de ambas estrelas.

\end{Enunciado}
Solución

\vspace*{2em}

\begin{Enunciado}
	\subsubsection{Pregunta 38}

	Unha cefeida ten $V = 14{,}30$, $B = 15$, sendo o seu período de 8 días. A que distancia se atopa? Onde é probable que se atope, a esa distancia?

\end{Enunciado}
Solución

\vspace*{2em}

\begin{Enunciado}
	\subsubsection{Pregunta 39}

	Unha estrela está a unha distancia de 20 pc calculada usando a súa paralaxe trigonométrica, cunha incerteza de $0{,}005''$. Se a magnitude non ten sesgo, cal é a incerteza asociada á súa magnitude absoluta?

\end{Enunciado}
Solución

\vspace*{2em}

\begin{Enunciado}
	\subsubsection{Pregunta 40}

	Rigel ten unha magnitude visual $V = 0{,}18$, $B - V = -0{,}03$ e está a 250 pc. Determinar o seu radio.

\end{Enunciado}
Solución

\vspace*{2em}

\begin{Enunciado}
	\subsubsection{Pregunta 41}

	Determina a masa do burato negro do centro galáctico sabendo que a estrela S2 que orbita ao seu redor ten un período orbital de $P = 15{,}2$ anos, unha excentricidade $e = 0{,}87$ e unha distancia no periastro de $r_p = 1{,}8 \cdot 10^{13}\,\text{m}$.

\end{Enunciado}
Solución

\vspace*{2em}

\begin{Enunciado}
	\subsubsection{Pregunta 42}

	Deriva as leis de Kepler a partir de principios fundamentais. Para a primeira, aplica que o momento angular dun sistema é constante baixo unha forza central. Para a terceira, podes integrar a segunda sobre unha órbita.

\end{Enunciado}
Solución

\vspace*{2em}
