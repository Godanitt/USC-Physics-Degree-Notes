		
	El máximo de horas ocurre cuando estamos el solsticio de verano. En este caso sabemos que $\delta=\epsilon$. Usando las ecuaicones del primer ejercicio:
		
	 \begin{equation}
	 	H_0 = 7^h 34^m 57^s \Rightarrow 2H_0 = 15^h 9^m 54^s
	 \end{equation}
	 El mínimo de horas del sol es en el solsticio de invierno.  En este caso
		
	 \begin{equation}
	 	\delta = - \varepsilon \Rightarrow 2H_0 = 8^h 50^m 4^s
	 \end{equation}
