
	Tenemos la ecuación para conocer el ángulo:

	\begin{equation}
		r(f) = \frac{c^2/ \mu}{1+e\cos(f)}
	\end{equation}
	Cuando $e=1$ tenemos perihelio $f=0$. El valor de $r_p = c^2/2\mu$. El valor de la masa reducida $$\mu=G(m_{\text{sol}}+m_{\text{cometa}})\simeq G m_{\text{sol}} = 2.95\cdot 10^{-4} \ua^3/d$$(ddespreciamos la masa del cometa).Ahora calculamos

	\begin{equation}
		c=|\rn\wedge\vn|=0.015 \*ua^2/
	\end{equation}
	de lo que puedo obtener el valor de $r_p$:
	\begin{equation}
		r_p = \frac{c^2}{2\mu} = 0.379 \ua
	\end{equation}
