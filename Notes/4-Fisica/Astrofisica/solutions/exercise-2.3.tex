    Para este ejercicio tenemos que aplicar la tercera ley de Kepler. Dado que $M_{Sol}\gg M_{Marte} \gg M_{Fobos}$, tenemos que la leyes de Kepler de Marte orbitando al Sol y Fobos orbitando a Marte son (donde hemos supuesto que la masa reducida es igual a la masa del objeto masivo, siendo una buena aproximación si tenemos en cuenta los muchos ordenes de magnitud que las separan entre sí):

    \begin{equation}
        T_{\Marte}^2 = \frac{4\pi a_{\Marte}^2}{GM_{\Sol}}   \tquad
        T_{\Fobos}^2 = \frac{4\pi a_{\Fobos}^2}{GM_{\Marte}}
    \end{equation}
    Es evidente que podemos despejar $M_{Marte}$:

    \begin{equation}
        M_{\Marte} = \frac{a_\Fobos^3}{a_\Marte^3} \frac{T_\Marte}{T_\Fobos} M_\Sol
    \end{equation}
    de lo que se deduce $M_\Marte\approx 6.475 \cdot 10^{23}$ kg.
