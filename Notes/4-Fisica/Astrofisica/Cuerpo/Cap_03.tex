\chapter{Fundamentos de la astrofísica}

\section{Medidas de las distancias}

\subsection{Escalas galácticas}

Existen otras maneras que permiten las distancias cuando ya no funciona el paralaje. Los métodos se van solapando en una escalera de distancas cósmicas mediante la utilzación de candelas estándar, objetos qeu tienen luminosidad $L$ conocida o una característica que permite usarlas para medir la distancia en función del flujo observado. A escala galáctica tendremos:

\begin{itemize}
    \item \textit{Paralaje dinámico de estrellas binarias}.
    \item \textit{Variables Cefeida}. 
    \item \textit{Variables RR Lyrae}.
    \item \textit{Paralaje espectroscópico}.
    \item \textit{Binarias eclipsantes}.
\end{itemize}

\subsection{Escalas extragalácticas}

Para completar la escalera de distancas cósmicas tenemos las escalas extragalácticas:

\begin{itemize} 
    \item \textit{Candela estelar}. Son cefeidas que se pueden detectar, sobretodo las más luminosas.
    \item \textit{Características comúns}. Podemos supoenr que algunos cúmulos son parecidos a los que ya conocemos y relacionar entonces la luminosdidad vista con la distancia.
    \item \textit{Supernovas tipo I}. Este es el método que más se usa junto con el corrimiento al rojo. Usa las violentas explosiones de supernovas, de las cuales conocemos el brillo esperado, para poder determinar cuan lejos está. 
    \item \textit{Corrimiento al rojo}.
\end{itemize}

\section{Medidas de masa}

\subsection{Estrellas binarias}

Las estrellas dobles o binarias son pares de estrellas que orbitan alrededor del centro común de masas. Estímase que la mitad de las estrellas observables en el cielo agrúpanse en estrellas binarioas o sistemas múltiples. El estudio de las estrellas binarias permite calcular la masa y radios estelares y lasd istancias relativas de las estrellas, mediante procedimientos básicos de determinación de elementos orbitales. 

\begin{itemize}
    \item Binarias visuales: las dos estrellas son detectables opticamente y sus elementos orbitaels pueden determinarse a través de observaiones seperaadas en el tiempo. Coñecese alrededor de mil setecientos binarias visuales.
    \item Binarias astrométricas: solo una de las estrellas es visible (por ejemplo una gigante roja y una estrella de neutrones, una será visible y otra no, pero la masa será similar, lo que hace que orbiten respecto un centro de masas, lo que si permitirá detectarlas, pero solo con el movimiento oscilatorio).
    \item Binarias eclipsantes: si las estrellas están orientadas con su plano orbital en la línea de visión cara la tierra, la curvatura de luz detectada mostrará variaciones (mínimos) periódicos cuando una de las estrellas eclipse total o parcialmente a la luz de la compañera. En muchas ocasiones permite medir las deformaciones y distorsiones o incluso la transferencia de masa entre estrellas. 
    \item Binarias especcrtrales: en la media del espectro de las estrellas se observa características que corresponden a dos tipos de espectralesdiferentes, en ocasiones con corrimientos Doppler diferentes
    \item Binarias espectroscópicas: las líena espectrales de cada componente del par se desplazan en diferente sentido (cara azul o rojo) alterantivamente. 
\end{itemize}
Esta clasificación no es mutuamente exclusiva: ciertas binarias se estudian por diferentes procedimientos.

\subsection{Estrellas binarias: las leyes de kepler}

La primera ley de kepler nos dice que todos los objetos en una línea vinaria se mueve alrededor del centro de masas en órbitas elípticas, siendo el centro de masa uno de los focos de la elipse. La segunda ley de Kepler nos dice que el radio vcto entre dos objetos en órbita y el centro de masa recorre áreas iguales en tiempos iguales 

\begin{equation*}
    \derivadas{A}{t} = \frac{1}{2} \frac{L}{\mu}
\end{equation*}
La tercera ley de kepler nos dice que el cuadrado del período orbital es directamente proporcional al cubo de la longitud del semieje mayor de la órbita elíptica alrededor del centro de masa e inversamente propircional a la masa total del sistema. 

\begin{equation*}
    P^2 = \frac{4\pi^2}{G(m_1+m_2)} a^3
\end{equation*}

\subsection{Cociente de las masas}

Partiendo de los elementos orbitales del par de estrellas se puede determinar el \textbf{coceinte de las masas}. Para este cálculo tenemos qeu determinar el plano orbital y proyectar las observaciones en ese plano, por definición del centro de masas. 

\begin{equation*}
    \frac{m_1}{m_2} = \frac{a_2}{a_1}
\end{equation*}

\subsection{Binarias eclipasntes, espectroscópicas y espectrales}

Existe una clara relcaión entre la medida de la masa y la luminosidad (en magnitudes absolutas o en normalizada a Sol). Nótese que la escala logarítmica en ambos casos y en ambos ejes (la magnitud es una escala logarítmica). Más adelante veremos qeu esto fue uno de los indicativos para construir el diagrama de Herzsprung-Russel (luminosidad-temperatura).

\section{Composición del Universo y el Sistema Solar}

El orgen de los elementos en la creación del sistema solar: el sistema solar se formo a partri del colapso de una nebulosa gaseosa con abundancias químicas e isotópicas uniformes y coincidentes con las observadas en otras partes del universo (obtenidas mediante mediciones espectroscópicas, granos presolares en meteoritos primitivos). Las abundancias químicas se obtiene la forma independiente y complementaria:

\begin{itemize}
    \item Observacioens de la fotoesfera solar.
    \item Análisis de una clase particualr de meteoritos.
\end{itemize}


