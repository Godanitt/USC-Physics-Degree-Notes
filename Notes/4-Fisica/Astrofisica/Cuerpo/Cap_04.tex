

\chapter{Teoría de la radiación}

\section{Propiedades corpusculares de la radiación}

Para explicar el cuerpo negro es improtante recordar las siguiente deficiones y conceptos: 

\begin{itemize}
    \item Radiación térmica: la emitida por un cuerpo como consecuencia de su temperatura.
    \item Cuerpo negro: emisor ideal, aquel cuya superifice abosrbe toda radiación que incide sobre él.
    \item Radiación espectral: energía emitida en forma de radiación por un cuerpo negro a una temperatura $T$ constante, con frecuencia en el intervalo ($v$,$v+\D v$), por unidad de ángulo sólido, área proyectada y tiempo. 
\end{itemize}
A finales del siglo XIX las observaciones de los espectros térmicos no concordaban con con las prediciones:

\begin{itemize}
    \item Ley de Stefan-Boltzman. La referencia total emiitda por $B_T = \sigma T^4$. 
    \item Ley del desplazamiento de Wien. $\lambda_{\max} = C / T$ siendo $C$ una constante y $\lambda_{\max}$ es la longitud de onda del máximo de $B(\lambda)$. 
    \item Teoría de Plank: la densidad de energía (energía por volujmen y longitud de onda/frecuencia) en una caja es:
    \begin{equation}
        \rho_\tau (\lambda) \D \lambda = \frac{8 \pi \hbar c^2}{\lambda^5} \frac{\D \lambda}{e^{hc/k_BT}-1}
    \end{equation}
    qeu se deduce a partir del postulado de Plank: los entes físicos con un grado de liberatad, dode su coordenada característica es función sinosuidal con el timepo, solo pueden tener energías totales cuantizadas, según $E=n h \nu$ con $n=1,2,3...$ El ajuste de los datos experimentales con la teoría es impresionante. 
\end{itemize}

\subsection{Aplicaciones astrofísicas}

La radiancia espectrtal (intensidad de radiación o energía por unidad de área, tiempo y ángulo sólidio emitida en el rango de una longiturd de onda entre $\lambda$ y $\lambda + \D \lambda$). Una vez integrada podemos obtener la luminosidad:

\begin{equation}
    L = 4 \pi R^2 \sigma T^4_e
\end{equation}