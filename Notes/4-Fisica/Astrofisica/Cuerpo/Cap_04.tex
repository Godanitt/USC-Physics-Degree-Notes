
\chapter{Cosmología}

\section{¿Qué es la cosmología?}

La cosmología es la ciencia qeu estudia el Universo como un todo.

\subsection{Pilares básicos de la cosmología}

Hay 5 pilares básicos en los que está fundametnada la cosmología:

\begin{itemize}
	\item El principio cosmológico de Albert Einstein, 1917.
	\item La ley de hubble y el descubrimiento de la expansión del Universo de Edwin Hubble, 1929.
	\item La Teoría de la Relatividad General de Albert Eistein, 1915.
	\item El descubrimiento del Fondo de Microondas por Arno Pezias y Robert Wilson en 1965.
	\item La Teoría de la Nucleosíntesis de ralph ALpher, George Gamov y otros, 1948.
\end{itemize}

\section{El principio cosmológico}

El \textbf{principio cosmológico} dice así: el universo es homogéneo, isótropo a grandes escalas. Fue introducido por Einstein en 1917 por la simmplicidad matemática en la resolución de ecuaciones de la Relatividad General. Edward A. Milne fue quien le otorgo dicho nombre en 1933. La homogeneidad e isotropía no se peuden inferir directamente de las observaciones, auqneu haya indiciones experimentales a favor como veremos a continuación. Son, sin embargo, dos hipótesis que deberían ser confirmadas o desmentidas por las consecuencias observables que se deriven de su utilización. Veamos entonces por separado qeu significa homogeo y que significa isótropo.


\subsection{Homogeneidad}

Se le llama \textbf{homogeneidad} a la invariancia bajo traslaciones en el espacio. No hay lugares preferidos en el universo, no se peude distinguir un punto de otro. Ahora bien, ¿Que consideramos por un ``universo homogéneo''? Está claro que a distancias pequeñas el universo no es homogéneo ya que hay masas ``puntuales'' muy grandes (planetas, estrellas) mientras que la mayor parte está vacío. Cuando Einstein propuso dicha homogegeindad se refería a que \textit{todos los puntos tienen las mismas propiedades a grandes escalas} entorno a 10 Mpc - 1 Gpc (misma densidad, temperatura). Las evidencias de homogeneidad:

\begin{itemize}
	\item \textbf{Fluctuaciones relativas de la densidad}. Mediante observaciones astronómicas de diversos tipos se peude medir la densidad promedio $\rho_i$ en un cubo de arista la $L$ centrado en un determinado punto $i$ del universo. Repitiendo la medida para muchos puntos distintos, calculando la media $\langle \rho \rangle $ y la dispersión $\delta \rho$, podremos obtener que:
	      \begin{equation}
		      \left| \frac{\delta \rho}{\langle \rho \rangle} \right| \rightarrow 0
	      \end{equation}
	\item \textbf{Compendios de galaxias}. Con potentes telescopios y software para el contaje automático se obtienen los ``surveys'' de galaxias. Actualmente la inspección lleva a concluir que la hipótesis es razonable.
\end{itemize}
\subsection{Isotropía}

Se le llama \textbf{isotropía} a la invariancia bajo rotacioens en el espacio, es decir, que las propiedades del Universo son las mismas en todas las direcciones. Las evidencias son:

\begin{itemize}
	\item \textbf{El fondo de microondas.} La radiación de fondoo (1989) George Smoot y su equipo utilizó el satélite COBE y descubrieron que la radiación de fondo tiene un espectro que corresponde a la de un cuerpo negro caracterizado por una temperatura promedio. Observado a distintas direcciones, se determinó que las fluctuaciones respecto al valor medio es de una parte en 10$^5$, es decir, extremadamente pequeñas.
\end{itemize}

Para ser exactos la isotropía de fondo de microondas solo garantiza que el universo es isótropo en torno a nuestro Sistema Solar, para aplicar esto universalemnte necesitaríamos que fuera homogéneo: así la Tierra sería un punto más.

\subsection{Implicaciones y comentarios sobre el tiempo}

Es importante tener en cuenta que ni la homogeneidad implica isotropía, ni la isotropía en torno a un punto implica homogeneidad. Sobre el tiempo:

\begin{itemize}
	\item Si en cada punto del Universo ponemos un reloj, todos deberían estar sincronizados (marcar el mismo tiempo) ya que sino no tendríamos una manera de distinguir un punto de otro. A este tiempo lo llamaremos \textbf{tiempo cósmico} $t$ y será medido en el llamado \textit{sistema de referencia comóvil}. Es también el tiempo qeu ha transcurrido desde el Big Bang despreciando a grandes escalas los efectos que sobre el tiempo tiene la Relatividad Especial así como efectos locales debidos a la Relatividad General. El tiempo cósmico sería también el tiempo que mediría un observador que ve el Universo expandiéndose de manera homogénea a su alrededor.
	\item El Universo es homogéneo e isótropo en cada instante $t$ de tiempo cósmico pero no es estático (evoluciona y varía con el tiempo). La densidad $\rho(t)$ es la misma en todos los puntos del espacio, pero no así igual en todos los instantes del tiempo $t$.
\end{itemize}

\section{La ley de de Hubble}

Si una fuente de luz con longitud de odna $\lambda_e$ que está en movimiento relativo respecto a otro sufre desplazamiento Doppler, tal que la longitud de onda observada $\lambda_o$ es diferente. En función de si se acerca o se aleja veremos una disminución o aumento de la longitud de onda (i.e. aumento o disminución de la frecuencia) respectivamente. En términos de la longitud de onda visible diríamos que se desplaza al azul al disminuir la longitud de onda o que se desplaza al rojo si aumenta. Para esto usamos el parámetro adimensional $z$ llamado \textbf{parámetro z}.

\begin{Resaltar}
	\begin{center}
		\textbf{Definición corrimiento al rojo}
	\end{center}
	\begin{equation}
		z = \frac{\lambda_o - \lambda_e}{\lambda_e}
	\end{equation}
\end{Resaltar}
que es capaz de decirnos si se desplaza al rojo $z>0$ o al azul $z<0$, o si no hay movimiento relativo $z=0$.

Tras diferentes medidas se observó que la mayor parte de los cuerpos presentan corrimiento al rojo. Usando los resultados, en 1929 relacionó el corrimeinto al rojo y la distancia a la que se encontraba el objeto en la llamada \textbf{ley de Hubble empírica}

\begin{Resaltar}
	\begin{center}
		\textbf{Ley de Hubble empírica}
	\end{center}
	\begin{equation}
		z = \frac{H_0}{c}d
	\end{equation}
\end{Resaltar}
donde $H_0$ es la \textit{cosntante de Hubble} siendo válida para distancias pequeñas y $z$ pequeñas, ya que luego se desvía de la linealidad. Usando que el corrimento al rojo $z=V/c$ siendp $V$ la \textit{velocidad de recesión medio a distancia d}. Así tenemos la \textbf{ley de Hubble teórica}

\begin{Resaltar}
	\begin{center}
		\textbf{Ley de Hubble empírica}
	\end{center}
	\begin{equation}
		V= H_0 d
	\end{equation}

\end{Resaltar}
El valor dado por Hubble (actual) es de $H_0=\SI{70}{km/s\cdot Mpc}$, auqneu es difícil de determinar (requiere muchas medidas de objetos con longitud de onda conocida y con $d$ controlada).

\subsection{Expansión cosmológica y Big Bang}

La interpretación de la ley de Hubble no se hizo esperar, cuyo nombre es bastante descriptivo: \textbf{expansión cosmológica}. La expansión cosmológica es el alejamiento de las galaxias entre sí debido a la expansión del espacio.

De acuerdo con esto (y ojo, es una de las mas importantes), \textit{las galaxias en realidad son estacionarias, no se alejan por movimiento relativo, sino porque las distancias entre ellas se hace cada vez más grande}. Las galaxias son ``arrastradas'' por la expansión del universo, siendo esta la única manera de explicar que el desplazamiento de las longitudes de onda es casi siempre hacia longitudes de onda mayores, no dependiendo de la longirud de onda solo de la distancia al objeto.

Poco después llego otra de las grandes implicaciones: si las galxias es están alejando, entonces hubo un momento en el pasado en el que estaban todas más próximas, en universo extremadamente pequeño y extremadamente denso, momento que se llamaría \textbf{Big Bang}.

Hay que hacer varios comentarios:

\begin{itemize}
	\item Además de las velocidades de recesión $V_{\text{recesion}}$ también tendríamos efectivamente un término de velocidad cinética (de natuarelza aleatoria) que suele ser del orden de cien km/s y se debe a fenómenos de otra naturaleza, a saber, gravitatorios. Así la velocidad neta:
	      \begin{equation}
		      \Vn = \Vn_{\text{recesion}} + \Vn_{\text{peculiar}}
	      \end{equation}
	      si la distnacia $d$ a la galaxia es pequeña entonces peude ocurrir que $\Vn_{\text{recesion}}<\Vn_{\text{peculiar}}$, por lo que si puede haber galaxias que se acercan a nosotros. Ejemplo: Andrómeda.
\end{itemize}

\section{Nociones generales de Relatividad General}

La teoría de la relatividad general se fundamenta en el \textbf{principio de equivalencia}: no existe un experimento físico qeu permita distinguir entre un sistema de referencia uniformemente acelerado de otro en un campo gravitatorio uniforme, el cual tiene como consecuencia directa que \textit{la trayectoria de un rayo de luz se curva en presencia de un campo gravitatorio}, siendo esta curva la trayectoria más corta entre dos puntos del espacio. Es decir, la gravedad está influyendo directamente en la curvatura del espacio tiempo.

El experimento de Eddington  realizado en Santo Tomé y Príncipe en 1919 tenía como intención estudiar fotografías de las estrellas y determinar la posición de las mismas en la línea visual al Sol durante el eclipse, para luego compararlas con la posición conocida. la hipótesis sometida a las observaciones: el campo gravitatorio del Sol debía curvar los rayos de luz (según la relatividad general) por lo que debía haber un movimiento aparente de las estrellas, el cual ocurrió y coincidía con el mismo dado por la relatividad general.

Otras evidencias experimentales: lentes gravitacionales (distorsión de objetos lejanos en su camino hacia nosotros debido a efectos graviatcionales), micro-lentes gravitacionales (lentes gravitacionales pero para estrellas individuales), corrimiento al rojo gravitacional (aumento de la longitud de onda debido al aumento al superar un pozo de potencial), precesión del planeta mercurio (incompatible con la mecánica Newtoniana y predicha con la de la relatividad general) y ondas gravitacionales.

\subsection{Métrica}

La métrica nos permite obtener distancias entre 4-vectores, distancias que serán invariantes relativistas (en cualquier sistema de referencia obtendremos el mismo resultado, matemáticamente significa que podemos aplicar un boost de Lorentz y una rotación y obtendremos el mismo escalar). Basicamente definimos el diferencial de distancia en un 4-vector como:

\begin{equation}
	\D s^2 : = g_{\mu \nu} (x) \D x^\mu \D x^\nu
\end{equation}
donde $d$ es la distancia

\begin{equation}
	d^2 = \int_{x_1}^{x_2} \D s^2  =  g_{\mu \nu} (x) ( x_2^\mu x_2^\nu - x_1^\mu x_1^\nu)
\end{equation}
que

\begin{Resaltar}
	\begin{center}
		\textbf{Corrimiento al rojo}
	\end{center}
	\begin{equation}
		1 + z = \frac{a(t_0)}{a(t_1)}
	\end{equation}

\end{Resaltar}
\subsection{Espacios con curvatura constante}

La curvatura es complicada de definirse, y más difícil es calcular distancias, tiempos... con una curvatura no constante. Además si el espacio tuviera curvatura no constante, habría una manera de diferenciar un punto y otro, en contra del principio cosmológico. El espacio-tiempo es homogéneo e isótropo. Veamos entonces los 3 espacios con curvatura constante más típicos para definir la cosmología:

\begin{itemize}
	\item El \textbf{espacio plano} usando coordenadas polares:
	      \begin{equation}
		      \D s^2 = c^2 \D t^2 - \D l^2 \qquad \D l^2  = R^2 (\D r^2 + r^2 \D \theta^2 )
	      \end{equation}
	\item El \textbf{espacio esférico} usando coordenadas polares:
	      \begin{equation}
		      \D s^2 = c^2 \D t^2 - \D l^2 \qquad \D l^2 = \frac{R^2}{1-r^2}\D r^2 +R^2 r^2 \D \phi^2 + r^2 \sin^2 (\theta) \D \theta^2
	      \end{equation}
	\item El \textbf{espacio hiperbólico} usando coordenadas polares:
	      \begin{equation}
		      \D s^2 = c^2 \D t^2 - \D l^2 \qquad \D l^2 = \frac{R^2}{1-kr^2}\D r^2 +R^2 r^2 \D \phi^2 + r^2 \sin^2 (\theta) \D \theta^2
	      \end{equation}
\end{itemize}

\subsection{Covariancia e invariancia}

El \textbf{principio de Covariancia General} dice que las leyes de la física tienen la misma forma en todos los sistemas de referencia y son independientes de la elección de coordenadas. La manera más sencilla de implementar este principio es escribir las ecuacioens de manera covariante (con superíndices).

Las ecuaciones escritas de manera covariante conservan su forma en todos los sistemas de referencia inerciales, incluyendo los no inerciales.

\subsection{Ecuacion de la Relatividad General}

La \textbf{ecuación de Einstein de la relatividad general} relaciona la curvatura/métrica del espacio y el tensor de energía momento (distribución de energía-masa en el universo)

\begin{Resaltar}
	\begin{center}
		\textbf{Ecuación de Einstein}
	\end{center}
	\begin{equation}
		G_{\mu \nu} = R_{\mu \nu} - \frac{1}{2} g_{\mu \nu} R = \frac{8 \pi G}{c^4} T_{\mu \nu}
	\end{equation}
\end{Resaltar}
Esta ecuación en realidad son 10 ecuaciones diferenciales acoplatadas con 10 incógnitas: o $G_{\mu \nu}$ dado $T_{\mu \nu}$ o al revés. Es importante saber que esta ecuación es no lineal: la suma de dos tensores de energía momento no da como resultado la suma de dos métricas correspondientes a la solución de cada uno de los tensores. La propia curvatura del espacio es una fuente de energía que entra en el tensor energía-momento. Se encuentran pocas soluciones de esta ecuación (analíticas).


\section{Métrica de Robertson-Walker}

La solución de la ecuació n de Einstein de la Relatividad General para una distribución de energía-momento homogénea e isótropoa se conoce como \textbf{métrica de Robertson-Walker}.

\begin{Resaltar}
	\begin{center}
		\textbf{Métrica de Robertson-Walker}
	\end{center}
	\begin{equation}
		\D s^2 = c^2 \D t^2 - a^2(t) \ccorchetes{\frac{\D r^2}{1-k^2 r^2} + r^2 (\D \theta^2 + \sin^2 \theta \D \phi^2)}
	\end{equation}
\end{Resaltar}
En esta métrica:

\begin{itemize}
	\item A las coordenadas $(r,\theta,\phi)$ se les llama coordenadas comóviles.
	\item A $t$ se le llama tiempo cósmico.
	\item A $k$ se le llama curvatura, y por tanto es constante y no depende de las coordenadas. Existen 3 posibilidades para un espacio homgéneo e isótropo: $k=1$ (esférico), $k=0$ (plano) y $k=-1$ (hiperbólico).
	\item A $a(t)$ se le llama \textit{parámetro de escala} o \textit{factor de expansión} y nos dice cuanto se ha expandido/contraido la parte espacial de la métrica. En general depende del tiempo cósmico, y tiene unidades de distancia.
	      La forma de este funcional está determinada por el \textit{modelo cosmológico} y no por la métrica.
\end{itemize}


\subsection{Distancia comóvil y distancia propia}

Las coordenadas $(r,\theta,\phi)$ en las que se expresa la métrica RW se denominan coordenadas comóviles, coordenadas en un sistema de reposo con la expansión del universo, se mueve con esta expansión. En este sistema las galaxias están en reposo. Un ejemplo para entenderlo: las coordendas latitud y longitud de una esfera son coordenadas comóviles, ya que no cambian al aumentar o disminuir el radio de la esfera.

Se define como \textbf{distancia comóvil} entre $a$ y $b$ como:

\begin{equation}
	\bar{r}_{ab} = \int_{r_a}^{r_b} \frac{\D r}{\sqrt{1-kr^2}}
\end{equation}
dependiendo únicamente de la coordenada $r$ ya que al ser homogéneo e isótropo la distancia no puede depender de la dirección.

La distancia comóvil entre dos galaxias es siempre la misma, no depende de $t$. Sin embargo la  \textbf{distancia real} o \textbf{distancia propia} entre las galaxias $a$ y $b$ cambia debido a la expansión del universo, tal que la distancia propia $d_{ab}$ varía como:

\begin{equation}
	d_{ab} = a(t) \bar{r}_{ab}
\end{equation}
La distancia propia entre dos galaxias varía con $t$ de la misma manera que $a(t)$.


De esta manera el aumento de distancia entre dos puntos del Universo no se debe al aumento de sus coordenadas relativas sino a que el espacio se está expandiendo, es decir, a la variación de la parte espacial de la métrica dada por $a(t)$.

\subsection{Propagación de la luz en el espacio RW}

La métrica es lo único que necesitamos para determinar como se propaga un rayo de luz por el espacio. Para un rayo de luz se verifica que:

\begin{equation}
	\D s^2 = 0  \Rightarrow c^2 \D t^2 = a^2(t) \ccorchetes{\frac{\D r^2}{1-kr^2}+r^2 \parentesis{\D \theta^2 + \sin^2 \theta \D \phi^2}}
\end{equation}
En el espacio homogéneo e isótropo 3D de Robertson-Walker la distancia entre dos puntos no puede depender de la dirección dada por las coordenadas ($\theta,\phi$) y puede considerarse cualquier dirección con $(\theta,\phi)$ constantes, es decir, $\D \theta = \D \phi = 0$.

\begin{equation}
	\frac{c^2 \D t^2}{a^2(t)} = \frac{\D r^2}{1-kr^2} \Rightarrow  \frac{c \D  t}{a(t)} = \pm\frac{\D r}{\sqrt{1-kr^2}}
\end{equation}
donde el signo $\pm$ se elige dependiendo de si $r$ aumenta o disminuye cuando $t$ aumento.


\subsection{Corrimiento al rojo}

Obtengamos ahora una relación de gran importancia entre el corrimiento al rojo $z$ de una galaxia a una distancia comóvil $r$ y el tamaño del Universo en el instante $t$ en el que se emite la luz de la galaxia que llega a nosotros hoy en día.

La relación entre la logitud de odna y la aceleración en un instante del tiempo es:


\begin{equation}
	\frac{\lambda_0}{\lambda_1} = \frac{a(t_0)}{a(t_1)}
\end{equation}
Recordando la definición de corrimiento al rojo $1+z=\lambda_0/\lambda_1$ es tal que

\begin{equation}
	1 + z = \frac{a(t_0)}{a(t_1)}
\end{equation}
de tal modo que el corrimiento al rojo de una galaxia es igual al cociente entre el tamaño del Universo en el momento de recepción y emisión de la luz procedente de ella.

\subsection{Parámetro de Hubble y el factor de escala de Universo}

La ley de Hubble teórica:

\begin{equation}
	V = H(t) d_p
\end{equation}
donde $d_p$ es la distancia propia y $V$ es la velocidad de recesión del as galxias. Sustituyendo $d$ por $d=a(t)$ y $V=\dot{a}(t)$, tenemos que:

\begin{Resaltar}
	\begin{center}
		\textbf{Parámetro de Hubble}
	\end{center}
	\begin{equation}
		H(t) = \frac{\dot{a}(t)}{a(t)}
	\end{equation}
\end{Resaltar}


El parámetro de Hubbble, $H(t)=\dot{a}(t)/a(t)$ representa la velocidad de expansión del universo normalizada a su tamaño.

\subsection{El parámetro de deceleración del universo}

El parámetro de deceleración del Universo se define como:

\begin{Resaltar}
	\begin{center}
		\textbf{Parámetro de deceleración}
	\end{center}
	\begin{equation}
		q(t) = - \frac{\ddot{a}(t)a(t)}{[\dot{a}(t)]^2}
	\end{equation}
\end{Resaltar}
siendo $q(t)$ un indicador de la aceleración en la expansión del Universo. Aunque sea trivial, si $q(t)>0$ la aceleración es decelerada, si $q(t)=0$ la expansión es a velocidad constante, y si $q(t)<0$ entonces es aceleración decelerada.



\section{Ecuaciones de Friedman}


En la métrica RW el parámetro de escala del Universo $a(t)$ y la curvatura $k$ no esan determinadas, es decir, la métrica RW no aporta acerca de la forma funcional de $a(t)$. En otras palabras la métrica RW sólo nos da la cinemática de la propagación de un rayo de luz, pero necesitamos unas ecuaciones \textit{dinámicas} que relacionan $a(t)$ y $k$ con el contenido de masa-energía del Universo. Estas ecuaciones se denominan \textit{ecuaciones de Friedman-Lemaître}.

\subsection{Derivación relativista}

La obtención formal de las ecuaciones de Friedmann se basa en el hecho de que la métrica de Robertson-Walker es:

\begin{equation}
	g_{\mu \nu} = \mqty(\dmat{1,-\frac{a^2}{1-kr},-a^2 r^2,-a^2 r^2 \sin^2 \theta})
\end{equation}
Y el cálculo del tensor de Ricci:
\begin{equation}
	R_{\mu \nu} = \mqty(\dmat{-3 \frac{\ddot{a}}{a}, - g_{rr} \parentesis{\frac{\ddot{a}a + 2 \dot{a}+2k}{a^2}},- g_{\phi \phi} \parentesis{\frac{\ddot{a}a + 2 \dot{a}+2k}{a^2}},- g_{\theta \theta} \parentesis{\frac{\ddot{a}a + 2 \dot{a}+2k}{a^2}}})
\end{equation}
y el escalar de curvatura:

\begin{equation}
	R = - 6 \parentesis{\frac{\ddot{a}a + 2 \dot{a}+2k}{a^2}}
\end{equation}
Trivialmente el tensor de Einstein:

\begin{equation}
	G_{\mu \nu} = \mqty(\dmat{-3 \frac{\ddot{a}}{a}+ 3 \frac{k}{a^2},
		- g_{rr} \parentesis{\frac{2\ddot{a}a +  \dot{a}+2k}{a^2}},
		- g_{\phi \phi} \parentesis{\frac{2\ddot{a}a +  \dot{a}+2k}{a^2}},
		- g_{\theta \theta} \parentesis{\frac{2\ddot{a}a + 2 \dot{a}+2k}{a^2}}})
\end{equation}

\subsection{Tensor de energía-momento del Universo}

Ahora tenemos que definir el tensror de energía momento, cuyos términos tienen un significado claro:

\begin{itemize}
	\item $T_{00}$ es la densidad de masa o de energía.
	\item $T_{0i}$ es el flujo de energía en la dirección $x^i$.
	\item $T_{i0}$ es la densidad de componente $p^i$ del momento.
	\item $T_{ij}$ es el flujo de $p^i$ en la dirección $x^j$ (compontente $i$ de la fuerza por unidad de área sobre la normal a $x^j$. Si $i\neq j$ es la fuerza de cizalladura).
	\item $T_{ii}$ es la fuerza por unidad de área a la superficie $x^i$ o lo que es lo mismo, presión.
\end{itemize}
Por definición estamos en un univereso isótropo, por lo que $T_{ij}=T_{ji}$; $T_{ii}\propto P$ para $i=1,2,3$. Así pues, si $\rho c^2$ es la densidad de energía, $P$ la presión y $U_{\mu}$ es la velocidad del fluido (que en estaría en reposo al trabajar en un sistema de referencia cómovil en el que el universo al moverse con la materia, esta última esta en repooso tal qeu $U^{\mu}=(1,0,0,0)$) y $g_{\mu \nu}$ la méttrica. Así pues:

\begin{equation}
	T_{\mu \nu} = (\rho c^2 + P) U_\mu U_\nu - P g_{\mu \nu}
\end{equation}
\begin{equation}
	T_{\mu \nu} = \mqty({\rho c^2 + P - P g_{00},
		- g_{rr} P,
		- g_{\phi \phi} P,
		- g_{\theta \theta} P})
\end{equation}
Vemos, entonces, que las ecuaciones diferenciales, dadas por la \textbf{ecuación general de la relatividad general} $G_{\mu \nu} = C T_{\mu \nu}$ donde $C=8\pi G / c^4$. Así tenemos las ecuaciones:

\begin{equation}
	3 \frac{\ddot{a}}{a}+ 3 \frac{k}{a^2} =  \frac{8 \pi G}{c^4} \rho c^2
\end{equation}
\begin{equation}
	2\frac{\ddot{a}}{a} + 2 \frac{\dot{a}}{a^2}+2\frac{k}{a^2} =  \frac{8 \pi G}{c^4}  P
\end{equation}
Que debemos reescribir considerando además qeu $\rho c^2 = \epsilon$ (densidad de energía y densidad de masa sería igual si $c=1$). Así pùes, vemos que ambas ecuacioens están acopladas. En cualquier caso, las podemos separar, oteniendo así las ecuaciones:

\begin{Resaltar}
	\begin{center}
		\textbf{Ecuación de Friedman para la velocidad}:
		\begin{equation}
			\parentesis{\frac{\dot{a}}{a}} + \frac{kc^2}{a^2} = \frac{8 \pi G}{3c^2} \epsilon
		\end{equation}
	\end{center}
\end{Resaltar}
\begin{Resaltar}
	\begin{center}
		\textbf{Ecuación de Friedman para la aceleración}:
		\begin{equation}
			\parentesis{\frac{\ddot{a}}{a}} =  - \frac{4 \pi G}{3c^2} (3P + \epsilon)
		\end{equation}
	\end{center}
\end{Resaltar}

Además de las ecuación de Einstein, tendrá que cumplirse otras ecuaciones como por ejemplo que la primera ley de la termodinámica $\D Q = \D U + P \D V$. Como además no hay intercambio de calor (todo está a la misma temperatura en un universo homgéneo) entonces la ecuació se convierte en $\D U + P \D V = 0$, la cual nos lleva de manera sencilla y directa a la ecuación del fluido (recordamos que $U=\epsilon V$) tal que:

\begin{Resaltar}
	\begin{center}
		\textbf{Ecuación del fluido}:
	\end{center}
	\begin{equation}
		\dot{\epsilon} = - 3 \frac{\dot{a}}{a} (\epsilon + P)
	\end{equation}
\end{Resaltar}


\section{Modelos cosmológicos}

Un modelo cosmológico que consiste en especificar la curvatura $k$ y las compontentes del tensor energía-momento $T_{\mu \nu}$ es decir, el contenido de masa-energía del Universo ($\epsilon$ y $P$). Con estos \textit{inputs} uno puede resolver las ecuaciones de Friedman y contrar la forma funcional de $a(t)$. Conocido $a(t)$ podemos obtener consecuencias observables del modelo cosmológico propueesto y comprobar si se ajusta a los modelos descartables.

\subsection{Relacion entre $k$ y $\epsilon$. Densidad de energía crítica.}

Tenemos que la constante de Hubble\footnote{La cual, en realidad, debería ser llamada \textit{parámetro de Hubble} al depender del tiempo cosmológico.} nos puede ayudar a reexpresar la ecuación de Friedman:

\begin{equation}
	H(t)^2 + \frac{kc^2}{a^2(t)} = \frac{8 \pi G}{3c^2} \epsilon (t)
\end{equation}
Definiendo entonces la \textbf{densidad de energía crítica}

\begin{equation}
	\epsilon_c : = \frac{3H^2c^2}{8 \pi G}
\end{equation}
y $\Omega(t)$, definida como el cociente entre la energía y la energía crítica
\begin{equation}
	\Omega(t) : = \frac{\epsilon}{\epsilon_c}
\end{equation}
pudiendo reescribir la ecuaciónde Friendman para la velocidad
\begin{equation}
	\frac{kc^2}{a^2} = H^2(\Omega -1)
\end{equation}
Tal que si $k=0 \Leftrightarrow \Omega = 1$, $k=1 \Leftrightarrow \Omega > 1$ y $k=--1 \Leftrightarrow \Omega < 1$. De ello se deduce que $\Omega$, que es el contenido de energía en relación a la energía crítica del universo, determina la curvatura del espacio tiempo.

Esto no debería ser sorprendente: la idea central de la relatividad General es que el contenido de energía del sistema determina la geometría del espacio tiempo y esta determina como se mueve una partícula en el espacio. Veamos qeu:

\begin{itemize}
	\item La curvatura $k$ del Universo es cosntante, y por tanto el signo de $\Omega-1$ no puede variar con el tiempo.
	\item La densidad dde energía crítica sí depende del tiempo cósmico, hoy en día se define a partir de $H_0$, tal que actualmente $\epsilon_{c,0} = \epsilon_c (H_0) = \SI{5200}{MeV/m^3} \simeq 1 \text{protón/200 litros}$, mucho menor que la densidad del medio interestelar de nuestra galaxia ($ \sim 1 \text{protón/cm}^3$)
\end{itemize}

\subsection{Ecuaciones de estado}

Una ecuación de estado de un sistema es simplemente una relación entre la presión $P$ del sistema y la densidad de energía $\epsilon$. Las ecuaciones pueden ser complicadas, pero no en nuestro caso, que asumiremos que

\begin{equation}
	P = \omega \epsilon
\end{equation}
la cual se correspondería con un gas diluido (la galaxia tiene poca densidad) y un gas perfecto en el que $P=\epsilon$. La ecuación de Fredman de la aceleración en el caso de un \textit{gas diluido} será:


\begin{equation}
	\frac{\ddot{a}}{a} = - \frac{4\pi G}{3c^2} \epsilon (3 \omega +1)
\end{equation}
Veamos que en función de si $3\omega$ es menor, igual o mayor a -1 tendremos qeu el universo se expandirá aceleradamente ($\ddot{a}>0$), a velocidad constante  ($\ddot{a}=0$) o deceleradamente ($\ddot{a}<0$) respectivamente.

La gracia de esto es que un determinado modelo cosmológico (al menos en esta asignatura) dará un valor de $\omega$ en función de que consdieremos: universo dominado por materia, radiación o radiación y materia simultáneamente.  Posteriormente también trataremos la influencia de la constante cosmológica. Ahora veamos que valores de $\omega$ se le dan tanto a la materia como a la radiación.


\subsection{Materia: ``gas'' de galaxias}

La materia en el Universo dominado por gas de galaxias, en el que el sistema de referencia que se expande con el Universo (\textit{sistema comóvil}) son comóviles (salvo por lo que hemos visto de las velocidades peculiares). El movimieto aleatorio será, en promedio, cero (al ser aleatorio y el universo isótropo), y como están en reposo respecto a la expansión del universo (precisamente por ser comóviles), tenemos que no se mueven y por tanto la presión será cero:

\begin{equation}
	\text{Materia} \rightarrow P_m \simeq 0 \Rightarrow \omega_m \simeq 0
\end{equation}
lo que implica que \textit{en un universo dominado por materia el universo se expande deceleradamente} al verificar $\ddot{a}<0$.


\subsection{Radiación: ``gas de fotones''}


Al contrario que las galaxias, los fotones nunca pueden ser comóviles (la velocidad de la luz no depende del sistema de referencia elegido). La presión de radiación es:

\begin{equation}
	\text{Radiacion} \rightarrow P_\gamma = \frac{1}{3} \epsilon_\gamma \Rightarrow \omega_\gamma = \frac{1}{3}
\end{equation}
que implicaría una expansión decelerada, decelerando más rápido que en un universo dominado por materia.

\subsection{Comportamiento de materia y radiación con $a(t)$}

Utilizando la ecuación del fluido

\[ \dot{\epsilon} = -3 \frac{\dot{a}}{a} (\epsilon + P) \]

podemos hallar \textit{la dependencia de la densidad de energía en función de a},  que dependerá de si tratemos con materia o con radiación, ya que $P(\epsilon)$ varía.

\begin{itemize}
	\item Si suponemos que estamos con \textbf{materia} tenemos que $P_m\simeq 0$ y por tanto:
	      \begin{equation}
		      \frac{\dot{\epsilon_m}}{\epsilon_m} = -3\frac{\dot{a}}{a}
	      \end{equation}
	      cuya solución trivial es:

	      \begin{equation}
		      \epsilon_m = \epsilon_{m,0} \parentesis{\frac{a}{a_0}}^{-3}
	      \end{equation}
	\item Si suponemos que estamos trabajando con \textbf{radiación} tenemos que $P_\gamma\simeq \epsilon_\gamma/3$ y por tanto:
	      \begin{equation}
		      \frac{\dot{\epsilon_\gamma}}{\epsilon_\gamma} = -4\frac{\dot{a}}{a}
	      \end{equation}
	      cuya solución trivial es:

	      \begin{equation}
		      \epsilon_\gamma = \epsilon_{\gamma,0} \parentesis{\frac{a}{a_0}}^{-4}
	      \end{equation}
\end{itemize}

\subsection{Costante cosmológica}

En 1917, cuando Einstein trató de solcionar un modelo particular del universo (estático, dominado por materia y esférico), concluyó que este colapsaría inexorablemente (debido a la interacción gravitatoria), por lo que introdujo un término ad hoc llamado \textbf{constante cosmológica} en sus ecuaciones, obteniendo así la \textit{ecuación de la relatividad general con constante cosmológica}

\begin{equation}
	R_{\mu \nu} - \frac{1}{2} R g_{\mu \nu} + \Lambda g_{\mu \nu} = \frac{8 \pi G}{c^4} T_{\mu \nu}
\end{equation}
Al igual que no hay razones para que exista, tampoco hay razones para negarla, salvo las que pueda dar un experimento. Tanto en el experimento de Eddington como muchos otros, no es detectable. La pregunta ahora es: ¿Cómo se modfiica las ecauciones de Friemdan con constante cosmológica? Veamos que, sencillamente es introducir un $\Lambda$:

\begin{equation}
	\frac{\ddot{a}}{a} = - \frac{4 \pi G}{3c^2} (3P + \epsilon) + \frac{\Lambda}{3}
\end{equation}
\begin{equation}
	\dot{\epsilon} = - 3 \frac{\dot{a}}{a} (\epsilon + P) + \frac{\Lambda}{3}
\end{equation}
que definiéndo $\epsilon_\Lambda = \Lambda c^2 / 8 \pi G$  que significa \textit{densidad de energía asociada a la cosntante cosmológica}, lo que supone interpretar la $\Lambda$ como una nueva fuente de energía relevante en el universo. Así tendremos que $\epsilon=\epsilon_m + \epsilon_\gamma + \epsilon_\Lambda$, $P=P_m + P_\gamma + P_\Lambda$.


La ecuación de estado asociado a la constante cosmológica es, por definición, aquella que $\dot{\epsilon_\Lambda}=0$, es decir:

\begin{equation}
	P_\Lambda = - \epsilon_\Lambda \Rightarrow \omega_\Lambda = -1
\end{equation}

¿Cuál es la interpretación física de $\epsilon_\Lambda$? Una de los mejores candidatos es la llamada eenergía del vacío, procedente de las fluctuaciones cuánticas fruto del principio de incertidumbre de heisenberg. Las razones por las que es un buen candidato: se hace más grande cuando aumenta el tamaño del espacio (lo que hace que la densidad de energía permanezca constante), además que la presión disminuye al expandirse. Sin embargo numéricamente no tiene sentido.


\section{Cosmología observacional}

En la sección anterior hemos visto varios modelos cosmológicos que representan destinos futuros, velocidades de expansión y aceleraciones para nuestro Universo. En esta sección veremos cual es el mejor modelo, i.e., el que es más compatible con las medidas experimentales.

A partir de la medida de la distancia de las galaxias obtenemos $H_0$. Sin embargo la distancia propia $d_p$ a una galaxia no es medible ya que la luz se emitió en $t=t_0$ no es igual a la que recibimos $t_1$, ya que en el recorrido hecho esta distancia se expandió, por lo qeu $d_p>d_t=c(t_1-t_0)$. Para determinar $d_p$ necesitamos saber la forma funcional de $a(t)$: necesitamos saber el modelo cosmológico.

Para resolver este problema usaremos la llamada \textit{distancia de luminosidad}

\subsection{Distancia de luminosidad $d_L$}tiene una luminosidad intrínseca $\Lcal$

Supongamos que la galaxia $G_1$ con luminosidad intrínseca $\Lcal$, con un flujo que medimos $\Fcal$. Entonces:

\begin{equation}
	\Fcal = \frac{\Lcal}{4\pi d_L^2}
\end{equation}
siendo $d_L$ medible y por tanto $\Fcal$ (si $\Lcal$ es conocida). Definimos la \textbf{distancia de luminosidad} como

\begin{equation}
	d_L = d_p (1+z)^2
\end{equation}
Será entonces, fundamental, experimentalmente hablando, medir el factor de corrimiento al rojo $z$ y flujo $\Fcal$ para objetos con flujo $\Lcal$ conocido (candelas estelares como supernovas).

Una vez tenemos un valor de $d_L$ y elegido un modelo, $d_p$ tendrá una fórma funcional que podremos representar. Así podremos ajustar los valores $H_0$ y $q_0$ a los datos obteniendo dichos valores ``si nuestro universo siguiera el  modelo elegido''. Veamos que:

\begin{equation*}
	d_p = a_0 \int_{0}^{1_2} \frac{\D r}{\sqrt{1 - k r^2}} = a_0 \int_0^{t_1} \frac{\D t}{a(t)}
\end{equation*}
pudiendo calcular la integral haciendo aproximaciones de taylor a primer orden (obteniendo la ley de Hubble) u otras (obteniendo diferentes separaciones de la ley de Hubble).

\section{Resolución de diferentes modelos cosmológicos}

Cuando decimos ``resolver'' un modelo cosmológico decimos que, para una determinada geometría (por ejemplo $k=0,1,-1$ o $k$ arbitario), una determinada componsición (mezcla de materia, radiación y constante cosmológica) usando sendas ecuaciones de estado (las de este curso u otras), y otras consideraciones (estático, aceleración nula...) obtenemos ecuaciones analíticas (o numéricas) de los siguientes parámetros:

\begin{itemize}
	\item Forma funcional de $a(t)$ (o representación numérica).
	\item Parámetro de deceleración $q(t)$.
	\item Parámetro de Hubble $H(t)$.
	\item Relación entre $t$ y $z$.
	\item Obtener la forma funciona de la distancia de luminosidad $d_L$.
\end{itemize}
Y todas las consecuencias/interpretaciones posteriores: ¿Se está acelerando o decelerando?¿Hubo Big-Bang?... 


\section{Tamaño del Universo Observable}

En esta sección trataremos de dar respuesta a la pregunta de cuál es el límite del Universo Observable. Para ello definiremos \textit{el horizonte de partículas} y el \textit{horizonte de sucesos}.

\subsection{Horizonte de partículas}

El Universo tiene una edad $t_0$ desde el Big Bang. Llamaremos \textbf{horizonte de partículas} a la distanccia máxima desde la que puede llegar la luz desde $t=0$. Si el universo fuera plano y estático está claro que la distancia máxima sería $d=ct_0$. Sin embargo el Universo se expande, por lo que la fuente más distante que podríamos observar, aquella que emitió su luz en el instante $t=0$ y se aleja de nosotros mientras la luz viaja y por tanto la distancia a la que podemos observar es en realidad mayor que $ct_0$. En pocas palabras: habrá que tener en cuenta la expansión del universo.

\begin{equation}
	d_H = c a_0 \int_0^{t_0} \frac{\D t}{a(t)}
\end{equation}
Cuanto más rápida sea la expansión del universo la distancia será más grande.

\subsection{El horizonte de sucesos}

El horizonte de partículas se puede entender como un límite a las comunicaciones desde el pasado: nada más allá que el horizonte de partículas no es conocido. El llamado \textbf{horizonte de sucesos} es el límite qeu establece las comunicaciones hacia el futuro. Consideremos un observador en $(t_0,r_0=0)$ que envía un rayo de luz a un obejeto en coordenada comóvil $(t_1,r_1)$. Si el Universo se expande muy rápidamente es posible qeu no le llegue nunca el rayo de luz (se necesitaría un tiempo infinito). El limite a paertir del cual esto ocurre es llamado el horizointe de sucesos, un límite a las comunicaciones futuras, tal que esta región estaría completamente aislada del mundo.

El requisito eseical es que:

\begin{equation}
	\int_{t_0}^{t_1} \frac{c\D t}{a(t)} = r_1
\end{equation}
tal qeu $t_1\rightarrow \infty, r_1 \rightarrow r_{S}$.

\subsection{¿Es el Universo finito o infinito?}

Un espacio 3D plano $k=0$ o hiperbólido $k=-1$ descrito por la métrica de Robertson-Walker es inifito. Sin embargo un espacio 3D esférico es finito. En este caso se poddría producir un fenómeno curioso: la posibilidad de observaar a través de dos un objeto astronómico distante llegando a nosotros por dos caminos opuestos. Como hasta la fecha el objeto $k$ es proximo a cero, pero no cero, no se descarta la posibilidad de que en realidad sea ligereamente esférico, lo que supondría un universo finito, o ligeramente hiperbólico, lo que supondría inifinito. Incluso si fuera finito es posible que no haya ocurriedo suficiente tiempo para que observemos desde dos puntos distintos del cielo el mismo objeto.




\newpage 

\section{Modelos cosmológicos}


\newpage
\section*{Formulario}
\addcontentsline{toc}{section}{Formulario}


\begin{multicols}{2}

\begin{Formulario}
    Corrimiento al rojo
	\begin{equation*}
		z = \frac{\lambda_o - \lambda_e}{\lambda_e}
	\end{equation*}
\end{Formulario}

\begin{Formulario}
    Corrimiento al rojo
	\begin{equation*}
		z = \frac{H_0}{c}d
	\end{equation*}
\end{Formulario}

\begin{Formulario}
    Velocidad de recesión
	\begin{equation*}
		V = H_0 d
	\end{equation*}
\end{Formulario}

\begin{Formulario}
    Velocidad de mov. total
	\begin{equation*}
		\vec{V} = \vec{V}_{\text{recesion}} + \vec{V}_{\text{peculiar}}
	\end{equation*}
\end{Formulario}

\begin{Formulario}
    Métrica Robertson-Walker
	\begin{equation*}
		\D s^2 = c^2 \D t^2 - a^2(t) \ccorchetes{\frac{\D r^2}{1-k^2 r^2} + r^2 (\D \theta^2 + \sin^2 \theta \D \phi^2)}
	\end{equation*}
\end{Formulario}

\begin{Formulario}
    Distancia propia
	\begin{equation*}
		\bar{r}_{ab} = \int_{r_a}^{r_b} \frac{\D r}{\sqrt{1-kr^2}}
	\end{equation*}
\end{Formulario}

\begin{Formulario}
    Distancia real
	\begin{equation*}
		d_{ab} = a(t) \bar{r}_{ab}
	\end{equation*}
\end{Formulario}

\begin{Formulario}
    Corrimiento al rojo
	\begin{equation*}
		1 + z = \frac{a(t_{\text{recibido}})}{a(t_{\text{emitido}})}
	\end{equation*}
\end{Formulario}

\begin{Formulario}
    Parámetro de Hubble
	\begin{equation*}
		H (t)= \frac{\dot{a}(t)}{a(t)}
	\end{equation*}
\end{Formulario}

\begin{Formulario}
    Parámetro de deceleración
	\begin{equation*}
		q(t) = - \frac{\ddot{a}(t)a(t)}{[\dot{a}(t)]^2}
	\end{equation*}
\end{Formulario}

\begin{Formulario}
    Ecuación de Friedman para la velocidad
	\begin{equation*}
		\parentesis{\frac{\dot{a}}{a}}^2 + \frac{kc^2}{a^2} = \frac{8 \pi G}{3c^2} \epsilon
	\end{equation*}
\end{Formulario}

\begin{Formulario}
    Ecuación de Friedman para la aceleración
	\begin{equation*}
		\frac{\ddot{a}}{a} =  - \frac{4 \pi G}{3c^2} (3P + \epsilon)
	\end{equation*}
\end{Formulario}

\begin{Formulario}
    Ecuación de un fluido
	\begin{equation*}
		\dot{\epsilon} = - 3 \frac{\dot{a}}{a} (\epsilon + P)
	\end{equation*}
\end{Formulario}

\begin{Formulario}
    Densidad de energía crítica
    \begin{equation*}
        \epsilon_c : = \frac{3H^2c^2}{8 \pi G}
    \end{equation*}
\end{Formulario}

\begin{Formulario}
    Factor Omega: 
\begin{equation*}
	\Omega(t) : = \frac{\epsilon}{\epsilon_c}
\end{equation*}
\end{Formulario}

\begin{Formulario}
    Ecuación de Friemdan para la velocidad  $H(t),\Omega$.
\begin{equation*}
	\frac{kc^2}{a^2} = H^2(\Omega -1)
\end{equation*}
\end{Formulario}

\begin{Formulario}
    Ecuación de Eistein con constante cosmológica 
	\begin{equation*}
		R_{\mu \nu} - \frac{1}{2} R g_{\mu \nu} + \Lambda g_{\mu \nu} = \frac{8 \pi G}{c^4} T_{\mu \nu}
	\end{equation*}
\end{Formulario}

\begin{Formulario}
    Ecuacones de estado: 
    \begin{equation*}
    P_m \simeq 0 \quad P_\gamma = \frac{\epsilon_\gamma}{3} \quad P_\Lambda = - \epsilon_\Lambda
    \end{equation*}
\end{Formulario}    

\begin{Formulario}
    Distancia de luminosidad 
	\begin{equation*}
		d_L = d_p (1+z)^2
	\end{equation*}
	\begin{equation*}
		d_p = a_0 \int_{0}^{t_1} \frac{\D r}{\sqrt{1 - k r^2}} = a_0 \int_0^{t_1} \frac{\D t}{a(t)}
	\end{equation*}
\end{Formulario}

\begin{Formulario}
    Horizonte de partículas
	\begin{equation*}
		d_H = c a_0 \int_0^{t_0} \frac{\D t}{a(t)}
	\end{equation*}
\end{Formulario}
\end{multicols}



\newpage
\section*{\textit{Ejercicios}}
\addcontentsline{toc}{section}{\textit{Ejercicios}}

El trabajo es, y ahora la prueba de fuego.