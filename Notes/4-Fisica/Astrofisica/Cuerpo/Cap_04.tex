

\chapter{Teoría de la radiación}

\section{Propiedades corpusculares de la radiación}

Para explicar el cuerpo negro es improtante recordar las siguiente deficiones y conceptos: 

\begin{itemize}
    \item Radiación térmica: la emitida por un cuerpo como consecuencia de su temperatura.
    \item Cuerpo negro: emisor ideal, aquel cuya superifice abosrbe toda radiación que incide sobre él.
    \item Radiación espectral: energía emitida en forma de radiación por un cuerpo negro a una temperatura $T$ constante, con frecuencia en el intervalo ($v$,$v+\D v$), por unidad de ángulo sólido, área proyectada y tiempo. 
\end{itemize}
A finales del siglo XIX las observaciones de los espectros térmicos no concordaban con con las prediciones:

\begin{itemize}
    \item Ley de Stefan-Boltzman. La referencia total emiitda por $B_T = \sigma T^4$. 
    \item Ley del desplazamiento de Wien. $\lambda_{\max} = C / T$ siendo $C$ una constante y $\lambda_{\max}$ es la longitud de onda del máximo de $B(\lambda)$. 
    \item Teoría de Plank: la densidad de energía (energía por volujmen y longitud de onda/frecuencia) en una caja es:
    \begin{equation}
        \rho_\tau (\lambda) \D \lambda = \frac{8 \pi \hbar c^2}{\lambda^5} \frac{\D \lambda}{e^{hc/k_BT}-1}
    \end{equation}
    qeu se deduce a partir del postulado de Plank: los entes físicos con un grado de liberatad, dode su coordenada característica es función sinosuidal con el timepo, solo pueden tener energías totales cuantizadas, según $E=n h \nu$ con $n=1,2,3...$ El ajuste de los datos experimentales con la teoría es impresionante. 
\end{itemize}

\subsection{Aplicaciones astrofísicas}

La radiancia espectrtal (intensidad de radiación o energía por unidad de área, tiempo y ángulo sólidio emitida en el rango de una longiturd de onda entre $\lambda$ y $\lambda + \D \lambda$). Una vez integrada podemos obtener la luminosidad:

\begin{equation}
    L = 4 \pi R^2 \sigma T^4_e
\end{equation}

\section{Espectros atómicos}

Los átomos pueden estar en diferentes estados de ionización con espectros característicos diferenciaddos. Los espectroes observadores correspodnen a sistemas gaseosos en equilibrio local. En particular, a ua t empreatura $T$ dada, el cociente de posibildia de ocupación d los estados $a$ y $b$ con degernación  $g_a$ y $g_b$ vendrá dada por el \textit{factor de Boltzmann}: 

\begin{equation}
    \frac{N_b}{N_a} \simeq \frac{P(S_b)}{P(S_a)} = \frac{g_b}{g_a} e^{-(E_a-E_b)/k_BT}
\end{equation}
De las relaciones entre las diferentes entre las energías de los estados y de la tempratura podremos obtener el cociente de las ocupaciones de cadda uno de los estados accesibles. Es preciso hacer un cálculo cuántico riguroso para determinar la degenración $g$ de un estado determinado. 

La ecuación de Saha calcula el grado de ionización del gas cuando está en equlibrio térmico a partir de la funión de partición para cada estado ionizado. El cocientre enter el número de átomos en cada posible estado ionizado viene dado por:

\begin{equation}
    \frac{N_{i+1}}{N_i} = \frac{2Z_{i+1}}{n_eZ_i} \parentesis{\frac{2\pi m_eK_B T}{\hbar^2}}^{3/2} e^{-\frac{x_i}{kT}}
\end{equation}
donde $n_e$ es la densidad numérica de elctrones libres, y la función de partición $Z_i$ para lelgar al estado ionizado:

\begin{equation}
    Z_i = \sum_{j=0}^\infty g_j e^{-(E_j-E_i)/kT}
\end{equation}
Con todo, el significado de las leyes espectroscópicas de Krichoff para la astrofíscia ahora resultan evidentes:

\begin{itemize}
    \item Un gas denso y caliente o un objeto sólido produce un espectro continuo de radiaciíon de acuerdo con la emisión de un cuerpo negro para una determianda tempratura, descrito por la raidanza espectral descrita por Plank.
    \item Un gas difuso produce líneas de emisión birllantes cuando se producen transiciones de electrones de una órbita a otra mas ligada. Las relaciones se Saha y Boltzmann permiten avaliar la itensidade relativa entre líneas tanto en los espectros de emisión como en los de absorción.
    \item Un gas difuso más frío qeu una fuente espectral captura de lso fotones y ciertas(...)
    \item (...)
\end{itemize}
El \textbf{corrimiento Doppler} se produce cuando un movimiento del emisor de una onda modifica la longitud de onda observada. Llamando $v_f$ a la velocidad radial entre emisión y receptor $v_s$ a la velocidad de trasmisión de la onda en el medio:

\begin{equation}
    \frac{\lambda_{obs}-\lambda_{rest}}{\lambda_{rest}} = \frac{v_r}{v_s}
\end{equation}
Esta relación es válida para ondas de sonido, pero no se puede aplciar a la luz. Para la luz necesitamos una formualción relativista que tenga en cuenta la dilatación temporal relativista y la distancai adicional que percorre la luz. Si la fuente se mueve, respecto al observador, $u$ ($\beta=u/c$) y la luz se emite entre los tiempos $t_1$ y $t_2=t_1+\Delta t$:

\begin{equation}
    \Delta t = \frac{\Delta t_{rest}}{\sqrt{1-\beta^2}}
\end{equation}
Teniendo que tener en cuenta el incremrento de la distancia:

\begin{equation}
    \Delta D = \frac{u\Delta t_{rest}\cos(\theta)}{\sqrt{1-\beta^2}}
\end{equation}
Con todo, el efecto global es:

\begin{equation}
    \Delta t_{obs} = \frac{\Delta_{rest}}{\sqrt{1-\beta^2}}(1+\beta \cos (\theta))
\end{equation}

El corrimiento Doppler relativista se obtiene de la diferencia temporal $\Delta t_{obs}$. Aplicándolas para determinar las frecuencias según el observador:

\begin{equation}
    v_{obs} = \frac{v_{rest}}{\gamma(1+\beta \cos (\theta))}
\end{equation}
para el desplazamiento radial:

\begin{equation}
    v_{obs} = v_{res} \sqrt{\frac{1-\beta}{1+\beta}}
\end{equation}
Observamos un desplazmaiento al rojo de las longitudesde onda cuando los objetos que meiten la luz se lejan a altas velcoidades de nosotros, de igual forma, en el caso de acercaser, veríamos un desplazamiento al azul. Definimos el parmaétro $z$ como el \textbf{parámetro del corrimiento al rojo} como:

\begin{equation}
    z = \frac{\Delta \lambda}{\lambda_{res}}
\end{equation}
Los efectos tienen que tener en cuenta las variaciones de las frecuencias de las líneas espectrales observadas pero tamibén en los efectos periódicos, en los que ha que calcular el tiempo propio del sistema qeu recede. El parámetro $z$ del corrimento al rojo puede venir motivado por el corrimiento doppler relativista, pero también por la meisión de radiación en un campo gravitatorio (corrimiento al rojo gravitacional) y también por la expansión propia del Universo (corrimiento al rojo cosmológico). 

\section{Tipos espetrales de las estrellas}

La clasificación moderna de Harvadr o Morgan-Keenan asigna letras a lso tipos espctrales en una secuencia decreciente de temperaturas: O, B, A, F, G, K, M, L, T. Cdaa clase posee subdivisiones (10, del 0 al 9) en función las características que paulatinamente se van suciediendo. Existen otras clases (D,S,C) para espectros de estrellas especiales que no se enmarcan en la categorización anterior. De forma adicional (veremos más adelante el diagrama HR y su importancia), diferntes características de espectros diferentes entrellas que llevan a añadir:


\begin{itemize}
    \item 0 o Ia+: hipergigantes o supergigantes estremadamente luminosas.
    \item Ia: supergigantes muy luminosas.
    \item Iab: supergigantes luminosas de tamaño intermedio.
    \item Ib: supergigantes menos luminosas.
    \item II: gigantes brillantes.
    \item III:
\end{itemize}
Hay muchos detalles más que a veces aparecen como subíndices, notas, paréntesis...  Además hay clases adicionales para estrellas especiales novas o que no entran bien en la categoriazación: W (tipo Wolf-Rayet), L, T, Y (especroes infrarrojos), estrellas con carbón (C-R,C-J,C-H,C-Hd) clase $S$ similar a la $M$ pero con fuertes líneas de absorción de ZrO, y clases P y Q para objetos estelares. Además una completa caracterización para las ennanas blancas. 

\subsection{Tipo O}

El tipo O son gigantes (más de 16 masas solares) azules, extremadamente calientes ($3.3\times 10^4$ K) y luminosas, que radian ampliamente en el UV. Son escassas pero comprenden alguna de las estrellas más conocidas. Tienen líneas de absorción dominantes para el He II. Son estrellas que no tienen atmósferas grandes debido al viento estelar que producen. Tienen enormes cores calientes y quemddan de forma rápida su hidrógeno, qeudando poco tiempo en la secuencia princicpal.

\subsection{Tipo B}

Son estrellas calientes ($1-3.3 \times 10^4$ K) azules de tamaño medio-gradne (mas de dos masas solares), muy luminosas. El $0.125\%$ de las estrellas en la secuencia principal son del tipo A.

\subsection{Tipo A}

Son estrellas blancas-azuladas calientes (7500-10000 K), grandes (del promedio de dos masas solares, de 1 a 15) cuando están en la secuencia principal y de las más brillates en el cielo nocturno.  El $0.625\%$ de las estrellas en la secuencia principal son del tipo A.

\subsection{Tipo F}

Estrella blanco amarillenta (6000-7500 K) del tamñaño del sol. El $3\%$ de las estrellas en la secuencia principal son del tipo F. 

\subsection{Tipo G}

Son estrellas amarillas (5200-6000 K) del tamaño del Sol 

\subsection{Tipo K}

\subsection{TIpo M}

Son estrellas rojas, enannas en la secuencia principal con bajo brillo qeu las hacen invisibles al ojo humano. Cerca del $76\%$ de las estrellas. 

\subsection{Ordenando en la Información}


