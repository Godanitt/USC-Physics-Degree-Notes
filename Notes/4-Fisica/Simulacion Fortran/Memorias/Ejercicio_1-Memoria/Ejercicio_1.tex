\documentclass[12pt]{article}
\usepackage[spanish]{babel}
\usepackage[colorlinks=true,allcolors=blue]{hyperref} % Hiperreferencias
\usepackage{geometry} % Nos permite modificar el tamaño de la páginca
\usepackage{fancyhdr} % Cambiar el layout de la pagina
\usepackage{lastpage}
\usepackage{amsmath}
\usepackage{amsthm}
\usepackage{amsfonts}
\usepackage{amssymb}
\usepackage{makeidx}
\usepackage{graphicx}
\usepackage{lmodern}

\title{Memoria Ejercicio 1 \\ Simulación en Fisica de Materiales} % Titulo
\author{Daniel Vázquez Lago} % Autor



\geometry{a4paper, total={170mm,237mm}, left=20mm, top=35mm}

%\renewcommand{\leftmark}[1]{\markboth{\sectionname \thesection.\ #1}{}}


\pagestyle{fancy}
\fancyhf{}
\rhead{}
\chead{}
\lhead{\leftmark}
\rfoot{}
\cfoot{\thepage/\pageref{LastPage}}
\lfoot{}
\renewcommand{\headrulewidth}{1.5pt}
\renewcommand{\footrulewidth}{1pt}

\begin{document}

\maketitle

\newpage

\tableofcontents

%\begin{abstract}
%En este documento es la memoria para el ejercicio 1 de la asignatura Simulación en Física de Materiales.
%\end{abstract}

\newpage

\section{Desarrollo teórico}

\section{Programa principal}

\subsection{Colocación de las partículas}

\subsection{Cálculo de la energía potencial y las fuerzas}

\subsection{Cálculo de las velocidades}

\subsection{Almacenamiento de los datos}

\end{document}
