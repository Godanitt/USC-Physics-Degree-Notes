    Las simetrías subyacentes a la fuerza nuclear, como bien sabemos (véase \cref{Subsec:02-01-03} para mayor explicación):

    \begin{itemize}
        \item Invarianza traslacional: las dependencias deben corresponder a la distancia relativa $\rn=\rn_1-\rn_2$.
        \item Invarianza de Loretz: la interacción debe ser la misma para cualquier sistema de referencia inercial, y por tanto la dependencia con el momento debe ser también relativa $\pn=\pn_1-\pn_2$.
        \item Invarianza rotacional: todos los términos de la intearcción deben tener un momento angular total nulo.
        \item Invarianza de iesoespín. La interacción debe ser isoescalar en el espacio de isoespín: solo términos $(\tau_1\cdot\tau_2)^n$ deben estar permitidos.
        \item Invarianza de paridad: los términos con $\rn$ y $\pn$ deben tener potencias pares.
        \item Invarianza de inversión teporal: los términos con $\pn$ y espín $\sigma$ deben tener potencias pares.
    \end{itemize}
    Ahora solo tenemos que ver que la interacción de Skyrme no posee las diferentes dependencias prohibidas (distancias no relativas, términos de momento angular...) o aplicar invarianza de paridad/temporal (la invairanza de paridad es $\rn\rightarrow -\rn$ y la temporal $t\rightarrow -t$). La itneracción de Skyrme
