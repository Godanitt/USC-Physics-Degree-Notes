
\chapter{Ejercicios y soluciones}

\section{Boletín 1:} \label{Sec:Ej-01}

\subsection{Ejercicios}
\tcbstartrecording

\begin{texercise}
    Encuentra dos ejemplos de triplete de isoespín, otro ejemplo de ``cuatriplete'', y otro de ``quintuplete'' en núcleos ligeros. Discute las diferencias de energía entre los estados análogos de isoespín de cada grupo e intenta dar una configuración de nucleones que explique sus estructuras.
\tcblower
    Los estados triplete de isoespín son aquellos que tienen $T=1$, los estados cuatripletes son aquellos con $T=3/2$ y los quintupletes con $T=2$. Por ejemplo el carbono 10 tiene $Z=6$ y $N=4$ 
\end{texercise}

\begin{texercise}
    Comprueba si la interacción de Skyrme cumple las simetrías subyacentes de la fuerza nuclear.
\tcblower
    Las simetrías subyacentes a la fuerza nuclear, como bien sabemos (véase \cref{Subsec:02-01-03} para mayor explicación): 

    \begin{itemize}
        \item Invarianza traslacional: las dependencias deben corresponder a la distancia relativa $\rn=\rn_1-\rn_2$.
        \item Invarianza de Loretz: la interacción debe ser la misma para cualquier sistema de referencia inercial, y por tanto la dependencia con el momento debe ser también relativa $\pn=\pn_1-\pn_2$.
        \item Invarianza rotacional: todos los términos de la intearcción deben tener un momento angular total nulo.
        \item Invarianza de iesoespín. La interacción debe ser isoescalar en el espacio de isoespín: solo términos $(\tau_1\cdot\tau_2)^n$ deben estar permitidos. 
        \item Invarianza de paridad: los términos con $\rn$ y $\pn$ deben tener potencias pares. 
        \item Invarianza de inversión teporal: los términos con $\pn$ y espín $\sigma$ deben tener potencias pares. 
    \end{itemize}
    Ahora solo tenemos que ver que la interacción de Skyrme no posee las diferentes dependencias prohibidas (distancias no relativas, términos de momento angular...) o aplicar invarianza de paridad/temporal (la invairanza de paridad es $\rn\rightarrow -\rn$ y la temporal $t\rightarrow -t$). La itneracción de Skyrme
\end{texercise}

\begin{texercise}
    Demuestra que los estados permitidos en el acoplamiento de tres fonones cuadrupolares son $0^+$, $2^+$, $3^+$, $4^+$, y $6^+$.
\tcblower
    Queremos acoplar tres fonones cuadrupolares, es decir, queremos hacer un acoplamiento $2\oplus 2\oplus 2$. Lógicamente la paridad va a ser positiva, ya que 

    \begin{equation}
        \pi = (-)^2 (-)^2 (-)^2 = +
    \end{equation}
    Ahora solo resta acoplar los modos de vibración. Esto exige un acoplamiento un tanto diferente al habitual. Primero usamos el acoplamiento de $2\oplus 2$ dado por los apuntes, que nos dice que 

    \begin{equation*}
        2 \oplus 2 = 0^+,1^-,2^+,,4^+
    \end{equation*}
    
\end{texercise}

%%%%%%%%%%%%%%%%%%%%%%%%%%%%%%%%%%%%%%%%%%%
%%%%%%%%%%%%%%%%%%%%%%%%%%%%%%%%%%%%%%%%%%%
%%%%%%%%%%%%%%%%% EJERCICIO 1.4 %%%%%%%%%%%
%%%%%%%%%%%%%%%%%%%%%%%%%%%%%%%%%%%%%%%%%%%
%%%%%%%%%%%%%%%%%%%%%%%%%%%%%%%%%%%%%%%%%%%

\begin{texercise}
    Encuentra los posibles estados (espín y paridad) que resultarían del acoplamiento de un fonón cuadrupolar y un octupolar.
\tcblower
    Todavía no está redactado. 
\end{texercise}

\begin{texercise}
    \textbf{Ejercicio 5}
    El núcleo de $^{68}$Er tiene su primer estado excitado 2+ en 91.4 keV.
    \begin{enumerate}
        \item Sabiendo que es el primer estado de una banda rotacional, determina su momento de inercia.
        \item ¿Cuáles serían las energías de los siguientes cuatro estados de la misma banda? Compara con los valores experimentales (208.1, 315.0, 410.2, 493.5 keV).
    \end{enumerate}
\tcblower
Todavía no está redactado.
\end{texercise}

\begin{texercise}
    \textbf{Ejercicio 6}
    ¿Por qué dos nucleones en la misma capa se acoplan sólo a $j$ pares ($j = 0, 2, 4, \dots$)?
\tcblower
    Porque minimizan la energía del apareamiento de nucleones.
\end{texercise}

\begin{texercise}
    \textbf{Ejercicio 7}
    \begin{enumerate}
        \item Construye la línea yrast de dos núcleos con $A>150$ y otros dos con $A<100$.
        \item Deduce el momento de inercia en función del cuadrado de la frecuencia de rotación y discute la aparición de backbending e identifica las diferentes bandas rotacionales.
        \item Compara los resultados con lo esperado en un sólido rígido y un fluido sin viscosidad.
        \item En los núcleos con $A<100$, intenta identificar bandas vibracionales.
    \end{enumerate}
\tcblower
Todavía no está redactado.
\end{texercise}

\begin{texercise}
    \textbf{Ejercicio 8}
    Usando el esquema de niveles tradicional del modelo de capas mostrado en la tabla 1,
    \begin{enumerate}
        \item Calcula la corrección de Strutinsky (sin incluir apareamiento) para $Z=50, N=82$ y para $Z=40, N=68$.
        \item Estudia el resultado en función de la anchura $\gamma$ de la Gaussiana.
    \end{enumerate}
\tcblower
Todavía no está redactado.
\end{texercise}


\tcbstoprecording


\subsection{Solucion}
\tcbinputrecords