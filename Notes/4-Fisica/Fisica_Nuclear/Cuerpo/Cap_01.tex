\chapter{Modelos nucleares}

\section{Teorías efectivas}


En el núcleo participan tres de las cuatro fuerzas fundamentales: la fuerza nuclear fuerte, la fuerza nuclear débil y la fuerza electromagnética, aunque solo laf fuerza fuerte es la responsable de mantener unido el núcleo atómico. Aún así, describir el núcleo con el Lagrangiano de QCD es prácticamente imposible, debido a la complejidad del problema. 

Por esa misma razón, Steve Weinberg propuso en 1979 el uso de \textbf{teorías efectivas}. Las teorías efectivas son modelos que describen un sistema en una escala de energía determinada sin necesidad de conocer todos los detalles microscópicos de la teoría subyacente. En otras palabras, capturan los efectos relevantes a bajas energías sin requerir una descripción completa de las interacciones fundamentales a energías más altas, preocupándonos únicamente de las simetrías relevantes de la teoría subyacente (QCD). 

\subsection{Simetría quiral formal}

La simetría más importante es la \textit{simetría quiral formal}. La simetría quiral formal de la cromodinámica cuántica nos dice que si los quarks no tuvieran masa podríamos suponer que los quarks a izquierdas y a derechas son completamente indepndientes, es decir, que bajo ningún tipo de interacción podríamos transformar un quark a izquiredas y a derechas. Sin embargo, la simetría quiral formal no se cumple, ya que los quarks si tienen masa, aunque a bajas energías (quarks ligeros) puede asumirse que se mantiene la simetría. 

Sin embargo en la cromodinámica cuántica el vacío no respeta la simetría quiral, ya que por culpa de las fluctuaciones cuánticas del vacío QCD aparecen espontáneamente pares quark-antiquark que no desaparecen (es decir, aparecen piones). Esta es una de las propiedades fundamentales de QCD. Sin embargo esto es un proceso aleatorio, por eso se le dice \textit{rotura espontánea de simetría}. Esta rotura espontánea de simetŕia es la que le da masa a los núcleos y la que hace que las interacciones fuertes tengan alcance de núcleo a núcleo. 

La simetría quiral formal también se puede romper a través de lo que se llama \textit{rotura explícita}, que le da masas a los quarks, la cual es la manera correcta de entender la física nuclear y la interacción entre quarks. 

\subsection{Teoría de Campo Efectiva Quiral}

La teoría de campo efectiva quiral (ChEFT) es una generalización de la intearcción entre quarks pero aplicada a núcleos, que nos dice que cualquier elemento de matriz se puede expresar como un desarrollo en potencias de la razón entre escalas:

\begin{equation}
    V \propto \sum_\nu \parentesis{\frac{Q}{\Lambda}}^\nu F_\nu (C_i(\Lambda))
\end{equation}
siendo $Q$ el momento típico del sistema (como el momento relativo de los nucleones) aproximadamente $\sim 140$ MeV$/c^2$ y $\Lambda$ la escala de rotura de la simetría quiral aproximadamente  $\sim 800$ MeV$/c^2$. Los $C_i (\Lambda)$ son los coeficientes de baja energía, que encapsulan información sobre la física de corto alcance y deben determinarse empíricamente. Además tenemos $\nu$, que es el orden en la expansión quiral, y $F_\nu(C_i(\Lambda))$  que  representa funciones de los coeficientes de baja energía, que dependen de la escala de corte $\Lambda$. Esta sistemática permite:
\begin{itemize}
    \item Ordenar las contribuciones según su importancia relativa. Las bajas irían ajustadas por $C_i(\Lambda)$ y las de largo alcance las dadas por el intercambio de piones.
    \item Separar las interacciones de corto alcance (ajustadas por de piones). Mejorar la precisión de los cálculos agregando órdenes superiores de la expansión.
\end{itemize}
\subsection{Simetrías subyacentes} \label{Subsec:02-01-03}

Las simetrías subyacentes a la fuerza nuclear, como bien sabemos para mayor explicación): 

\begin{itemize}
    \item Invarianza traslacional: las dependencias deben corresponder a la distancia relativa $\rn=\rn_1-\rn_2$.
    \item Invarianza de Loretz: la interacción debe ser la misma para cualquier sistema de referencia inercial, y por tanto la dependencia con el momento debe ser también relativa $\pn=\pn_1-\pn_2$.
    \item Invarianza rotacional: todos los términos de la intearcción deben tener un momento angular total nulo.
    \item Invarianza de iesoespín. La interacción debe ser isoescalar en el espacio de isoespín: solo términos $(\tau_1\cdot\tau_2)^n$ deben estar permitidos. 
    \item Invarianza de paridad: los términos con $\rn$ y $\pn$ deben tener potencias pares. 
    \item Invarianza de inversión teporal: los términos con $\pn$ y espín $\sigma$ deben tener potencias pares. 
\end{itemize}



\subsection{Intercambio de piones}

\section{Modelos nucleares \textit{ab initio}: No-Core Shell mode}


\section{Modelos nuclares microscópicos}

\subsection{Campo medio}

\subsection{Interacción efectiva de Skyrme}

Una de las interacciones más usadas en los cálculos de Hartre-Fock es un tipo de potencial de contacto, i.e., la interacción solo actúa en un rango finito. Este tipo de interacción de contacto se puede simular con una depednencia en el momento de los participantes. Definiendo $\rn=\rn_1-\rn_2$, la forma más sencilla de la interacción $v(\rn)$ con invarianza rotacional es
\begin{equation}
    v(\rn) = v_0 \delta (\rn) + v_1 \ccorchetes{\hnp^2 \delta (\rn) + \delta (\rn) \hnp^2 } + v_2 \hnp \delta (\rn) \hnp
\end{equation}
La interacción de Skyrme se basa en esta propiedad e incluye un término de interacción de tres cuerpos:

\begin{equation}
    v_{Sk} = \sum_{i<j} v(i,j) + \sum_{i<j<k} v(i,j,k)
\end{equation}
El \textit{término de dos cuerpos} actúa también sobre el espín de los nucleones con las matrices de Dirac:

\begin{equation}
    v(1,2)    
\end{equation}

\begin{itemize}
    \item El término de contacto con intercambio de espín:
    \begin{equation*}
        t_0 
    \end{equation*}
    \item El término de rango efectivo:
    
    \begin{equation*}
        \frac{1}{2} t_1
    \end{equation*}
    \item El término de espín órbita de dos cuerpos:
    
    \begin{equation*}
        i W_0
    \end{equation*}
\end{itemize}
donde hemos usado que 

\begin{equation}
    \hnk
\end{equation}
El \textit{término de tres cuerpos} se describe como un potencial de rango 0:

\begin{equation}
    v(1,2,3) = t_3 \delta (\rn_1-\rn_2)\delta (\rn_2-\rn_3)
\end{equation}
Las constantes $t_0,t_1,t_2,t_3,x_0$ y $W_0$ se ajustan a datos experimentales. Consecuentemente el Hamiltoniano obtenido por el método Hartree-Fock usando la interacciión efectiva de Skyrme es:

\begin{equation}
    \ccorchetes{-\nabla} = 
\end{equation}
donde definimos:
\begin{equation}
    U (\rn) = 
\end{equation}
\begin{equation}
    m^* = 
\end{equation}
\begin{equation}
    \rho (\rn) = 
\end{equation}



\subsection{Teoría relativista de campo medio}
