\chapter{Física Nuclear}

\section{Introducción a la fuerza nuclear}

\subsection{Potencial de Yukawa}

\subsection{Intercambio de piones}

\section{Isoespín}

\subsection{¿Qué es el isoespín?}

Ignorando la interacción electromagnética, se pueden entender los protones y neutrones como estados de la misma partícula: \textbf{el nucleón}. En función del isoespín de esta partícula tendremos una manifestación u otra. Asignamos arbitrariamente $T_3=1/2$ para el protón y $T_3=-1/2$ para el neutrón. Matemáticamente, el isoespín vive en el espacio SU(2), lo que le confiere propiedades matemáticas análogas al espín o momento angular.

\subsection{Estados singlete y triplete de isoespín}

Un núcleo en el estado fundamental tiene un valor de isoespín $T_3=\frac{1}{2} (Z-N)$, tal que $T=|T_3|$. Por ejemplo, con dos nucleones, tendríamos dos posibles configuraciones de isoespín, como podrían ser el estado triplete $T=1$ y el estado singlete $T=0$. El estado singlete (en el que los espines están anti-alineados) solo podría darse cuando los dos nucleones son diferentes, mientras que el triplete podría darse cuando los dos nucleones son diferentes e iguales. En la naturaleza se observa que los estados de dos nucleones iguales no son estables ($^2$n y $^2$He), por lo que \textit{el estado fundamental de isoespín para dos nucleones es el singlete}.  

Mientras que podemos encontrar estados singlete ($T=0$) estables ya con dos nucleones, solo a partir de estados con 4 partículas o más podemos decir que los estados tripletes ($T=1$) son estables.  Estados por ejemplo cuadrupletes ($T=3/2$) son más difíciles de encontrar, así como estados quintupletes ($T=2$). La razón de estos nombres es obvia: la degeneración por isoespín viene dada por $g(T)=2T+1$ (como en el espín) y por tanto para $T=2$ hay 5 estados posibles. 

\subsection{Reglas de selección}

La conservación de isoespín introduce reglas de selección suaves en desintegraciones y reacciones nucleares. Veamos algunas de ellas:

\begin{itemize}
    \item Para transiciones E1, $\Delta T = 0,\pm 1$, aunque $\Delta T=0$ está suprimido para $T=0$ y para $T_3=0$.
    \item En desintegración $\beta$, la componente Fermi está suprimida para $\Delta T\neq 0$.
    \item En reacciones nucleares, $T$ permanece invariante, ya que la fuerza fuerte no distingue entre protones y neutrones. $T_3$ se conserva mientras no cambien el número de protones y neutrones.
\end{itemize}