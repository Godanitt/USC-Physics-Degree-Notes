\chapter{Estructura cristalina}

Las sustancias cristalinas se caracterizan por una periodicidad espacial perfecta, que facilita enormemente la tarea de comprender y calcular sus propiedades físicas. Las sustancias cristalinas se encuentran comúnmente en forma de policristales (aglomerados de pequeñas cristales orientados desordenadamenet llamads cristalitos o granos). Existe una categoría importante de sólidos, que no se tratará aquí denominados amorfos, como el vidrio común  y muchos polímeros, que no pertenecen a los sólidos cristalinos, pues aunque poseen cierto ordden de corto alcance carecen del orden de largo alcance característico de los critales.

\section{Conceptos básicos}

En esta sección introduciremos las definiciones más importantes que usaremos a lo largo del tema. \\

\begin{definition}[{\bf Red}]
    Cojunto de puntos discretos del espacio con vectores posición dados por la combinación lineal: 

    \begin{equation}
        \rn = u_1 \an_1 + u_2 \an_2 + u_3 \an_3
    \end{equation}
    donde los $u_i$ barren {\it todos} los enteros. Los $\an_i$ se denominan {\it vectores base primitivos}.
\end{definition}

\begin{definition}[{\bf Base atómica}]
    conjunto de átomos que se asocia a todos y cada uno de los puntos de la red.
\end{definition}

\begin{definition}[{\bf Estructura cristalina o cristal}]
    Es la combinación red+base atómica. Un ejemplo en 2D sería el representado por la figura.
\end{definition}

\begin{definition}[{\bf Celda unitaria primitiva}]
    Es un volumen del espacio que por traslaciones en vectores de la red cube todo el espacio (sin solapamientos). Una posible forma de construirla es por el paralepípedo definido por los vectores base primitivos.    
\end{definition}

\begin{definition}[{\bf Vectores base y celdas no primitivas}]
    Es un volumen del espacio que por traslaciones en vectores de la red cube todo el espacio (sin solapamientos). Una posible forma de construirla es por el paralepípedo definido por los vectores base primitivos.    
\end{definition}

\begin{definition}[{\bf Primeras consecuencias}]
    Es un volumen del espacio que por traslaciones en vectores de la red cube todo el espacio (sin solapamientos). Una posible forma de construirla es por el paralepípedo definido por los vectores base primitivos.    
\end{definition}

\begin{definition}[{\bf Otras simetrías}]
    Es un volumen del espacio que por traslaciones en vectores de la red cube todo el espacio (sin solapamientos). Una posible forma de construirla es por el paralepípedo definido por los vectores base primitivos.    
\end{definition}


\begin{definition}[{\bf Número de coordinación}]
    Es un volumen del espacio que por traslaciones en vectores de la red cube todo el espacio (sin solapamientos). Una posible forma de construirla es por el paralepípedo definido por los vectores base primitivos.    
\end{definition}

\begin{definition}[{\bf Celda de Wigner-Seitz}]
    Es un volumen del espacio que por traslaciones en vectores de la red cube todo el espacio (sin solapamientos). Una posible forma de construirla es por el paralepípedo definido por los vectores base primitivos.    
\end{definition}