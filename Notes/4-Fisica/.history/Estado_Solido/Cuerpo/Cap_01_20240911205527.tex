\chapter{Estructura cristalina}

Las sustancias cristalinas se caracterizan por una periodicidad espacial perfecta, que facilita enormemente la tarea de comprender y calcular sus propiedades físicas. Las sustancias cristalinas se encuentran comúnmente en forma de policristales (aglomerados de pequeñas cristales orientados desordenadamenet llamads cristalitos o granos). Existe una categoría importante de sólidos, que no se tratará aquí denominados amorfos, como el vidrio común  y muchos polímeros, que no pertenecen a los sólidos cristalinos, pues aunque poseen cierto ordden de corto alcance carecen del orden de largo alcance característico de los critales.

\section{Conceptos básicos}

En esta sección introduciremos las definiciones más importantes que usaremos a lo largo del tema. \\

\begin{definition}[{\bf Red}]
    Cojunto de puntos discretos del espacio con vectores posición dados por la combinación lineal: 

    \begin{equation}
        \rn = u_1 \an_1 + u_2 \an_2 + u_3 \an_3
    \end{equation}
    donde los $u_i$ barren {\it todos} los enteros. Los $\an_i$ se denominan {\it vectores base primitivos}.
\end{definition}

