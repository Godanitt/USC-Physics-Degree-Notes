\documentclass[12pt,a4paper]{article}
\usepackage[utf8]{inputenc}
\usepackage[english]{babel}
\usepackage{amsmath}
\usepackage{amsfonts}
\usepackage{amssymb}
\usepackage{latexsym}
\usepackage{makeidx}
\usepackage{graphicx}
\usepackage{graphics}
\usepackage{lmodern}
\usepackage{hyperref}
\usepackage{subcaption}
\usepackage{pgfplots}
\usepackage{dsfont}
\usepackage{multicol}
\usepackage{xcolor}
\usepackage{booktabs}
\usepackage{float}
\usepackage{subcaption}
\pgfplotsset{width=10cm,compat=1.9}
\usepgfplotslibrary{external}
\usepackage{fancybox}


\setlength{\parindent}{0px}
\usepackage[left=2cm,right=2cm,top=4cm,bottom=2cm]{geometry}

\author{Daniel Vázquez Lago}
\title{Placa solar}

\newcommand{\parentesis}[1]{\left( #1  \right)}
\newcommand{\parciales}[2]{\frac{\partial #1}{\partial #2}}
\newcommand{\pparciales}[2]{\parentesis{\parciales{#1}{#2}}}
\newcommand{\ccorchetes}[1]{\left[ #1  \right]}
\newcommand{\D}{\mathrm{d}}
\newcommand{\sech}{\mathrm{sech} \ }
\newcommand{\csch}{\mathrm{csch} \ }

\begin{document}

\maketitle

\newpage

\tableofcontents

\newpage

\section{Caudal número 1} 
 
Los datos para este caudal:  
 
\begin{equation} 
\begin{array}{lllllll}
P_c & = & 600 KPa &  \ \ &  s(P_c) & =  & 50  KPa \\ 
 P_v & = & 250 KPa &  \ \ &  s(P_v) & =  & 25  KPa\\ 
 Q & = & 50.0 g/s &  \ \ &  s(Q) & =  & 2.0  g/s \\ 
 W & = & 268 J & & & & \\ 
\end{array} 
\end{equation} 
 
\begin{table}[h!] 	 \centering 
\begin{tabular}{|c|c|c|c|c|c|c|c|} 
\hline 
$T_1$ & $T_2$ & $T_3$ & $T_4$ & $T_5$ & $T_6$ & $T_7$ & $T_8$ \\ \hline 
17.7 & 53.8 & 24.8 & 6.8 & 20.0 & 26.0 & 20.1 & 12.1  \\  
17.7 & 54.0 & 24.9 & 6.7 & 20.0 & 26.0 & 19.9 & 12.1  \\  
17.6 & 54.1 & 24.9 & 6.8 & 20.1 & 26.2 & 20.2 & 12.3  \\  
17.7 & 54.1 & 24.9 & 6.7 & 20.0 & 26.0 & 20.3 & 12.3  \\  
17.8 & 54.1 & 24.9 & 6.7 & 20.0 & 25.9 & 20.5 & 12.2  \\  
\hline 
\end{tabular} 
\label{tab:regresion1} 
\end{table} 
 
 
\section{Caudal número 2} 
 
Los datos para este caudal:  
 
\begin{equation} 
\begin{array}{lllllll}
P_c & = & 675 KPa &  \ \ &  s(P_c) & =  & 50  KPa \\ 
 P_v & = & 268 KPa &  \ \ &  s(P_v) & =  & 25  KPa\\ 
 Q & = & 40.0 g/s &  \ \ &  s(Q) & =  & 2.0  g/s \\ 
 W & = & 277 J & & & & \\ 
\end{array} 
\end{equation} 
 
\begin{table}[h!] 	 \centering 
\begin{tabular}{|c|c|c|c|c|c|c|c|} 
\hline 
$T_1$ & $T_2$ & $T_3$ & $T_4$ & $T_5$ & $T_6$ & $T_7$ & $T_8$ \\ \hline 
17.5 & 55.2 & 27.6 & 27.6 & 21.9 & 29.2 & 20.4 & 12.8  \\  
17.5 & 55.3 & 27.6 & 27.6 & 21.8 & 29.1 & 20.5 & 12.7  \\  
17.5 & 55.2 & 27.6 & 27.6 & 21.8 & 29.0 & 20.4 & 12.8  \\  
17.6 & 55.2 & 26.5 & 26.5 & 21.8 & 29.0 & 20.5 & 12.8  \\  
17.7 & 55.3 & 27.5 & 27.5 & 21.8 & 29.3 & 20.4 & 12.8  \\  
17.7 & 55.3 & 27.6 & 27.6 & 21.7 & 29.1 & 20.4 & 12.9  \\  
\hline 
\end{tabular} 
\label{tab:regresion2} 
\end{table} 
 
 
\section{Caudal número 3} 
 
Los datos para este caudal:  
 
\begin{equation} 
\begin{array}{lllllll}
P_c & = & 800 KPa &  \ \ &  s(P_c) & =  & 50  KPa \\ 
 P_v & = & 275 KPa &  \ \ &  s(P_v) & =  & 25  KPa\\ 
 Q & = & 30.0 g/s &  \ \ &  s(Q) & =  & 2.0  g/s \\ 
 W & = & 289 J & & & & \\ 
\end{array} 
\end{equation} 
 
\begin{table}[h!] 	 \centering 
\begin{tabular}{|c|c|c|c|c|c|c|c|} 
\hline 
$T_1$ & $T_2$ & $T_3$ & $T_4$ & $T_5$ & $T_6$ & $T_7$ & $T_8$ \\ \hline 
17.9 & 57.6 & 32.0 & 8.9 & 24.1 & 34.5 & 20.8 & 13.7  \\  
17.9 & 57.7 & 32.1 & 8.9 & 24.0 & 34.0 & 20.7 & 13.8  \\  
17.9 & 57.7 & 32.0 & 8.7 & 24.0 & 33.8 & 21.2 & 13.6  \\  
18.0 & 57.7 & 31.8 & 8.7 & 24.0 & 34.0 & 21.0 & 13.7  \\  
18.1 & 57.7 & 31.9 & 8.9 & 24.1 & 34.3 & 20.7 & 13.7  \\  
18.0 & 57.7 & 32.1 & 9.0 & 24.3 & 34.8 & 20.4 & 13.6  \\  
\hline 
\end{tabular} 
\label{tab:regresion3} 
\end{table} 
 
 
\section{Caudal número 4} 
 
Los datos para este caudal:  
 
\begin{equation} 
\begin{array}{lllllll}
P_c & = & 850 KPa &  \ \ &  s(P_c) & =  & 50  KPa \\ 
 P_v & = & 335 KPa &  \ \ &  s(P_v) & =  & 25  KPa\\ 
 Q & = & 25.0 g/s &  \ \ &  s(Q) & =  & 2.0  g/s \\ 
 W & = & 293 J & & & & \\ 
\end{array} 
\end{equation} 
 
\begin{table}[h!] 	 \centering 
\begin{tabular}{|c|c|c|c|c|c|c|c|} 
\hline 
$T_1$ & $T_2$ & $T_3$ & $T_4$ & $T_5$ & $T_6$ & $T_7$ & $T_8$ \\ \hline 
17.8 & 17.5 & 32.0 & 9.1 & 25.2 & 37.3 & 20.6 & 13.9  \\  
17.8 & 17.5 & 32.1 & 9.3 & 25.2 & 37.5 & 20.6 & 14.0  \\  
17.7 & 17.5 & 32.0 & 9.3 & 25.2 & 37.2 & 20.6 & 14.0  \\  
17.7 & 17.6 & 31.8 & 9.2 & 25.3 & 37.6 & 20.6 & 13.9  \\  
17.7 & 17.7 & 31.9 & 9.1 & 25.2 & 37.2 & 20.5 & 13.9  \\  
\hline 
\end{tabular} 
\label{tab:regresion4} 
\end{table} 
 
 
\section{Caudal número 5} 
 
Los datos para este caudal:  
 
\begin{equation} 
\begin{array}{lllllll}
P_c & = & 950 KPa &  \ \ &  s(P_c) & =  & 50  KPa \\ 
 P_v & = & 300 KPa &  \ \ &  s(P_v) & =  & 25  KPa\\ 
 Q & = & 20.0 g/s &  \ \ &  s(Q) & =  & 2.0  g/s \\ 
 W & = & 307 J & & & & \\ 
\end{array} 
\end{equation} 
 
\begin{table}[h!] 	 \centering 
\begin{tabular}{|c|c|c|c|c|c|c|c|} 
\hline 
$T_1$ & $T_2$ & $T_3$ & $T_4$ & $T_5$ & $T_6$ & $T_7$ & $T_8$ \\ \hline 
17.4 & 61.6 & 38.0 & 9.5 & 26.8 & 41.8 & 20.4 & 14.3  \\  
17.4 & 61.7 & 38.2 & 9.6 & 26.8 & 41.8 & 20.4 & 14.2  \\  
17.4 & 61.7 & 38.2 & 9.6 & 26.8 & 41.7 & 20.2 & 14.2  \\  
17.3 & 61.8 & 38.2 & 9.5 & 26.8 & 41.0 & 20.2 & 14.2  \\  
17.3 & 61.7 & 38.2 & 9.6 & 26.9 & 7.0 & 20.2 & 14.2  \\  
\hline 
\end{tabular} 
\label{tab:regresion5} 
\end{table} 
 
 
\section{Caudal número 6} 
 
Los datos para este caudal:  
 
\begin{equation} 
\begin{array}{lllllll}
P_c & = & 1100 KPa &  \ \ &  s(P_c) & =  & 50  KPa \\ 
 P_v & = & 300 KPa &  \ \ &  s(P_v) & =  & 25  KPa\\ 
 Q & = & 15.0 g/s &  \ \ &  s(Q) & =  & 2.0  g/s \\ 
 W & = & 318 J & & & & \\ 
\end{array} 
\end{equation} 
 
\begin{table}[h!] 	 \centering 
\begin{tabular}{|c|c|c|c|c|c|c|c|} 
\hline 
$T_1$ & $T_2$ & $T_3$ & $T_4$ & $T_5$ & $T_6$ & $T_7$ & $T_8$ \\ \hline 
17.6 & 66.0 & 43.6 & 10.4 & 28.7 & 48.4 & 21.0 & 14.9  \\  
17.7 & 66.1 & 43.6 & 10.4 & 28.6 & 48.3 & 20.8 & 14.9  \\  
17.6 & 66.1 & 43.5 & 10.3 & 28.6 & 48.1 & 20.8 & 14.8  \\  
17.6 & 66.1 & 43.5 & 10.2 & 28.6 & 48.0 & 20.8 & 14.8  \\  
17.6 & 66.1 & 43.4 & 10.3 & 28.6 & 48.0 & 20.7 & 14.8  \\  
\hline 
\end{tabular} 
\label{tab:regresion6} 
\end{table} 
 
 
\section{Caudal número 7} 
 
Los datos para este caudal:  
 
\begin{equation} 
\begin{array}{lllllll}
P_c & = & 1350 KPa &  \ \ &  s(P_c) & =  & 50  KPa \\ 
 P_v & = & 325 KPa &  \ \ &  s(P_v) & =  & 25  KPa\\ 
 Q & = & 10.0 g/s &  \ \ &  s(Q) & =  & 2.0  g/s \\ 
 W & = & 343 J & & & & \\ 
\end{array} 
\end{equation} 
 
\begin{table}[h!] 	 \centering 
\begin{tabular}{|c|c|c|c|c|c|c|c|} 
\hline 
$T_1$ & $T_2$ & $T_3$ & $T_4$ & $T_5$ & $T_6$ & $T_7$ & $T_8$ \\ \hline 
17.6 & 72.3 & 50.4 & 10.7 & 30.4 & 57.0 & 20.7 & 15.1  \\  
17.7 & 72.3 & 50.4 & 10.7 & 30.5 & 57.2 & 20.8 & 15.1  \\  
17.7 & 72.4 & 50.5 & 10.7 & 30.5 & 57.1 & 20.9 & 15.2  \\  
17.7 & 72.4 & 50.5 & 10.7 & 30.5 & 56.9 & 20.7 & 15.1  \\  
17.7 & 72.4 & 50.4 & 10.7 & 30.4 & 56.7 & 20.6 & 15.0  \\  
\hline 
\end{tabular} 
\label{tab:regresion7} 
\end{table} 
 
 

\end{document}
