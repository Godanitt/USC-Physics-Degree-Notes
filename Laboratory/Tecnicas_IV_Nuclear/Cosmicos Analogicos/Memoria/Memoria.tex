\documentclass[11pt]{article}
\usepackage[utf8]{inputenc}
\usepackage[spanish,es-tabla,es-nodecimaldot]{babel}

% Paquetess

\usepackage{amsmath}
\usepackage{amsthm}
\usepackage{amsfonts}
\usepackage{amssymb}
\usepackage{makeidx}
\usepackage{graphicx}
\usepackage{lmodern}
\usepackage[dvipsnames]{xcolor} 
\usepackage{fancyhdr}
\usepackage{geometry}
\usepackage{lastpage}		
\usepackage{array}			 % Para fjar tamaño de columnas
\usepackage{tikz}
\usepackage{subcaption}
\usepackage{caption}
\usepackage{pgfplots} % Para controlar la perspectiva
\RequirePackage{siunitx}
\usepackage{extramarks} % Para poder usar firstleftmarks
\usepackage[version=4]{mhchem} % Para poder usar formulas de reacciones nucleares
\usepackage{chemfig}
\usepackage{xcolor}
\RequirePackage[most]{tcolorbox}
\usepackage{enumitem}
\usepackage{physics}
%\usepackage{background}
\usepackage{eso-pic} % Para insertar imágenes de fondo específicas
\usepackage[absolute,overlay]{textpos} % Paquete para colocar elementos en posiciones absolutas
\usepackage{wrapfig}
\usepackage{booktabs}
\usepackage{float} % en el preámbulo

\setlength{\parindent}{0pt} % Elimina la sangría
\newtcolorbox{mybox}{colback=black!5!white,
	colframe=black!75!black}

\newtcolorbox{Anotacion}{colback=red!5!white,
	colframe=red!75!red}


%##############################################################################
%######### Ponemos el decimal con . ###########################################
%##############################################################################

\sisetup{output-decimal-marker={.},
	% exponentes ------------------------
	exponent-mode=threshold,
	exponent-thresholds=-3:4, % non usar exponentes 10^{-2,-1, 0, 1,2,3}
	% redondear -------------------------
	% round-mode=figures, % cifras sig
	% round-mode=places, % cantos decimales
	round-mode=uncertainty, % cifras sig da incerteza (necesario usar erro)
	round-precision=2,
	%uncertainty-mode = separate,
	print-unity-mantissa=false,
	% unidades --------------------------
	inter-unit-product = \ensuremath{{}\cdot{}}, % separacion entre unidades
	% per-mode=power-positive-first, % so furrula con metodo interpretado puro
	inline-per-mode=single-symbol,
	display-per-mode=fraction,
}

%##############################################################################
%######### Para codigo python #################################################
%##############################################################################

\definecolor{codegreen}{rgb}{0,0.6,0}
\definecolor{codegray}{rgb}{0.5,0.5,0.5}
\definecolor{codepurple}{rgb}{0.58,0,0.82}

\usepackage{listings}


%\lstdefinestyle{mystyle}{	backgroundcolor=\color{backcolour},   	commentstyle=\color{codegreen},	keywordstyle=\color{magenta},	numberstyle=\tiny\color{codegray},	stringstyle=\color{codepurple},	basicstyle=\ttfamily\footnotesize,	breakatwhitespace=false,         	breaklines=true,                 	captionpos=b,                    	keepspaces=true,                 	numbers=left,                    	numbersep=5pt,                  	showspaces=false,                	showstringspaces=false,	showtabs=false,                  	tabsize=2}

%\lstset{style=mystyle}
%\usepackage{background}     % Para manejar el fondo

%%%%%%%%%%%%%%%%%%%%%%%%%%%%%%%%%%%%%%%%%%
%%%%%%%%%%%%%%%%%% BIBLIOGRAFIA %%%%%%%%%%
%%%%%%%%%%%%%%%%%%%%%%%%%%%%%%%%%%%%%%%%%%


\usepackage{biblatex} %Imports biblatex package
\addbibresource{sample.bib} %Import the bibliography file

%##############################################################################
%######### Tipo de fuente #################################################
%##############################################################################

\usepackage{newtxtext,newtxmath} % Cambia la fuente (pero mola)
%\usepackage{kpfonts}

%\usepackage{helvet} 
%\renewcommand{\familydefault}{\sfdefault}.

%\usepackage{fontspec} % Paquete necesario para seleccionar fuentes
%\setmainfont{Verdana} % Cambia la fuente principal a Verdana


%##############################################################################
%######### Geometría #################################################
%##############################################################################

\geometry{a4paper, total={165mm,237mm}, left=22mm, top=30mm}



%##############################################################################
%######### Formatos capítulo #################################################
%##############################################################################

%\usepackage[lmodern]{quotchap}
%\usepackage[options]{fncychap}
% Configuración de la imagen de fondo solo para la portada



%##############################################################################
%######### Hiperreferenias #################################################
%##############################################################################


\usepackage[colorlinks=true, linkcolor=RoyalBlue, citecolor=ForestGreen, urlcolor=BrickRed]{hyperref} % Crea las
\usepackage[nameinlink]{cleveref}
\crefname{figure}{fig.}{Figs.}

%##############################################################################
%######### Formato de pagina #################################################
%##############################################################################

\pagestyle{fancy}
\fancyhf{} % Limpia encabezados y pies
\fancyhead[L]{\small \textbf{Memoria Rayos Cósmicos Analógicos}}    % Encabezado izquierdo
\fancyhead[R]{\small \textbf{Daniel Vázquez Lago}}     % Encabezado derecho
\fancyfoot[C]{\thepage}      % Pie de página centrado con el número de página
\renewcommand{\headrulewidth}{0.4pt}  % Grosor de la línea del encabezado
\renewcommand{\footrulewidth}{0pt}    % Sin línea en el pie



%##############################################################################
%#########  Modificar caption #################################################
%##############################################################################

\usepackage[font=small, justification=centering]{caption}  % Configura las captions



%##############################################################################
%######### Comandos propios #################################################
%##############################################################################


\newcommand{\parentesis}[1]{\left( #1  \right)}
\newcommand{\parciales}[2]{\frac{\partial #1}{\partial #2}}
\newcommand{\pparciales}[2]{\parentesis{\parciales{#1}{#2}}}
\newcommand{\ccorchetes}[1]{\left[ #1  \right]}
\newcommand{\D}{\mathrm{d}}
\newcommand{\derivadas}[2]{\frac{\D #1}{\D #2}}

\newcommand{\tquad}{\quad \quad \quad}
%\newcommand{\vnabla}{\vec{\nabla}}

\newcommand{\Ocal}{\mathcal{O}}
\newcommand{\Jcal}{\mathcal{J}}
\newcommand{\Mcal}{\mathcal{M}}
\newcommand{\Fcal}{\mathcal{F}}
\newcommand{\Hcal}{\mathcal{H}}
\newcommand{\Ecal}{\mathcal{E}}
\newcommand{\Ncal}{\mathcal{N}}

\newcommand{\cmm}{\text{cm}^{-1}}
\newcommand{\fcc}{\textit{fcc}}
\newcommand{\bcc}{\textit{bcc}}
\renewcommand{\sc}{\textit{sc}}
\newcommand{\hcp}{\textit{hcp}}


\newcommand{\PZB}{\text{{\tiny PZB}}}
\newcommand{\gap}{\text{{\tiny gap}}}
\newcommand{\SZB}{\text{{\tiny SZB}}}
\newcommand{\inicial}{\text{{\tiny inicial}}}
\newcommand{\final}{\text{{\tiny final}}}
\newcommand{\atomico}{\text{{\tiny atómico}}}

\newcommand{\arctanh}{\text{{arctanh}}}



\newcommand{\Namas}{\text{Na}^+}
\newcommand{\Clmenos}{\text{Cl}^-}

\newcommand{\cm}{\text{cm}}
\newcommand{\eV}{\text{eV}}

\newcommand{\arr}{\text{arr}}
\newcommand{\diff}{\text{diff}}

\newcommand{\er}{$^{\text{er}}$}
\newcommand{\cte}{\text{cte}}
\newcommand{\expo}{\text{exp}}
\newcommand{\simu}{\text{sim}}


% Comandos vectoriales

\newcommand{\an}{\mathbf{a}}
\newcommand{\bn}{\mathbf{b}}
\newcommand{\dn}{\mathbf{d}}
\newcommand{\fn}{\mathbf{f}}
\newcommand{\jn}{\mathbf{j}}
\newcommand{\kn}{\mathbf{k}}
\newcommand{\pn}{\mathbf{p}}
\newcommand{\qn}{\mathbf{q}}
\newcommand{\rn}{\mathbf{r}}
\newcommand{\sn}{\mathbf{s}}
\newcommand{\un}{\mathbf{u}}
\newcommand{\vn}{\mathbf{v}}
\newcommand{\xn}{\mathbf{x}}
\newcommand{\wn}{\mathbf{w}}
\newcommand{\yn}{\mathbf{y}}
\newcommand{\qndot}{\dot{\qn}}

\newcommand{\alphan}{\boldsymbol{\alpha}}
\newcommand{\sigman}{\boldsymbol{\sigma}}
\newcommand{\pin}{\boldsymbol{\pi}}
\newcommand{\rhon}{\boldsymbol{\rho}}
\newcommand{\epsilonn}{\boldsymbol{\epsilon}}
\newcommand{\omegan}{\boldsymbol{\omega}}
\newcommand{\mun}{\boldsymbol{\mu}}



\newcommand{\An}{\mathbf{A}}
\newcommand{\Bn}{\mathbf{B}}
\newcommand{\En}{\mathbf{E}}
\newcommand{\Fn}{\mathbf{F}}
\newcommand{\Jn}{\mathbf{J}}
\newcommand{\Hn}{\mathbf{H}}
\newcommand{\Gn}{\mathbf{G}}
\newcommand{\Kn}{\mathbf{K}}
\newcommand{\Ln}{\mathbf{L}}
\newcommand{\Mn}{\mathbf{M}}
\newcommand{\Pn}{\mathbf{P}}
%\newcommand{\Rn}{\mathbf{R}}
\newcommand{\Sn}{\mathbf{S}}
\newcommand{\Tn}{\mathbf{T}}
\newcommand{\In}{\mathbf{1}}
\newcommand{\Encal}{\boldsymbol{\mathcal{E}}}

\newcommand{\hnn}{\hat{\mathbf{n}}}
\newcommand{\hnr}{\hat{\mathbf{r}}}
\newcommand{\hnz}{\hat{\mathbf{z}}}
\newcommand{\hnv}{\hat{\mathbf{v}}}
\newcommand{\hnx}{\hat{\mathbf{x}}}
\newcommand{\hny}{\hat{\mathbf{y}}}
\newcommand{\hnu}{\hat{\mathbf{u}}}
\newcommand{\hnR}{\hat{\mathbf{R}}}
\newcommand{\hnp}{\hat{\mathbf{p}}}
\newcommand{\hnk}{\hat{\mathbf{k}}}
\newcommand{\hni}{\hat{\mathbf{i}}}
\newcommand{\hnj}{\hat{\mathbf{j}}}
\renewcommand{\hnk}{\hat{\mathbf{k}}}

 
\title{\textbf{\Huge Cósmicos Analógicos}}
\author{Daniel Vázquez Lago}
\date{\today}
\begin{document}
\maketitle
\newpage 
\tableofcontents
\newpage

\section{Objetivos}

Los objetivos basicamente son rascarme los huevos con \cite{Estadistica}.

\section{Caracterización de los detectores}


\subsection{Determinación de la ventana temporal}

\begin{center}
\begin{table}[H]
\caption{Ventana de coincidencias}
\label{Tab:ventana_01}
\small
\begin{tabular}{cccccccccccccccccccccc}
\toprule
$N_1$  & $N_2$ & $N_{acc}$ & $t$ [s] & $n_1$ [s$^{-1}$] & $n_2$  [s$^{-1}$] & $n_{acc}$  [s$^{-1}$] & $\tau$ [$\mu$s] \\
\midrule
\num{45323.0000000000(212.8919913947)} & \num{10854.0000000000(104.1825321251)} & \num{154.0000000000(12.4096736460)} & \num{94.2100000000(0.3000000000)} & \num{481.0848105297(2.7300916193)} & \num{115.2106995011(1.1651224443)} & \num{1.6346460036(0.1318263382)} & \num{14.7461709041(1.2014395137)} \\
\num{84902.0000000000(291.3794776576)} & \num{21458.0000000000(146.4854941624)} & \num{268.0000000000(16.3707055437)} & \num{184.6800000000(0.3000000000)} & \num{459.7249296080(1.7455667243)} & \num{116.1901667750(0.8153325141)} & \num{1.4511587611(0.0886749683)} & \num{13.5836818887(0.8370943900)} \\
\num{43165.0000000000(207.7618829333)} & \num{16088.0000000000(126.8384799657)} & \num{181.0000000000(13.4536240471)} & \num{101.7400000000(0.3000000000)} & \num{424.2677413014(2.3948291109)} & \num{158.1285630037(1.3310341506)} & \num{1.7790446236(0.1323393577)} & \num{13.2588699141(0.9954110424)} \\
\bottomrule
\end{tabular}
\end{table}
\end{center}


\subsection{Determinación de la zona de trabajo}

\begin{center}
\begin{table}[H]
\small
\caption{Medidas variando el alto voltaje $U_1$.}
\label{Tab:plateu_U}
\begin{tabular}{cccccccccccccccccccccc}
\toprule
$U_1$  [V] & $N_1$  & $N_2$ & $N_{12}$ & $t$ [s] & $n_1$ [s$^{-1}$] & $n_2$  [s$^{-1}$] & $n_{acc}$ [s$^{-1}$] & $n_{r}$ [s$^{-1}$] \\
\midrule
\num{-0.0590000000(0.0026563509)} & \num{1822.0000000000(42.6848919408)} & \num{40815.0000000000(202.0272258880)} & \num{350.0000000000(18.7082869339)} & \num{15.5000000000(0.1300000000)} & \num{117.5483870968(2.9250205822)} & \num{2633.2258064516(25.6444545001)} & \num{8.5059241473(0.2098661907)} & \num{14.0747210139(1.2396478182)} \\
\num{-0.1000000000(0.0031622777)} & \num{9944.0000000000(99.7196068985)} & \num{3040.0000000000(55.1361950084)} & \num{223.0000000000(14.9331845231)} & \num{29.7200000000(0.1300000000)} & \num{334.5895020188(3.6606051754)} & \num{102.2880215343(1.9083793522)} & \num{0.9404892102(0.0220000044)} & \num{6.5628755274(0.5040136339)} \\
\num{-0.2090000000(0.0045930600)} & \num{6250.0000000000(79.0569415042)} & \num{5667.0000000000(75.2794792756)} & \num{323.0000000000(17.9722007556)} & \num{47.5200000000(0.1300000000)} & \num{131.5235690236(1.7021202912)} & \num{119.2550505051(1.6174089460)} & \num{0.4310196743(0.0098108114)} & \num{6.3661183728(0.3787867973)} \\
\num{-0.3140000000(0.0060232217)} & \num{3764.0000000000(61.3514466007)} & \num{6482.0000000000(80.5108688315)} & \num{318.0000000000(17.8325545001)} & \num{51.4500000000(0.1300000000)} & \num{73.1584062196(1.2066905359)} & \num{125.9863945578(1.5968879207)} & \num{0.2532821661(0.0058771540)} & \num{5.9274758514(0.3470011313)} \\
\num{-0.4060000000(0.0072943266)} & \num{8208.0000000000(90.5980132232)} & \num{3060.0000000000(55.3172667438)} & \num{320.0000000000(17.8885438200)} & \num{66.5100000000(0.1300000000)} & \num{123.4100135318(1.3833640148)} & \num{46.0081190798(0.8365609881)} & \num{0.1560276642(0.0036402062)} & \num{4.6552789062(0.2691491886)} \\
\bottomrule
\end{tabular}
\end{table}
\end{center}


\begin{center}
\begin{table}[H]
\caption{Medidas variando el alto voltaje $V_1$.}
\label{Tab:plateu_V}
\small
\begin{tabular}{cccccccccccccccccccccc}
\toprule
$V_1$  [kV] & $N_1$  & $N_2$ & $N_{12}$ & $t$ [s] & $n_1$ [s$^{-1}$] & $n_2$  [s$^{-1}$] & $n_{acc}$ [s$^{-1}$] & $n_{r}$ [s$^{-1}$] \\
\midrule
\num{1.4950000000(0.0226009956)} & \num{1026.0000000000(32.0312347561)} & \num{7436.0000000000(86.2322445492)} & \num{300.0000000000(17.3205080757)} & \num{65.2900000000(0.1300000000)} & \num{15.7145045183(0.4915961731)} & \num{113.8918670547(1.3400840999)} & \num{0.0491824470(0.0013100570)} & \num{4.5457019150(0.2654467188)} \\
\num{1.7220000000(0.0258057513)} & \num{4737.0000000000(68.8258672303)} & \num{5695.0000000000(75.4652237789)} & \num{308.0000000000(17.5499287748)} & \num{48.9200000000(0.1300000000)} & \num{96.8315617334(1.4302446101)} & \num{116.4145543745(1.5733391449)} & \num{0.3097711334(0.0071344971)} & \num{5.9862223253(0.3592083114)} \\
\num{1.8990000000(0.0283054800)} & \num{9944.0000000000(99.7196068985)} & \num{3040.0000000000(55.1361950084)} & \num{223.0000000000(14.9331845231)} & \num{29.7200000000(0.1300000000)} & \num{334.5895020188(3.6606051754)} & \num{102.2880215343(1.9083793522)} & \num{0.9404892102(0.0220000044)} & \num{6.5628755274(0.5040136339)} \\
\num{1.9970000000(0.0296897592)} & \num{24917.0000000000(157.8511957509)} & \num{3904.0000000000(62.4819974073)} & \num{338.0000000000(18.3847763109)} & \num{33.7900000000(0.1300000000)} & \num{737.4075170169(5.4655229849)} & \num{115.5371411660(1.9018031033)} & \num{2.3412398416(0.0529615384)} & \num{7.6617196139(0.5480137737)} \\
\bottomrule
\end{tabular}
\end{table}
\end{center}



\section{Caracterización estadística de la radiación cósmica secundaria}

\subsection{Numero de cuentas en 1 segundo}
\subsection{Numero de cuentas en 2 segundo}
\subsection{Numero de cuentas en 5 segundo}
\subsection{Numero de cuentas en 10 segundo}

\section{Atenuación de la radiación cósmica secundaria}

\subsection{Componente blanda}

\begin{center}
\begin{table}[H]
\caption{Medidas de atenuación blanda usando únicamente placas de hierro}
\small
\label{Tab:hierro}
\begin{tabular}{cccccccccccccccccccccc}
\toprule
$x_{\text{Fe}}$ (mm) & $N_1$  & $N_2$ & $N_{12}$ & $t$ [s] & $n_1$ [s$^{-1}$] & $n_2$  [s$^{-1}$] & $n_{acc}$ [s$^{-1}$] & $n_{r}$ [s$^{-1}$] \\
\midrule
\num{0.0(0.0)} & \num{10326.0000000000(101.6169277237)} & \num{3988.0000000000(63.1506136154)} & \num{313.0000000000(17.6918060130)} & \num{36.6000000000(0.1300000000)} & \num{282.1311475410(2.9517310291)} & \num{108.9617486339(1.7682995938)} & \num{0.8447765074(0.0193277084)} & \num{7.7071360609(0.4847216248)} \\
\num{0.0(0.0)} & \num{17115.0000000000(130.8243096676)} & \num{6221.0000000000(78.8733161468)} & \num{462.0000000000(21.4941852602)} & \num{52.5000000000(0.1300000000)} & \num{326.0000000000(2.6193810628)} & \num{118.4952380952(1.5307336676)} & \num{1.0615372206(0.0234526395)} & \num{7.7384627794(0.4106627555)} \\
\num{3.2000000000(0.0280000000)} & \num{11183.0000000000(105.7497044913)} & \num{3966.0000000000(62.9761859753)} & \num{341.0000000000(18.4661853126)} & \num{39.0700000000(0.1300000000)} & \num{286.2298438700(2.8693421242)} & \num{101.5101100589(1.6468886806)} & \num{0.7984375268(0.0182276878)} & \num{7.9294867118(0.4738856349)} \\
\num{3.2000000000(0.0280000000)} & \num{16251.0000000000(127.4794101022)} & \num{5668.0000000000(75.2861208989)} & \num{406.0000000000(20.1494416796)} & \num{51.4100000000(0.1300000000)} & \num{316.1058159891(2.6053132330)} & \num{110.2509239448(1.4907268469)} & \num{0.9577043335(0.0212630528)} & \num{6.9395919124(0.3930202552)} \\
\num{6.4000000000(0.0560000000)} & \num{11621.0000000000(107.8007421125)} & \num{4059.0000000000(63.7102817448)} & \num{325.0000000000(18.0277563773)} & \num{41.4800000000(0.1300000000)} & \num{280.1591128255(2.7431758356)} & \num{97.8543876567(1.5662459388)} & \num{0.7533586609(0.0171504681)} & \num{7.0817425927(0.4356440790)} \\
\num{6.4000000000(0.0560000000)} & \num{17252.0000000000(131.3468690148)} & \num{6003.0000000000(77.4790294209)} & \num{461.0000000000(21.4709105536)} & \num{54.5300000000(0.1300000000)} & \num{316.3763066202(2.5240364525)} & \num{110.0861910875(1.4448864916)} & \num{0.9570916468(0.0211693460)} & \num{7.4969703375(0.3948283303)} \\
\num{9.6000000000(0.0840000000)} & \num{10886.0000000000(104.3359957062)} & \num{4153.0000000000(64.4437739429)} & \num{323.0000000000(17.9722007556)} & \num{45.5300000000(0.1300000000)} & \num{239.0951021305(2.3911138061)} & \num{91.2145837909(1.4391750825)} & \num{0.5993102270(0.0136332789)} & \num{6.4949133618(0.3954875892)} \\
\num{9.6000000000(0.0840000000)} & \num{9617.0000000000(98.0663041008)} & \num{4161.0000000000(64.5058136915)} & \num{314.0000000000(17.7200451467)} & \num{39.8000000000(0.1300000000)} & \num{241.6331658291(2.5872972464)} & \num{104.5477386935(1.6563337160)} & \num{0.6942052857(0.0158555214)} & \num{7.1952419505(0.4462541707)} \\
\num{12.8000000000(0.1120000000)} & \num{14495.0000000000(120.3951826279)} & \num{3481.0000000000(59.0000000000)} & \num{314.0000000000(17.7200451467)} & \num{37.4900000000(0.1300000000)} & \num{386.6364363830(3.4800178171)} & \num{92.8514270472(1.6063510536)} & \num{0.9865249889(0.0226110857)} & \num{7.3890418289(0.4740915349)} \\
\num{12.8000000000(0.1120000000)} & \num{15148.0000000000(123.0772115381)} & \num{5310.0000000000(72.8697468089)} & \num{408.0000000000(20.1990098767)} & \num{53.3800000000(0.1300000000)} & \num{283.7766953915(2.4070276202)} & \num{99.4754589734(1.3864428969)} & \num{0.7757278917(0.0172901331)} & \num{6.8675842102(0.3792522298)} \\
\num{16.0000000000(0.1400000000)} & \num{12982.0000000000(113.9385799455)} & \num{4297.0000000000(65.5515064663)} & \num{333.0000000000(18.2482875909)} & \num{46.1800000000(0.1300000000)} & \num{281.1173668255(2.5910781648)} & \num{93.0489389346(1.4434440800)} & \num{0.7188128458(0.0162592430)} & \num{6.4921009697(0.3960106112)} \\
\num{16.0000000000(0.1400000000)} & \num{14880.0000000000(121.9836054558)} & \num{5181.0000000000(71.9791636517)} & \num{421.0000000000(20.5182845287)} & \num{54.3600000000(0.1300000000)} & \num{273.7306843267(2.3375286663)} & \num{95.3090507726(1.3435940822)} & \num{0.7169260413(0.0160014179)} & \num{7.0277391537(0.3782446331)} \\
\num{19.2000000000(0.1680000000)} & \num{17022.0000000000(130.4683869755)} & \num{5593.0000000000(74.7863623932)} & \num{415.0000000000(20.3715487875)} & \num{55.7100000000(0.1300000000)} & \num{305.5465805062(2.4480514328)} & \num{100.3949021720(1.3627113219)} & \num{0.8429577677(0.0187044269)} & \num{6.6063332034(0.3665617727)} \\
\num{22.4000000000(0.1960000000)} & \num{14875.0000000000(121.9631091765)} & \num{5291.0000000000(72.7392603757)} & \num{426.0000000000(20.6397674406)} & \num{54.0300000000(0.1300000000)} & \num{275.3100129558(2.3525086275)} & \num{97.9270775495(1.3667384585)} & \num{0.7408691811(0.0165198341)} & \num{7.1436394252(0.3828330434)} \\
\num{22.4000000000(0.1960000000)} & \num{17365.0000000000(131.7763256431)} & \num{5844.0000000000(76.4460594145)} & \num{411.0000000000(20.2731349327)} & \num{58.0600000000(0.1300000000)} & \num{299.0871512229(2.3663919009)} & \num{100.6544953496(1.3358223515)} & \num{0.8272707332(0.0183110694)} & \num{6.2516131800(0.3500144492)} \\
\num{25.6000000000(0.2240000000)} & \num{15848.0000000000(125.8888398548)} & \num{5617.0000000000(74.9466476902)} & \num{417.0000000000(20.4205778567)} & \num{57.8800000000(0.1300000000)} & \num{273.8078783690(2.2602683419)} & \num{97.0456116102(1.3130798835)} & \num{0.7301945210(0.0162137772)} & \num{6.4743666400(0.3535517447)} \\
\num{28.8000000000(0.2520000000)} & \num{14736.0000000000(121.3919272439)} & \num{5358.0000000000(73.1983606374)} & \num{417.0000000000(20.4205778567)} & \num{53.9300000000(0.1300000000)} & \num{273.2430928982(2.3453059047)} & \num{99.3510105693(1.3782512408)} & \num{0.7459989392(0.0166271635)} & \num{6.9862465642(0.3794726040)} \\
\num{28.8000000000(0.2520000000)} & \num{17707.0000000000(133.0676519670)} & \num{5842.0000000000(76.4329771761)} & \num{413.0000000000(20.3224014329)} & \num{61.2600000000(0.1300000000)} & \num{289.0466862553(2.2571226086)} & \num{95.3640222005(1.2639873382)} & \num{0.7574767085(0.0167573529)} & \num{5.9842797394(0.3324710753)} \\
\num{32.0000000000(0.2800000000)} & \num{16089.0000000000(126.8424219258)} & \num{5557.0000000000(74.5452882482)} & \num{415.0000000000(20.3715487875)} & \num{58.8400000000(0.1300000000)} & \num{273.4364377974(2.2387687945)} & \num{94.4425560843(1.2839832053)} & \num{0.7096445124(0.0157600682)} & \num{6.3433806406(0.3469280454)} \\
\num{32.0000000000(0.2800000000)} & \num{16089.0000000000(126.8424219258)} & \num{5557.0000000000(74.5452882482)} & \num{415.0000000000(20.3715487875)} & \num{58.8400000000(0.1300000000)} & \num{273.4364377974(2.2387687945)} & \num{94.4425560843(1.2839832053)} & \num{0.7096445124(0.0157600682)} & \num{6.3433806406(0.3469280454)} \\
\bottomrule
\end{tabular}
\end{table}
\end{center}


\begin{figure} \centering
    \includegraphics[width=0.5\linewidth]{../Graficas/Atenuacion_Fe.pdf}
\end{figure}

\subsection{Componente blanda y dura}


\begin{center}
\begin{table}[H]
\caption{Medidas de atenuación dura usando únicamente placas de plomo con 20 de hierro}
\label{Tab:plomo_1}
\begin{tabular}{cccccccccccccccccccccc}
\toprule
$x_{\text{Pb}}$ (mm) & $N_1$ & $N_2$ & $N_{12}$ & $t$ (s) & $n_1$ (s$^{-1}$) & $n_2$ (s$^{-1}$) & $n_{12}$ (s$^{-1}$) \\
\midrule
\num{0.0000000000(0.0000000000)} & \num{16089.0000000000(126.8424219258)} & \num{5557.0000000000(74.5452882482)} & \num{415.0000000000(20.3715487875)} & \num{58.8400000000(0.3000000000)} & \num{273.4364377974(2.5672420945)} & \num{94.4425560843(1.3553367372)} & \num{7.0530251530(0.3480819042)} \\
\num{0.0000000000(0.0000000000)} & \num{16089.0000000000(126.8424219258)} & \num{5557.0000000000(74.5452882482)} & \num{415.0000000000(20.3715487875)} & \num{58.8400000000(0.3000000000)} & \num{273.4364377974(2.5672420945)} & \num{94.4425560843(1.3553367372)} & \num{7.0530251530(0.3480819042)} \\
\num{7.5000000000(0.7500000000)} & \num{17091.0000000000(130.7325514170)} & \num{5418.0000000000(73.6070648783)} & \num{406.0000000000(20.1494416796)} & \num{62.2200000000(0.3000000000)} & \num{274.6865959499(2.4837222319)} & \num{87.0781099325(1.2553081568)} & \num{6.5252330440(0.3253665994)} \\
\num{7.5000000000(0.7500000000)} & \num{16888.0000000000(129.9538379579)} & \num{5633.0000000000(75.0533143838)} & \num{412.0000000000(20.2977831302)} & \num{60.6600000000(0.3000000000)} & \num{278.4042202440(2.5466391512)} & \num{92.8618529509(1.3197634257)} & \num{6.7919551599(0.3362973625)} \\
\num{15.0000000000(1.5000000000)} & \num{17940.0000000000(133.9402852020)} & \num{5424.0000000000(73.6478105581)} & \num{409.0000000000(20.2237484162)} & \num{65.3100000000(0.3000000000)} & \num{274.6899402848(2.4079106681)} & \num{83.0500689022(1.1904462321)} & \num{6.2624406676(0.3109910512)} \\
\num{15.0000000000(1.5000000000)} & \num{18321.0000000000(135.3550885634)} & \num{6061.0000000000(77.8524244966)} & \num{415.0000000000(20.3715487875)} & \num{67.1300000000(0.3000000000)} & \num{272.9182183822(2.3564967997)} & \num{90.2875018621(1.2279123758)} & \num{6.1820348577(0.3047191383)} \\
\num{15.0000000000(1.5000000000)} & \num{18008.0000000000(134.1938895777)} & \num{5570.0000000000(74.6324326282)} & \num{406.0000000000(20.1494416796)} & \num{64.8600000000(0.3000000000)} & \num{277.6441566451(2.4351261755)} & \num{85.8772741289(1.2172995241)} & \num{6.2596361394(0.3120067910)} \\
\num{22.5000000000(2.2500000000)} & \num{16645.0000000000(129.0155029444)} & \num{5288.0000000000(72.7186358508)} & \num{414.0000000000(20.3469899494)} & \num{62.5000000000(0.3000000000)} & \num{266.3200000000(2.4280162538)} & \num{84.6080000000(1.2323393018)} & \num{6.6240000000(0.3271008021)} \\
\num{37.5000000000(3.7500000000)} & \num{17336.0000000000(131.6662447251)} & \num{5496.0000000000(74.1350119714)} & \num{408.0000000000(20.1990098767)} & \num{66.0000000000(0.3000000000)} & \num{262.6666666667(2.3249277959)} & \num{83.2727272727(1.1853183637)} & \num{6.1818181818(0.3073328414)} \\
\num{45.0000000000(4.5000000000)} & \num{17645.0000000000(132.8344834747)} & \num{5710.0000000000(75.5645419493)} & \num{405.0000000000(20.1246117975)} & \num{66.6600000000(0.3000000000)} & \num{264.7014701470(2.3216496378)} & \num{85.6585658566(1.1973382375)} & \num{6.0756075608(0.3031350617)} \\
\num{52.5000000000(5.2500000000)} & \num{18037.0000000000(134.3018987208)} & \num{5662.0000000000(75.2462623656)} & \num{419.0000000000(20.4694894905)} & \num{67.5700000000(0.3000000000)} & \num{266.9379902324(2.3141194930)} & \num{83.7945833950(1.1741060011)} & \num{6.2009767648(0.3041860078)} \\
\num{60.0000000000(6.0000000000)} & \num{18094.0000000000(134.5139397981)} & \num{5698.0000000000(75.4850978671)} & \num{410.0000000000(20.2484567313)} & \num{67.7400000000(0.3000000000)} & \num{267.1095364629(2.3113898185)} & \num{84.1157366401(1.1749542544)} & \num{6.0525538825(0.3001137805)} \\
\num{67.5000000000(6.7500000000)} & \num{17038.0000000000(130.5296901092)} & \num{5252.0000000000(72.4706837280)} & \num{404.0000000000(20.0997512422)} & \num{63.4900000000(0.3000000000)} & \num{268.3572216097(2.4155043709)} & \num{82.7216884549(1.2065199504)} & \num{6.3632068042(0.3180059696)} \\
\num{67.5000000000(6.7500000000)} & \num{18248.0000000000(135.0851583261)} & \num{5732.0000000000(75.7099729230)} & \num{416.0000000000(20.3960780544)} & \num{67.0600000000(0.3000000000)} & \num{272.1145243066(2.3536518099)} & \num{85.4756934089(1.1919870456)} & \num{6.2033999404(0.3054101631)} \\
\num{75.0000000000(7.5000000000)} & \num{19374.0000000000(139.1905169184)} & \num{5953.0000000000(77.1556867638)} & \num{407.0000000000(20.1742410018)} & \num{72.1900000000(0.3000000000)} & \num{268.3751212079(2.2274393250)} & \num{82.4629450062(1.1223821007)} & \num{5.6378999861(0.2804407463)} \\
\bottomrule
\end{tabular}
\end{table}
\end{center}


\begin{figure}[h!] \centering
    \includegraphics[width=0.5\linewidth]{../Graficas/Atenuacion_Pb.pdf}
\end{figure}


\subsection{Componente dura}

\begin{center}
\begin{table}[H]
\caption{Medidas de atenuación dura usando únicamente placas de plomo sin planchas de hierro}
\small
\label{Tab:plomo_2}
\begin{tabular}{cccccccccccccccccccccc}
\toprule
$x_{\text{Pb}}$ (mm) & $N_1$  & $N_2$ & $N_{12}$ & $t$ [s] & $n_1$ [s$^{-1}$] & $n_2$  [s$^{-1}$] & $n_{acc}$ [s$^{-1}$] & $n_{r}$ [s$^{-1}$] \\
\midrule
\num{0.0(0.0)} & \num{10326.0000000000(101.6169277237)} & \num{3988.0000000000(63.1506136154)} & \num{313.0000000000(17.6918060130)} & \num{36.6000000000(0.1300000000)} & \num{282.1311475410(2.9517310291)} & \num{108.9617486339(1.7682995938)} & \num{0.8447765074(0.0193277084)} & \num{7.7071360609(0.4847216248)} \\
\num{0.0(0.0)} & \num{17115.0000000000(130.8243096676)} & \num{6221.0000000000(78.8733161468)} & \num{462.0000000000(21.4941852602)} & \num{52.5000000000(0.1300000000)} & \num{326.0000000000(2.6193810628)} & \num{118.4952380952(1.5307336676)} & \num{1.0615372206(0.0234526395)} & \num{7.7384627794(0.4106627555)} \\
\num{7.5000000000(0.0140000000)} & \num{17450.0000000000(132.0984481362)} & \num{5952.0000000000(77.1492060879)} & \num{448.0000000000(21.1660104885)} & \num{61.6800000000(0.1300000000)} & \num{282.9118028534(2.2231321958)} & \num{96.4980544747(1.2672254112)} & \num{0.7502160517(0.0165863226)} & \num{6.5130783711(0.3438999098)} \\
\num{15.0000000000(0.0280000000)} & \num{16334.0000000000(127.8045382606)} & \num{5467.0000000000(73.9391641825)} & \num{429.0000000000(20.7123151772)} & \num{59.2600000000(0.1300000000)} & \num{275.6328045899(2.2398351409)} & \num{92.2544718191(1.2640145202)} & \num{0.6987712999(0.0155260323)} & \num{6.5405132090(0.3502208765)} \\
\num{22.5000000000(0.0420000000)} & \num{19751.0000000000(140.5382510209)} & \num{6151.0000000000(78.4283112148)} & \num{466.0000000000(21.5870331449)} & \num{77.3000000000(0.1300000000)} & \num{255.5109961190(1.8681796595)} & \num{79.5730918499(1.0233839183)} & \num{0.5587178630(0.0123039893)} & \num{5.4697426804(0.2797177548)} \\
\num{37.5000000000(0.0700000000)} & \num{18875.0000000000(137.3863166403)} & \num{5947.0000000000(77.1167945392)} & \num{413.0000000000(20.3224014329)} & \num{69.0000000000(0.1300000000)} & \num{273.5507246377(2.0567268633)} & \num{86.1884057971(1.1293696820)} & \num{0.6478932357(0.0143022089)} & \num{5.3376140107(0.2950901671)} \\
\num{52.5000000000(0.0980000000)} & \num{21546.0000000000(146.7855578727)} & \num{6703.0000000000(81.8718510845)} & \num{471.0000000000(21.7025344142)} & \num{78.0700000000(0.1300000000)} & \num{275.9830920968(1.9355276044)} & \num{85.8588446266(1.0583986971)} & \num{0.6511547974(0.0142739850)} & \num{5.3818924679(0.2785355961)} \\
\num{67.5000000000(0.1260000000)} & \num{18783.0000000000(137.0510853660)} & \num{5993.0000000000(77.4144689318)} & \num{444.0000000000(21.0713075057)} & \num{69.2600000000(0.1300000000)} & \num{271.1954952353(2.0432147453)} & \num{86.5290210800(1.1294752541)} & \num{0.6448533943(0.0142311445)} & \num{5.7657732300(0.3048051343)} \\
\bottomrule
\end{tabular}
\end{table}
\end{center}


\begin{figure}[h!] \centering
    \includegraphics[width=0.5\linewidth]{../Graficas/Atenuacion_Pb2.pdf}
\end{figure}


\section{Flujo en la superficie}


\section{Eficiencia geométrica}


\subsection{Dependencia con la distancia}

\begin{center}
\begin{table}[H]
\caption{Medidas de coincidencias a una distancia $d$ entre los detectores.}
\label{Tab:distancia}
\begin{tabular}{cccccccccccccccccccccc}
\toprule
$d$ (cm) & $N_1$ & $N_2$ & $N_{12}$ & $t$ (s) & $n_1$ (s$^{-1}$) & $n_2$ (s$^{-1}$) & $n_{12}$ (s$^{-1}$) \\
\midrule
\num{3.3000000000(0.3300000000)} & \num{8569.0000000000(92.5688932633)} & \num{3684.0000000000(60.6959636220)} & \num{212.0000000000(14.5602197786)} & \num{35.7400000000(0.3000000000)} & \num{239.7593732513(3.2800469858)} & \num{103.0777839955(1.9059709017)} & \num{5.9317291550(0.4104242075)} \\
\num{3.3000000000(0.3300000000)} & \num{16455.0000000000(128.2770439323)} & \num{7216.0000000000(84.9470423264)} & \num{408.0000000000(20.1990098767)} & \num{69.1100000000(0.3000000000)} & \num{238.0986832586(2.1244923243)} & \num{104.4132542324(1.3100613898)} & \num{5.9036318912(0.2933946975)} \\
\num{8.3000000000(0.8300000000)} & \num{22589.0000000000(150.2963738751)} & \num{9313.0000000000(96.5038859321)} & \num{411.0000000000(20.2731349327)} & \num{80.6500000000(0.3000000000)} & \num{280.0867947923(2.1350271435)} & \num{115.4742715437(1.2713372478)} & \num{5.0960942343(0.2520855407)} \\
\num{8.3000000000(0.8300000000)} & \num{24440.0000000000(156.3329779669)} & \num{9925.0000000000(99.6242942259)} & \num{430.0000000000(20.7364413533)} & \num{89.9100000000(0.3000000000)} & \num{271.8273829385(1.9611153594)} & \num{110.3881659437(1.1676596702)} & \num{4.7825603381(0.2311869466)} \\
\num{13.6000000000(1.3600000000)} & \num{29857.0000000000(172.7917822120)} & \num{12252.0000000000(110.6887528162)} & \num{416.0000000000(20.3960780544)} & \num{113.4800000000(0.3000000000)} & \num{263.1036305957(1.6740050574)} & \num{107.9661614381(1.0163059997)} & \num{3.6658442016(0.1799938826)} \\
\num{18.8000000000(1.8800000000)} & \num{37079.0000000000(192.5590818424)} & \num{15964.0000000000(126.3487237767)} & \num{406.0000000000(20.1494416796)} & \num{141.6800000000(0.3000000000)} & \num{261.7094861660(1.4677453063)} & \num{112.6764539808(0.9231532239)} & \num{2.8656126482(0.1423473512)} \\
\num{23.5000000000(2.3500000000)} & \num{48584.0000000000(220.4177851263)} & \num{20692.0000000000(143.8471410908)} & \num{419.0000000000(20.4694894905)} & \num{190.6800000000(0.3000000000)} & \num{254.7933710929(1.2234920051)} & \num{108.5168869310(0.7734687648)} & \num{2.1973987833(0.1074056105)} \\
\num{28.2000000000(2.8200000000)} & \num{37162.0000000000(192.7744796388)} & \num{16595.0000000000(128.8215820428)} & \num{313.0000000000(17.6918060130)} & \num{152.2400000000(0.3000000000)} & \num{244.1014188124(1.3545399851)} & \num{109.0055176038(0.8730128749)} & \num{2.0559642669(0.1162805722)} \\
\num{33.3000000000(3.3300000000)} & \num{44041.0000000000(209.8594767934)} & \num{18297.0000000000(135.2664038111)} & \num{324.0000000000(18.0000000000)} & \num{172.8000000000(0.3000000000)} & \num{254.8668981481(1.2925595921)} & \num{105.8854166667(0.8040869813)} & \num{1.8750000000(0.1042175169)} \\
\num{38.2000000000(3.8200000000)} & \num{53537.0000000000(231.3806387752)} & \num{23429.0000000000(153.0653455228)} & \num{338.0000000000(18.3847763109)} & \num{218.3000000000(0.3000000000)} & \num{245.2450755841(1.1122141139)} & \num{107.3247824095(0.7165143022)} & \num{1.5483279890(0.0842448156)} \\
\num{48.2000000000(4.8200000000)} & \num{59969.0000000000(244.8856876177)} & \num{26718.0000000000(163.4564162093)} & \num{325.0000000000(18.0277563773)} & \num{243.6300000000(0.3000000000)} & \num{246.1478471453(1.0498593138)} & \num{109.6662972540(0.6843760113)} & \num{1.3339900669(0.0740146854)} \\
\num{54.5000000000(5.4500000000)} & \num{44514.0000000000(210.9834116702)} & \num{19942.0000000000(141.2161463856)} & \num{210.0000000000(14.4913767462)} & \num{180.4000000000(0.3000000000)} & \num{246.7516629712(1.2394283628)} & \num{110.5432372506(0.8040901313)} & \num{1.1640798226(0.0803524616)} \\
\bottomrule
\end{tabular}
\end{table}
\end{center}


\begin{figure}[h!] \centering
    \includegraphics[width=0.5\linewidth]{../Graficas/Distancias.pdf}
\end{figure}


\subsection{Dependencia con el ángulo}

\begin{center}
\begin{table}[H]
\scriptsize
\caption{Medidas de coincidencias a una distancia $33.3$ cm entre los detectores a diferentes ángulos. Factor de cobertura al ángulo de $k=3$.}
\label{Tab:angulo}
\begin{tabular}{cccccccccccccccccccccc}
\toprule
$\theta$ ($^\circ$) & $N_1$  & $N_2$ & $N_{12}$ & $t$ [s] & $n_1$ [s$^{-1}$] & $n_2$  [s$^{-1}$] & $n_{acc}$ [s$^{-1}$] & $n_{r}$ [s$^{-1}$] \\
\midrule
\num{0.00} & \num{44041.0000000000(209.8594767934)} & \num{18297.0000000000(135.2664038111)} & \num{324.0000000000(18.0000000000)} & \num{172.8000000000(0.1300000000)} & \num{254.8668981481(1.2295074061)} & \num{105.8854166667(0.7868344241)} & \num{0.7415941781(0.0157292873)} & \num{1.1334058219(0.1053569869)} \\
\num{10.0000000000(0.8400000000)} & \num{55792.0000000000(236.2033022631)} & \num{25452.0000000000(159.5368296037)} & \num{399.0000000000(19.9749843554)} & \num{227.8300000000(0.1300000000)} & \num{244.8843435895(1.0461263694)} & \num{111.7148751262(0.7031404583)} & \num{0.7517765118(0.0158536022)} & \num{0.9995292864(0.0891023711)} \\
\num{25.0000000000(0.8400000000)} & \num{51215.0000000000(226.3073131828)} & \num{22894.0000000000(151.3076336475)} & \num{331.0000000000(18.1934053987)} & \num{210.6000000000(0.1300000000)} & \num{243.1861348528(1.0850181877)} & \num{108.7084520418(0.7215867426)} & \num{0.7264719499(0.0153462314)} & \num{0.8452279552(0.0877462853)} \\
\num{45.0000000000(0.8400000000)} & \num{65255.0000000000(255.4505823051)} & \num{27374.0000000000(165.4508990607)} & \num{316.0000000000(17.7763888346)} & \num{259.8700000000(0.1300000000)} & \num{251.1063223920(0.9909874155)} & \num{105.3372840266(0.6388449225)} & \num{0.7268695779(0.0153046083)} & \num{0.4891230338(0.0700987542)} \\
\num{60.0000000000(0.8400000000)} & \num{52123.0000000000(228.3046210658)} & \num{23612.0000000000(153.6619666671)} & \num{303.0000000000(17.4068951855)} & \num{219.3800000000(0.1300000000)} & \num{237.5923055885(1.0501617100)} & \num{107.6305953141(0.7033352378)} & \num{0.7027240915(0.0148373861)} & \num{0.6784410101(0.0807253596)} \\
\num{67.0000000000(0.8400000000)} & \num{56803.0000000000(238.3337995333)} & \num{27445.0000000000(165.6653252796)} & \num{315.0000000000(17.7482393493)} & \num{218.2500000000(0.1300000000)} & \num{260.2657502864(1.1029711076)} & \num{125.7502863688(0.7627488804)} & \num{0.8993789775(0.0189496636)} & \num{0.5439199916(0.0835037875)} \\
\num{77.0000000000(0.8400000000)} & \num{68655.0000000000(262.0209915255)} & \num{29397.0000000000(171.4555335940)} & \num{301.0000000000(17.3493515729)} & \num{267.5000000000(0.1300000000)} & \num{256.6542056075(0.9874271286)} & \num{109.8953271028(0.6431764675)} & \num{0.7750760897(0.0163044246)} & \num{0.3501575552(0.0668776062)} \\
\num{90.0000000000(0.8400000000)} & \num{72191.0000000000(268.6838290631)} & \num{30162.0000000000(173.6721048413)} & \num{292.0000000000(17.0880074906)} & \num{280.0100000000(0.1300000000)} & \num{257.8157922931(0.9669875466)} & \num{107.7175815149(0.6222482657)} & \num{0.7631551487(0.0160463202)} & \num{0.2796647506(0.0631026352)} \\
\bottomrule
\end{tabular}
\end{table}
\end{center}


\begin{figure}[h!] \centering
    \includegraphics[width=0.5\linewidth]{../Graficas/Angulos.pdf}
\end{figure}

\subsection{Montecarlo}


\appendix

\section{Tablas}



\printbibliography
\addcontentsline{toc}{section}{Bibliografía}


\end{document}