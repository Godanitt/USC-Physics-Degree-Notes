\begin{lstlisting}[language=C++, caption={Simulación Monte Carlo de eficiencia de detección en C++ROOT \cite{Root}. Formato \href{https://ctan.org/pkg/listings}{lstlisting} basado en \href{https://gist.github.com/mpdehnel/cc43e89a09c18ef6a602}{
mpdehnel}} Martin Dehnel-Wil.]
#include <iostream>
#include <fstream>
#include <sstream>
#include <vector>
#include <cmath>
#include <string>
#ifndef M_PI
#define M_PI 3.14159265358979323846
#endif
#include "TRandom3.h"
#include "TCanvas.h"
#include "TGraphErrors.h"
#include "TGraph.h"
#include "TLegend.h"
#include "TAxis.h"
#include "TROOT.h"
#include "TApplication.h"
#include "TString.h"

const int Nparticles = 10000000;
const double detectorRadius = 10.0;

void SetEstiloPublicacion() {
    gStyle->SetOptStat(0);
    gStyle->SetOptTitle(0);
    gStyle->SetTitleFont(42, "XYZ");
    gStyle->SetLabelFont(42, "XYZ");
    gStyle->SetTitleSize(0.05, "XYZ");
    gStyle->SetLabelSize(0.045, "XYZ");
    gStyle->SetFrameLineWidth(1.3);
    gStyle->SetLineWidth(2.3);
    gStyle->SetHistLineWidth(1.3);
    gStyle->SetTickLength(0.03, "X");
    gStyle->SetTickLength(0.03, "Y");
}

bool LeerDatosCSV(const char* filename,
                  std::vector<double> &dist, std::vector<double> &errDist,
                  std::vector<double> &rate, std::vector<double> &errRate)
{
    std::ifstream infile(filename);
    if (!infile.is_open()) {
        std::cerr << "No se pudo abrir el archivo: " << filename << std::endl;
        return 1;
    }

    std::string linea;
    std::getline(infile, linea); // saltar cabecera

    std::vector<double> x, sx, y, sy;

    while (std::getline(infile, linea)) {
        std::stringstream ss(linea);
        std::string campo;
        std::vector<double> fila;

        while (std::getline(ss, campo, ',')) {
            fila.push_back(std::stod(campo));
        }

        if (fila.size() == 4) {
            x.push_back(fila[0]);
            sx.push_back(fila[1]);
            y.push_back(fila[2]);
            sy.push_back(fila[3]);
        }
    }

    dist = x; errDist = sx;
    rate = y; errRate = sy;

    return true;
}

double SimularEficiencia(double n, double d, TRandom3 &randGen)
{
    long long countCoincident = 0;
    for (int i = 0; i < Nparticles; ++i)
    {
        double x1 = randGen.Uniform()*30;
        double y1 = randGen.Uniform()*10;
        double r = detectorRadius * std::sqrt(std::pow(x1,2)+std::pow(y1,2));
        double u2 = randGen.Uniform();
        double cosTheta = std::pow(u2, 1.0 / (n + 2.0));
        double sinTheta = std::sqrt(1 - cosTheta * cosTheta);
        double phi_dir = randGen.Uniform(0, 2 * M_PI);
        double x2 = x1 + d * (sinTheta * std::cos(phi_dir));
        double y2 = y1 + d * (sinTheta * std::sin(phi_dir));
        if (x2 > 0 && x2 < 30 && y2 > 0 && y2 < 10) {
            countCoincident++;
        }
    }
    return (double)countCoincident / (double)Nparticles;
}

int Montecarlo()
{
    const char* nombreArchivo = "Distancias.csv";

    std::vector<double> dist, errDist, rate, errRate;
    if (!LeerDatosCSV(nombreArchivo, dist, errDist, rate, errRate)) return 1;
    int Npoints = dist.size();
    if (Npoints == 0) return 1;

    TRandom3 randGen;
    randGen.SetSeed(0);
    double n1 = 2.0;
    double n2 = 4.63;
    std::vector<double> eff_n1, eff_n2;
    eff_n1.reserve(Npoints);
    eff_n2.reserve(Npoints);

    for (int i = 0; i < Npoints; ++i) {
        eff_n1.push_back(SimularEficiencia(n1, dist[i], randGen));
        eff_n2.push_back(SimularEficiencia(n2, dist[i], randGen));
    }

    double chi2_n1 = 0.0, chi2_n2 = 0.0;
    double scale_n1 = 1.0, scale_n2 = 1.0;

    {
        double num = 0.0, den = 0.0;
        for (int i = 0; i < Npoints; ++i) {
            double sigma = (errRate[i] == 0.0 ? 1.0 : errRate[i]);
            num += rate[i] * eff_n1[i] / (sigma * sigma);
            den += eff_n1[i] * eff_n1[i] / (sigma * sigma);
        }
        if (den != 0.0) scale_n1 = num / den;

        for (int i = 0; i < Npoints; ++i) {
            double sigma = (errRate[i] == 0.0 ? 1.0 : errRate[i]);
            double diff = eff_n1[i] * scale_n1 - rate[i];
            chi2_n1 += (diff * diff) / (sigma * sigma);
        }
    }

    {
        double num = 0.0, den = 0.0;
        for (int i = 0; i < Npoints; ++i) {
            double sigma = (errRate[i] == 0.0 ? 1.0 : errRate[i]);
            num += rate[i] * eff_n2[i] / (sigma * sigma);
            den += eff_n2[i] * eff_n2[i] / (sigma * sigma);
        }
        if (den != 0.0) scale_n2 = num / den;

        for (int i = 0; i < Npoints; ++i) {
            double sigma = (errRate[i] == 0.0 ? 1.0 : errRate[i]);
            double diff = eff_n2[i] * scale_n2 - rate[i];
            chi2_n2 += (diff * diff) / (sigma * sigma);
        }
    }

    std::cout << "Resultado del ajuste:\n";
    std::cout << "  n = " << n1 << " -> escala = " << scale_n1 << ", chi-cuadrado = " << chi2_n1 << std::endl;
    std::cout << "  n = " << n2 << " -> escala = " << scale_n2 << ", chi-cuadrado = " << chi2_n2 << std::endl;

    TGraphErrors *grData = new TGraphErrors(Npoints, dist.data(), rate.data(), errDist.data(), errRate.data());
    grData->GetXaxis()->SetTitle("d_{real} [cm]");
    grData->GetYaxis()->SetTitle("n_{r} [1/s]");
    grData->SetMarkerStyle(20);
    grData->SetMarkerColor(kBlack);
    grData->SetLineColor(kBlack);
    grData->SetLineWidth(1);

    int NcurvePoints = 100;
    double d_min = *std::min_element(dist.begin(), dist.end());
    double d_max = *std::max_element(dist.begin(), dist.end());
    double step = (d_max - d_min) / (NcurvePoints - 1);
    std::vector<double> curve_d, curve_sim1, curve_sim2;

    for (int j = 0; j < NcurvePoints; ++j)
    {
        double d_val = d_min + j * step;
        double eff1 = SimularEficiencia(n1, d_val, randGen);
        double eff2 = SimularEficiencia(n2, d_val, randGen);
        curve_d.push_back(d_val);
        curve_sim1.push_back(eff1 * scale_n1);
        curve_sim2.push_back(eff2 * scale_n2);
    }

    TGraph *grSim1 = new TGraph(NcurvePoints, curve_d.data(), curve_sim1.data());
    TGraph *grSim2 = new TGraph(NcurvePoints, curve_d.data(), curve_sim2.data());
    grSim1->SetLineColor(kRed); grSim1->SetLineWidth(2); grSim1->SetLineStyle(1);
    grSim2->SetLineColor(kBlue); grSim2->SetLineWidth(2); grSim2->SetLineStyle(2);

    TCanvas *c1 = new TCanvas("c1", "Simulacion vs Datos", 800, 600);
    c1->SetGrid();
    grData->Draw("AP");
    grSim1->Draw("L SAME");
    grSim2->Draw("L SAME");

    TLegend *legend = new TLegend(0.55, 0.7, 0.88, 0.85);
    legend->SetBorderSize(0); legend->SetFillColor(0);
    legend->AddEntry(grData, "Datos experimentales", "PE");
    legend->AddEntry(grSim1, TString::Format("Simulacion n=%.2f", n1), "L");
    legend->AddEntry(grSim2, TString::Format("Simulacion n=%.2f", n2), "L");
    legend->Draw();

    c1->SetRightMargin(0.03);
    c1->SetTopMargin(0.03);
    c1->Update();
    c1->SaveAs("../Graficas/Montecarlo.pdf");

    std::ofstream outfile("../Datos Crudos/Eficiencias.csv");
    if (!outfile.is_open()) {
        std::cerr << "No se pudo crear el archivo de salida Montecarlo_resultados.csv" << std::endl;
    } else {
        outfile << "distancia,sim_n1,err_sim_n1,sim_n2,err_sim_n2\n";
        for (int i = 0; i < Npoints; ++i) {
            double e1 = eff_n1[i], e2 = eff_n2[i];
            double sim_n1 = e1 * scale_n1;
            double sim_n2 = e2 * scale_n2;
            double err_sim_n1 = scale_n1 * std::sqrt(e1 * (1.0 - e1) / Nparticles);
            double err_sim_n2 = scale_n2 * std::sqrt(e2 * (1.0 - e2) / Nparticles);
            outfile << dist[i] << "," << sim_n1 << "," << err_sim_n1 << ","
                    << sim_n2 << "," << err_sim_n2 << "\n";
        }
        outfile.close();
        std::cout << ">> Resultados guardados en Montecarlo_resultados.csv" << std::endl;
    }

    return 0;
}
\end{lstlisting}
