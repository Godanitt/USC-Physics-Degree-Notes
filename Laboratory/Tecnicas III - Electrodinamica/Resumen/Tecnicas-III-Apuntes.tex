\documentclass[12pt,a4paper]{article}
\usepackage[utf8]{inputenc}
\usepackage[spanish]{babel}
\usepackage{amsmath}
\usepackage{amsfonts}
\usepackage{amssymb}
\usepackage{latexsym}
\usepackage{makeidx}
\usepackage{graphics}
\usepackage{lmodern}
\usepackage{hyperref}
\usepackage{subcaption}
\usepackage{pgfplots}
\usepackage{dsfont}
\usepackage{multicol}
\usepackage{xcolor}
\usepackage{booktabs}
\usepackage{float}
\usepackage{subcaption}
\pgfplotsset{width=10cm,compat=1.9}
\usepgfplotslibrary{external}
\usepackage{graphicx}
\usepackage{wrapfig}
\author{Daniel Vázquez Lago}
\usepackage{fancybox}
\title{Apuntes Tecnicas III: Electrodinamica}

\numberwithin{equation}{section}
\numberwithin{figure}{section}



\setlength{\parindent}{15px}
\usepackage[left=2.25cm,right=2cm,top=4cm,bottom=2cm]{geometry}


\newcommand{\parentesis}[1]{\left( #1  \right)}
\newcommand{\parciales}[2]{\frac{\partial #1}{\partial #2}}
\newcommand{\pparciales}[2]{\parentesis{\parciales{#1}{#2}}}
\newcommand{\ccorchetes}[1]{\left[ #1  \right]}
\newcommand{\D}{\mathrm{d}}
\newcommand{\derivadas}[2]{\frac{\D #1}{\D #2}}
\newcommand{\sech}{\mathrm{sech} \ }
\newcommand{\csch}{\mathrm{csch} \ }
\newcommand{\cotanh}{\mathrm{cotanh}}
\newcommand{\cotan}{\ \mathrm{cotan}}
\newcommand{\Res}{\mathrm{Res}}
\newcommand{\Arg}{\mathrm{arg}}
\newcommand{\Real}{\mathrm{Re}}
\newcommand{\tquad}{\quad \quad \quad}
\newcommand{\dB}{\mathrm{dB}}


\newcommand{\rota}{\nabla \times}
\newcommand{\dive}{\nabla \cdot}

\newcommand{\Vin}{V_{in}}
\newcommand{\Vout}{V_{out}}


\newtheorem{theorem}{Teorema}[section]
\newtheorem{corollary}{Corolario}[theorem]
\newtheorem{lemma}{Lema}[section]
\newtheorem{ejemplo}{Ejemplo}[section]

\begin{document}

\maketitle

\newpage

\tableofcontentsa

\newpage

\section{Practica I: Guia de Ondas}

Supongamos que tenemos un generador de ondas electromangéticas en el principio de un tubo metálico vacío en su interior, que a partir de ahora llamaremos \textit{guida de ondas}. En este práctica vamos a estudiar con precisión que ocurre al cambiar el final de la guía, que ocurre cuando trasformamos las condiciones de contorno en las que viven las ondas dentro de la guía. Estudiaremos 3 casos principales: un conductor metálico al final de la guía, una apertura directa el aire y colocar una bucina. \\

Además de esto estudiaremos otros fenómenos como que ocurre fuera de la guía, que ocurre cuando colocamos un metal a una distancia $d$ de la apertura dejando parte al aire, cuando cambiamos las condiciones dentro de la guía...



\subsection{Parte I: condiciones de contorno}

Como hemos dicho la principal misión de esta práctica es estudiar el comportamiento de la onda bajo varias condiciones de contorno. Estas condiciones de contorno que obligaremos a que cumpla a la onda serán parecidas a las condiciones que sufre la cuerda de una guitarra: ¿Que ocurre cuando impedimos el movimiento en dos puntos? Pues que se formarán ondas estacionarias. ¿Que pasa si cambiamos el material de la cuerda de un punto a otro? Que aparecerá una onda reflejada y una trasmitida. ¿Que pasa cuando el cambio de material es progresivo? Pues que las ondas reflejadas se irán cancelando...

Exactamente lo mismo ocurrirá 





\section{Practica II: Linea de Trasmision}


\end{document}